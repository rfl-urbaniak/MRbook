% Options for packages loaded elsewhere
\PassOptionsToPackage{unicode}{hyperref}
\PassOptionsToPackage{hyphens}{url}
\PassOptionsToPackage{dvipsnames,svgnames,x11names}{xcolor}
%
\documentclass[
  10pt,
  dvipsnames,enabledeprecatedfontcommands]{scrartcl}
\usepackage{amsmath,amssymb}
\usepackage{lmodern}
\usepackage{iftex}
\ifPDFTeX
  \usepackage[T1]{fontenc}
  \usepackage[utf8]{inputenc}
  \usepackage{textcomp} % provide euro and other symbols
\else % if luatex or xetex
  \usepackage{unicode-math}
  \defaultfontfeatures{Scale=MatchLowercase}
  \defaultfontfeatures[\rmfamily]{Ligatures=TeX,Scale=1}
\fi
% Use upquote if available, for straight quotes in verbatim environments
\IfFileExists{upquote.sty}{\usepackage{upquote}}{}
\IfFileExists{microtype.sty}{% use microtype if available
  \usepackage[]{microtype}
  \UseMicrotypeSet[protrusion]{basicmath} % disable protrusion for tt fonts
}{}
\usepackage{xcolor}
\usepackage{graphicx}
\makeatletter
\def\maxwidth{\ifdim\Gin@nat@width>\linewidth\linewidth\else\Gin@nat@width\fi}
\def\maxheight{\ifdim\Gin@nat@height>\textheight\textheight\else\Gin@nat@height\fi}
\makeatother
% Scale images if necessary, so that they will not overflow the page
% margins by default, and it is still possible to overwrite the defaults
% using explicit options in \includegraphics[width, height, ...]{}
\setkeys{Gin}{width=\maxwidth,height=\maxheight,keepaspectratio}
% Set default figure placement to htbp
\makeatletter
\def\fps@figure{htbp}
\makeatother
\setlength{\emergencystretch}{3em} % prevent overfull lines
\providecommand{\tightlist}{%
  \setlength{\itemsep}{0pt}\setlength{\parskip}{0pt}}
\setcounter{secnumdepth}{5}
\newlength{\cslhangindent}
\setlength{\cslhangindent}{1.5em}
\newlength{\csllabelwidth}
\setlength{\csllabelwidth}{3em}
\newlength{\cslentryspacingunit} % times entry-spacing
\setlength{\cslentryspacingunit}{\parskip}
\newenvironment{CSLReferences}[2] % #1 hanging-ident, #2 entry spacing
 {% don't indent paragraphs
  \setlength{\parindent}{0pt}
  % turn on hanging indent if param 1 is 1
  \ifodd #1
  \let\oldpar\par
  \def\par{\hangindent=\cslhangindent\oldpar}
  \fi
  % set entry spacing
  \setlength{\parskip}{#2\cslentryspacingunit}
 }%
 {}
\usepackage{calc}
\newcommand{\CSLBlock}[1]{#1\hfill\break}
\newcommand{\CSLLeftMargin}[1]{\parbox[t]{\csllabelwidth}{#1}}
\newcommand{\CSLRightInline}[1]{\parbox[t]{\linewidth - \csllabelwidth}{#1}\break}
\newcommand{\CSLIndent}[1]{\hspace{\cslhangindent}#1}
%\documentclass{article}

% %packages
 \usepackage{booktabs}
\usepackage{subcaption}
\usepackage{multirow}
\usepackage{colortbl}
\usepackage{graphicx}
\usepackage{longtable}
\usepackage{ragged2e}
\usepackage{etex}
%\usepackage{yfonts}
\usepackage{marvosym}
%\usepackage[notextcomp]{kpfonts}
\usepackage[scaled=0.86]{helvet}
\usepackage{nicefrac}
\newcommand*{\QED}{\hfill \footnotesize {\sc Q.e.d.}}
\usepackage{floatrow}
%\usepackage[titletoc]{appendix}
%\renewcommand\thesubsection{\Alph{subsection}}

\usepackage[textsize=footnotesize]{todonotes}
\newcommand{\inbook}[1]{\todo[color=gray!40]{#1}}
\newcommand{\mar}[1]{\todo[color=blue!40]{#1}}
\newcommand{\raf}[1]{\todo[color=olive!40]{#1}}
%\linespread{1.5}
\newcommand{\indep}{\!\perp \!\!\! \perp\!}


\setlength{\parindent}{10pt}
\setlength{\parskip}{1pt}


%language
\usepackage{times}
\usepackage{t1enc}
%\usepackage[utf8x]{inputenc}
%\usepackage[polish]{babel}
%\usepackage{polski}




%AMS
\usepackage{amsfonts}
\usepackage{amssymb}
\usepackage{amsthm}
\usepackage{amsmath}
\usepackage{mathtools}

\usepackage{geometry}
 \geometry{a4paper,left=35mm,top=20mm,}


%environments
\newtheorem{fact}{Fact}



%abbreviations
\newcommand{\ra}{\rangle}
\newcommand{\la}{\langle}
\newcommand{\n}{\neg}
\newcommand{\et}{\wedge}
\newcommand{\jt}{\rightarrow}
\newcommand{\ko}[1]{\forall  #1\,}
\newcommand{\ro}{\leftrightarrow}
\newcommand{\exi}[1]{\exists\, {_{#1}}}
\newcommand{\pr}[1]{\mathsf{P}(#1)}
\newcommand{\cost}{\mathsf{cost}}
\newcommand{\benefit}{\mathsf{benefit}}
\newcommand{\ut}{\mathsf{ut}}

\newcommand{\odds}{\mathsf{Odds}}
\newcommand{\ind}{\mathsf{Ind}}
\newcommand{\nf}[2]{\nicefrac{#1\,}{#2}}
\newcommand{\R}[1]{\texttt{#1}}
\newcommand{\prr}[1]{\mbox{$\mathtt{P}_{prior}(#1)$}}
\newcommand{\prp}[1]{\mbox{$\mathtt{P}_{posterior}(#1)$}}

\newcommand{\s}[1]{\mbox{$\mathsf{#1}$}}


\newtheorem{q}{\color{blue}Question}
\newtheorem{lemma}{Lemma}
\newtheorem{theorem}{Theorem}



%technical intermezzo
%---------------------

\newcommand{\intermezzoa}{
	\begin{minipage}[c]{13cm}
	\begin{center}\rule{10cm}{0.4pt}



	\tiny{\sc Optional Content Starts}
	
	\vspace{-1mm}
	
	\rule{10cm}{0.4pt}\end{center}
	\end{minipage}\nopagebreak 
	}


\newcommand{\intermezzob}{\nopagebreak 
	\begin{minipage}[c]{13cm}
	\begin{center}\rule{10cm}{0.4pt}

	\tiny{\sc Optional Content Ends}
	
	\vspace{-1mm}
	
	\rule{10cm}{0.4pt}\end{center}
	\end{minipage}
	}
%--------------------






















\newtheorem*{reply*}{Reply}
\usepackage{enumitem}
\newcommand{\question}[1]{\begin{enumerate}[resume,leftmargin=0cm,labelsep=0cm,align=left]
\item #1
\end{enumerate}}

\usepackage{float}

% \setbeamertemplate{blocks}[rounded][shadow=true]
% \setbeamertemplate{itemize items}[ball]
% \AtBeginPart{}
% \AtBeginSection{}
% \AtBeginSubsection{}
% \AtBeginSubsubsection{}
% \setlength{\emergencystretch}{0em}
% \setlength{\parskip}{0pt}






\usepackage[authoryear]{natbib}

%\bibliographystyle{apalike}



\usepackage{tikz}
\usetikzlibrary{positioning,shapes,arrows}

\ifLuaTeX
  \usepackage{selnolig}  % disable illegal ligatures
\fi
\IfFileExists{bookmark.sty}{\usepackage{bookmark}}{\usepackage{hyperref}}
\IfFileExists{xurl.sty}{\usepackage{xurl}}{} % add URL line breaks if available
\urlstyle{same} % disable monospaced font for URLs
\hypersetup{
  pdftitle={Second-order Probability, Accuracy and Weight of Evidence},
  pdfauthor={Rafal Urbaniak and Marcello Di Bello},
  colorlinks=true,
  linkcolor={Maroon},
  filecolor={Maroon},
  citecolor={Blue},
  urlcolor={blue},
  pdfcreator={LaTeX via pandoc}}

\title{Second-order Probability, Accuracy and Weight of Evidence}
\author{Rafal Urbaniak and Marcello Di Bello}
\date{January 31, 2023}

\begin{document}
\maketitle

{
\hypersetup{linkcolor=}
\setcounter{tocdepth}{2}
\tableofcontents
}
\vspace{2cm}

\noindent \textbf{DISCLAIMER:}
\textbf{This is a draft of work in progress, please do not cite or distribute without permission.}

\thispagestyle{empty}

\newpage

\begin{quote} \textbf{Abstract.}  \todo{need to write one when done}

\end{quote}

\inbook{this is what a comment about what will go into book looks like; will be globally supressed when generating latex for the journal paper, don't worry about deleting them.}

\hypertarget{introduction}{%
\section{Introduction}\label{introduction}}

\todo{see M'scomment in boldface}

\textbf{M's comment: This introductory section needs some work. This section is not well-grounded in the literature. Two main things.}

\textbf{FIRST: There is a big debate among forensic scientists about whether intervals shoud be used in the presentation of the value of the evidence. There was a 2016 special issue of Science and Justice on this topic. We should mention it. See, in particular, Stewart Morrison (2016) "Special issue on measuring and reporting the precision of forensic likelihood ratios: Introduction to the debate" and also, in that special issue, Ommen, Saunders, and Neumann (2016) "An argument against presenting interval quantifications as a surrogate for the value of evidence". Both papers were added to the folder. I am not sure our framing of the debate is faithful to the complexities that are brought up in the forensic science literature.}

\textbf{SECOND: The last paragraph at the end of the section in response to Kadane's remark seems too generic to me. If we are not proposing anything technically new but simply applying existing ideas, we need to cite the technical work we are relying on. Kadane mentioned "Bayesian hierarchical models". I don’t know what they are, but I looked up the wikipedia entry on the topic and talked to a statistician. They are a common thing in Bayesian statistics. So we should cite references discussing them and say how we are applying them (if we are just applying them) or say in what way our approach differs from them.}

A defendant in a criminal case may face multiple items of incriminating
evidence whose strength can at least sometimes be assessed using
probabilities. For example, consider a murder case in which the police
recover trace evidence that matches the defendant. Hair found at the
crime scene matches the defendant's hair (call this evidence
\textsf{hair}). In addition, the defendant owns a dog whose fur matches
the dog fur found in a carpet wrapped around one of the bodies (call
this evidence \textsf{dog}).\footnote{The hair evidence and the dog fur
  evidence are stylized after two items of evidence in the notorious
  1981 Wayne Williams case (Deadman, 1984b, 1984a).} The two matches
suggest that the defendant (and the defendant's dog) must be the source
of the crime traces (call this hypothesis \(\mathsf{source}\)). But how
strong is this evidence, really? What are the fact-finders to make of
it?

The standard story among legal probabilists goes something like this. To
evaluate the strength of the two items of match evidence, we must find
the value of the likelihood ratio:
\[\frac{\pr{\s{dog}\wedge \s{hair} \vert \s{source}}}{\pr{\s{dog}\wedge \s{hair} \vert \neg \s{source}}}\]
For simplicity, the numerator can be equated to one. To fill in the
denominator, an expert provides the relevant random match probabilities.
Suppose the expert testifies that the probability of a random person's
hair matching the reference sample is about 0.0253, and the probability
of a random dog's hair matching the reference sample happens to be about
the same, 0.0256.\footnote{Probabilities have been slightly but not
  unrealistically modified to be closer to each other in order to make a
  conceptual point. The original probabilities were 1/100 for the dog
  fur, and 29/1148 for Wayne Williams' hair. We modified the actual
  reported probabilities slightly to emphasize the point that we will
  elaborate further on: the same first-order probabilities, even when
  they sound precise, may come with different degrees of second-order
  uncertainty.} Presumably, the two matches are independent lines of
evidence. In other words, their random match probabilities must be
independent of each other conditional on the source hypothesis. Then, to
evaluate the overall impact of the evidence on the source hypothesis,
you calculate: \begin{align*}
\pr{\s{dog}\wedge \s{hair} \vert \neg \s{source}} & = \pr{\s{dog} \vert \neg \s{source}} \times \pr{\s{hair} \vert \neg \s{source}} \\
& =  0.0252613 \times  0.025641 = \ensuremath{6.4772626\times 10^{-4}}
\end{align*} This is a very low number. Two such random matches would be
quite a coincidence. Following our advice from Chapter 5, the expert
facilitates your understanding of how this low number should be
interpreted. They show you how the items of match evidence change the
probability of the source hypothesis given a range of possible priors
(Figure \ref{fig:impactOfPoint}). The posterior of .99 is reached as
soon as the prior is higher than 0.061.\footnote{These calculations
  assume that the probability of a match if the suspect and the
  suspect's dog are the sources is one.} While perhaps not sufficient
for outright belief in the source hypothesis, the evidence seems
extremely strong: a minor additional piece of evidence could make the
case against the defendant overwhelming.

\begin{figure}[H]

\begin{center}\includegraphics[width=0.6\linewidth]{paper-outline_files/figure-latex/impactOfPoint4-1} \end{center}
\caption{Impact of dog fur and human hair evidence on the prior, point estimates.}
\label{fig:impactOfPoint}
\end{figure}

Unfortunately, this analysis leaves out something crucial. You reflect
on what you have been told and ask the expert: how can you know the
random match probabilities with such precision? Shouldn't we also be
mindful of the uncertainty that may affect these numbers? The expert
agrees, and tells you that in fact the random match probability for the
hair evidence is based on 29 matches found in a database of size 1148,
while the random match probability for the dog evidence is based on
finding two matches in a reference database of size 78.

The expert's answer makes apparent that the precise random match
probabilities do not tell the whole story. Perhaps, the information
about sample sizes is good enough and now you know how to use the
evidence properly.\footnote{This is what, effectively, CITE TARONI seem
  to suggest when they insist the fact-finders should be simply given
  point estimates and information about the study set-up, such as sample
  size. As will transpire, we disagree.} But if you are like most human
beings, you can't. What to do, then?\\
\todo{added this bit to draw attention to this aspect of the Taroni debate, to come back to this}

You ask the expert for guidance: what are reasonable ranges of the
random match probabilities? What are the worst-case and best-case
scenarios? The expert responds with 99\% credible
intervals---specifically, starting with uniform priors, the ranges of
the random match probabilities are (.015,.037) for hair evidence and
(.002, .103) for fur evidence.\footnote{Roughly, the 99\% credible
  interval is the narrowest interval to which the expert thinks the true
  parameter belongs with probability .99. For a discussion of what
  credible intervals are, how they differ from confidence intervals, and
  why confidence intervals should not be used, see Chapter 3.} With this
information, you redo your calculations using the upper bounds of the
two intervals: \(.037\) and \(.103\). The rationale for choosing the
upper bounds is that these numbers result in random match probabilities
that are most favorable to the defendant. Your new calculation yields
the following: \begin{align*}
\mathsf{P}(\s{dog}\wedge \s{hair} \vert \neg \s{source})   & =  .037 \times .103 =.003811.
\end{align*} This number is around 5.88 times greater than the original
estimate. Now the prior probability of the source hypothesis needs to be
higher than 0.274 for the posterior probability to be above .99 (Figure
\ref{fig:impactOfCharitable}). So you are no longer convinced that the
two items of match evidence are strongly incriminating.

\begin{figure}[H]

\begin{center}\includegraphics[width=0.6\linewidth]{paper-outline_files/figure-latex/fig:charitableImpact7-1} \end{center}
\caption{Impact of dog fur and human hair evidence on the prior, charitable reading.}
\label{fig:impactOfCharitable}
\end{figure}

This result is puzzling. Are the two items of match evidence strongly
incriminating evidence (as you initially thought) or somewhat weaker (as
the new calculation suggests)? For one thing, using precise random match
probabilities might be too unfavorable toward the defendant. On the
other hand, your new assessment of the evidence based on the upper
bounds might be too \emph{favorable} toward them. Is there a middle way
that avoids overestimating and underestimating the strength of the
evidence?

To see what this middle path looks like, we should reconsider the
calculations you just did. You made an important blunder: you assumed
that because the worst-case probability for one event is \(x\) and the
worst-case probability for another independent event is \(y\), the
worst-case probability for their conjunction is \(xy\). But this
conclusion does not follow if the margin of error (credible interval) is
fixed. The intuitive reason is simple: just because the probability of
an extreme (or larger absolute) value \(x\) for one variable \(X\) is
.01, and so it is for the value \(y\) of another independent variable
\(Y\), it does not follow that the probability that those two
independent variables take values \(x\) and \(y\) simultaneously is the
same. This probability is actually much smaller. The interval
presentation instead of doing us good led us into error.

In general, it is impossible to calculate the credible interval for the
joint distribution based solely on the individual credible intervals
corresponding to the individual events. We need additional information:
the distributions that were used to calculate the intervals for the
probabilities of the individual events. In our example, if you
additionally knew, for instance, that the expert used beta distributions
(as, arguably, they should in this context), you could in principle
calculate the 99\% credible interval for the joint distribution. It
usually will not be the same as whatever the results of multiplication
of individual interval edges, and it is unlikely that a human
fact-finder would be able to correctly run such calculations in their
head even if they knew the functional form of the distributions used.
\footnote{Also, in principle, in more complex contexts, we need further
  information about how the items of evidence are related if we cannot
  take them to be independent.} So providing the fact-finder with
individual intervals, even if further information about the
distributions is provided, might easily mislead.\footnote{Investigation
  of the extent to which the individual interval presentation is
  misleading would be an interesting psychological study.}
\todo{Can you google to see if there is any such study?}

As it turns out, given the reported sample sizes, the 99\% credible
interval for the probability
\(\mathsf{P}(\s{dog}\wedge \s{hair} \vert \neg \s{source})\) is
\((0.000023, 0.002760)\). \todo{the fn was repetitive, compare to fn 5}

The upper bound of this interval would then require the prior
probability of the source hypothesis to be above .215 for the posterior
to be above .99. On this interpretation, the two items of match evidence
are still not quite as strong as you initially thought, but stronger
than what your second calculation indicated.

Still, the interval approach---even the corrected version just
outlined---suffers from a more general problem. Working with intervals
might be useful if the underlying distributions are fairly symmetrical.
But in our case, they might not be. For instance, Figure
\ref{fig:densities} depicts beta densities for dog fur and human hair,
together with sampling-approximated density for the joint evidence. The
distribution for the joint evidence is not symmetric. If you were only
informed about the edges of the interval, you would be oblivious to the
fact that the most likely value (and the bulk of the distribution,
really) does not simply lie in the middle between the edges. Just
because the parameter lies in an interval with some posterior
probability, it does not mean that the ranges near the edges of the
interval are equally likely---the bulk of the density might very well be
closer to one of the edges. Therefore, only relying on the edges can
lead one to either overestimate or underestimate the probabilities at
play. This also means that---following our advice on how to illustrate
the impact of evidence on prior probabilities---a better representation
of the dependence of the posterior on the prior should comprise multiple
possible sampled lines whose density mirrors the density around the
probability of the evidence (Figure \ref{fig:lines}).

\begin{figure}[H]

\begin{center}\includegraphics[width=0.8\linewidth]{paper-outline_files/figure-latex/fig:densities-1} \end{center}
\caption{Beta densities for individual items of evidence and the resulting joint density with .99 and .9 highest posterior density intervals, assuming the sample sizes as discussed and independence, with uniform priors.}
\label{fig:densities}
\end{figure}

\begin{figure}[H]

\begin{center}\includegraphics[width=0.6\linewidth]{paper-outline_files/figure-latex/fig:lines5-1} \end{center}

\caption{300 lines illustrating the uncertainty about the dependence of the posterior on the prior given aleatory uncertainty about the evidence, with the distribution of the minimal priors required for the posterior to be above .99.}

\label{fig:lines}

\end{figure}

This, then, is the main claim of this chapter: whenever density
estimates for the probabilities of interest are available (and they
should be available for match evidence and many other items of
scientific evidence if the reliability of a given type of evidence has
been properly studied), those densities should be reported for assessing
the strength of the evidence. This approach avoids hiding actual
aleatory uncertainties under the carpet. It also allows for a balanced
assessment of the evidence, whereas using point estimates or intervals
may exaggerate or underestimate the value of the evidence.

In what follows, we expand on this idea in different directions. Section
\ref{sec:three-probabilism} engages with the philosophical debate about
precise and imprecise probabilism. We argue that both options are
problematic and should be superseded by a higher-order approach to
probability whenever possible. Section \ref{sec:objections} revisits a
recent discussion in the forensic science literature. A prominent view
has it that trial experts, even when they use densities, should present
only first-order probabilities. We disagree and show that reasons of
accuracy maximization sometimes recommend relying on higher-order
probabilities. Section \ref{sec:legal-applications} turns to some legal
applications of higher-order probabilism. We focus on two topics: first,
the role of higher-order probabilities and false positive rates in the
evaluation of DNA evidence; second, how complex bodies of evidence can
be represented by what we call higher-order Bayesian networks.

Before we dive in, one more remark: ost of the time, mathematically, we
do not propose anything radically new---we just put together some of the
items from the standard Bayesian toolkit. The novelty is rather in our
arguing that that these tools are under-appreciated in the legal
scholarship and should be properly used to incorporate second-order
uncertainties in evidence evaluation and incorporation. Perhaps a minor
exception is our explication of the notion of weight, but even here many
related notions are available in information theory, and the novelty
here is not technical, but rather in the argument that they also are
under-appreciated in legal scholarship.
\todo{added this par to preemt Kadane's style pickiness}

\hypertarget{three-probabilisms}{%
\section{Three probabilisms}\label{three-probabilisms}}

\label{sec:three-probabilism}

\todo{see M'scomment in boldface}

\textbf{M's comment: this section looks good more or less, but now -- given Kadane's remark -- there is the question whether the novel proposal -- higher-order probabilism -- is really new. Is it just an application of hierarchical Bayesian models to discussions in philosophy? In some sense, "probabilism" in epistemology is an application of standard probability theory to questions in epistemology, so perhpas "higher order probabilism" is also an application of existing hiearchical bayesian models to problems in epistemology. If this is right, we need to make these connections clear. }

In introduction we sketched three probabilistic approaches that one
might take for assessing the value of the evidence presented at trial.
The first approach uses precise probabilities; the second uses
intervals; the third uses distributions over probabilities. By relying
on an example featuring two items of match evidence, we suggested that
the third approach is preferable. This section buttresses this claim by
providing principled, philosophical reasons in favor of the third
approach.

The three approaches we considered correspond (roughly) to three ways in
which probabilities can be deployed to model a rational agent's fallible
and evidence-based beliefs about the world. The first approach, known in
the philosophical literature as precise probabilism, posits that an
agent's credal state is modeled by a single, precise probability
measure. The second approach, known as imprecise probabilism, replaces
precise probabilities by sets of probability measures. The third
approach, what we call higher-order probabilism, relies on distributions
over parameter values. There are good reasons to abandon precise
probabilism and endorse higher-order probabilism. Imprecise probabilism
is a step in the right direction, but also suffers from too many
difficulties of its own.

\hypertarget{precise-probabilism}{%
\subsection{Precise probabilism}\label{precise-probabilism}}

Precise probabilism (\textsf{PP}) holds that a rational agent's
uncertainty about a hypothesis is to be represented as a single, precise
probability measure. This is an elegant and simple theory. But
representing our uncertainty about a proposition in terms of a single,
precise probability runs into a number of difficulties. Precise
probabilism fails to capture an important dimension of how our fallible
beliefs reflect the evidence we have (or have not) obtained. A couple of
stylized examples should make the point clear. (For the sake of
simplicity, we will use examples featuring coins, but biases of coins
can be thought of as random match probabilities in the forensic
context.)

\begin{quote}
\textbf{No evidence v. fair coin}
You are about to toss a coin, but have no evidence 
whatsoever about its bias. You are completely ignorant. 
Compare this to the situation in which you know, 
based on overwhelming evidence, that the coin is fair. 
\end{quote}

\noindent On precise probabilism, both scenarios are represented by
assigning a probability of .5 to the outcome \emph{heads}. If you are
completely ignorant, the principle of insufficient evidence suggests
that you assign .5 to both outcomes. Similarly, if you know for sure the
coin is fair, assigning .5 seems the best way to quantify the
uncertainty about the outcome. The agent's evidence in the two scenario
is quite different, but precise probabilities cannot capture this
difference.

\begin{quote}
\textbf{Learning from ignorance}
You toss a coin with unknown bias. You toss it 10 times and observe \emph{heads} 5 times. Suppose you toss it further and observe 50 \emph{heads} in 100 tosses. 
\end{quote}

\noindent Since the coin initially had unknown bias, you should
presumably assign a probability of .5 to both outcomes. After the 10
tosses, you end up again with an estimate of .5. You must have learned
something, but whatever that is, it is not modeled by precise
probabilities. When you toss the coin 100 times and observe 50 heads,
you learn something. But your precise probability assessment will again
be .5.

These examples suggest that precise probabilism is not appropriately
responsive to evidence. It ends up assigning the same probability in
situations in which one's evidence is quite different: when no evidence
is available about the coin's bias; when there is little evidence that
the coin is fair (say, after only 10 tosses); and when there is strong
evidence that the coin is fair (say, after 100 tosses). The general
problem is, precise probability captures the value around which your
uncertainty should be centered, but fails to capture how centered it
should be given the evidence.\footnote{Precise probabilism suffers from
  other difficulties. For example, it has problems with formulating a
  sensible method of probabilistic opinion aggregation Stewart \&
  Quintana (2018). A seemingly intuitive constraint is that if every
  member agrees that \(X\) and \(Y\) are probabilistically independent,
  the aggregated credence should respect this. But this is hard to
  achieve if we stick to \s{PP} (Dietrich \& List, 2016). For instance,
  a \emph{prima facie} obvious method of linear pooling does not respect
  this. Consider probabilistic measures \(p\) and \(q\) such that
  \(p(X) = p(Y) = p(X\vert Y) = 1/3\) and
  \(q(X) = q(Y) = q(X\vert Y) = 2/3\). On both measures, taken
  separately, \(X\) and \(Y\) are independent. Now take the average,
  \(r=p/2+q/2\). Then \(r(X\cap Y) = 5/18 \neq r(X)r(Y)=1/4\).}

\hypertarget{imprecise-probabilism}{%
\subsection{Imprecise probabilism}\label{imprecise-probabilism}}

What if we give up the assumption that probability assignments should be
precise? Imprecise probabilism (\textsf{IP}) holds that an agent's
credal stance towards a hypothesis is to be represented by means of a
\emph{set of probability measures}, typically called a representor
\(\mathbb{P}\), rather than a single measure \(\mathsf{P}\). The
representor should include all and only those probability measures which
are compatible with the evidence. For instance, if an agent knows that
the coin is fair, their credal state would be represented by the
singleton set \(\{\mathsf{P}\}\), where \(\mathsf{P}\) is a probability
measure which assigns \(.5\) to \emph{heads}. If, on the other hand, the
agent knows nothing about the coin's bias, their credal state would be
represented by the set of all probabilistic measures, since none of them
is excluded by the available evidence. Note that the set of probability
measures does not represent admissible options that the agent could
legitimately pick from. Rather, the agent's credal state is essentially
imprecise and should be represented by means of the entire set of
probability measures.\footnote{For the development of imprecise
  probabilism, see Keynes (1921); Levi (1974); Gärdenfors \& Sahlin
  (1982); Kaplan (1968); Joyce (2005); Fraassen (2006); Sturgeon (2008);
  Walley (1991). S. Bradley (2019) is a good source of further
  references. Imprecise probabilism shares some similarities with what
  we might call \textbf{interval probabilism} (Kyburg, 1961; Kyburg Jr
  \& Teng, 2001). On interval probabilism, precise probabilities are
  replaced by intervals of probabilities. On imprecise probabilism,
  instead, precise probabilities are replaced by sets of probabilities.
  This makes imprecise probabilism more general, since the probabilities
  of a proposition in the representor set do not have to form a closed
  interval. As we have already noted, intervals do not contain
  probabilistic information sufficient to guide reasoning with multiple
  items of evidence. So we focus on \s{IP}, which is the more promising
  approach.}

Imprecise probabilism, at least \emph{prima facie}, offers a
straightforward picture of learning from evidence, that is a natural
extension of the classical Bayesian approach. When faced with new
evidence \(E\) between time \(t_0\) and \(t_1\), the representor set
should be updated point-wise, running the standard Bayesian updating on
each probability measure in the representor:
\begin{align*} \label{eq:updateRepresentor}
\mathbb{P}_{t_1} = \{\mathsf{P}_{t_1}\vert \exists\, {\mathsf{P}_{t_0} \!\in  \mathbb{P}_{t_0}}\,\, \forall\, {H}\,\, \left[\mathsf{P}_{t_1}(H)=\mathsf{P}_{t_0}(H \vert E)\right] \}.
\end{align*}

\noindent The hope is that, if we start with a range of probabilities
that is not extremely wide, point-wise learning will behave
appropriately. For instance, if we start with a prior probability of
\emph{heads} equal to .4 or .6, then those measure should be updated to
something closer to \(.5\) once we learn that a given coin has already
been tossed ten times with the observed number of heads equal 5 (call
this evidence \(E\)). This would mean that if the initial range of
values was \([.4,.6]\) the posterior range of values should be more
narrow. But even this seemingly straightforward piece of reasoning is
hard to model without using densities. For to calculate
\(\pr{\s{heads}\vert E}\) we need to calculate
\(\pr{E \vert \s{heads}}\pr{\s{heads}}\) and divide it by
\(\pr{E} = \pr{E \vert \s{heads}}\pr{\s{heads}} + \pr{E} = \pr{E \vert \neg \s{heads}}\pr{\neg \s{heads}}\).
The tricky part is obtaining the conditional probabilities
\(\pr{E \vert \s{heads}}\) and \(\pr{E \vert \neg \s{heads}}\) in a
principled manner without explicitly going second-order, estimating the
parameter value and using beta distributions.

The situation is even more difficult if we start with complete lack of
knowledge, as imprecise probabilism runs into the problem of
\textbf{belief inertia} (Levi, 1980). Say you start tossing a coin
knowing nothing about its bias. The range of possibilities is \([0,1]\).
After a few tosses, if you observed at least one tail and one heads, you
can exclude the measures assigning 0 or 1 to \emph{heads}. But what else
have you learned? If you are to update your representor set point-wise,
you will end up with the same representor set. Consequently, the edges
of your resulting interval will remain the same. In the end, it is not
clear how you are supposed to learn anything if you start from complete
ignorance.\footnote{Here's another example from Rinard (2013). Either
  all the marbles in the urn are green (\(H_1\)), or exactly one tenth
  of the marbles are green (\(H_2\)). Your initial credence \([0,1]\) in
  each. Then you learn that a marble drawn at random from the urn is
  green (\(E\)). After conditionalizing each function in your
  representor on this evidence, you end up with the the same spread of
  values for \(H_1\) that you had before learning \(E\), and no matter
  how many marbles are sampled from the urn and found to be green.}

Some downplay the problem of belief inertia. They insist that vacuous
priors should not be used and that imprecise probabilism gives the right
results when the priors are non-vacuous. After all, if you started with
knowing truly nothing, then perhaps it is right to conclude that you
will never learn anything. Another strategy is to say that, in a state
of complete ignorance, a special updating rule should be
deployed.\footnote{Elkin (2017) suggests the rule of
  \emph{credal set replacement} that recommends that upon receiving
  evidence the agent should drop measures rendered implausible, and add
  all non-extreme plausible probability measures. This, however, is
  tricky. One needs a separate account of what makes a distribution
  plausible or not, as well as a principled account of why one should
  use a separate special update rule when starting with complete
  ignorance.} But no matter what we think about belief inertia, other
problems plague imprecise probabilism. Two more problems are
particularly pressing.

One problem is that imprecise probabilism fails to capture intuitions we
have about evidence and uncertainty in a number of scenarios. Consider
this example:

\begin{quote}
\textbf{Even v. uneven bias:}
 You have two coins and you know, for sure, that the probability of getting heads is .4, if you toss one coin, and .6, if you toss the other coin. But you do not know which is which. You pick one of the two at random and toss it.  Contrast this with an uneven case. You have four coins and you know that three of them have bias $.4$ and one of them has bias $.6$. You pick a coin at random and plan to toss it. You should be three times more confident that the probability of getting heads is .4. rather than .6.
\end{quote}

\noindent The first situation can be easily represented by imprecise
probabilism. The representor would contain two probability measures, one
that assigns .4. and the other that assigns .6 to the hypothesis `this
coin lands heads'. But imprecise probabilism cannot represent the second
situation, at least not without moving to higher-order probabilities or
assigning probabilities to chance hypotheses, in which case it is no
longer clear whether the object-level imprecision performs any valuable
task.\footnote{Other scenarios can be constructed in which imprecise
  probabilism fails to capture distinctive intuitions about evidence and
  uncertainty; see, for example, (Rinard, 2013). Suppose you know of two
  urns, \textsf{GREEN} and \textsf{MYSTERY}. You are certain
  \textsf{GREEN} contains only green marbles, but have no information
  about \textsf{MYSTERY}. A marble will be drawn at random from each.
  You should be certain that the marble drawn from \textsf{GREEN} will
  be green (\(G\)), and you should be more confident about this than
  about the proposition that the marble from \textsf{MYSTERY} will be
  green (\(M\)). In line with how lack of information is to be
  represented on \textsf{IP}, for each \(r\in [0,1]\) your representor
  contains a \(\mathsf{P}\) with \(\pr{M}=r\). But then, it also
  contains one with \(\pr{M}=1\). This means that it is not the case
  that for any probability measure \(\mathsf{P}\) in your representor,
  \(\mathsf{P}(G) > \mathsf{P}(M)\), that is, it is not the case that RA
  is more confident of \(G\) than of \(M\). This is highly
  counter-intuitive.}

Second, besides descriptive inadequacy, an even deeper, foundational
problem exists for imprecise probabilism. This problem arises when we
attempt to measure the accuracy of a representor set of probability
measures. Workable \emph{scoring rules} exists for measuring the
accuracy of a single, precise credence function, such as the Brier
score. These rules measure the distance between one's credence function
(or probability measure) and the actual value. A requirement of scoring
rules is that they be \emph{proper}: any agent will score their own
credence function to be more accurate than every other credence
function. After all, if an agent thought a different credence was more
accurate, they should switch to it. Proper scoring rules are then used
to formulate accuracy-based arguments for precise probabilism. These
arguments show (roughly) that, if your precise credence follows the
axioms of probability theory, no other credence is going to be more
accurate than yours whatever the facts are. Can the same be done for
imprecise probabilism? It seems not. Impossibility theorems demonstrate
that no proper scoring rules are available for representor sets. So, as
many have noted, the prospects for an accuracy-based argument for
imprecise probabilism look dim (Campbell-Moore, 2020; Mayo-Wilson \&
Wheeler, 2016; Schoenfield, 2017; Seidenfeld, Schervish, \& Kadane,
2012). Moreover, as shown by Schoenfield (2017), if an accuracy measure
satisfies certain plausible formal constraints, it will never strictly
recommend an imprecise stance, as for any imprecise stance there will be
a precise one with at least the same accuracy.

\hypertarget{higher-order-probabilism}{%
\subsection{Higher-order probabilism}\label{higher-order-probabilism}}

There is, however, a view in the neighborhood that fares better: a
second-order perspective. In fact, some of the comments by the
proponents of imprecise probabilism tend to go in this direction. For
instance, Seamus Bradley compares the measures in a representor to
committee members, each voting on a particular issue, say the true bias
of a coin. As they acquire more evidence, the committee members will
often converge on a specific chance hypothesis. He writes (S. Bradley,
2012, p. 157):

\begin{quote}
\dots the committee members are ''bunching up''. Whatever measure you put over the set of probability functions---whatever ''second order probability'' you use---the ''mass'' of this measure gets more and more concentrated around the true chance hypothesis'.
\end{quote}

\noindent Note, however, that such bunching up cannot be modeled by
imprecise probabilism. Joyce (2005), in a paper defending imprecise
probabilism, in fact uses a density over chance hypotheses to account
for the notion of evidential weight. The idea that one should use
higher-order probabilities has also been suggested by critics of
imprecise probabilism. For example, Carr (2020) argues that sometimes
evidence requires uncertainty about what credences to have. Carr,
however, does not articulate this suggestion more fully, does not
develop it formally, and does not explain how her approach would fare
against the difficulties affecting precise ad imprecise probabilism.

The key idea of the higher-order approach we propose is that uncertainty
is not a single-dimensional thing to be mapped on a single
one-dimensional scale such as a real line. It is the whole shape of the
whole distribution over parameter values that should be taken under
consideration.\footnote{Bradley admits this much (S. Bradley, 2012, p.
  90), and so does Konek (Konek, 2013, p. 59). For instance, Konek
  disagrees with: (1) \(X\) is more probable than \(Y\) just in case
  \(p(X)>p(Y)\), (2) \(D\) positively supports \(H\) if
  \(p_D(H)> p(H)\), or (3) \(A\) is preferable to \(B\) just in case the
  expected utility of \(A\) w.r.t. \(p\) is larger than that of \(B\).}
From this perspective, when an agent is asked about their credal stance
towards \(X\), they can refuse to summarize it in terms of a point value
\(\mathsf{P}(X)\). They can instead express their credal stance in terms
of a probability (density) distribution \(f_x\) treating
\(\mathsf{P}(X)\) as a random variable. To be sure, an agent's credal
state toward \(X\) could sometimes be usefully represented by the
expectation\\
\[\int_{0}^{1} x f(x) \, dx\] as the precise, object-level credence in
\(X\), where \(f\) is the probability density over possible object-level
probability values. But this need not always be the case. If the
probability density \(f\) is not sufficiently concentrated around a
single value, a one-point summary might fail to do justice to the
nuances of the agent's credal state.\footnote{This approach lines up
  with common practice in Bayesian statistics, where the primary role of
  uncertainty representation is assigned to the whole distribution.
  Summaries such as the mean, mode standard deviation, mean absolute
  deviation, or highest posterior density intervals are only succinct
  ways for representing the uncertainty of a given scenario. Whether the
  expectation should be used in betting behavior is a separate problem.
  Here we focus on epistemic issues.} For example, consider again the
scenario in which the agent knows that the bias of the coin is either .4
or .6 but the former is three times more likely. Representing the
agent's credal state with the expectation
\(\mathsf{P}(X) = .75 \times .4 + .25 \times .6 = .45\) would be
inadequate as it would fail to capture the agent's belief that the two
biases are uneven.

The higher-order approach can easily model all the challenging scenarios
we discussed so far in the manner illustrated in Figure
\ref{fig:evidenceResponse}. In particular, the scenario in which the two
biases of the coin are not equally likely---which imprecise probabilism
cannot model---can be easily modeled within high-order probabilism by
assigning different probabilities to the two biases.

\begin{figure}[t]

\begin{center}\includegraphics[width=0.8\linewidth]{paper-outline_files/figure-latex/fig:evidenceResponse2-1} \end{center}
\caption{Examples of higher-order distributions for scenarios brought up in the literature.}
\label{fig:evidenceResponse}
\end{figure}

Besides its flexibility in modelling uncertainty, higher-order
probabilism does not fall prey to belief inertia. Consider a situation
in which you have no idea about the bias of a coin. So you start with a
uniform density over \([0,1]\) as your prior. By using binomial
probabilities as likelihoods, observing any non-zero number of heads
will exclude 0 and observing any non-zero number of tails will exclude 1
from the basis of the posterior. The posterior distribution will become
more centered around the parameter estimate as the observations come in.
Figure \ref{fig:intertia2} shows---starting with a uniform prior
distribution--- how the posterior distribution changes after successive
observations of heads, heads again, and then tails.\footnote{More
  generally, learning about frequencies, assuming independence and
  constant probability for all the observations, is modeled the Bayes
  way. You start with some prior density \(p\) over the parameter
  values. If you start with complete lack of information, \(p\) should
  be uniform. Then, you observe the data \(D\) which is the number of
  successes \(s\) in a certain number of observations \(n\). For each
  particular possible value \(\theta\) of the parameter, the probability
  of \(D\) conditional on \(\theta\) follows the binomial distribution.
  The probability of \(D\) is obtained by integration. That is:
  \begin{align*}
  p(\theta \vert D) & = \frac{p(D\vert \theta)p(\theta)}{p(D)}\\
  & = \frac{\theta^s (1-\theta)^{(n - s)}p(\theta)}{\int (\theta')^s (1-\theta')^{(n - s)}p(\theta')\,\, d\theta'}.
  \end{align*}}

\begin{figure}[t]

\begin{center}\includegraphics[width=0.8\linewidth]{paper-outline_files/figure-latex/fig:inertia3-1} \end{center}
\caption{As observations of heads, heads and tails come in, extreme parameter values drop out of the picture and the posterior is shaped by the evidence.}
\label{fig:intertia2}
\end{figure}

A further advantage of high-order probabilism over imprecise probabilism
is that the prospects for accuracy-based arguments are not foreclosed.
This is a significant shortcoming of imprecise probabilism, especially
because such arguments exist for precise probabilism. One can show that
there exist proper scoring rules for higher-order probabilism. These
rules can then be used to formulate accuracy-based arguments. Another
interesting feature of the framework is that the point made by
Schoenfield against imprecise probabilism does not apply: there are
cases in which accuracy considerations recommend an imprecise stance
(that is, a multi-modal distribution) over a precise one (Urbaniak, 2022
manuscript).

All in all, higher-order probabilism outperforms both precise and
imprecise probabilism, at the descriptive as well as the normative
level. From a descriptive standpoint, higher-order probabilism can
easily model a variety of scenarios that cannot be adequately modeled by
the other versions of probabilism. From a normative standpoint, accuracy
maximization may sometimes recommend that a rational agent represent
their credal state with a distribution over probability values rather
than a precise probability measure (more on this in the next section).

\hypertarget{objections}{%
\section{Objections}\label{objections}}

\label{sec:objections}

This section addresses a number of conceptual difficulties that may
arise in using higher-order probabilities, with focus on those brought
up by prominent legal evidence scholars. In discussing these conceptual
issues, we will formulate an accuracy-based argument that higher-order
probabilities are preferable to precise probabilities.

\hypertarget{the-taroni-sjerps-debate}{%
\subsection{The Taroni-Sjerps debate}\label{the-taroni-sjerps-debate}}

Our treatment will be centered around a discussion initiated by Taroni,
Bozza, Biedermann, \& Aitken (2015), who argue extensively that trial
experts should avoid report higher-order densities, and should only
report point estimates. Their point of departure is a reflection on
match evidence.

Say an expert reports at trial that the sample from the crime scene
matches the defendant. The significance of this match should be
evaluated in light of the population frequency \(\theta\) of the
matching profile. This frequency, however, cannot be known for sure and
must instead be estimated.

The expert will estimate the true parameter \(\theta\) by means of a
probability distribution \(p(\theta)\) over its possible values. For
example, if the observations are realizations of independent and
identically distributed Bernoulli trails given \(\theta\), the expert's
uncertainty about \(\theta\) can be modeled as
\(\s{beta}(\alpha + s + 1 ,\beta + n - s)\), where \(s\) is the number
of observed successes, \(n\) the number of observations in the database
(1 is added to the first shape parameter to include the match with the
suspect), and \(\alpha\) and \(\beta\) reflect the expert's priors.

Nothing so far should be controversial. However, the question arises of
how the expert should report their own uncertainty about \(\theta\),
especially in the light of the usual practice of reporting likelihood
ratios.

To fix the notation, let the prosecution hypothesis \(H_p\) be that the
suspect is the source of the trace, and the defense hypothesis \(H_d\)
that another person, unrelated to the suspect, is the source. For
simplicity, assume that if \(H_p\) holds, the laboratory will surely
report a match \(M\), so that \(\pr{M\vert H_p}=1\). The likelihood
ratio, then, reduces to \(\nicefrac{1}{\pr{M \vert H_d}}\)---but given
that \(\theta\) was estimated using density over its possible values, it
is not obvious how a single value \(\pr{M \vert H_d}\) is to be obtained
and whether its use in the reporting does not hide the uncertainty
involved in the estimation of \(\theta\) under the carpet.

Taroni et al. (2015) claim that the point estimate for the match
evidence given the defense hypothesis should be calculated as follows:
\begin{align*}\pr{M \vert H_d} & = \int_{\theta} \pr{M\vert \theta} \pr{\theta}\,\, d\theta \\
& =  \int_\theta  \theta \pr{\theta}\,\, d\theta
\end{align*} In case of a DNA match, they recommend that the expert
report the expected value of the \(\s{beta}\) distribution, which
reduces to \(\nicefrac{\alpha + s + 1}{\alpha + \beta +n + 1}\). They
claim that this number satisfactorily expresses the posterior
uncertainty about \(\theta\). For them, it is this probability alone
that should be used in the denominator in the calculation and reporting
of the likelihood ratio.

Sjerps et al. (2015) disagree. In reporting a single value, the expert
would refrain from providing the fact-finders with relevant information
that can make a difference in the proper evaluation of the evidence.
There is a difference between (a) an expert who is certain \(\theta\) is
\(.1\); (b) an expert whose best estimate of \(\theta\) is \(.1\) based
on thousands of observations; and (c) an expert whose best estimate of
\(\theta\) is again \(.1\) but based on only ten observations.

These three scenarios mirror scenarios we discussed earlier: (a) the
bias of a coin is known for sure; (b) the bias is estimated on the basis
of a large number of tosses; and (c) the bias is estimated using a small
set of observations. As our critique of precise probabilism makes clear,
a simple point estimate (or precise probability) would fail to capture
the differences among the three scenarios. This concern might be
slightly mitigated by the fact that Taroni et al. (2015) admits that the
expert, besides providing a point estimate, should also informally
explain how the estimate was arrived at. They grant that this additional
information can be helpful so long as the recipients are instructed on
``the nature of probability, the importance of an understanding of it
and its proper use in dealing with uncertainty'' {[}p.~16{]}. But why
stop at an informal presentation? It is unclear why the fact-finders
should be deprived of quantifiable information about the aleatory
uncertainty of the parameter of interest and only be given an informal
description of what the expert did, along with some remarks about the
nature of probability. It is wildly optimistic to assume that an
informal description of how the point estimate has been arrived at is
enough to secure a proper assessment of the evidence. We hope to have
convinced the reader already in the introduction that informal treatment
and bare intuitions are not good enough even when it comes to the
evaluation of the impact of a rather simple combination of two items of
evidence if all the fact-finder has to go by is point estimates and and
informal description of how the estimates have been obtained.

Somewhat surprisingly, most of the concerns raised by Taroni et al.
(2015) are philosophical. They argue that if probabilities express an
agent's epistemic attitude towards a proposition probabilities are not
states of nature, but states of mind associated with individuals. They
think this claim has two consequences. First, it makes no sense sense to
talk about second-order uncertainty about subjective probabilities, as
there is no ``underlying state of the nature'' to estimate. Second, if
these subjective probabilities can be elicited by examining an agent's
betting preferences, a proper elicitation will lead to a single
number.\footnote{They write: ``Clearly, one can adjust the measure of
  belief of success in the reference gamble in such a way that one will
  be indifferent with respect to the truth of the event about which one
  needs to give one's probability. This understanding is fundamental, as
  it implies that probability is given by a single number. It may be
  hard to define, but that does not mean that probability does not exist
  in an individual's mind. One cannot logically have two different
  numbers because they would reflect different measures of belief.''
  (Taroni et al., 2015, p. 7)}

In response to the philosophical argument, Dahlman \& Nordgaard (2022)
have also emphasized that the distinction is not so clear-cut. They
argue that, if a probability assessment is a subjective attitude that is
elicited via a betting preference, a probability assessment is itself a
state of nature, ``the formation of a betting preference by a certain
person at a certain time'' {[}p.~15{]}. While we will have something to
say about the philosophical dimension of this debate, let us first
develop a less philosophically involved argument for the position taken
by Sjerps et al. (2015).

\hypertarget{an-accuracy-based-argument}{%
\subsection{An accuracy-based
argument}\label{an-accuracy-based-argument}}

\todo{see M'scomment in boldface}

\textbf{M's comment: This accuracy-based argument is evocative and intriguing, but what does it show really? What is the significance of using PMF based on a point estimate versus a posterior predictive PMF? Does this correspond to something that is done in court? How? The more interesting question is whether using higher-order probabilities reduces errors, say the rate of false convictions or false acquittals. Does it? If so, how?}

With this argument, we hope to break the stalemate in the debate by
proving an argument to which both parties should be receptive. It is an
accuracy-based argument in favor of using higher-order
probabilities---roughly, it says, if you discard relevant information
that you already have contained in the densities resulting from the
estimation and rely on point estimates only, your predictions about the
world will be less accurate in a very precise and quantifiable sense.

First, let us go over a particular example. Suppose we randomly draw a
true population frequency from the uniform distribution. In our
particular case, we obtained 0.632. Then, we randomly draw a sample size
as a natural number between 10 and 20. In our particular case, it is 16.
Next, we simulate an experiment in which we draw that number of
observations from the true distribution. We observe 8 successes and use
this number to calculate the point estimate of the parameter, which is
0.5.

What is the probability mass function (PMF) for all possible outcomes of
an observation of the same size? Two PMF are initially relevant: first,
the true probability mass based on the true parameter; second, the
probability mass function based on the point estimate which is binomial
around the point estimate. This latter PMF, however, does not take into
account the uncertainty about the point estimate. To take this
uncertainty seriously, continuing our example, we take a sample
distribution of size 16 of possible parameter values from the posterior
\(\s{beta}(1+\s{successes}, 1+\s{sample size} - \s{successes})\)
distribution (we assume uniform prior for the sake of an example). Then,
we use this sample of parameter values to simulate observations, one
simulation for each parameter value in the sample. This simulation
yields the so-called \emph{posterior predictive distribution} (or
posterior predictive PMF), which instead of a point estimate, propagates
the uncertainty about the parameter value into the predictions about the
outcomes of possible observations. Finally, we take simulated
frequencies as our estimates of probabilities. This distribution is more
honest about uncertainty and wider than the one obtained using the point
estimate. The three PMFs are displayed in Figure
\ref{fig:posteriorPrediction}.

\begin{figure}[H]

\begin{center}\includegraphics[width=0.8\linewidth]{paper-outline_files/figure-latex/fig:posteriorPrediction2-1} \end{center}


\caption{Real probability mass, probability mass calculated using a point estimate, sampling distribution from the posterior, and the posterior predictive distribution based on this sampling distribution.}
\label{fig:posteriorPrediction}
\end{figure}

The PMF based on a point estimate is further off from the real PMF than
the posterior predictive distribution. For instance, if we ask about the
probability of the outcome being at least 9 successes, the true answer
is 0.7984, the point estimate PMF tells us it is 0.4056, while the
posterior predictive distribution gives a somewhat better guess at
0.4277. A similar thing happens when we ask about the probability of the
outcome being at most 9 successes. The true answer is 0.3681, the
point-estimate-based answer is 0.778, while the posterior predictive
distribution yields 0.7051. More generally, we can use an
information-theoretic measure, Kullback-Leibler divergence, to quantify
how far the point-estimate PMF and the posterior predictive PMF are from
the true
PMF.\footnote{A bit of explanation of this divergence measure. Suppose we are dealing with a variable $X$ with $n$ distinct possible discrete states $x_1, \dots, x_n$ and consider two probability mass functions $p$ and $q$ which express uncertainty about the true value of $X$ so that, say, on $p$, $\pr{X=x_i}=p_i$. First, the uncertainty of a given distribution $p$, its \emph{entropy}, is given by the sum of the logarithms of surprise $\nicefrac{1}{p_i}$ for all the possible values,  $H(p) = \sum x_i \log \frac{1}{p_i} = - \sum p_i \log p_i$.  Next, suppose events arise according to $p$, but we predict them
using $q$. The \emph{cross-entropy}  is then  $\mathsf{H}(p, q)  = \sum p_i \log(q_i)$. This value is going to be higher than the entropy of $p$ if $q$ is different from it. Think of it as the uncertainty involved in using $q$ to predict events that arise according to $p$.  Third, \emph{Kullback-Leibler}  divergence is the additional entropy introduced by
using $q$ instead of $p$ itself, that is, the difference between cross-entropy and entropy:
\begin{align*}
\mathsf{DKL}(p, q) & = H(p, q) - H(p)\\
&= - \sum p_i \log q_i   - \left(   - \sum p_i \log p_i \right) \\
& = - \sum p_i\left( \log q_i - \log p_i\right)\\
& =  \sum p_i\left( \log p_i - \log q_i\right)\\
& = \sum p_i \log \left( \frac{p_i}{q_i}\right)
\end{align*} \noindent  As it turns out, KL divergence is also the expected
difference in log probabilities. In particular, if
\(p=q\) we get
\(DKL(p,p) = \sum p_i (\log p_i - \log p_i) = 0\),
which works out as it intuitively should be.}

In our particular case, the former distance is 0.7905638 and the latter
is 0.5681121. The posterior predictive distribution is
information-theoretically closer to the true distribution.

This was just one example, but the phenomenon generalizes. We repeat the
simulation 1000 times, each time with a new true parameter, a new sample
size, and a new sample. Every time the three PMFs are constructed using
the methods we described and their KL divergence from the true
distribution is calculated. Figure \ref{fig:kldsPlots} displays the
empirical distribution of the results of such a simulation. A positive
value indicates that the distribution based on the point-estimate was
further from the true PMF than the posterior predictive distribution
based on the same observed sample. Notably, the mean difference is
0.865, the median difference is 0.044, and the distribution is
asymmetrical, as there are multiple cases of large differences favoring
posterior predictive distributions over point-based predictions. All in
all, accuracy-wise, point-estimate-based PMFs are systematically worse
than the posterior predictive distribution.

\begin{figure}[H]

\begin{center}\includegraphics[width=0.7\linewidth]{paper-outline_files/figure-latex/fig:kldsPlots-1} \end{center}
\caption{Differences in Kullback-Leibler divergencies from the true distributions, comparing the distributions obtained using point estimates and posterior predictive distributions. Positive values indicate the point-estimate-based PMF was further from the true distribution than the posterior predictive distribution.}
\label{fig:kldsPlots}
\end{figure}

\hypertarget{conceptual-issues}{%
\subsection{Conceptual issues}\label{conceptual-issues}}

\todo{M's comment in boldface}

\textbf{M's comments: this subsection looks muddled to me. It goes around in circles. I cannot follow the argument. What is the key point made in this subection? The key point seems to be this: just like we compare the plausibility/probability of propositions like "defendant was at crime scene" and "defedant was not at crime scenbe", we compare the plausibility/probability of propositions like "random match probability is .0001" and "random match probability of .0002". The second comparison requires higher-order probabilities. Is this the point? But Taroni is saying that we can average over all possible random match probabilities and get a point estimate. What do we say in response to that?}

Accuracy considerations aside, we will now engage with the more
conceptual points. Taroni et al. (2015) argue that since first-order
probabilities capture your uncertainty about a proposition of interest,
second-order probabilities are supposed to capture your uncertainty
about how uncertain you are, and that ``estimating'' your first-order
uncertainties is unnecessary. They think that you can simply figure out
your fair odds in a suitable bet on the proposition in question, and the
fair odds track your unique, first-order uncertainty without any
uncertainty about it. But this point can be questioned. For one thing,
the betting interpretation of probability is not
uncontroversial.\footnote{See textbooks in formal epistemology (D.
  Bradley, 2015; Titelbaum, 2020).} Even assuming the betting
interpretation, there seems to be nothing wrong in saying that sometimes
we are\todo{add ref to Williamson} uncertain about what we think the
fair bets are.\footnote{On a related note, the introspective axioms in
  epistemic logic---that is, if an agent knows (or doesn't know) \(p\),
  they also know that they (don't) know \(p\)---are by no means
  uncontroversial. See, for example, Williamson 2000 (chapter 5)'s
  argument against the KK principle of positive introspection.} But
admittedly, this answer while undermines the betting argument for the
sufficiency of point estimates, does not cast much light on what the
appropriate relatively uncontroversial interpretation of higher-order
uncertainty should be.
\todo{M's: I don't understand the argument here. What is a "relatively uncontroversial interpretation"?}

Think again about an expert who gathers information about the allelic
frequency \(f\) of DNA matches in an available database, and starts with
a defensible \s{beta} prior with parameters \(\alpha, \beta\). Say the
expert observes \(s\) matches in a database of size \(n\). So the
population relative frequency the experts is estimating should follow
the \(\s{beta}(\alpha + s + 1 ,\beta + n - s)\) distribution. So far,
nothing controversial happens---the expert is estimating the relevant
population frequency.

But subjective uncertainty that is to be reported by the expert, Taroni
et al. (2015) complain, is not about the frequency, but about their
attitude towards a proposition (supposedly expressing a ``state of
nature'')---and, they insist, it makes no sense for an agent to attach
uncertainty to their own uncertainty about a proposition.

Assuming the conditions are pristine (the expert has no modeling
uncertainty, rules out laboratory errors, and so on), the beta
distribution can be used to pretty directly inform the expert's
subjective uncertainty. But uncertainty about what? The (estimated)
population frequency, for instance, can underlie a probability
assignment to the proposition
\emph{a match is observed  if another person, unrelated to the suspect, is the source of the trace}.
Admittedly, if only this proposition is being considered, it is yet not
clear what second-order uncertainties would be uncertainties about. But
the expert also considers a continuum of propositions, each of the form
\emph{the true population frequency is $\theta$} for each
\(\theta\in [0,1]\). A density over \(\theta\) models the comparative
plausibility that the expert assigns to such propositions in light of
the
evidence.\footnote{Moreover, and normalization allows them to calculate their subjective probabilities for $\theta$ belonging to various sub-intervals of $[0,1]$.}
So if one were worried that there were no propositions that the expert
could be ``second-order'' uncertain about, there actually are plenty.
\todo{M: who was worried that there were no second-order propositions? What argument in the literature is this a response to? I cannot follow.}
In particular, if \(\theta\) is a population frequency, gauging which
density captures the extent to which the evidence justifies various
estimates of that frequency is the same as gauging the comparative
plausibility of the corresponding propositions about possible population
frequencies.\footnote{Perhaps, this should no longer be called
  ``estimation'', but the the connection with estimation is strong
  enough to justify this terminology. In the end, this is a verbal
  discussion that we will not get into.}

More generally, in many contexts, evidence justifies first-order
probability assignments (population frequency estimates) to various
degrees. Suppose there is no evidence about the bias of a coin. Then,
each first-order uncertainty about it would be equally (un)-justified.
(If you like to think in terms of bets, the evidence would give no
reason to prefer any particular odds as fair.) If, instead, we know the
coin is fair, the evidence clearly selects one preferred value, .5.
(Again, if you like the betting metaphor, 1:1 would be the unique
recommended betting odds.) But often the evidence is stronger than the
former case and weaker than the latter case. Consider, for example,
propositions about population frequencies in light of the results of
observations. In such circumstances, the evidence justifies different
values of first-order uncertainty to various degrees, and densities
simply capture the extent to which different first-order uncertainties
are supported by the evidence.

We conclude this section by examining two additional points raised by
Taroni et al. (2015). The first---which we already alluded to
earlier---is that first-order probabilities are not ``states of nature''
and so cannot be estimated. It is unclear why the authors insist that
only states of nature can be estimated. Mathematicians use approximate
methods to estimate answers to fairly abstract questions, not obviously
related to ``states of nature'', whatever these are. So, estimation
should make sense whenever there are some objective answers that we can
approximate to a greater or lesser extent. If there is some objectivity
to what the ideal evidence would support, or to the extent to which the
actual evidence supports various competing hypotheses, we can be more or
less wrong about such things, and so it is not implausible to say that
there is a clear sense in which we can estimate them.\footnote{Taroni et
  al. (2015) make the same point for likelihood ratios. They argue that
  there is no ``meaningful state of nature equivalent for the likelihood
  ratio in its entirety, as it is given by a ratio of two conditional
  probabilities?'' But if it is meaningful to estimate two conditional
  probabilities (that is, frequencies in the population), or to compare
  the relative plausibility of various propositions about them in terms
  of density, it is equally meaningful to estimate any function of the
  numbers involved. Otherwise it would also be meaningless to try to
  estimate the body mass index (BMI) of an average 21 years old male
  student in the USA just because BMI is a ratio of other quantities.
  There are reasons not to care about BMI, but it not being a state of
  nature because it is a function of other values is not one of them.}

\todo{added back the claim about not worrying about BMI, as some readers might be sensitive to anyone bringing BMI up}

Second, Taroni et al. (2015) argue that once we allow second-order
probability, we run into the threat of infinite regress. But do we?
Surely, they would agree that one can be uncertain about a statistical
model. But this can be the case even if this model spits out a point
estimate rather than a density. If you think the possibility of putting
uncertainty on top of propositions about possible values of a
first-order parameter leaves us in an epistemically hopeless situation,
you might have hard time explaining why your point estimation is in a
better situation. After all, if asking further questions about
probabilities up the hierarchy is always justified, we can keep asking
about the probability of a point-estimate-spitting model, the
probability of that probability, and so on.

Perhaps the problem at issue is just one of complexity. Admittedly,
second-order estimation is more complex than relying on point estimates.
But we hope to have convinced the reader this complexity is worth the
effort. What about more complex models going third-order? If a workable
approach can accomplish that---and the additional complexity pays
off---we are all for going third-order. The fact that more complex
models can always be built hardly lead us into a vicious infinite
regress. Rather, it is an indication that our models of uncertainty
can---in principle---always be improved.

\hypertarget{legal-applications}{%
\section{Legal Applications}\label{legal-applications}}

\label{sec:legal-applications}

\inbook{Add carpet evidence in the Wayne Williams case}

Our discussion so far has been mostly theoretical. We made a case that
higher-order probabilism outperforms precise probabilism on both
descriptive and normative grounds. We also staved off a number of
conceptual difficulties with going higher-order. It is time to extend
our discussion from the introduction to a further illustration of how
higher-order probabilism can be of service in evaluating evidence at
trial. We present here two examples.

\hypertarget{false-positives-in-dna-identification}{%
\subsection{False Positives in DNA
Identification}\label{false-positives-in-dna-identification}}

One important topic is that of errors in the process of DNA match
evidence evaluation. As already known, the probability of a false
positive caused by contamination, laboratory or evidence collection or
storage error has serious impact on the value of DNA match evidence. As
Thompson, Taroni, \& Aitken (2003) have shown, the probability of false
positives, even when seemingly low, has a non-negligible impact:

\begin{quote}
If, as commentators have suggested, the rate of false positives is between 1 in 100 and 1 in 1000, or even less, then one might argue that the jury can safely rule out
the prospect that the reported match in their case is due to error and can proceed to consider the probability of a coincidental match \dots this argument is fallacious and profoundly misleading \dots  the probability that a reported match occurred due to error in a particular case can be much higher, or lower, than the false positive probability. 
\end{quote}

\noindent We are particularly interested in the passing remark that the
rate of false positives is between 1 in 100 and 1 in 1000. This
difference is not negligible. The simplest option would be to use the
upper bound of the {[}0.001, 0.01{]} interval. This choice would be the
most favorable toward the defendant. But, as already noted in the
introduction, doing so would lead to an overly conservative evaluation
of the evidence. It is much preferable to have a sensible distribution
to work with.

To fix ideas, the posterior probability of the source hypothesis (\(S\))
conditional on the match evidence (\(E\)) is, with some idealization, as
follows:

\begin{align*}
\pr{S \vert E} &   =  \frac{\pr{E\vert S} \pr{S} } {\pr{E}}\\
& = \frac{\overbrace{\pr{E\vert S}}^1 \pr{S}}{\underbrace{\pr{E\vert S}}_1 \pr{S} + \underbrace{\pr{E \vert RM}}_1 \pr{RM} + \underbrace{\pr{E \vert FP}}_1 \pr{FP}} \\ & = \frac{\pr{S}}{\pr{S} + \pr{RM} + \pr{FP}} 
\end{align*}

\noindent For simplicity, the false negative rate is assumed to be zero,
or in other words, \(\pr{E\vert S} =1\). The other assumption is that
the evidence could come about if: (1) the source hypothesis is true; (2)
a random match (\(RM\)) occurred; or (3) a false positive match occurred
(\(FP\)).

Suppose the random match probability for the DNA match evidence is
rather low, say \(10^{-9}\), and there is no uncertainty associated with
this number. Consider now two ways of assessing the DNA match. First,
disregarding the possibility of a false positive---setting \(FP\) to
\(0\)---makes the match evidence appear extremely strong. In this case,
the minimal prior sufficient for the posterior to be above .99 is only
0.001, where the relation between the prior probabilities and the
posterior probabilities of the source hypothesis is given by the dashed
orange line in Figure \ref{fig:fplinesPlot}. What happens after taking
into account the possibility of a false positive match? This depends on
how this possibility of error is quantified. Assume the false positive
rate corresponds to the upper bound of the {[}0.001, 0.01{]} interval.
This assumption completely changes the assessment of the match evidence.
Now the posterior of \(.99\) is reached only if the prior is above .99.
The match evidence appears to be extremely weak. So which is it? As
already seen in the introduction, the point estimate exaggerates the
value of the match evidence, while using the upper bound of the false
positive rate has the opposite effect. What happens within the {[}0.001,
0.01{]} interval cannot be ignored.

To take into consideration the values within the edges, it would be best
to have a good density estimate of the false positive errors frequency,
as we should, if the issue had been properly studied. But we do not. For
now, we will illustrate the consequences of taking two different
approaches. On one approach, any value between the edges is considered
equally likely (and we add a little leeway on top). On another approach,
not all values are equally likely---for example, suppose you think it is
50\% likely that the false positive rate is below .0033. In addition,
suppose the distribution, while being centered closer to zero, is
long-tailed (we used a truncated normal distribution here). These two
distributions are displayed in Figure \ref{fig:fppdistros}. On both
approaches, we assume that the false positive rate is between 0.001 and
0.01 with 99\% certainty. The uniform distribution---which regards all
false positive rates in the interval as equally likely---leads to a
rather conservative evaluation of the match evidence, much more so than
the truncated normal distribution. This is apparent from Figure
\ref{fig:fppMinima} and \ref{fig:fplinesPlot}, which show the prior
probabilities of the source hypothesis needed to secure a posterior
probability above .99. Working with a distribution---more so if it is
not a uniform distribution---affords a more balanced assessment of the
evidence than simply relying on the edges of an interval.

The lingering question, however, is how these distributions can be
obtained. Admittedly, studies on false positives are limited and only
give an incomplete picture. More studies are needed. This does not mean,
however, that until then using point estimates and interval edges is
preferable. After deciding on the functional form of a
distribution---such as truncated normal or beta---only a few numbers
need to be elicited from experts for constructing a density.\footnote{For
  instance, assuming the distribution is a truncated normal, it is
  enough for the expert to assert that both the 99\% interval is as the
  one we used, and that they believe with more than 50\% confidence the
  false positive rates to be below \(.033\) for the curve to be
  determined.} Having to rely on such elicitation is not without
problems, but it is better than asking experts for single point
estimates and relying on these (O'Hagan et al., 2006).

\begin{figure}[H]



\begin{center}\includegraphics[width=0.8\linewidth]{paper-outline_files/figure-latex/fig:fppdistros-1} \end{center}


\caption{Two examples of assumptions about the false positive rates, both having pretty much the same 99\% highest density intervals. Left: all error rates are equally likely. Right: the most likely values are closer to 0, but also some high values while unlikely are possible.}

\label{fig:fppdistros}

\end{figure}

\begin{figure}[H]



\begin{center}\includegraphics[width=0.8\linewidth]{paper-outline_files/figure-latex/fig:fppMinima-1} \end{center}


\caption{The distribution of minimal priors sufficient for obtaining a posterior above .99 on the two distributions of false positive rates. The truncated normal distribution has its bulk towards the left, but at the same time has higher ratio of evens in which this posterior is never reached. }

\label{fig:fppMinima}

\end{figure}

\begin{figure}[H]

\begin{center}\includegraphics[width=0.8\linewidth]{paper-outline_files/figure-latex/fig:fplinesPlot3-1} \end{center}
\caption{Impact of prior on the posterior assumign two different densitites for false positive rates. Note how both the "pristine" error-free point estimate (orange) and the charitable version (blue) are quite far from where the bulks of the distributions in fact are. Note also how the trnormal density allows for even more charitable cases, which results from it being long-tailed.}
\label{fig:fplinesPlot}
\end{figure}

\hypertarget{higher-order-bayesian-networks}{%
\subsection{Higher-order Bayesian
Networks}\label{higher-order-bayesian-networks}}

The higher-order framework we are advocating is not only applicable to
the evaluation of individual pieces of evidence. Complex bodies of
evidence---for example, those represented by Bayesian networks---can
also be assessed using higher-order probabilities. One fairly
straightforward way to go about this is to stochastically generate
Bayesian networks using our uncertainty about the parameter values,
update with the evidence, and propagate uncertainty to approximate the
marginal posterior for nodes of interest.

As an illustration, let us start with a simplified Bayesian network
developed by Fenton \& Neil (2018). The network is reproduced in Figure
\ref{fig:scBNplot} and represents the key items of evidence in the
infamous British case R. v. Clark (EWCA Crim 54, 2000).\footnote{Sally
  Clark's first son died in 1996 soon after birth, and her second son
  died in similar circumstances a few years later in 1998. At trial, the
  pediatrician Roy Meadow testified that the probability that a child
  from such a family would die of Sudden Infant Death Syndrome (SIDS)
  was 1 in 8,543. Meadow calculated that therefore the probability of
  both children dying of SIDS was approximately 1 in 73 million. Sally
  Clark was convicted of murdering her infant sons. The conviction was
  reversed on appeal. The case of appeal was based on new evidence:
  signs of a potentially lethal disease were found in one of the bodies.}

In a Bayesian network the arrows depict direct relationships of
influence between variables, and nodes---conditional on their
parents---are taken to be independent of their non-descendants.
\textsf{Amurder} and \textsf{Bmurder} are binary nodes corresponding to
whether Sally Clark's sons, call them A and B, were murdered. These
nodes influence whether signs of disease (\textsf{Adisease} and
\textsf{Bdisease}) and bruising (\textsf{Abruising} and
\textsf{Bbruising}) were present. Also, since A's death preceded in time
B's death, whether A was murdered casts some light on the probability
that B was also murdered.

The choice of the probabilities in the network is quite specific, and it
is not clear where such precise values come from. The standard response
invokes \emph{sensitivity analysis}: a range of plausible values is
tested. As already discussed, this approach ignores the shape of the
underlying distributions. Sensitivity analysis does not make any
difference between probability measures (or point estimates) in terms of
their plausibility, but some will be more plausible than others.
Moreover, if the sensitivity analysis is guided by extreme values, these
might play an undeservedly strong role. These concerns can be addressed,
at least in part, by recourse to higher-order probabilities. In a
precise Bayesian network, each node is associated with a probability
table determined by a finite list of numbers (precise probabilities).
But suppose that, instead of precise numbers, we have densities over
parameter values for the numbers in the probability tables.\footnote{The
  densities of interests can then be approximated by (1) sampling
  parameter values from the specified distributions, (2) plugging them
  into the construction of the BN, and (3) evaluating the probability of
  interest in that precise BN. The list of the probabilities thus
  obtained will approximate the density of interest. In what follows we
  will work with sample sizes of 10k.}\\
An example of a higher-order Bayesian network for the Sally Clark case
is given in Figure \ref{fig:SCwithHOP}.

\begin{figure}[H]

\begin{center}\includegraphics[width=0.5\linewidth]{paper-outline_files/figure-latex/scBNplot2-1} \end{center}
\caption{Bayesian network for the Sally Clark case, with marginal prior probabilities.}
\label{fig:scBNplot}
\end{figure}

\begin{figure}[H]

\begin{center}\includegraphics[width=1.1\linewidth,height=2\textheight,angle=90]{paper-outline_files/figure-latex/SCwithHOP-1} \end{center}

\caption{Example of a higher-order Bayesian network for the Sally Clark Case.}
\label{fig:SCwithHOP}
\end{figure}

With the help of the higher-order Bayesian network, we can investigate
the impact of different items of evidence on Sally Clark's probability
of guilt (Figure \ref{fig:SCwithHOP2}). The starting point is the prior
density for the \s{Guilt} node (first graph). Next, the network is
updated with evidence showing signs of bruising on both children (second
graph). Next, the assumption that both children lack signs of
potentially lethal disease is added (third graph). Finally, we consider
the state of the evidence at the time of the appellate case: signs of
bruising existed on both children, but signs of lethal disease were
discovered only on the first child. Interestingly, in the strongest
scenario against Sally Clark (third graph), the median of the posterior
distribution is above .95, but the uncertainty around that median is
still too wide to warrant a conviction.\footnote{The lower limit of the
  89\% Highest Posterior Density Intervals (HPDI) is at .83.} This
underscores the fact that relying on point estimates can lead to
overconfidence. Paying attention to the higher-order uncertainty about
the first-order probability can make a difference to trial decisions.

\begin{figure}[H]

\begin{center}\includegraphics[width=0.9\linewidth]{paper-outline_files/figure-latex/SCwithHOP2-1} \end{center}


\caption{Impact of incoming evidence in the Sally Clark case.}
\label{fig:SCwithHOP2}
\end{figure}

One question that arises is how this approach relates to the standard
method of using likelihood ratios to report the value of the evidence.
On this approach, the conditional probabilities that are used in the
likelihood ratio calculations are estimated and come in a package with
an uncertainty about them. Accordingly, these uncertainties propagate:
to estimate the likelihood ratio while keeping track of the uncertainty
involved, we can sample probabilities from the selected distributions
appropriate for the conditional probabilities needed for the
calculations, then divide the corresponding samples, obtaining a sample
of likelihood ratios, thus approximating the density capturing the
recommended uncertainty about the likelihood ratio. Uncertainty about
likelihood ratio is just propagated uncertainty about the involved
conditional probabilities. For instance, we can use this tool to gauge
our uncertainty about the likelihood ratios corresponding to the signs
of bruising in son A and the presence of the symptoms of a potentially
lethal disease in son A (Figure \ref{fig:SClrs}).

\begin{figure}[H]


\begin{center}\includegraphics[width=0.9\linewidth]{paper-outline_files/figure-latex/SClrs-1} \end{center}

\caption{Likelihood ratios forbruising and signs of disease in child A in the Sally Clark case.}
\label{fig:SClrs}

\end{figure}

\hypertarget{weight-of-evidence}{%
\section{Weight of Evidence}\label{weight-of-evidence}}

\label{sec:weight}

\todo{M's comments in boldface.}

\textbf{M's comment: The discussion about weight seems the start of a new paper. We should probably work on the previous sections and make them more defensible and better grounded in the literature.}

After sketching how the legal applications of the higher-order approach
should go, we turn to another payoff of higher-order probabilism: it
allows us to develop a theory of the weight of evidence that outperforms
the existing proposals. We will start with an informal sketch of the
concept of the weight of evidence, as opposed to the balance of the
evidence. We will then explore a few attempts at modeling this idea,
first from the preciser's and then from the impreciser's perspective.
Finally, we will show that higher-order probabilism can offer a better
theory.

\hypertarget{examples-and-desiderata}{%
\subsection{Examples and Desiderata}\label{examples-and-desiderata}}

In the 1872 manuscript \emph{The Fixation of Belief} (W3 295), C. S.
Peirce makes the following observation about sampling from a bag of
beans that are either black or white:

\begin{quote} When we have drawn a thousand times, if about half have been white, we have great confidence in this result ... a confidence which would be entirely wanting if, instead of sampling the bag by 1000 drawings, we had done so by only two.
\end{quote}

\noindent In both cases, our best assessment of the probability that the
next draw will be a black bean is .5, but how sure we should be of that
assessment is quite different depending on whether it is based on two or
one thousands draws. In other words, the \emph{weight} of the evidence
seems much greater after drawing a thousand beans and finding out that
half are black, compared to drawing just two beans and again finding out
that half are black. The weight of the evidence is different in the two
cases, but its \emph{balance}---understood here as the empirical
proportion of black-to-white beans---is the same.\footnote{Similar
  remarks can be found in Peirce's 1878 \emph{Probability of Induction}.
  There, he also proposes to represent uncertainty by at least two
  numbers, the first depending on the inferred probability, and the
  second measuring the amount of knowledge obtained. For the latter,
  Peirce proposes to use some dispersion-related measure of error (but
  then suggests that an error of that estimate should also be estimated
  and so on, so that ideally more numbers representing errors would be
  needed).}

Peirce did not use the expression the weight of evidence (and, in fact,
he used the phrase to refer to the balance of evidence, W3 294) (Kasser,
2016). However, his remarks anticipated what came to be called weight of
evidence by Keynes in his 1921 \emph{A Treatise on Probability}:

\begin{quote}
As the relevant evidence at our disposal increases, the magnitude of the
probability of the argument may either increase or decrease, according as the new knowledge strengthens the unfavourable or the favourable evidence; but something seems to have increased in either case---we have a more substantial basis upon which to rest our conclusion. I express this by saying that an accession of new evidence increases the weight of an argument. (p. 71)
\end{quote}

\noindent The key point is the same (Levi, 2011): since the balance of
probability alone cannot characterize all important aspects of
evidential appraisal, another dimension---the weight of an
argument---must be deployed to quantify uncertainty.\footnote{Keynes
  entertained the possibility of measuring weight of evidence in terms
  of the variance of the posterior distribution of a certain parameter,
  but was quite attached to the idea that weight should increase with
  new information, even if the dispersion may increase with new evidence
  {[}TP 80-82{]}. So he proposed only a very rough sketch of a positive
  proposal. Moreover, he was unclear how a measure of weight could be
  part of decision-making. He was ultimately skeptical about the
  practical significance of the notion {[}TP 83{]}.}

It is instructive to examine more closely Keynes' claim that the weight
of evidence, unlike its balance, always increases as more evidence is
taken into account. While balance can oscillate one direction or the
other, weight would seem to always increase. We can state this
requirement as follows:

\vspace{1mm}
\begin{tabular}{lp{9cm}}
(Monotonicity) & If $E$ is relevant to $X$ given $K$, where $K$ is background knowledge, $V(X\vert K \wedge E) > V(X\vert K)$, where $V$ is the weight of evidence.
\end{tabular}
\vspace{1mm}

\noindent Monotonicity is consistent with Peirce's example involving
drawing from a bag of beans. As the sample size increases and the
relative proportion of black-to-white beans remains constant, the weight
of the evidence increases. That is:

\vspace{1mm}
\begin{tabular}{p{3cm}p{10cm}}
(Weak Increase) & In urn-like cases, the evidential weight obtained by a larger sample is greater, if the relative frequencies in the samples remain the same.
\end{tabular}
\vspace{1mm}

\noindent This formulation can be strengthened by dropping the
assumption of equal relative frequencies:

\vspace{1mm}
\begin{tabular}{p{3cm}p{10cm}}
(Strong Increase) & In urn-like cases, the evidential weight obtained by a larger sample is higher.
\end{tabular}

\vspace{1mm}

\noindent We think there are good reasons to reject Monotonicity and
Strong Increase, but we agree with Weak Increase. Monotonicity and
Strong Increase are consistent with a certain conception of the weight
of evidence, what we might call \emph{quantity} of evidence. There is no
doubt that, as more evidence is taken into account, the quantity of the
evidence must increase. But the weight of evidence need not be
identified with its quantity alone. As more evidence is taken into
account, the new evidence may speak less clearly in favor or against a
hypothesis. For consider this example:

\begin{quote}
\textbf{A (possibly) rigged lottery:} Initially, you think the lottery  is  fair. You have no reason to doubt that. So, you calculate precisely the probability that a certain ticket number will be drawn. Then, rumors begin to surface that the lottery is rigged and that only numbers that satisfy a complicated equation will be drawn. You now have more relevant evidence at your disposal, but that evidence is more confusing and muddled than before. 
\end{quote}

\noindent Arguably, this is a scenario in which the quantity of evidence
has increased, but the weight of evidence has not. If this is right,
weight and quantity can come apart.

So, we seek a theory of evidential weight that can model two intuitions.
The first is that, as the sample size increases, the weight of the
evidence must also increase (under certain conditions, along the lines
of Weak Increase). The second intuition is that that, even when the
quantity of evidence increases, the weight of evidence might not (as
illustrated by the example of the rigged lottery).

\hypertarget{weight-and-precise-probabilism}{%
\subsection{Weight and Precise
Probabilism}\label{weight-and-precise-probabilism}}

An obvious place to look for a theory of the weight of evidence is
within precise probabilism. The earliest account of the weight of
evidence is given by Good (1950), as follows: \begin{align*}
W(H:E) & = \log \frac{\pr{E \vert H}}{\pr{E\vert \neg H}},
\end{align*}

\noindent where \(E\) is the evidence and \(H\) a hypothesis of
interest. This account follows naturally from the assumption that that
\(W(H:E)\) must be a function of \(\pr{E\vert H}\) and of
\(\pr{E\vert \neg H}\). As Good (1985) put it, ``I cannot see how
anything can be relevant to the weight of evidence other than the
probability of the evidence given guilt and the probability given
innocence'' {[}p
250{]}.\footnote{We can: how good those estimates are and what our uncertainty about them is. And if you want to have a context-relative notion of weight of evidence, so that weight considerations tell you when further investigation is undesirable as the potential weight of evidence is not that high given what you already know, also the weight of evidence and the posterior resulting from   the evidence you have obtained so far.}

One important question is whether Good's weight satisfies the desiderata
we already discussed. We can investigate further developing Good's own
example. If, in an experiment, the observations \(E_1, \dots, E_K\) are
independent given \(H\) and given \(\neg H\), the resulting joint
likelihood is the result of multiplying the individual likelihoods.
Thus, the joint weight is the result of adding the individual weights.
Now, suppose a die is selected at random from a hat containing nine fair
dice and one loaded die with bias \(\nicefrac{1}{3}\) of obtaining a
six. Every time you throw the die and obtain a six, the weight for the
hypothesis that the die is biased increases by
\(log_{10}(\frac{\nicefrac{1}{3}}{\nicefrac{1}{6}})= log_{10}(2)\), that
is 0.30103, and every time you throw it and obtain something else, the
weight changes by
\(log_{10}(\frac{\nicefrac{2}{3}}{\nicefrac{5}{6}})= log_{10}(.8)\),
that is -0.09691. The weights in db (that is, multiplied by 10) for all
possible outcomes of up to 20 tosses are displayed in Figure
\ref{fig:goodWeight}.
\todo{Hey, why wasn't it here in the first place? Can you reinstate it, Marcello?}

Two facts are notable. First, Good's weight can drop with sample size.
For instance, the weight for 5 sixes and 4 other numbers is 1.2db, and
it is .2db for 5 sixes and 5 other numbers. In addition, weight can drop
while the sample size increases even if the proportion of sixes remains
the same. For instance, if none of the observations are sixes, the
weights go from -10 to -19.7 as the sample size goes from 0 to 10. Less
trivially, the observation of one six in five leads to weight of -10.9,
while the observation of two sixes in ten tosses leads to weight -11.7.
That is, (Monotonicity) (Weak Increase) and (Strong Increase) all fail
on Good's measure. This suggests measure is too closely connected to
likelihood ratio for our purpose and as such does not capture the notion
of weight of evidence that we are after.

One way to think about this feature of Good's weight is that the (log of
the) likelihood ratio is a \emph{directional} measure: the evidence may
be favorable or unfavorable to a hypothesis (compared to another). By
contrast, Keynes' remarks suggests that weight is a non-directional
measure. It appears to always increase no matter the balance (though we
will ultimately reject this claim trying to find a middle ground between
these approaches).

So one might think that all there is to be done is stripping the
likelihood ratio of its directionality, say by taking the absolute value
of the natural log (Nance, 2016, sec. 3.5). The weight of evidence \(E\)
relative to the pair of hypotheses \(H, H'\) would be
\[\vert \ln (LR_{H, H'}(E)) \vert, \] where
\(LR_{H, H'}(E)=\frac{P(E \vert H)}{P(E \vert H')}\). By the properties
of logarithms, \(ln(1/x) = -ln(x)\), so two items of evidence of
equivalent strength---but opposite directionality---would have the same
weight. So, for example, \(|ln(1/3)|= |ln(3)| = 1.61\).

This account, while simple and elegant, faces difficulties. As it is
ordinally equivalent to the standard ordering of the absolute values of
Good's weights, our objections, mutatis mutandis, apply. Moreover, as
Nance points out, there is a decomposition problem. Consider two items
of evidence that, taken together, have a likelihood ratio of one, say
one has a likelihood ratio of 1/3 and the other of 3. Assuming they are
probabilistically independent given a hypothesis of interest, their
combined likelihood ratio results from multiplying the individual
likelihood ratios. Thus, their combined weight would be zero since
\(\ln (LR_H(E1 \wedge E_2))=\ln (1) = 0\). However, by adding the
weights one by one, the combined weight would be different from zero,
since \(|ln(1/3)| + |ln(3)| = 3.22\). So, do the two items of evidence
have zero weight or not? Depending on how evidence is decomposed, it
appears to have different weights.
\todo{To think about: does the decomposition problem apply to Good's weight?}
\todo{M: the decomposition problem does not apply to Good who has a directional notion of weight.}

\todo{R: the bit I commended out was false, I think. See the revised verson of the above passage, and think hard about the argument}
\todo{M: Why was it false?}

\hypertarget{weight-and-imprecise-probabilism}{%
\subsection{Weight and Imprecise
Probabilism}\label{weight-and-imprecise-probabilism}}

Precise probabilism was not very successful at delivering an account of
weight. Does imprecise probabilism fare any better? Weatherson (2002)
proposes the following: a body of evidence is weightier whenever the
representor set of probability measures compatible with the evidence is,
in some sensible sense, smaller.
\todo{In what sense? How is this defined?} Consider a case of complete
ignorance about the bias of a coin. The representor set will contain
\emph{any} probability measure. This corresponds to complete lack of
evidence and null weight, as expected. At the other extreme, consider a
case in which the fairness of the coin is known for sure. The supporting
evidence here would have maximal weight and the representor set would
only contain the precise probability measure that assigns .5 to the two
outcomes. All other intermediate cases would fall somewhere in
between.\footnote{Let \(x_1\) and \(x_2\) the two extreme probability
  assignments in the representator set compatible with the available
  evidence \(E\). Then, the weight of \(E\) would be 1-\(x_1-x_2\). As
  expected, if there is only one probability measure, the weight would
  be one.}

This account accommodates the intuition underlying the rigged lottery
example. As more evidenced is accumulated, the representor set can
become larger and include more probability measures than before. Thus,
weight of evidence can decrease even when the quantity of evidence
increases. This is promising. The problem, however, is that this theory
of weight inherits the difficulties of imprecise probabilism, which we
have already mentioned. For example, recall that imprecise probabilism
cannot model a situation in which an epistemic agent believes that the
coin could have bias .8 or .4, but thinks that one bias is more likely
than the other. Arguably, evidence compatible with the coin having two
equally likely biases should possess different weight than evidence
compatible with the coin having two biases one more likely than the
other. The latter situation is closer to full weight---the situation in
which the exact bias of the coin is known for sure. But a theory of
weight based on imprecise probabilism is ill-equipped to accommodate
these nuances.

\todo{R: I reinstated but abridged a discussion of Peden and Joyce}

There are two proposals on the market that are somehow related to
imprecise probabilism. One, due to Peden (2018) follows a suggestion
from (Kyburg, 1961). Kyburg proposed using the degree of imprecision of
the intervals in his probability system called Evidential Probability
(EP), and a range of rules describing how such intervals are to be
shaped by the evidence. With this system in the background, Peden
proposes the following definition of the weight of the argument for
\(H\) given \(E\) and \(K\), where
\(\mathsf{EP}(H \vert E \wedge K) = [x,y]\): \begin{align}
\tag{WK}  \mathsf{WK(H\vert E\wedge K)} & = 1 - (y-x)
\end{align} \noindent That is, the weight of the evidence is the spread
of the evidential probability, transformed to scale between 0 and 1,
reaching 1 when the spread is 0 and 0 when the spread is 1. A similar
line is taken within standard imprecise probabilism by Walley (1991),
who proposes to take the edges of the resulting interval that captures
changes in the weight of evidence.

We already criticized the focus on the edges of the intervals in this
paper. Relatedly, on this proposal the edges of the intervals are what
contributes to \textsf{WK}, and these are highly sensitive to the choice
of the margin of error, but what margin of error to choose and why
remains a mystery, and what margin of error has been chosen does not
function anywhere in the \(\mathsf{EP}\) representation of uncertainty.
It may easily happen that for two different distributions the \(1\%\)
intervals will be identical while the 78\% intervals will not. Such
differences will obviously not be captured by the 1\% margin of error
intervals.

Another stab at explicating weight of evidence within the \textsf{IP}
framework has been made by Joyce (2005). Joyce uses a density over
chance hypotheses to account for the notion of evidential weight. He
conceptualizes the weight of evidence as an increase of concentration of
smaller subsets of chance
hypotheses:\footnote{This looks a bit complicated, so let us take a  slow look at an example. Suppose you only consider three chance hypotheses, that the coin bias is one of $.4, .5,$ and $.6$, that is, the hypotheses are $ch(X) = .4, ch(X)=.5,$ and $ch(X)=.6$. For each $x\in \{.4, .5, .6\}$) you attach a prior credence  $c(ch(X) = x)$ to the corresponding hypothesis. Say you start with equal priors, that is for all $x\in \{.4, .5, .6\}$ you have $c(ch(X) =x) = \nicefrac{1}{3}$. Then, your expected value of $X$, which Joyce takes to be your credence in $X$ simpliciter is $\sum_x c(Ch(X)=x)x$, which is .5. 


Now consider your evidence: you tossed the coin and observed, say, seven heads out of ten tosses. We need $c(ch(X)=x \vert E)$. By Bayes, we have:
\begin{align*}
c(ch(X)=x \vert E) & = \frac{c(E \vert ch(X) =x) c(ch(X)=x )}{c(E)},
\end{align*}
\noindent so we need to calculate the likelihoods, $c(E \vert ch(X) =x)$. We assume you are probabilistically coherent, that you defer to chances, and know the experimental setup, so that the likelihoods are calculated using the binomial distribution, i.e. if the evidence is $a$ heads and $b$ tails: 
\begin{align*}
c(E \vert ch(X) =x) & = {a+b \choose a}\,x ^a (1-x)^{b}
\end{align*}
\noindent In our example, the likelihoods (rounded) are $.042, .117$, and $.214$ respectively. The denominator is calculated by taking $c(E) =  \sum_x c(E \vert ch(X) =x) c(ch(X)=x)$, which in our case turns out to be $.124$. Putting these together, the values of $c(ch(X)=x \vert E)$ are $.113$ $.312$, and  $.573$ (rounded). Then, your expected value, which Joyce to be your credence in $X$ simpliciter conditional on $E$ is $\sum_x c(Ch(X)=x\vert E)x$, which is  $.54$.}
\begin{align}
\tag{Joyce} w(X,E) & = \sum_x \vert c(ch(X) = x  \vert E) \times (x - c(X\vert E))^2 - c(ch(X) = x) \times (x - c(X))^2\vert
\end{align} The idea here is that weighty evidence should make the
credence resilient, and resilience makes the difference between the
posterior credence in chances \(c(ch(X)=x \vert E)\) and the prior
credence in chances \(c(ch(X)=x)\). The complication is that the impact
of this difference should be lower for for those values of \(x\) that
are close to \(c(X\vert E)\) for the posterior and close to \(c(X)\) for
the prior. Hence, the formula for \(w\) is takes (the absolute value of)
the difference between posteriors and priors weighed by, these (squared)
distances. The weightier the evidence, the smaller \(w\) is supposed to
be.\footnote{Accordingly, in our example the weights for the prior are $-.1^2, 0, .1^2 = 0.01, 0, .01$, the weights for the posterior are   $.021330539, .002120582$, and  $.0029$ and w is $0.003241822$. For comparison, if instead we observed 70 heads in 100 tosses, $w$ would be $.006689603$.}

There are various issues with this approach. One is that now to evaluate
the weight of evidence \(E\) with respect to proposition \(X\) now you
need to have and use in your calculations your estimation of chances of
\(X\). Let us put aside the worry that it is not obvious that we can
meaningfully talk about chances of arbitrary propositions. Even then,
the name of the game for the imprecise probabilist was to express the
uncertainty about \(X\) in terms of a representor, a set of probability
measures. However, one can have a representor with respect to a set of
object-level propositions including \(X\) without having a single
credence about chances, so now the calculations of weight of \(E\) with
respect to \(X\) do not fall out whatever was supposed to capture the
agent's uncertainty about \(X\), \(E\) and their relationship.

Crucially, the measure displays strange behaviors when run on even
straightforward cases. Expanding a bit on an example that we've been
discussing in a footnote, observe that measure might result in drastic
shift in weights even if the observed frequencies are not too far from
the chance hypotheses, and that the weight of evidence might drop as the
sample size increases, even if the observed success frequency remains
the same, which means that (Weak Increase) fails for this measure (see
Figure \ref{fig:joyce2}).\footnote{
What is the reason for this strange behavior? The shaping of Joyce's weight is a balancing act. For instance, for frequency .1 with equal priors the weight is maximized at $n=90$ and starts dropping at $n=100$. Why? We start with three chance hypotheses, $.4, .5, .6$ with equal priors. Once the observations have been made, the posterior for the chance hypotheses is focused at the chance hypothesis .4 (its posterior is more or less .99999), and so is the credence in $X$ simpliciter (this expected value is .4000003). Now the weights are build by measuring square distance from the credence in $X$ simpliciter and since the expected value is nearly equal to the lowest chance hypothesis under consideration, the weight for the lowest chance hypothesis is $8.260660e-14$, so while the posterior for this hypothesis is very high, the weight is very low and its contribution to the weight calculation is severely limited. Ultimately, what happens with weight is now a matter of balancing the uneven posteriors with squared penalties (or rewards, really) for the distance from the expected value (which is pretty much the most likely hypothesis once you have made enough observations). Once you observe 10 successes in 1000 trials, the credence in $X$ simpliciter becomes $.4000001$, so the distance of the lowest chance hypothesis to it drops, and this weight drops "faster" than the resulting increase in the probability of the lowest chance hypothesis itself. The stabilization is achieved because further on the posterior for this hypothesis can only get closer to one (and closer to zero for all the other hypotheses).}

\begin{figure}

\begin{center}\includegraphics[width=1\linewidth]{paper-outline_files/figure-latex/joyce2b-1} \end{center}

\caption{(top) Joyce's $w$ (the lower it is, the higher the weight) for various observed successes in 100 Bernoulli trials. Three chance chypotheses: $.4, .5, .6$, and two sets of priors: equal and $.5, .3, .2$ respectively. Agan, the weightiest evidence is obtained with successes close to the expected value, with  large variation for observed frequencies not too far from the expected values, fairly flat otherwise. (bottom) Joyce's $w$ (the lower it is, the higher the weight) for two fixed success ratio across various observed successes in  Bernoulli trials (lines are used for smoothing).  Note large shifts with possible decrease in the beginning, and a flattening afterwards.}
\label{fig:joyce2}
\end{figure}

Moreover, the approach is of limited applicability. For one thing, as
Joyce admits, it is supposed to work when RA's credence is mediated by
chance hypotheses. Depending on applications, such a mediation might be
unavailable. Another issue is that this might work for unimodal
distributions when we only consider the influx of new data points, but
it's unlikely to give desired results if, say, the evidence obtained is
the testimony of disagreeing witnesses. This is because an essential
part of the calculations relies on taking the expected value, and it is
not too hard to imagine cases of diverging items of evidence resulting
only in a small chance of the expected value. This is related to the
assumption that an agent's stance towards a proposition should be
represented as the expected value of the chance hypothesis---which we
already extensively argued against.

Finally, note that the proposal employs probabilities or probability
densities (if we go continuous) over parameter values. This is in line
with what we propose if we do not assume these are chances and treat
them as, say, parameters that are potentially rational to accept in
light of the evidence. But then, there are useful ways to go this way
without turning to IP,
\footnote{After all, notice how the notion of a representor plays no role in Joyce's explication of weight whatsoever!}
to which we now proceed.

\hypertarget{weight-and-higher-order-probabilism}{%
\subsection{Weight and Higher-order
Probabilism}\label{weight-and-higher-order-probabilism}}

We now present our own account of the weight of evidence. This account
has two distinctive features. First, it is based on higher-order
probabilism. Second, it is information-theoretic. To develop this
account, we will begin with a short introduction to Shannon's theory of
information.

\hypertarget{entropy-of-a-distribution}{%
\subsubsection{Entropy of a
Distribution}\label{entropy-of-a-distribution}}

Let \(X\) be a random variable and \(P\) a probability distribution over
its values.\footnote{Since working with continuous distributions is not
  straightforward, we will be using \emph{grid approximations} of
  continuous distributions: we will split \(X\) into a 1000 bins and use
  the normalized densities for their centers to obtain their
  corresponding probabilities. As long as we do not change our level of
  precision (which would inevitably lead to changes in entropy) in our
  comparisons, this is not a problem.} Shannon's measure of information,
\(H(X)\), reads: \begin{align*}
H(X)  & =
- \sum \mathsf{P}(x_i) \log_2 \mathsf{P}(x_i)
\end{align*} \noindent  Consider the simple case in which \(X\) can take
two values---outcome 1 and outcome 0---whose probabilities are \(p\) and
\(1-p\). Figure \ref{} \todo{Add plot of H(X)} shows \(H(X)\) as a
function of \(p\). Entropy \(H(X)\) is greatest when the two outcomes
have an equal probability of .5. The more the probabilities deviate from
.5, the more \(H(X)\) approaches zero. To make sense of this, \(H(X)\)
can be thought as the \textit{entropy} (i.e.~the lack of information)
contained in the distribution associated with random variable \(X\).
When the two outcomes have equal probability, entropy is greatest. When
they have different probabilities, one outcome will be more probable
than the other. The more probable one of the outcomes (and thus the less
likely the other), the lower the entropy. Intuitively, \(H(X)\) captures
the idea that entropy is greatest when the indecision about which
outcome will occur is maximal, and the entropy decreases when such
indecision decreases.

If \(H(X)\) is a measure of entropy---that is, a measure of lack of
information---why call it a measure of information? \(H(X)\) is also a
measure of information in the following sense: it describes the
\textit{expected amount of information} one would receive upon learning
the actual value of \(X\). After all, the higher the entropy, the less
informative a distribution, the more you expect to learn upon finding
out the actual value of \(X\). Conversely, the lower the entropy of the
distribution, the more informative the distribution, the less one
expects to learn upon finding out the actual value of \(X\).

\hypertarget{weight-of-a-distribution}{%
\subsubsection{Weight of a
Distribution}\label{weight-of-a-distribution}}

Since \(H(X)\) can be thought as the entropy of a distribution, we will
switch to the notation \(H(P)\). This notation emphasizes the
distribution \(P\) rather than the random variable \(X\). The entropy of
a distribution is to be contrasted with its informativeness, which we
will denote by \(W(P)\), the weight of the distribution. How should we
measure the weight (or informativeness) of a distribution?

\todo{This temptation didn't make sense, entropy can often be higher than 1.}

BEfore we proced, one technical remark. The move to continuous
distributions is not straightforward,\footnote{One might expect that
  entropy in the continuous case could be made by binning and taking the
  limit. For instance, suppose we divide \(X\) into bins \(x_i\) of
  length \(\Delta\), so that we discretize \(X\) into \(X^\Delta\). The
  discrete case definition applies: \begin{align*}
  H(X^\Delta) & = \sum \left[\mathsf{P}(X \mbox{ is in the $i$-th bin}) \log_2 \frac{1}{\mathsf{P}(X \mbox{ is in the $i$-th bin})}\right]
  \end{align*} \noindent If you think of the histogram of the
  distribution of \(X^\Delta\) with total area \(A\), each bin has area
  \(a_i\) and height \(p_i\). Suppose we normalize so that \(A =1\),
  then the probability of each bin is \(\mathsf{P}_i = p_i \Delta\) and
  \(p_i\) can be thought of probability density. Then we have:
  \begin{align*}
  H(X^\Delta) & = \sum \mathsf{P}_i \log_2 \frac{1}{\mathsf{P}_i}\\
  & = \sum p_i \Delta \log_2 \frac{1}{p_i \Delta}\\
  & = \sum \left[ p_i \Delta \left(\log_2 \frac{1}{p_i} + \log_2\frac{1}{\Delta}\right)\right]\\
  & = \sum p_i \Delta \log_2 \frac{1}{p_i} +    \underbrace{\sum \underbrace{p_i \Delta}_{\mathsf{P}_i}}_1 \log_2\frac{1}{\Delta} \\
  & = \sum p_i \Delta \log_2 \frac{1}{p_i} +  \log_2\frac{1}{\Delta}
  \end{align*} \noindent Accordingly, when we try to go continuous by
  taking the limit, we get: \begin{align*}
  H(X) & = \left[\int_{-\infty}^\infty p(x) \log_2 \frac{1}{p(x)}\, dx  \right] + \infty
  \end{align*} \noindent This is as it should: the entropy of a
  continuous variable increases with the precision of measurement, so
  infinite precision gives infinite information. For this reason, for
  the continuous case it is usual to drop the rightmost part of the
  equation and talk about \emph{differential entropy}: \begin{align*}
  \mathsf{H}(X) & = \left[\int_{-\infty}^\infty p(x) \log_2 \frac{1}{p(x)}\, dx  \right] 
  \end{align*}}, so in what follows we prefer to stick to entropy proper
and discretize. One reason is that we will want to meaningfully compare
information conveyed by discrete distributions to that conveyed by
continuous ones. A convenient way to do so is to abandon the idea that
we should be infinitely precise, fix a certain number of bins (that is a
certain level of precision) and keep it fixed in our comparison. This is
what we will do: effectively, we will be using
\emph{grid approximations} of continuous distributions: we will split
\(X\) into a 1000 bins and use the normalized densities for their
centers to obtain their corresponding probabilities. As long as we do
not change our level of precision (which would inevitably lead to
changes in entropy) in our comparisons, this is not a problem. An
additional advantage is that now we do not have to deal with the
intricacies of explicit analytic calculations for continuous variables
and comparing apples (entropy) with oranges (differential entropy). One
side-effect of this is that the exact absolute values of entropy will
depend on the levels of precision we choose, and so we should not be too
attached to differences, focusin on proportions instead.

From this perspective, the weight (or informativeness) of a distribution
is modeled by comparing it to the least informative distribution, the
uniform distribution, which expresses complete uncertainty. The more
informative a distribution, the more it departs from the uniform
distribution, the more weight it has, on a scale from 0 to 1. If the
drop from uncertainty is complete, the entropy drops to zero, and thus
the weight should be 1; if the drop is null, the entropy remains the
same, and thus the weight should be zero; if the drop is half, the
weight should be .5; and so on for other intermediate cases. This
pattern can be captured by the following definition of weight (or
informativeness) of a distribution: \begin{align*}
\mathsf{w(P)} & = 1 - \left( \frac{H(\mathsf{P})}{H(\mathsf{uniform})}\right)
\end{align*} \noindent where \(\mathsf{P}\) is the probability
distribution of interest and \(\mathsf{uniform}\) is the baseline
uniform
distribution.\footnote{Note that the entropy of a uniform distribution is pretty straightforward, so we can simplify:
\begin{align*}
H(\mathsf{uniform}) & = \sum_{i=1}^n \nicefrac{1}{n} \log_2 \frac{1}{\nicefrac{1}{n}} \\
& = \log_2(n) \\
\mathsf{w(P)} & = 1 - \left( \frac{H(\mathsf{P})}{\log_2(n)}\right)
\end{align*}
}

This measure captures the two key intuitions we identified earlier.
First, the weight of a distribution increases as the number of
observations increases provided the relative frequency is fixed (Weak
Increase; see Figure \ref{fig:entropyJoyceExampleSampleSize}). As usual,
we can think of a coin that is tossed a number of times and that lands
heads or tails with a certain observed frequency. At the same time---and
this is the second intuition---weight may diminish when the quantity of
evidence increases, as the Rigged Lottery example suggest. Figure
\ref{fig:entropyJoyceExampleSampleSize} shows that weight can
drop---despite a larger number of observations---so long as the observed
relative frequency changes from more extreme (say .1) to less extreme
(say .5). That weight is not strictly tied to the quantity of evidence
(number of observations) is also apparent from Figure
\ref{fig:entropyJoyceExamplePlot}. This shows that weight can vary
dramatically depending on the observed relative frequency, all else
being equal. The behavior of the proposed measure of weight can be
illustrated more generally using distributions of various of shapes,
displayed in Figure \ref{fig:weightsWeird}.

\begin{figure}

\begin{center}\includegraphics[width=0.7\linewidth]{paper-outline_files/figure-latex/entropyJoyceExampleSampleSize-1} \end{center}

\caption{Entropy-based weight for two observed frequencies for various sample sizes (lines used instead of points for smoothing). Three chance chypotheses: $.4, .5, .6$, and two sets of priors: equal and $.5, .3, .2$ respectively.}
\label{fig:entropyJoyceExampleSampleSize}
\end{figure}

\begin{figure}

\begin{center}\includegraphics[width=0.7\linewidth]{paper-outline_files/figure-latex/entropyJoyceExamplePlot10-1} \end{center}

\caption{Entropy-based weight for for various observed successes in 10 Bernoulli trials. Three chance chypotheses: $.4, .5, .6$, and two sets of priors: equal and $.5, .3, .2$ respectively.}
\label{fig:entropyJoyceExamplePlot}
\end{figure}

\begin{figure}[H]

\begin{center}\includegraphics[width=1\linewidth]{paper-outline_files/figure-latex/fig:weightsWeird-1} \end{center}
\caption{Examples of various distributions with their entropies and weights, ordered by weights. (1) beta(4,4), (2) uniform starting from .5 to 1, (3), uniform strating from .6 to 1, (4) two normal distributions centered around .4 and .6 with standard deviation .05, glued at .5. (5) normal centered around .5 with the same standard deviation, (6) one that assigns .5 to each of .4  and .6, (7) One that assigns .3 to .4 and .7 to .6., (8) one that assigns all weight to .5, and (9) one that assigns all weight to .7.}

\label{fig:weightsWeird}
\end{figure}

\hypertarget{weight-of-evidence-1}{%
\subsubsection{Weight of Evidence}\label{weight-of-evidence-1}}

So far we talked about the weight of a distribution, and indirectly the
weight of the evidence so long as a distribution reflects the evidence.
The notion of weight of evidence, however, be made more precise.

Suppose a distribution \(P\) depicts what an agent thinks, at some point
in time, about the probability of the possible values of a random
variable \(X\), say the possible biases of a coin or first-order
probabilities. In this sense, the weight \(W(P)\) (or informativeness)
of a distribution measures how informed an agent is about \(X\). But it
measures the information level of an agent only relative to the state of
full uncertainty represented by the uniform distribution. There are two
things missing from this account: one, not every agent starts with a
uniform prior; two, how informed an agent is must depend on the evidence
available to that agent. What we need, then, is an account of how
informed agents are \textit{on the basis of the evidence} they have, or
in other words, an account of the weight (or informativeness) of the
evidence they have.

If the agent starts with a uniform prior over the values of a random
variable of interest, \(W(P)\) would be a good enough approximation of
how informed the evidence made them. In general, however, how much more
information is obtained is context-dependent. The weight of evidence,
then, must depend on what the agent already knows. Here is a general
recipe. In a given context, consider the prior distribution \(P_0\) for
the target random variables \(X\) given what the agent already knows.
Then, the agent updates by a body of evidence \(E\). Call this posterior
distribution \(P_E\), where the updating is done by standard Bayesian
conditionalization. Take the difference between the weight of the prior
distribution, \(W(P_0)\), and the weight of the posterior distribution,
\(W(P_E)\). The difference between the two--\(W(P_E)-W(P_0)\) or more
succinctly \(\Delta W\)---measures the impact that evidence \(E\) has on
the information level of the evidence.\footnote{More precisely, the
  calculation follows the following schema:} The difference
\(\Delta W\), then, is our proposed measure of the weight of the
evidence.\footnote{If you prefer to think that weight of evidence should
  always be positive as a result of adding evidence, you might prefer
  the absolute value of the difference. We, however, prefer to keep
  track of whether the evidence makes the agent more or less informed
  about an issue.}

Notice that our account of weight of evidence, in principle, is
independent of the proposal to adopt second-order probabilities. The
notions of \(W(P)\) and \(\Delta W\) could be applied to first-order
probability distributions. Even if you just considered two competing
hypotheses and the likelihood ratio, you could deploy our account of
weight. But this would bring us back to Good's notion of weight, which
does not capture the intuitions about weight of evidence that we wanted
to capture. So it is crucial for our account of weight that it comprises
two components: the information-theoretic and the higher-order
component.

\todo{I will add a section on expected weight of evidence, and I didn't get the importance of the final challenge section, so I haven't copied it yet.}

\hypertarget{references}{%
\section*{References}\label{references}}
\addcontentsline{toc}{section}{References}

\hypertarget{refs}{}
\begin{CSLReferences}{1}{0}
\leavevmode\vadjust pre{\hypertarget{ref-bradley2015critical}{}}%
Bradley, D. (2015). \emph{A critical introduction to formal
epistemology}. Bloomsbury Publishing.

\leavevmode\vadjust pre{\hypertarget{ref-bradley2012scientific}{}}%
Bradley, S. (2012). \emph{Scientific uncertainty and decision making}
(PhD thesis). London School of Economics; Political Science (University
of London).

\leavevmode\vadjust pre{\hypertarget{ref-bradley2019imprecise}{}}%
Bradley, S. (2019). {Imprecise Probabilities}. In E. N. Zalta (Ed.),
\emph{The {Stanford} encyclopedia of philosophy} ({S}pring 2019).
\url{https://plato.stanford.edu/archives/spr2019/entries/imprecise-probabilities/};
Metaphysics Research Lab, Stanford University.

\leavevmode\vadjust pre{\hypertarget{ref-CampbellMoore2020accuracy}{}}%
Campbell-Moore, C. (2020). \emph{Accuracy and imprecise probabilities}.

\leavevmode\vadjust pre{\hypertarget{ref-Carr2020impreciseEvidence}{}}%
Carr, J. R. (2020). Imprecise evidence without imprecise credences.
\emph{Philosophical Studies}, \emph{177}(9), 2735--2758.
\url{https://doi.org/10.1007/s11098-019-01336-7}

\leavevmode\vadjust pre{\hypertarget{ref-Dahlman2022Information}{}}%
Dahlman, C., \& Nordgaard, A. (2022). \emph{Information economics in the
criminal standard of proof}.

\leavevmode\vadjust pre{\hypertarget{ref-deadman1984fiber2}{}}%
Deadman, H. A. (1984a). Fiber evidence and the wayne williams trial
(conclusion). \emph{FBI L. Enforcement Bull.}, \emph{53}, 10--19.

\leavevmode\vadjust pre{\hypertarget{ref-deadman1984fiber1}{}}%
Deadman, H. A. (1984b). Fiber evidence and the wayne williams trial
(part i). \emph{FBI L. Enforcement Bull.}, \emph{53}, 12--20.

\leavevmode\vadjust pre{\hypertarget{ref-Dietrich2016pooling}{}}%
Dietrich, F., \& List, C. (2016). Probabilistic opinion pooling. In A.
Hajek \& C. Hitchcock (Eds.), \emph{Oxford handbook of philosophy and
probability}. Oxford: Oxford University Press.

\leavevmode\vadjust pre{\hypertarget{ref-Lee2017impreciseEpistemology}{}}%
Elkin, L. (2017). \emph{Imprecise probability in epistemology} (PhD
thesis). Ludwig-Maximilians-Universit{ä}t;
Ludwig-Maximilians-Universität München.

\leavevmode\vadjust pre{\hypertarget{ref-Elkin2018resolving}{}}%
Elkin, L., \& Wheeler, G. (2018). Resolving peer disagreements through
imprecise probabilities. \emph{Noûs}, \emph{52}(2), 260--278.
\url{https://doi.org/10.1111/nous.12143}

\leavevmode\vadjust pre{\hypertarget{ref-Fenton2018Risk}{}}%
Fenton, N., \& Neil, M. (2018). \emph{Risk assessment and decision
analysis with bayesian networks}. Chapman; Hall.

\leavevmode\vadjust pre{\hypertarget{ref-VanFraassen2006vague}{}}%
Fraassen, B. C. V. (2006). Vague expectation value loss.
\emph{Philosophical Studies}, \emph{127}(3), 483--491.
\url{https://doi.org/10.1007/s11098-004-7821-2}

\leavevmode\vadjust pre{\hypertarget{ref-Gardenfors1982unreliable}{}}%
Gärdenfors, P., \& Sahlin, N.-E. (1982). Unreliable probabilities, risk
taking, and decision making. \emph{Synthese}, \emph{53}(3), 361--386.
\url{https://doi.org/10.1007/bf00486156}

\leavevmode\vadjust pre{\hypertarget{ref-good1950probability}{}}%
Good, I. J. (1950). \emph{Probability and the weighing of evidence}. C.
Griffin London.

\leavevmode\vadjust pre{\hypertarget{ref-good1985}{}}%
Good, I. J. (1985). Weight of evidence: A brief survey. In J. M.
Bernardo, M. H. DeGroot, D. V. Lindley, \& A. F. M. Smith (Eds.),
\emph{Bayesian statistics}. Elsevier Science.

\leavevmode\vadjust pre{\hypertarget{ref-joyce2005probabilities}{}}%
Joyce, J. M. (2005). How probabilities reflect evidence.
\emph{Philosophical Perspectives}, \emph{19}(1), 153--178.

\leavevmode\vadjust pre{\hypertarget{ref-Kaplan1968decision}{}}%
Kaplan, J. (1968). Decision theory and the fact-finding process.
\emph{Stanford Law Review}, \emph{20}(6), 1065--1092.

\leavevmode\vadjust pre{\hypertarget{ref-kasser2016two}{}}%
Kasser, J. (2016). Two conceptions of weight of evidence in peirce's
illustrations of the logic of science. \emph{Erkenntnis}, \emph{81}(3),
629--648.

\leavevmode\vadjust pre{\hypertarget{ref-keynes1921treatise}{}}%
Keynes, J. M. (1921). \emph{A treatise on probability, 1921}. London:
Macmillan.

\leavevmode\vadjust pre{\hypertarget{ref-konek2013foundations}{}}%
Konek, J. (2013). \emph{New foundations for imprecise bayesianism} (PhD
thesis). University of Michigan.

\leavevmode\vadjust pre{\hypertarget{ref-Kyburg1961}{}}%
Kyburg, H. E. (1961). \emph{Probability and the logic of rational
belief}. Wesleyan University Press.

\leavevmode\vadjust pre{\hypertarget{ref-kyburg2001uncertain}{}}%
Kyburg Jr, H. E., \& Teng, C. M. (2001). \emph{Uncertain inference}.
Cambridge University Press.

\leavevmode\vadjust pre{\hypertarget{ref-Levi1974ideterminate}{}}%
Levi, I. (1974). On indeterminate probabilities. \emph{The Journal of
Philosophy}, \emph{71}(13), 391. \url{https://doi.org/10.2307/2025161}

\leavevmode\vadjust pre{\hypertarget{ref-Levi1980enterprise}{}}%
Levi, I. (1980). \emph{The enterprise of knowledge: An essay on
knowledge, credal probability, and chance}. MIT Press.

\leavevmode\vadjust pre{\hypertarget{ref-levi2011weight}{}}%
Levi, I. (2011). The weight of argument. In \emph{Fundamental
uncertainty} (pp. 39--58). Springer.

\leavevmode\vadjust pre{\hypertarget{ref-Mayo-Wilson2016scoring}{}}%
Mayo-Wilson, C., \& Wheeler, G. (2016). Scoring imprecise credences: A
mildly immodest proposal. \emph{Philosophy and Phenomenological
Research}, \emph{92}(1), 55--78.
\url{https://doi.org/10.1111/phpr.12256}

\leavevmode\vadjust pre{\hypertarget{ref-nance2016}{}}%
Nance, D. A. (2016). \emph{The burdens of proof: Discriminatory power,
weight of evidence, and tenacity of belief}. Cambridge University Press.

\leavevmode\vadjust pre{\hypertarget{ref-o2006uncertain}{}}%
O'Hagan, A., Buck, C. E., Daneshkhah, A., Eiser, J. R., Garthwaite, P.
H., Jenkinson, D. J., \ldots{} Rakow, T. (2006). \emph{Uncertain
judgements: Eliciting experts' probabilities}.

\leavevmode\vadjust pre{\hypertarget{ref-peden2018imprecise}{}}%
Peden, W. (2018). Imprecise probability and the measurement of keynes's
{``weight of arguments.''} \emph{Journal of Applied Logics---IFCoLog
Journal of Logics and Their Applications}, \emph{5}(3).

\leavevmode\vadjust pre{\hypertarget{ref-Rinard2013against}{}}%
Rinard, S. (2013). Against radical credal imprecision. \emph{Thought: A
Journal of Philosophy}, \emph{2}(1), 157--165.
\url{https://doi.org/10.1002/tht3.84}

\leavevmode\vadjust pre{\hypertarget{ref-Schoenfield2017accuracy}{}}%
Schoenfield, M. (2017). The accuracy and rationality of imprecise
credences. \emph{Noûs}, \emph{51}(4), 667--685.
\url{https://doi.org/10.1111/nous.12105}

\leavevmode\vadjust pre{\hypertarget{ref-seidenfeld2012forecasting}{}}%
Seidenfeld, T., Schervish, M., \& Kadane, J. (2012). Forecasting with
imprecise probabilities. \emph{International Journal of Approximate
Reasoning}, \emph{53}, 1248--1261.
\url{https://doi.org/10.1016/j.ijar.2012.06.018}

\leavevmode\vadjust pre{\hypertarget{ref-Sjerps2015Uncertainty}{}}%
Sjerps, M. J., Alberink, I., Bolck, A., Stoel, R. D., Vergeer, P., \&
Zanten, J. H. van. (2015). {Uncertainty and LR: to integrate or not to
integrate, that's the question}. \emph{Law, Probability and Risk},
\emph{15}(1), 23--29. \url{https://doi.org/10.1093/lpr/mgv005}

\leavevmode\vadjust pre{\hypertarget{ref-Stewart2018pooling}{}}%
Stewart, R. T., \& Quintana, I. O. (2018). Learning and pooling, pooling
and learning. \emph{Erkenntnis}, \emph{83}(3), 1--21.
\url{https://doi.org/10.1007/s10670-017-9894-2}

\leavevmode\vadjust pre{\hypertarget{ref-Sturgeon2008grain}{}}%
Sturgeon, S. (2008). Reason and the grain of belief. \emph{No{û}s},
\emph{42}(1), 139--165. Retrieved from
\url{http://www.jstor.org/stable/25177157}

\leavevmode\vadjust pre{\hypertarget{ref-Taroni2015Dismissal}{}}%
Taroni, F., Bozza, S., Biedermann, A., \& Aitken, C. (2015). {Dismissal
of the illusion of uncertainty in the assessment of a likelihood ratio}.
\emph{Law, Probability and Risk}, \emph{15}(1), 1--16.
\url{https://doi.org/10.1093/lpr/mgv008}

\leavevmode\vadjust pre{\hypertarget{ref-Thomason2003How-the-Probabi}{}}%
Thompson, W. C., Taroni, F., \& Aitken, C. G. G. (2003). How the
probability of a false positive affects the value of {DNA} evidence.
\emph{Journal of Forensic Science}, \emph{48}(1), 47--54.

\leavevmode\vadjust pre{\hypertarget{ref-Titelbaum2020Fundamentals-of}{}}%
Titelbaum, M. G. (2020). \emph{Fundamentals of bayesian epistemology}.

\leavevmode\vadjust pre{\hypertarget{ref-walley1991statistical}{}}%
Walley, P. (1991). \emph{Statistical reasoning with imprecise
probabilities}. Chapman; Hall London.

\leavevmode\vadjust pre{\hypertarget{ref-Weatherson2002}{}}%
Weatherson, B. (2002). Keynes, uncertainty and interest rates.
\emph{Cambridge Journal of Economics}, \emph{26}, 47--62.

\end{CSLReferences}

\end{document}
