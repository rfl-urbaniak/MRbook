% Options for packages loaded elsewhere
\PassOptionsToPackage{unicode}{hyperref}
\PassOptionsToPackage{hyphens}{url}
\PassOptionsToPackage{dvipsnames,svgnames,x11names}{xcolor}
%
\documentclass[
  10pt,
  dvipsnames,enabledeprecatedfontcommands]{scrartcl}
\usepackage{amsmath,amssymb}
\usepackage{lmodern}
\usepackage{iftex}
\ifPDFTeX
  \usepackage[T1]{fontenc}
  \usepackage[utf8]{inputenc}
  \usepackage{textcomp} % provide euro and other symbols
\else % if luatex or xetex
  \usepackage{unicode-math}
  \defaultfontfeatures{Scale=MatchLowercase}
  \defaultfontfeatures[\rmfamily]{Ligatures=TeX,Scale=1}
\fi
% Use upquote if available, for straight quotes in verbatim environments
\IfFileExists{upquote.sty}{\usepackage{upquote}}{}
\IfFileExists{microtype.sty}{% use microtype if available
  \usepackage[]{microtype}
  \UseMicrotypeSet[protrusion]{basicmath} % disable protrusion for tt fonts
}{}
\usepackage{xcolor}
\usepackage{graphicx}
\makeatletter
\def\maxwidth{\ifdim\Gin@nat@width>\linewidth\linewidth\else\Gin@nat@width\fi}
\def\maxheight{\ifdim\Gin@nat@height>\textheight\textheight\else\Gin@nat@height\fi}
\makeatother
% Scale images if necessary, so that they will not overflow the page
% margins by default, and it is still possible to overwrite the defaults
% using explicit options in \includegraphics[width, height, ...]{}
\setkeys{Gin}{width=\maxwidth,height=\maxheight,keepaspectratio}
% Set default figure placement to htbp
\makeatletter
\def\fps@figure{htbp}
\makeatother
\setlength{\emergencystretch}{3em} % prevent overfull lines
\providecommand{\tightlist}{%
  \setlength{\itemsep}{0pt}\setlength{\parskip}{0pt}}
\setcounter{secnumdepth}{-\maxdimen} % remove section numbering
%\documentclass{article}

% %packages
 \usepackage{booktabs}
\usepackage{subcaption}
\usepackage{multirow}
\usepackage{colortbl}
\usepackage{graphicx}
\usepackage{longtable}
\usepackage{ragged2e}
\usepackage{etex}
%\usepackage{yfonts}
\usepackage{marvosym}
%\usepackage[notextcomp]{kpfonts}
\usepackage[scaled=0.86]{helvet}
\usepackage{nicefrac}
\newcommand*{\QED}{\hfill \footnotesize {\sc Q.e.d.}}
\usepackage{floatrow}
%\usepackage[titletoc]{appendix}
%\renewcommand\thesubsection{\Alph{subsection}}

\usepackage[textsize=footnotesize]{todonotes}
\newcommand{\inbook}[1]{\todo[color=gray!40]{#1}}
\newcommand{\mar}[1]{\todo[color=blue!40]{#1}}
\newcommand{\raf}[1]{\todo[color=olive!40]{#1}}
%\linespread{1.5}
\newcommand{\indep}{\!\perp \!\!\! \perp\!}


\setlength{\parindent}{10pt}
\setlength{\parskip}{1pt}


%language
\usepackage{times}
\usepackage{t1enc}
%\usepackage[utf8x]{inputenc}
%\usepackage[polish]{babel}
%\usepackage{polski}




%AMS
\usepackage{amsfonts}
\usepackage{amssymb}
\usepackage{amsthm}
\usepackage{amsmath}
\usepackage{mathtools}

\usepackage{geometry}
 \geometry{a4paper,left=35mm,top=20mm,}


%environments
\newtheorem{fact}{Fact}



%abbreviations
\newcommand{\ra}{\rangle}
\newcommand{\la}{\langle}
\newcommand{\n}{\neg}
\newcommand{\et}{\wedge}
\newcommand{\jt}{\rightarrow}
\newcommand{\ko}[1]{\forall  #1\,}
\newcommand{\ro}{\leftrightarrow}
\newcommand{\exi}[1]{\exists\, {_{#1}}}
\newcommand{\pr}[1]{\mathsf{P}(#1)}
\newcommand{\cost}{\mathsf{cost}}
\newcommand{\benefit}{\mathsf{benefit}}
\newcommand{\ut}{\mathsf{ut}}

\newcommand{\odds}{\mathsf{Odds}}
\newcommand{\ind}{\mathsf{Ind}}
\newcommand{\nf}[2]{\nicefrac{#1\,}{#2}}
\newcommand{\R}[1]{\texttt{#1}}
\newcommand{\prr}[1]{\mbox{$\mathtt{P}_{prior}(#1)$}}
\newcommand{\prp}[1]{\mbox{$\mathtt{P}_{posterior}(#1)$}}

\newcommand{\s}[1]{\mbox{$\mathsf{#1}$}}


\newtheorem{q}{\color{blue}Question}
\newtheorem{lemma}{Lemma}
\newtheorem{theorem}{Theorem}



%technical intermezzo
%---------------------

\newcommand{\intermezzoa}{
	\begin{minipage}[c]{13cm}
	\begin{center}\rule{10cm}{0.4pt}



	\tiny{\sc Optional Content Starts}
	
	\vspace{-1mm}
	
	\rule{10cm}{0.4pt}\end{center}
	\end{minipage}\nopagebreak 
	}


\newcommand{\intermezzob}{\nopagebreak 
	\begin{minipage}[c]{13cm}
	\begin{center}\rule{10cm}{0.4pt}

	\tiny{\sc Optional Content Ends}
	
	\vspace{-1mm}
	
	\rule{10cm}{0.4pt}\end{center}
	\end{minipage}
	}
%--------------------






















\newtheorem*{reply*}{Reply}
\usepackage{enumitem}
\newcommand{\question}[1]{\begin{enumerate}[resume,leftmargin=0cm,labelsep=0cm,align=left]
\item #1
\end{enumerate}}

\usepackage{float}

% \setbeamertemplate{blocks}[rounded][shadow=true]
% \setbeamertemplate{itemize items}[ball]
% \AtBeginPart{}
% \AtBeginSection{}
% \AtBeginSubsection{}
% \AtBeginSubsubsection{}
% \setlength{\emergencystretch}{0em}
% \setlength{\parskip}{0pt}






\usepackage[authoryear]{natbib}

%\bibliographystyle{apalike}



\usepackage{tikz}
\usetikzlibrary{positioning,shapes,arrows}

\ifLuaTeX
  \usepackage{selnolig}  % disable illegal ligatures
\fi
\IfFileExists{bookmark.sty}{\usepackage{bookmark}}{\usepackage{hyperref}}
\IfFileExists{xurl.sty}{\usepackage{xurl}}{} % add URL line breaks if available
\urlstyle{same} % disable monospaced font for URLs
\hypersetup{
  pdftitle={Higher-order Legal Probabilism},
  pdfauthor={Rafal Urbaniak; Marcello Di Bello},
  colorlinks=true,
  linkcolor={Maroon},
  filecolor={Maroon},
  citecolor={Blue},
  urlcolor={blue},
  pdfcreator={LaTeX via pandoc}}

\title{Higher-order Legal Probabilism}
\author{Rafal Urbaniak\footnote{University of Gdansk and Basis.ai} \and Marcello
Di Bello\footnote{Arizona State University -
  \href{mailto:mdibello@asu.edu}{\nolinkurl{mdibello@asu.edu}}}}
\date{April 03, 2023}

\begin{document}
\maketitle

\paragraph{Type of contribution:}

Oral presentation type 1, 45 minutes

\paragraph{Keyworks:}

likelihood ratio; strength of evidence; Bayesian network; higher-order
probability

\paragraph{The problem}

It is standard to assess the strength of evidence using likelihood
ratios, at least for quantitative evidence such as genetic matches. For
example, the likelihood ratio of interest can be P(match evidence /
source hypothesis)/P(match evidence / alternative to source hypothesis).
Simplifying a bit, this likelihood ratio can be approximated by \(1/f\)
where \(f\) is the expected frequency (or proportion) of the matching
genetic profile in a reference population. The \(1/f\) ratio captures
some of the uncertainty associated with the match in relation to the
source hypothesis, but also leaves out crucial information. After all,
the expected frequency may have been arrived at in different ways, say,
via a larger or smaller sample. Should an additional measure of
uncertainty accompany the likelihood ratio itself?

\paragraph{The state of play}

A debate is ongoing in the forensic science literature on whether
likelihood ratios should also be accompanied by a measure of precision,
confidence or an interval estimation. (See, e.g., the 2016 special issue
of \emph{Science and Justice}, ``Special issue on measuring and
reporting the precision of forensic likelihood ratios'', edited by G.S.
Morrison.) Some scholars note that the likelihood ratio is not a
parameter to be estimated, and thus all the uncertainty -- including
sampling uncertainty -- should be encapsulated into a single number.
Others believe that sampling uncertainty should be modeled separately,
for example, via an interval estimation. Interestingly, this debate
among forensic scientists parallels a philosophical debate on how
probabilities can model a rational agent's evidence-based beliefs. One
approach, known in the philosophical literature as precise probabilism,
posits that an agent's credal state is modeled by a single, precise
probability measure. Another approach, known as imprecise probabilism,
replaces precise probabilities by sets of probability measures. The
philosophical literature contains arguments for and against each of
these views.

\paragraph{Contribution}

We favor a third approach, what we call higher-order probabilism, which
relies on distributions over probabilities. We show that there are good
theoretical reasons to abandon both precise and imprecise probabilism
and endorse higher-order probabilism. This claim has applications to the
debate in forensic science. We argue that a second-order uncertainty
should be taken into account when we assess, in probabilistic terms, the
strength of match evidence. Instead of single-number likelihood ratios
or intervals, we propose that higher-order likelihood ratios be used.
This approach is consistent with Bayesianism in epistemology and does
not require treating likelihood ratios as parameters to be estimated. In
addition, we show that higher-order probabilism can be scaled to model
complex bodies of evidence. Standard, single-number likelihood ratios
associated with different pieces can be combined together to model
complex structures of evidence by means of Bayesian networks. The same
is possible for higher-order likelihood ratios. To this end, we sketch a
formalism for constructing what we call higher-order Bayesian networks.
We illustrate this higher-order approach by revisiting familiar cases in
the literature such as Sally Clark and Charles Shonubi.

\end{document}
