% Options for packages loaded elsewhere
\PassOptionsToPackage{unicode}{hyperref}
\PassOptionsToPackage{hyphens}{url}
\PassOptionsToPackage{dvipsnames,svgnames,x11names}{xcolor}
%
\documentclass[
  10pt,
  dvipsnames,enabledeprecatedfontcommands]{scrartcl}
\usepackage{amsmath,amssymb}
\usepackage{lmodern}
\usepackage{iftex}
\ifPDFTeX
  \usepackage[T1]{fontenc}
  \usepackage[utf8]{inputenc}
  \usepackage{textcomp} % provide euro and other symbols
\else % if luatex or xetex
  \usepackage{unicode-math}
  \defaultfontfeatures{Scale=MatchLowercase}
  \defaultfontfeatures[\rmfamily]{Ligatures=TeX,Scale=1}
\fi
% Use upquote if available, for straight quotes in verbatim environments
\IfFileExists{upquote.sty}{\usepackage{upquote}}{}
\IfFileExists{microtype.sty}{% use microtype if available
  \usepackage[]{microtype}
  \UseMicrotypeSet[protrusion]{basicmath} % disable protrusion for tt fonts
}{}
\usepackage{xcolor}
\usepackage{graphicx}
\makeatletter
\def\maxwidth{\ifdim\Gin@nat@width>\linewidth\linewidth\else\Gin@nat@width\fi}
\def\maxheight{\ifdim\Gin@nat@height>\textheight\textheight\else\Gin@nat@height\fi}
\makeatother
% Scale images if necessary, so that they will not overflow the page
% margins by default, and it is still possible to overwrite the defaults
% using explicit options in \includegraphics[width, height, ...]{}
\setkeys{Gin}{width=\maxwidth,height=\maxheight,keepaspectratio}
% Set default figure placement to htbp
\makeatletter
\def\fps@figure{htbp}
\makeatother
\setlength{\emergencystretch}{3em} % prevent overfull lines
\providecommand{\tightlist}{%
  \setlength{\itemsep}{0pt}\setlength{\parskip}{0pt}}
\setcounter{secnumdepth}{5}
\newlength{\cslhangindent}
\setlength{\cslhangindent}{1.5em}
\newlength{\csllabelwidth}
\setlength{\csllabelwidth}{3em}
\newlength{\cslentryspacingunit} % times entry-spacing
\setlength{\cslentryspacingunit}{\parskip}
\newenvironment{CSLReferences}[2] % #1 hanging-ident, #2 entry spacing
 {% don't indent paragraphs
  \setlength{\parindent}{0pt}
  % turn on hanging indent if param 1 is 1
  \ifodd #1
  \let\oldpar\par
  \def\par{\hangindent=\cslhangindent\oldpar}
  \fi
  % set entry spacing
  \setlength{\parskip}{#2\cslentryspacingunit}
 }%
 {}
\usepackage{calc}
\newcommand{\CSLBlock}[1]{#1\hfill\break}
\newcommand{\CSLLeftMargin}[1]{\parbox[t]{\csllabelwidth}{#1}}
\newcommand{\CSLRightInline}[1]{\parbox[t]{\linewidth - \csllabelwidth}{#1}\break}
\newcommand{\CSLIndent}[1]{\hspace{\cslhangindent}#1}
%\documentclass{article}

% %packages
 \usepackage{booktabs}
\usepackage{subcaption}
\usepackage{multirow}
\usepackage{colortbl}
\usepackage{graphicx}
\usepackage{longtable}
\usepackage{ragged2e}
\usepackage{etex}
%\usepackage{yfonts}
\usepackage{marvosym}
\usepackage[notextcomp]{kpfonts}
\usepackage{nicefrac}
\newcommand*{\QED}{\hfill \footnotesize {\sc Q.e.d.}}
\usepackage{floatrow}
%\usepackage[titletoc]{appendix}
%\renewcommand\thesubsection{\Alph{subsection}}

\usepackage[textsize=footnotesize]{todonotes}
\newcommand{\ali}[1]{\todo[color=gray!40]{#1}}
\newcommand{\mar}[1]{\todo[color=blue!40]{#1}}
\newcommand{\raf}[1]{\todo[color=olive!40]{#1}}
%\linespread{1.5}
\newcommand{\indep}{\!\perp \!\!\! \perp\!}


\setlength{\parindent}{10pt}
\setlength{\parskip}{1pt}


%language
\usepackage{times}
\usepackage{t1enc}
%\usepackage[utf8x]{inputenc}
%\usepackage[polish]{babel}
%\usepackage{polski}




%AMS
\usepackage{amsfonts}
\usepackage{amssymb}
\usepackage{amsthm}
\usepackage{amsmath}
\usepackage{mathtools}

\usepackage{geometry}
 \geometry{a4paper,left=35mm,top=20mm,}


%environments
\newtheorem{fact}{Fact}



%abbreviations
\newcommand{\ra}{\rangle}
\newcommand{\la}{\langle}
\newcommand{\n}{\neg}
\newcommand{\et}{\wedge}
\newcommand{\jt}{\rightarrow}
\newcommand{\ko}[1]{\forall  #1\,}
\newcommand{\ro}{\leftrightarrow}
\newcommand{\exi}[1]{\exists\, {_{#1}}}
\newcommand{\pr}[1]{\mathsf{P}(#1)}
\newcommand{\cost}{\mathsf{cost}}
\newcommand{\benefit}{\mathsf{benefit}}
\newcommand{\ut}{\mathsf{ut}}

\newcommand{\odds}{\mathsf{Odds}}
\newcommand{\ind}{\mathsf{Ind}}
\newcommand{\nf}[2]{\nicefrac{#1\,}{#2}}
\newcommand{\R}[1]{\texttt{#1}}
\newcommand{\prr}[1]{\mbox{$\mathtt{P}_{prior}(#1)$}}
\newcommand{\prp}[1]{\mbox{$\mathtt{P}_{posterior}(#1)$}}

\newcommand{\s}[1]{\mbox{$\mathsf{#1}$}}


\newtheorem{q}{\color{blue}Question}
\newtheorem{lemma}{Lemma}
\newtheorem{theorem}{Theorem}



%technical intermezzo
%---------------------

\newcommand{\intermezzoa}{
	\begin{minipage}[c]{13cm}
	\begin{center}\rule{10cm}{0.4pt}



	\tiny{\sc Optional Content Starts}
	
	\vspace{-1mm}
	
	\rule{10cm}{0.4pt}\end{center}
	\end{minipage}\nopagebreak 
	}


\newcommand{\intermezzob}{\nopagebreak 
	\begin{minipage}[c]{13cm}
	\begin{center}\rule{10cm}{0.4pt}

	\tiny{\sc Optional Content Ends}
	
	\vspace{-1mm}
	
	\rule{10cm}{0.4pt}\end{center}
	\end{minipage}
	}
%--------------------






















\newtheorem*{reply*}{Reply}
\usepackage{enumitem}
\newcommand{\question}[1]{\begin{enumerate}[resume,leftmargin=0cm,labelsep=0cm,align=left]
\item #1
\end{enumerate}}

\usepackage{float}

% \setbeamertemplate{blocks}[rounded][shadow=true]
% \setbeamertemplate{itemize items}[ball]
% \AtBeginPart{}
% \AtBeginSection{}
% \AtBeginSubsection{}
% \AtBeginSubsubsection{}
% \setlength{\emergencystretch}{0em}
% \setlength{\parskip}{0pt}






\usepackage[authoryear]{natbib}

%\bibliographystyle{apalike}



\usepackage{tikz}
\usetikzlibrary{positioning,shapes,arrows}

\ifLuaTeX
  \usepackage{selnolig}  % disable illegal ligatures
\fi
\IfFileExists{bookmark.sty}{\usepackage{bookmark}}{\usepackage{hyperref}}
\IfFileExists{xurl.sty}{\usepackage{xurl}}{} % add URL line breaks if available
\urlstyle{same} % disable monospaced font for URLs
\hypersetup{
  pdftitle={Second-order Probability, Accuracy and Weight of Evidence},
  pdfauthor={Rafal Urbaniak and Marcello Di Bello},
  colorlinks=true,
  linkcolor={Maroon},
  filecolor={Maroon},
  citecolor={Blue},
  urlcolor={blue},
  pdfcreator={LaTeX via pandoc}}

\title{Second-order Probability, Accuracy and Weight of Evidence}
\author{Rafal Urbaniak and Marcello Di Bello}
\date{}

\begin{document}
\maketitle

{
\hypersetup{linkcolor=}
\setcounter{tocdepth}{2}
\tableofcontents
}
\vspace{2cm}

\noindent \textbf{DISCLAIMER:} This is a draft of work in progress,
please do not cite or distribute without permission.

\thispagestyle{empty}

\newpage

\hypertarget{introduction}{%
\section{Introduction}\label{introduction}}

A defendant in a criminal case may face multiple items of incriminating
evidence whose strength can at least sometimes be assessed using
probabilities. For example, consider a murder case in which the police
recover trace evidence that matches the defendant. Hair found at the
crime scene matches the defendant's hair (call this evidence
\textsf{hair}). In addition, the defendant owns a dog whose fur matches
the dog fur found in a carpet wrapped around one of the bodies (call
this evidence \textsf{dog}).\footnote{The hair evidence and the dog fur
  evidence are stylized after two items of evidence in the notorious
  1981 Wayne Williams case (Deadman, 1984b, 1984a).} The two matches
suggest that the defendant (and the defendant's dog) must be the source
of the crime traces (call this hypothesis, \(\mathsf{source}\)). But how
strong is this evidence, really? What are the fact-finders to make of
it?

The standard story among legal probabilists goes something like this. To
evaluate the strength of the two items of match evidence, we must find
the value of the likelihood ratio:
\[\frac{\pr{\s{dog}\wedge \s{hair} \vert \s{source}}}{\pr{\s{dog}\wedge \s{hair} \vert \neg \s{source}}}\]
For simplicity, the numerator can be equated to one. To fill in the
denominator, an expert provides the relevant random match probabilities.
Suppose the expert testifies that the probability of a random person's
hair matching the reference sample is about 0.0253, and the probability
of a random dog's hair matching the reference sample happens to be about
the same, 0.0256.\footnote{Probabilities have been slightly but not
  unrealistically modified to be closer to each other in order to make a
  conceptual point. The original probabilities were 1/100 for the dog
  fur, and 29/1148 for Wayne Williams' hair. We modified the actual
  reported probabilities slightly to emphasize the point that we will
  elaborate further on: the same first-order probabilities, even when
  they sound precise, may be affected by second-order uncertainty to
  different degrees.} Presumably, the two matches are independent lines
of evidence. In other words, their random match probabilities must be
independent of each other conditional on the source hypothesis. Then, to
evaluate the overall impact of the evidence on the source hypothesis,
you calculate: \begin{align*}
\pr{\s{dog}\wedge \s{hair} \vert \neg \s{source}} & = \pr{\s{dog} \vert \neg \s{source}} \times \pr{\s{hair} \vert \neg \s{source}} \\
& =  0.0252613 \times  0.025641 = \ensuremath{6.4772626\times 10^{-4}}
\end{align*} This is a very low number. Two such random matches would be
quite a coincidence. Following our advice from Chapter 5, the expert
facilitates your understanding of how this low number should be
interpreted. They show you how the items of match evidence change the
probability of the source hypothesis given a range of possible priors
(Figure \ref{fig:impactOfPoint}). The posterior of .99 is reached as
soon as the prior is higher than 0.061.\footnote{These calculations
  assume that the probability of a match if the suspect and the
  suspect's dog are the sources is one.} While perhaps not sufficient
for outright belief in the source hypothesis, the evidence seems
extremely strong: a minor additional piece of evidence could make the
case against the defendant overwhelming.

\begin{figure}[H]

\begin{center}\includegraphics[width=0.6\linewidth]{chapter-outline_files/figure-latex/impactOfPoint4-1} \end{center}
\caption{Impact of dog fur and human hair evidence on the prior, point estimates.}
\label{fig:impactOfPoint}
\end{figure}

Unfortunately, this analysis leaves out something crucial. You reflect
on what you have been told and ask the expert: how can you know the
random match probabilities with such precision? Shouldn't we also be
mindful of the uncertainty that may affect these numbers? The expert
agrees, and tells you that in fact the random match probability for the
hair evidence is based on 29 matches found in a database of size 1148,
while the random match probability for the dog evidence is based on
finding two matches in a reference database of size 78.

The expert's answer makes apparent that the precise random match
probabilities do not tell the whole story. What to do, then? You ask the
expert for guidance: what are reasonable ranges of the random match
probabilities? What are the worst-case and best-case scenarios? The
expert responds with 99\% credible intervals---specifically, starting
with uniform priors, the ranges of the random match probabilities are
(.015,.037) for hair evidence and (.002, .103) for fur
evidence.\footnote{Roughly, the 99\% credible interval is the narrowest
  interval to which the expert thinks the true parameter belongs with
  probability .99. For a discussion of what credible intervals are; how
  they differ from confidence intervals; and why confidence intervals
  should not be used, see Chapter XXX.} With this information, you redo
your calculations using the upper bounds of the two intervals: \(.037\)
and \(.103\). The rationale for choosing the upper bounds is that these
numbers result in random match probabilities that are most favorable to
the defendant. Your new calculation yields the following: \begin{align*}
\mathsf{P}(\s{dog}\wedge \s{hair} \vert \neg \s{source})   & =  .037 \times .103 =.003811.
\end{align*} This number is around 5.88 times greater than the original
estimate. Now the prior probability of the source hypothesis needs to be
higher than 0.274 for the posterior probability to be above .99 (Figure
\ref{fig:impactOfCharitable}). So you are no longer convinced that the
two items of match evidence are strongly incriminating.

\begin{figure}[H]

\begin{center}\includegraphics[width=0.6\linewidth]{chapter-outline_files/figure-latex/fig:charitableImpact7-1} \end{center}
\caption{Impact of dog fur and human hair evidence on the prior, charitable reading.}
\label{fig:impactOfCharitable}
\end{figure}

This result is puzzling. Are the two items of match evidence strongly
incriminating evidence (as you initially thought) or somewhat weaker (as
the new calculation suggests)? For one thing, using precise random match
probabilities might be too unfavorable toward the defendant. On the
other hand, your new assessment of the evidence based on the upper
bounds might be too \emph{favorable} toward them. Is there a middle way
that avoids overestimating and underestimating the strength of the
evidence?

To see what this middle path looks like, we should reconsider the
calculations you just did. Here you made an important blunder: you
assumed that because the worst-case probability for one event is \(x\)
and the worst-case probability for another independent event is \(y\),
the worst-case probability for their conjunction is \(xy\). But this
conclusion does not follow so long as the margin of error (or credible
interval) is kept fixed.\footnote{The intuitive reason is simple: just
  because the probability of an extreme (or larger absolute) value \(x\)
  for one variable is .01, and so it is for the value \(y\) of another
  independent variable, it does not follow that the probability that
  those two independent variables take values \(x\) and \(y\)
  simultaneously is the same. This probability is actually much smaller.}
The interval presentation instead of doing us good led us into error.

Even if we are right about the interval that we obtained for the
probability of a single event, it is not straightforward to calculate a
new interval once a second event is included. We need additional
information: the distributions that were used to calculate the intervals
for the probabilities of the individual events.\footnote{Also, in
  principle, we need further information about how these items of
  evidence are related if we cannot take them to be independent, which
  we here assume random matches are.} So, in the case of the two items
of match evidence, if you knew what distributions the expert used (it
should have been beta distributions in this context), you could work
your way back and calculate the 99\% credible interval for both items of
evidence. As it turns out, given the reported sample sizes, the credible
interval for the probability
\(\mathsf{P}(\s{dog}\wedge \s{hair} \vert \neg \s{source})\) is
\((0.000023, 0.002760)\).\footnote{The 99\% credible interval (or a 99\%
  margin of error) is not the 99\% confidence interval known from
  classical statistics. There are various reasons not to use these,
  already discussed in Chapter 3. Another sense, in which we mean it
  here, is the range to which the true value belongs with posterior
  probability of 99\% given the evidence. Normally we mean highest
  posterior density intervals, that is the narrowest intervals with this
  property.} Using the upper bound of this interval would then require
the prior probability of the source hypothesis to be above .215 for the
posterior to be above .99. On this interpretation, the two items of
match evidence are still not quite as strong as you initially thought,
but stronger than what your second calculation indicated.

Still, the interval approach---even the corrected version just
outlined---suffers from a more general problem. Working with intervals
might be useful if the underlying distributions are fairly symmetrical.
But in our case, they might not be. For instance, Figure
\ref{fig:densities} depicts beta densities for dog fur and human hair,
together with sampling-approximated density for the joint evidence. The
distribution for the joint evidence is not symmetric. If you were only
informed about the edges of the interval, you would be oblivious to the
fact that the most likely value (and the bulk of the distribution,
really) does not simply lie in the middle between the edges. Just
because the parameter lies in an interval with some posterior
probability, it does not mean that the ranges near the edges of the
interval are equally likely---the bulk of the density might very well be
closer to one of the edges. Therefore, only relying on the edges can
lead one to either overestimate or underestimate the probabilities at
play. This also means that---following our advice on how to illustrate
the impact of evidence on prior probabilities---a better representation
of the dependence of the posterior on the prior should comprise multiple
possible sampled lines whose density mirrors the density around the
probability of the evidence (Figure \ref{fig:lines}).

\begin{figure}[H]

\begin{center}\includegraphics[width=0.8\linewidth]{chapter-outline_files/figure-latex/fig:densities-1} \end{center}
\caption{Beta densities for individual items of evidence and the resulting joint density with .99 and .9 highest posterior density intervals, assuming the sample sizes as discussed and independence, with uniform priors.}
\label{fig:densities}
\end{figure}

\begin{figure}[H]

\begin{center}\includegraphics[width=0.6\linewidth]{chapter-outline_files/figure-latex/fig:lines3-1} \end{center}

\caption{100 lines illustrating the uncertainty about the dependence of the posterior on the prior given aleatory uncertainty about the evidence, with the distribution of the minimal priors required for the posterior to be above .99.}

\label{fig:lines}

\end{figure}

This, then, is the working hypothesis of this chapter: whenever density
estimates for the probabilities of interest are available (and they
should be available for match evidence whose reliability has been
properly studied), those densities should be reported for assessing the
strength of the evidence. This approach avoids hiding actual aleatory
uncertainties under the carpet. It also allows for a balanced assessment
of the evidence, whereas using point estimates or intervals may
exaggerate or underestimate the value of the evidence.

The rest of this chapter expands on this idea in a few dimensions.
First, it places it in the context of philosophical discussions
surrounding a proper probabilistic representation of uncertainty. The
main alternatives on the market are precise probabilism and imprecise
probabilism. We argue that both options are problematic and should be
superseded by a higher-order approach whenever possible. Second, having
gained this perspective, we revisit a recent discussion in the forensic
science literature, where a prominent proposal is that the experts, even
if they use densities, should integrate and present only point estimates
to the fact-finders. We disagree. Third, we explain how the approach can
be used in more complex situations in which multiple items of evidence
and multiple propositions interact---and the idea is that such
complexities can be handled by sampling from distributions and
approximating densities using multiple Bayesian Networks in the
calculations. Last but not least, we turn to the notion of weight of
evidence. Having distinguished quite a few notions in the vicinity, we
explain how the framework we propose allows for a more successful
explication and implementation of the notion of weight of evidence than
the ones currently available on the market.

\hypertarget{three-probabilisms}{%
\section{Three Probabilisms}\label{three-probabilisms}}

The introduction outlined three probabilistic approaches that one might
take for assessing the value of the evidence presented at trial. The
first approach uses precise probabilities; the second uses intervals;
the third uses distributions over probabilities. By relying on an
example featuring two items of match evidence, we suggested that the
third approach is preferable. This section buttresses this claim by
providing principled, philosophical reasons in favor of the third
approach.

The three approaches we considered correspond (roughly) to three ways in
which probabilities can be deployed to model a rational agent's fallible
and evidence-based beliefs about the world. The first approach, known in
the philosophical literature as precise probabilism, posits that an
agent's credal state is modeled by a single, precise probability
measure. The second approach, known as imprecise probabilism, replaces
precise probabilities by sets of probability measures. The third
approach, what we call higher-order probabilism, relies on distributions
over parameter values. There are good reasons to abandon precise
probabilism and endorse higher-order probabilism. Imprecise probabilism
is a step in the right direction, but also suffers from too many
difficulties of its own.

\hypertarget{precise-probabilism}{%
\subsection{Precise Probabilism}\label{precise-probabilism}}

Precise probabilism (\textsf{PP}) holds that a rational agent's
uncertainty about a hypothesis is to be represented as a single, precise
probability measure. This is an elegant and simple theory. But
representing our uncertainty about a proposition in terms of a single,
precise probability runs into a number of difficulties. Precise
probabilism fails to capture an important dimension of how our fallible
beliefs reflect the evidence we have (or have not) obtained. A couple of
stylized examples should make the point clear. (For the sake of
simplicity, we will use examples featuring coins, but biases of coins
can be thought of as random match probabilities in the forensic
context.)

\begin{quote}
\textbf{No evidence v. fair coin}
You are about to toss a coin, but have no evidence 
whatsoever about its bias. You are completely ignorant. 
Compare this to the situation in which you know, 
based on overwhelming evidence, that the coin is fair. 
\end{quote}

\noindent On precise probabilism, both scenarios are represented by
assigning a probability of .5 to the outcome \emph{heads}. If you are
completely ignorant, the principle of insufficient evidence suggests
that you assign .5 to both outcomes. Similarly, if you know for sure the
coin is fair, assigning .5 seems the best way to quantify the
uncertainty about the outcome. The agent's evidence in the two scenario
is quite different, but precise probabilities cannot capture this
difference.

\begin{quote}
\textbf{Learning from ignorance}
You toss a coin with unknown bias. You toss it 10 times and observe \emph{heads} 5 times. Suppose you toss it further and observe 50 \emph{heads} in 100 tosses. 
\end{quote}

\noindent Since the coin initially had unknown bias, you should
presumably assign a probability of .5 to both outcomes. After the 10
tosses, you end up again with an estimate of .5. You must have learned
something, but whatever that is, it is not modeled by precise
probabilities. When you toss the coin 100 times and observe 50 heads,
you learn something. But your precise probability assessment will again
be .5.

These examples suggest that precise probabilism is not appropriately
responsive to evidence. It ends up assigning the same probability in
situations in which one's evidence is quite different. For instance,
when no evidence is available about the coin's bias, when there is
little evidence that the coin is fair (say, after only 10 tosses), and
when there is strong evidence that the coin is fair (say, after 100
tosses). The general problem is, precise probability captures the value
around which your uncertainty should be centered, but fails to capture
how centered it should be given the evidence.\footnote{Precise
  probabilism suffers from other difficulties. For example, it has
  problems with formulating a sensible method of probabilistic opinion
  aggregation Stewart \& Quintana (2018). A seemingly intuitive
  constraint is that if every member agrees that \(X\) and \(Y\) are
  probabilistically independent, the aggregated credence should respect
  this. But this is hard to achieve if we stick to \s{PP} (Dietrich \&
  List, 2016). For instance, a \emph{prima facie} obvious method of
  linear pooling does not respect this. Consider probabilistic measures
  \(p\) and \(q\) such that \(p(X) = p(Y) = p(X\vert Y) = 1/3\) and
  \(q(X) = q(Y) = q(X\vert Y) = 2/3\). On both measures, taken
  separately, \(X\) and \(Y\) are independent. Now take the average,
  \(r=p/2+q/2\). Then \(r(X\cap Y) = 5/18 \neq r(X)r(Y)=1/4\).}

\hypertarget{imprecise-probabilism}{%
\subsection{Imprecise Probabilism}\label{imprecise-probabilism}}

What if we give up the assumption that probability assignments should be
precise? Imprecise probabilism (\textsf{IP}) holds that an agent's
credal stance towards a hypothesis is to be represented by means of a
\emph{set of probability measures}, typically called a representor
\(\mathbb{P}\), rather than a single measure \(\mathsf{P}\). The
representor should include all and only those probability measures which
are compatible with the evidence. For instance, if an agent knows that
the coin is fair, their credal state would be captured by the singleton
set \(\{\mathsf{P}\}\), where \(\mathsf{P}\) is a probability measure
which assigns \(.5\) to \emph{heads}. If, on the other hand, the agent
knows nothing about the coin's bias, their credal state would rather be
represented as the set of all probabilistic measures, as none of them is
excluded by the available evidence. Note that the set of probability
measures does not represent admissible options that the agent could
legitimately pick from. Rather, the agent's credal state is essentially
imprecise and should be represented by means of the entire set of
probability measures.\footnote{For the development of imprecise
  probabilism, see Keynes (1921); Levi (1974); Gärdenfors \& Sahlin
  (1982); Kaplan (1968); Joyce (2005); Fraassen (2006); Sturgeon (2008);
  Walley (1991). S. Bradley (2019) is a good source of further
  references. Imprecise probabilism shares some similarities with what
  we might call \textbf{interval probabilism} (Kyburg, 1961; Kyburg Jr
  \& Teng, 2001). On interval probabilism, precise probabilities are
  replaced by intervals of probabilities. On imprecise probabilism,
  instead, precise probabilities are replaced by sets of probabilities.
  This makes imprecise probabilism more general, since the probabilities
  of a proposition in the representor set do not have to form a closed
  interval. As we have already noted, intervals do not contain
  probabilistic information sufficient to guide reasoning with multiple
  items of evidence. So we focus on \s{IP}, which is the more promising
  approach.}

Imprecise probabilism, at least \emph{prima facie}, offers a
straightforward picture of learning from evidence, that is a natural
extension of the classical Bayesian approach. When faced with new
evidence \(E\) between time \(t_0\) and \(t_1\), the representor set
should be updated point-wise, running the standard Bayesian updating on
each probability measure in the representor:
\begin{align*} \label{eq:updateRepresentor}
\mathbb{P}_{t_1} = \{\mathsf{P}_{t_1}\vert \exists\, {\mathsf{P}_{t_0} \!\in  \mathbb{P}_{t_0}}\,\, \forall\, {H}\,\, \left[\mathsf{P}_{t_1}(H)=\mathsf{P}_{t_0}(H \vert E)\right] \}.
\end{align*}

\noindent The hope is that, if we start with a range of probabilities
that is not extremely wide, point-wise learning will behave
appropriately. For instance, if we start with a prior probability of
\emph{heads} equal to .4 or .6, then those measure should be updated to
something closer to \(.5\) once we learn that a given coin has already
been tossed ten times with the observed number of heads equal 5 (call
this evidence \(E\)). This would mean that if the initial range of
values was \([.4,.6]\) the posterior range of values should be more
narrow. But even this seemingly straightforward piece of reasoning is
hard to model without using densities. For to calculate
\(\pr{\s{heads}\vert E}\) we need to calculate
\(\pr{E \vert \s{heads}}\pr{\s{heads}}\) and divide it by
\(\pr{E} = \pr{E \vert \s{heads}}\pr{\s{heads}} + \pr{E} = \pr{E \vert \neg \s{heads}}\pr{\neg \s{heads}}\).
The tricky part is obtaining the conditional probabilities
\(\pr{E \vert \s{heads}}\) and \(\pr{E \vert \neg \s{heads}}\) in a
principled manner without explicitly going second-order, estimating the
parameter value and using beta distributions.

The situation is even more difficult if we start with complete lack of
knowledge, as imprecise probabilism runs into the problem of
\textbf{belief inertia} (Levi, 1980). Say you start tossing a coin
knowing nothing about its bias. The range of possibilities is \([0,1]\).
After a few tosses, if you observed at least one tail and one heads, you
can exclude the measures assigning 0 or 1 to \emph{heads}. But what else
have you learned? If you are to update your representor set point-wise,
you will end up with the same representor set. Consequently, the edges
of your resulting interval will remain the same. In the end, it is not
clear how you are supposed to learn anything if you start from complete
ignorance.\footnote{Here's another example from Rinard (2013). Either
  all the marbles in the urn are green (\(H_1\)), or exactly one tenth
  of the marbles are green (\(H_2\)). Your initial credence \([0,1]\) in
  each. Then you learn that a marble drawn at random from the urn is
  green (\(E\)). After conditionalizing each function in your
  representor on this evidence, you end up with the the same spread of
  values for \(H_1\) that you had before learning \(E\), and no matter
  how many marbles are sampled from the urn and found to be green.}

Some downplay the problem of belief inertia. They insist that vacuous
priors should not be used and that imprecise probabilism gives the right
results when the priors are non-vacuous. After all, if you started with
knowing truly nothing, then perhaps it is right to conclude that you
will never learn anything. Another strategy is to say that, in a state
of complete ignorance, a special updating rule should be
deployed.\footnote{Elkin (2017) suggests the rule of
  \emph{credal set replacement} that recommends that upon receiving
  evidence the agent should drop measures rendered implausible, and add
  all non-extreme plausible probability measures. This, however, is
  tricky. One needs a separate account of what makes a distribution
  plausible or not, as well as a principled account of why one should
  use a separate special update rule when starting with complete
  ignorance.} But no matter what we think about belief inertia, other
problems plague imprecise probabilism. Two more problems are
particularly pressing.

One problem is that imprecise probabilism fails to capture intuitions we
have about evidence and uncertainty in a number of scenarios. Consider
this example:

\begin{quote}
\textbf{Even v. uneven bias:}
 You have two coins and you know, for sure, that the probability of getting heads is .4, if you toss one coin, and .6, if you toss the other coin. But you do not know which is which. You pick one of the two at random and toss it.  Contrast this with an uneven case. You have four coins and you know that three of them have bias $.4$ and one of them has bias $.6$. You pick a coin at random and plan to toss it. You should be three times more confident that the probability of getting heads is .4. rather than .6.
\end{quote}

\noindent The first situation can be easily represented by imprecise
probabilism. The representor would contain two probability measures, one
that assigns .4. and the other that assigns .6 to the hypothesis `this
coin lands heads'. But imprecise probabilism cannot represent the second
situation, at least not without moving to higher-order probabilities or
assigning probabilities to chance hypotheses, in which case it is no
longer clear whether the object-level imprecision performs any valuable
task.\footnote{Other scenarios can be constructed in which imprecise
  probabilism fails to capture distinctive intuitions about evidence and
  uncertainty; see, for example, (Rinard, 2013). Suppose you know of two
  urns, \textsf{GREEN} and \textsf{MYSTERY}. You are certain
  \textsf{GREEN} contains only green marbles, but have no information
  about \textsf{MYSTERY}. A marble will be drawn at random from each.
  You should be certain that the marble drawn from \textsf{GREEN} will
  be green (\(G\)), and you should be more confident about this than
  about the proposition that the marble from \textsf{MYSTERY} will be
  green (\(M\)). In line with how lack of information is to be
  represented on \textsf{IP}, for each \(r\in [0,1]\) your representor
  contains a \(\mathsf{P}\) with \(\pr{M}=r\). But then, it also
  contains one with \(\pr{M}=1\). This means that it is not the case
  that for any probability measure \(\mathsf{P}\) in your representor,
  \(\mathsf{P}(G) > \mathsf{P}(M)\), that is, it is not the case that RA
  is more confident of \(G\) than of \(M\). This is highly
  counter-intuitive.}

Second, besides descriptive inadequacy, an even deeper, foundational
problem exists for imprecise probabilism. This problem arises when we
attempt to measure the accuracy of a representor set of probability
measures. Workable \emph{scoring rules} exists for measuring the
accuracy of a single, precise credence function, such as the Brier
score. These rules measure the distance between one's credence function
(or probability measure) and the actual value. A requirement of scoring
rules is that they be \emph{proper}: any agent will score their own
credence function to be more accurate than every other credence
function. After all, if an agent thought a different credence was more
accurate, they should switch to it. Proper scoring rules are then used
to formulate accuracy-based arguments for precise probabilism. These
arguments show (roughly) that, if your precise credence follows the
axioms of probability theory, no other credence is going to be more
accurate than yours whatever the facts are. Can the same be done for
imprecise probabilism? It seems not. Impossibility theorems demonstrate
that no proper scoring rules are available for representor sets. So, as
many have noted, the prospects for an accuracy-based argument for
imprecise probabilism look dim (Campbell-Moore, 2020; Mayo-Wilson \&
Wheeler, 2016; Schoenfield, 2017; Seidenfeld, Schervish, \& Kadane,
2012). Moreover, as shown by Schoenfield (2017), if an accuracy measure
satisfies certain plausible formal constraints, it will never strictly
recommend an imprecise stance, as for any imprecise stance there will be
a precise one with at least the same accuracy.

\hypertarget{higher-order-probabilism}{%
\subsection{Higher-order Probabilism}\label{higher-order-probabilism}}

There is, however, a view in the neighborhood that fares better: a
second-order perspective. In fact, some of the comments by the
proponents of imprecise probabilism tend to go in this direction. For
instance, Seamus Bradley compares the measures in a representor to
committee members, each voting on a particular issue, say the true bias
of a coin. As they acquire more evidence, the committee members will
often converge on a specific chance hypothesis. He writes (S. Bradley,
2012, p. 157):

\begin{quote}
\dots the committee members are ''bunching up''. Whatever measure you put over the set of probability functions---whatever ''second order probability'' you use---the ''mass'' of this measure gets more and more concentrated around the true chance hypothesis'.
\end{quote}

\noindent Note, however, that such bunching up cannot be modeled by
imprecise probabilism. Joyce (2005), in a paper defending imprecise
probabilism, in fact uses a density over chance hypotheses to account
for the notion of evidential weight. The idea that one should use
higher-order probabilities has also been suggested by critics of
imprecise probabilism. For example, Carr (2020) argues that sometimes
evidence requires uncertainty about what credences to have. Carr,
however, does not articulate this suggestion more fully, does not
develop it formally, and does not explain how her approach would fare
against the difficulties affecting precise ad imprecise probabilism.

The key idea of the higher-order approach we propose is that uncertainty
is not a single-dimensional thing to be mapped on a single
one-dimensional scale such as a real line. It is the whole shape of the
whole distribution over parameter values that should be taken under
consideration.\footnote{Bradley admits this much (S. Bradley, 2012, p.
  90), and so does Konek (Konek, 2013, p. 59). For instance, Konek
  disagrees with: (1) \(X\) is more probable than \(Y\) just in case
  \(p(X)>p(Y)\), (2) \(D\) positively supports \(H\) if
  \(p_D(H)> p(H)\), or (3) \(A\) is preferable to \(B\) just in case the
  expected utility of \(A\) w.r.t. \(p\) is larger than that of \(B\).}
From this perspective, when an agent is asked about their credal stance
towards \(X\), they can refuse to summarize it in terms of a point value
\(\mathsf{P}(X)\). They can instead express their credal stance in terms
of a probability (density) distribution \(f_x\) treating
\(\mathsf{P}(X)\) as a random variable. To be sure, an agent's credal
state toward \(X\) could sometimes be usefully represented by the
expectation\\
\[\int_{0}^{1} x f(x) \, dx\] as the precise, object-level credence in
\(X\), where \(f\) is the probability density over possible object-level
probability values. But this need not always be the case. If the
probability density \(f\) is not sufficiently concentrated around a
single value, a one-point summary might fail to do justice to the
nuances of the agent's credal state.\footnote{This approach lines up
  with common practice in Bayesian statistics, where the primary role of
  uncertainty representation is assigned to the whole distribution.
  Summaries such as the mean, mode standard deviation, mean absolute
  deviation, or highest posterior density intervals are only succinct
  ways for representing the uncertainty of a given scenario. Whether the
  expectation should be used in betting behavior is a separate problem.
  Here we focus on epistemic issues.} For example, consider again the
scenario in which the agent knows that the bias of the coin is either .4
or .6 but the former is three times more likely. Representing the
agent's credal state with the expectation
\(\mathsf{P}(X) = .75 \times .4 + .25 \times .6 = .45\) would be
inadequate as it would fail to capture the agent's belief that the two
biases are uneven.

The higher-order approach can easily model all the challenging scenarios
we discussed so far in the manner illustrated in Figure
\ref{fig:evidenceResponse}. In particular, the scenario in which the two
biases of the coin are not equally likely---which imprecise probabilism
cannot model---can be easily modeled within high-order probabilism by
assigning different probabilities to the two biases.

\begin{figure}[t]

\begin{center}\includegraphics[width=0.8\linewidth]{chapter-outline_files/figure-latex/fig:evidenceResponse2-1} \end{center}
\caption{Examples of higher-order distributions for scenarios brought up in the literature.}
\label{fig:evidenceResponse}
\end{figure}

Besides its flexibility in modelling uncertainty, higher-order
probabilism does not fall prey to belief inertia. Consider a situation
in which you have no idea about the bias of a coin. So you start with a
uniform density over \([0,1]\) as your prior. By using binomial
probabilities as likelihoods, observing any non-zero number of heads
will exclude 0 and observing any non-zero number of tails will exclude 1
from the basis of the posterior. The posterior distribution will become
more centered around the parameter estimate as the observations come in.
Figure \ref{fig:intertia2} shows---starting with a uniform prior
distribution--- how the posterior distribution changes after successive
observations of heads, heads again, and then tails.\footnote{More
  generally, learning about frequencies, assuming independence and
  constant probability for all the observations, is modeled the Bayes
  way. You start with some prior density \(p\) over the parameter
  values. If you start with complete lack of information, \(p\) should
  be uniform. Then, you observe the data \(D\) which is the number of
  successes \(s\) in a certain number of observations \(n\). For each
  particular possible value \(\theta\) of the parameter, the probability
  of \(D\) conditional on \(\theta\) follows the binomial distribution.
  The probability of \(D\) is obtained by integration. That is:
  \begin{align*}
  p(\theta \vert D) & = \frac{p(D\vert \theta)p(\theta)}{p(D)}\\
  & = \frac{\theta^s (1-\theta)^{(n - s)}p(\theta)}{\int (\theta')^s (1-\theta')^{(n - s)}p(\theta')\,\, d\theta'}.
  \end{align*}}

\begin{figure}[t]

\begin{center}\includegraphics[width=0.8\linewidth]{chapter-outline_files/figure-latex/fig:inertia3-1} \end{center}
\caption{As observations of heads, heads and tails come in, extreme parameter values drop out of the picture and the posterior is shaped by the evidence.}
\label{fig:intertia2}
\end{figure}

A further advantage of high-order probabilism over imprecise probabilism
is that the prospects for accuracy-based arguments are not foreclosed.
This is a significant shortcoming of imprecise probabilism, especially
because such arguments exist for precise probabilism. One can show that
there exist proper scoring rules for higher-order probabilism. These
rules can then be used to formulate accuracy-based arguments. Another
interesting feature of the framework is that the point made by
Schoenfield against imprecise probabilism does not apply: there are
cases in which accuracy considerations recommend an imprecise stance
(that is, a multi-modal distribution) over a precise one (Urbaniak, 2022
manuscript).

All in all, higher-order probabilism outperforms both precise and
imprecise probabilism, at the descriptive as well as the normative
level. From a descriptive standpoint, higher-order probabilism can
easily model a variety of scenarios that cannot be adequately modeled by
the other versions of probabilism. From a normative standpoint, accuracy
maximization may sometimes recommend that a rational agent represent
their credal state with a distribution over probability values rather
than a precise probability measure (more on this in the next section).

\hypertarget{objections-to-the-higher-order-approach}{%
\section{Objections to the Higher-order
Approach}\label{objections-to-the-higher-order-approach}}

This section addresses a number of conceptual difficulties that may
arise in using higher-order probabilities, with focus on those brought
up by prominent legal evidence scholars. In discussing these conceptual
issues, we will formulate an accuracy-based argument that using
higher-order probabilities is actually preferable to precise
probabilities.

\hypertarget{the-taroni-sjerps-debate}{%
\subsection{The Taroni-Sjerps Debate}\label{the-taroni-sjerps-debate}}

Our main polemic target is a discussion initiated by Taroni, Bozza,
Biedermann, \& Aitken (2015), who argue extensively that trial experts
should only report point estimates. Their point of departure is a
reflection on match evidence. Say an expert reports at trial that the
sample from the crime scene matches the defendant. The significance of
this match should be evaluated in light of the population frequency
\(\theta\) of the matching profile. This frequency, however, cannot be
known for sure and must instead be estimated. The expert will estimate
the true parameter \(\theta\) by means of a probabilistic distribution
\(p(\theta)\) over its possible values. For example, if the observations
are realizations of independent and identically distributed Bernoulli
trails given \(\theta\), the expert's uncertainty about \(\theta\) can
be captured as \(\s{beta}(\alpha + s + 1 ,\beta + n - s)\), where \(s\)
is the number of observed successes, \(n\) the number of observations in
the database (1 is added to the first shape parameter to include the
match with the suspect), and \(\alpha\) and \(\beta\) capture the
expert's priors.

Nothing so far should be controversial. However, the question arises of
how the expert should report their own uncertainty about \(\theta\). To
fix the notation, let the prosecution hypothesis \(H_p\) be that the
suspect is the source of the trace, and the defense hypothesis \(H_d\)
that another person, unrelated to the suspect, is the source. For
simplicity, assume that if \(H_p\) holds, the laboratory will surely
report a match \(M\), so that \(\pr{M\vert H_p}=1\). The likelihood
ratio, then, reduces to \(\nicefrac{1}{\pr{M \vert H_d}}\). Taroni et
al. (2015) claim that the probability of the match evidence given the
defense hypothesis should be calculated as follows:
\begin{align*}\pr{M \vert H_d} & = \int_{\theta} \pr{M\vert \theta} \pr{\theta}\,\, d\theta \\
& =  \int_\theta  \theta \pr{\theta}\,\, d\theta
\end{align*} In case of a DNA match, they recommend that the expert
report the expected value of the \(\s{beta}\) distribution, which
reduces to \(\nicefrac{\alpha + s + 1}{\alpha + \beta +n + 1}\). They
claim that this number satisfactorily expresses the posterior
uncertainty about \(\theta\). For them, it is this probability alone
that should be used in the denominator in the calculation and reporting
of the likelihood ratio.

Sjerps et al. (2015) disagree. In reporting a single value, the expert
would refrain from providing the fact-finders with relevant information
that can make a difference in the evaluation of the evidence. There is a
difference between (a) an expert who is certain \(\theta\) is \(.1\);
(b) an expert whose best estimate of \(\theta\) is \(.1\) based on a
thousand of observations; and (c) an expert whose best estimate of
\(\theta\) is again \(.1\) but based on only ten observations. These
three scenarios mirror scenarios we discussed earlier: (a) the bias of a
coin is known for sure; (b) the bias is estimated on the basis of a
large number of tosses; and (c) the bias is estimated using a small set
of observations. As our critique of precise probabilism makes clear, a
simple point estimate (or precise probability) would fail to capture the
differences among the three scenarios.\footnote{Taroni et al. (2015)
  admits that the expert, besides providing a point estimate, should
  also informally explain how the estimate was arrived at. They grant
  that this additional information can be helpful so long as the
  recipients are instructed on ``the nature of probability, the
  importance of an understanding of it and its proper use in dealing
  with uncertainty'' {[}p.~16{]}. But why stop at an informal
  presentation? It is unclear why the fact-finders should be deprived of
  quantifiable information about the aleatory uncertainty of the
  parameter of interest and only be given an informal description of
  what the expert did, along with some remarks about the nature of
  probability. It is wildly optimistic to assume this way of proceeding
  is enough to secure a proper assessment of the evidence.}

The debate between Taroni et al. (2015) and Sjerps et al. (2015) mostly
rests on diverging philosophical commitments about the nature of
probability and the relation between uncertainty and betting behavior.
Taroni et al. (2015) argues that, if probabilities express an agent's
epistemic attitude towards a proposition, probabilities are not states
of nature, but states of mind associated with individuals. For them,
this commitment to subjective probability has two consequences. First,
it makes no sense sense to talk about second-order uncertainty about
subjective probabilities, as there is no underlying state of the nature
to estimate. Second, if these subjective probabilities can be elicited
by examining an agent's betting preferences, a proper elicitation will
lead to a single number.\footnote{They write: ``Clearly, one can adjust
  the measure of belief of success in the reference gamble in such a way
  that one will be indifferent with respect to the truth of the event
  about which one needs to give one's probability. This understanding is
  fundamental, as it implies that probability is given by a single
  number. It may be hard to define, but that does not mean that
  probability does not exist in an individual's mind. One cannot
  logically have two different numbers because they would reflect
  different measures of belief.'' Taroni et al. (2015, p. 7)} On the
other hand, Sjerps et al. (2015) reject the sharp distinction between
states of nature (which can be estimated) and mental attitudes (which
cannot be in any meaningful way estimated because they are subjective).
More recently, Dahlman \& Nordgaard (2022) have also emphasized that the
distinction is not so clear-cut. They argue that, if a probability
assessment is a subjective attitude that is elicited via a betting
preference, a probability assessment is itself a state of nature: ``the
formation of a betting preference by a certain person at a certain
time'' {[}p.~15{]}.

\hypertarget{an-accuracy-based-argument}{%
\subsection{An Accuracy-based
Argument}\label{an-accuracy-based-argument}}

We sidestep these conceptual issues for now. In what follows, we hope to
break the stalemate in the debate by proving an argument to which both
parties should be receptive. This is a accuracy-based argument in favor
of using higher-order probabilities---roughly, if you discard relevant
information that you already have in the densities and rely on point
estimates, your predictions about the world will be less accurate in a
very precise quantifiable sense.

First, let us go over a particular example. Suppose we randomly draw a
true population frequency from the uniform distribution. In our
particular case, we obtained 0.632. Then, we randomly draw a sample size
as a natural number between 10 and 20 (our points holds with larger
samples, just the discrepancies get smaller). In our particular case, it
is 16. Next, we simulate an experiment in which we draw that number of
observations from the true distribution. We observe 8 successes and use
this number to calculate the point estimate of the parameter, which is
0.5.

What is the probability mass function (PMF) for all possible outcomes of
an observation of the same size? Two probabilities mass functions are
initially relevant: first, the true probability mass based on the true
parameter; second, the probability mass function based on the point
estimate which is binomial around the point estimate. This latter PMF,
however, does not take into account the uncertainty about the point
estimate. To take this uncertainty seriously, we take a sampling
distribution of size 16 of possible parameter values from the posterior
\(\s{beta}(1+\s{successes}, 1+\s{sample size} - \s{successes})\)
distribution (we assume uniform prior for the sake of an example). Then,
we use this sample of parameter values to simulate observations, one
simulation for each parameter value in the sample. This simulation
yields the so-called \emph{posterior predictive distribution} (or
posterior predictive PMF), which instead of a point estimate, propagates
the uncertainty about the parameter value into the predictions about the
outcomes of possible observations. Finally, we take simulated
frequencies as our estimates of probabilities. This distribution is more
honest about uncertainty and wider than the one obtained using the point
estimate. The three PMFs are illustrated in Figure
\ref{fig:posteriorPrediction}.

\begin{figure}[H]

\begin{center}\includegraphics[width=0.8\linewidth]{chapter-outline_files/figure-latex/fig:posteriorPrediction2-1} \end{center}


\caption{Real probability mass, probability mass calculated using a point estimate, sampling distribution from the posterior, and the posterior predictive distribution based on this sampling distribution.}
\label{fig:posteriorPrediction}
\end{figure}

The PMF based on a point estimate is further off from the real PMF than
the posterior predictive distribution. For instance, if we ask about the
probability of the outcome being at least 9 successes, the true answer
is 0.7984, the point estimate PMF tells us it is 0.4056, while the
posterior predictive distribution gives a somewhat better guess at
0.4277. A similar thing happens when we ask about the probability of the
outcome being at most 9 successes. The true answer is 0.3681, the
point-estimate-based answer is 0.778, while the posterior predictive
distribution yields 0.7051. More generally, we can use an
information-theoretic measure, Kullback-Leibler divergence, to quantify
how far the point-estimate PMF and the posterior predictive PMF are from
the true PMF.\footnote{DEFINE Kullback-Leibler divergence.} In our
particular case, the former distance is 0.7905638 and the latter is
0.5681121.

This result can be generalized. We repeat the simulation 1000 times,
each time with a new true parameter, a new sample size, and a new
sample. Every time the three PMFs are constructed using the methods we
described. Figure \ref{fig:kldsPlots} displays the empirical
distribution of the results of such a simulation, where the differences
are measured by the Kullback-Leibler divergence. A positive difference
indicates that the distribution based on the point-estimate was further
from the true PMF than the posterior predictive distribution. Notably,
the mean difference is 0.865, the median difference is 0.044, and the
distribution is asymmetrical, as there are multiple cases of huge
differences favoring posterior predictive distributions over point-based
predictions. All in all, accuracy-wise, point-estimate-based PMFs are
systematically worse than the posterior predictive distribution.

\begin{figure}[H]

\begin{center}\includegraphics[width=0.7\linewidth]{chapter-outline_files/figure-latex/fig:kldsPlots-1} \end{center}
\caption{Differences in Kullback-Leibler divergencies from the true distributions, comparing the distributions obtained using point estimates and posterior predictive distributions. Positive values indicate the point-estimate-based PMF was further from the true distribution than the posterior predictive distribution.}
\label{fig:kldsPlots}
\end{figure}

\hypertarget{conceptual-issues}{%
\subsection{Conceptual Issues}\label{conceptual-issues}}

Accuracy considerations aside, we will now engage with the more
conceptual points. Taroni et al. (2015) argue that since first-order
probabilities capture your uncertainty about a proposition of interest,
second-order probabilities are supposed to capture your uncertainty
about how uncertain you are, and that ``estimating'' your first-order
uncertainties is unnecessary. They think that you can simply figure out
your fair odds in a suitable bet on the proposition in question, and the
fair odds track your unique, first-order uncertainty without any
uncertainty about it. But this point can be questioned. For one thing,
the betting interpretation of probability is not
uncontroversial.\footnote{See textbooks in formal epistemology (D.
  Bradley, 2015; Titelbaum, 2020).} In addition, even granting the
betting interpretation, it need not be implausible to say that we
sometimes are uncertain about what we think the fair bets are or what
our first-order uncertainties are.\footnote{On a related score, the
  introspective axioms in epistemic logic---that is, if an agent knows
  (or doesn't know) \(p\), they also know that they (don't) know
  \(p\)---are by no means uncontroversial. See, for example, Williamson
  2000 (chapter 5)'s argument against the KK principle of positive
  introspection.}

Think again about an expert who gathers information about the allelic
frequency \(f\) of DNA matches in an available database, and starts with
a defensible \s{beta} prior with parameters \(\alpha, \beta\). Say the
expert observes \(s\) matches in a database of size \(n\). So the
population relative frequency the experts is estimating should follow
the \(\s{beta}(\alpha + s + 1 ,\beta + n - s)\) distribution. So far,
nothing controversial happens---the expert is estimating the relevant
population frequency. Assuming the conditions are pristine (the expert
has no modeling uncertainty, rules out laboratory errors, and so on),
the beta distribution can be used to inform the expert's subjective
uncertainty. But uncertainty about what? The (estimated) population
frequency can serve to attach a probability to the proposition
\emph{a match is observed  if another person, unrelated to the suspect, is the source of the trace}.
Admittedly, if only this proposition is being considered, it is yet not
clear what second-order uncertainties would be uncertainties about. But
the expert also considers a continuum of propositions, each of the form
\emph{the true population frequency is $\theta$} for each
\(\theta\in [0,1]\). A density over \(\theta\) models the comparative
plausibility that the expert assigns to such propositions in light of
the evidence. So if one were worried that there were no propositions
that the expert could be ``second-order'' uncertain about, there
actually are plenty. In particular, if \(\theta\) is a population
frequency, gauging which density captures the extent to which the
evidence justifies various estimates of that frequency is the same as
gauging the comparative plausibility of the corresponding propositions
about the population frequencies.\footnote{Perhaps, this should no
  longer be called ``estimation'', but the the connection with
  estimation is strong enough to justify this terminology. In the end,
  this is a verbal discussion that we will not get into.}

More generally, evidence justifies first-order probability assignments
to various degrees. Suppose there is no evidence about the bias of a
coin. Then, each first-order uncertainty about it would be equally
(un)-justified. (If you like to think in terms of bets, the evidence
would give no reason to prefer any particular odds as fair.) If,
instead, we know the coin is fair, the evidence clearly selects one
preferred value, .5. (Again, if you like the betting metaphor, 1:1 would
be the preferred betting odds.) But often the evidence is stronger than
the former case and weaker than the latter case. Consider, for example,
propositions about population frequencies in light of the results of
observations. In such circumstances, the evidence justifies different
values of first-order uncertainty to various degrees, and densities
simply capture the extent to which different first-order uncertainties
are supported by the evidence.

We conclude this section by examining two additional points raised by
Taroni et al. (2015). The first---which we already alluded to
earlier---is that first-order probabilities are not ``states of nature''
and so cannot be estimated. It is unclear why only states of nature can
be estimated. Mathematicians use approximate methods to estimate answers
to fairly abstract questions ,not obviously related to ``states of
nature'', whatever these are. So, estimation should make sense whenever
there are some objective answers that we can approximate to a greater or
lesser extent.\footnote{Taroni et al. (2015) makes the same point for
  likelihood ratios. They argue that there is no ``meaningful state of
  nature equivalent for the likelihood ratio in its entirety, as it is
  given by a ratio of two conditional probabilities?'' But if it is
  meaningful to estimate two conditional probabilities (that is,
  frequencies in the population), or to compare the relative
  plausibility of various propositions about them in terms of density,
  it is equally meaningful to estimate any function of the numbers
  involved. Otherwise it would also be meaningless to try to estimate
  the body mass index (BMI) of an average 21 years old male student in
  the USA just because BMI is a ratio of other quantities.}

Second, Taroni et al. (2015) argue that once we allow second-order
probability, we run into the threat of infinite regress. But do we?
Surely, they would agree that one can be uncertain about a statistical
model. But this can be the case even if this model spits out a point
estimate rather than a density. If you think the possibility of putting
uncertainty on top of propositions about possible values of a
first-order parameter leaves us in an epistemically hopeless situation,
you might have hard time explaining why your point estimation is in a
better situation. After all, if asking further questions about
probabilities up the hierarchy is always justified, we can keep asking
about the probability of a point-estimate-spitting model being adequate,
the probability of that probability , and so on. Perhaps the problem at
issue is just one of complexity. Admittedly, second-order estimation is
more complex than relying on point estimates. But we hope to have
convinced the reader this complexity is worth the effort. What about
more complex models going third-higher? If a workable approach can
accomplish that---and the additional complexity pays off---we are all
for going third-order. So, the fact that more complex models can always
be built hardly lead us into a vicious infinite regress. Rather, it is
an indication that our models of uncertainty can---in principle---always
be improved.

\hypertarget{legal-applications}{%
\section{Legal Applications}\label{legal-applications}}

Our discussion so far has been mostly theoretical. We made a case that
higher-order probabilism outperforms precise probabilism on both
descriptive and normative grounds. We also staved off a number of
conceptual difficulties with going higher-order. It is time to
illustrate how higher-order probabilism can be of service in evaluating
evidence at trial. We present here two examples.

\hypertarget{false-positives-in-dna-identification}{%
\subsection{False Positives in DNA
Identification}\label{false-positives-in-dna-identification}}

Let us return to a topic we discussed in Chapter 5: how the probability
of a false positive impacts the value of DNA match evidence. As
Thompson, Taroni, \& Aitken (2003) have shown, the probability of false
positives, even when it is seemingly low, has a non-negligible impact:

\begin{quote}
If, as commentators have suggested, the rate of false positives is between 1 in 100 and 1 in 1000, or even less, then one might argue that the jury can safely rule out
the prospect that the reported match in their case is due to error and can proceed to consider the probability of a coincidental match \dots this argument is fallacious and profoundly misleading \dots  the probability that a reported match occurred due to error in a particular case can be much higher, or lower, than the false positive probability. 
\end{quote}

\noindent We are particularly interested in the passing remark that the
rate of false positives is between 1 in 100 and 1 in 1000. This
difference is not negligible. The simplest option would be to use the
upper bound of the {[}0.001, 0.01{]} interval. This choice would be the
most favorable toward the defendant. But, as already noted in the
introduction, doing so will lead to an evaluation of the match evidence
that is too conservative. It is much preferable to have a sensible
distribution to work with.

To fix ideas, let the posterior probability of the source hypothesis
(\(S\)) conditional on the match evidence (\(E\)) be defined, as
follows:

\begin{align*}
\pr{S \vert E} &   =  \frac{\pr{E\vert S} \pr{S} } {\pr{E}}\\
& = \frac{\overbrace{\pr{E\vert S}}^1 \pr{S}}{\underbrace{\pr{E\vert S}}_1 \pr{S} + \underbrace{\pr{E \vert RM}}_1 \pr{RM} + \underbrace{\pr{E \vert FP}}_1 \pr{FP}} \\ & = \frac{\pr{S}}{\pr{S} + \pr{RM} + \pr{FP}} 
\end{align*}

\noindent For simplicity, the false negative rate is assumed to be zero,
or in other words, \(\pr{E\vert S} =1\). The other assumption is that
the evidence could come about if: (1) the source hypothesis is true; (2)
a random match (\(RM\)) occurred; or (3) a false positive match occurred
(\(FP\)).

Suppose the random match probability for the DNA match evidence is
rather low, say \(10^{-9}\) (as in some of the examples we discussed in
Chapter 5), and there is no uncertainty associated with this number.
Consider now two ways of assessing the DNA match. First, disregarding
the possibility of a false positive---so \(FP=0\)---makes the match
evidence appear extremely strong. In this case, the minimal prior
sufficient for the posterior to be above .99 is only 0.001, where the
relation between the prior probabilities and the posterior probabilities
of the source hypthesis is given by the dashed orange line in Figure
\ref{fig:fplinesPlot}. What happens after taking into account the
possibility of a false positive match? This depends on how this
possibility of error is quantified. Assume the false positive rate
corresponds to the upper bound of the {[}0.001, 0.01{]} interval. This
assumption completely changes the assessment of the match evidence. For
the posterior of \(.99\) is reached only once the prior is above .99.
The match evidence now appears to be extremely weak. As already seen in
the introduction, the point estimate exaggerates the value of the match
evidence, while using the upper bound of the false positive rate has the
opposite effect. What matters within these edges of the {[}0.001,
0.01{]} interval cannot be ignored.

To take into consideration the values within the edges, two approaches
can be pursued. On one approach, any value between the edges is regarded
as equally likely. On another approach, not all values are equally
likely. For example, suppose you think it is 50\% likely that the false
positive rate is below .0033. In addition, suppose the distribution,
while being centered closer to zero, is long-tailed (known as the
truncated normal distribution). Thee two distributions are illustrated
in Figure \ref{fig:fppdistros}.\footnote{On both approaches, we assume
  that the false positive rate is between 0.001 and 0.01 with 99\%
  certainty.} The uniform distribution---which regards all false
positive rates in the interval as equally likely---leads to a rather
conservative evaluation of the match evidence, much more so than the
truncated normal distribution. This is apparent from Figure
\ref{fig:fppMinima}) and Figure \ref{fig:fplinesPlot} which show the
prior probabilities of the source of hypothesis needed to secure a
posterior probability above. Working with a distribution---better if it
is not a uniform distribution---affords a more balanced assessment of
the evidence than simply relying on the edges of an interval.

We have suggested that the choice of a distribution over possible values
of false positive rates can make a difference for the assessment of DNA
evidence. The lingering question, however, is how these distributions
can be obtained. Admittedly, studies on false positives are limited and
only give a rough and foggy picture. More experiments and studies are
needed. This does not mean, however, that until then using point
estimates and interval edges is preferable. After deciding on the
functional form of a distribution---such as truncated normal or
beta---only a few numbers need to be elicited from experts for
constructing a density.\footnote{For instance, assuming the distribution
  is a truncated normal, it is enough for the expert to assert that they
  believes with more than 50\% confidence the false positive rates to be
  below \(.033\) for the curve to be determined.} Having to rely on such
elicitation is not without problems, but it is better than asking
experts for single point estimates and relying on these (O'Hagan et al.,
2006).

\begin{figure}[H]



\begin{center}\includegraphics[width=0.8\linewidth]{chapter-outline_files/figure-latex/fig:fppdistros-1} \end{center}


\caption{Two examples of assumptions about the false positive rates, both having pretty much the same 99\% highest density intervals. Left: all error rates are equally likely. Right: the most likely values are closer to 0, but also some high values while unlikely are possible.}

\label{fig:fppdistros}

\end{figure}

\begin{figure}[H]



\begin{center}\includegraphics[width=0.8\linewidth]{chapter-outline_files/figure-latex/fig:fppMinima-1} \end{center}


\caption{The distribution of minimal priors sufficient for obtaining a posterior above .99 on the two distributions of false positive rates. The truncated normal distribution has its bulk towards the left, but at the same time has higher ratio of evens in which this posterior is never reached. }

\label{fig:fppMinima}

\end{figure}

\begin{figure}[H]

\begin{center}\includegraphics[width=0.8\linewidth]{chapter-outline_files/figure-latex/fig:fplinesPlot2-1} \end{center}
\caption{Impact of prior on the posterior assumign two different densitites for false positive rates. Note how both the "pristine" error-free point estimate (orange) and the charitable version (blue) are quite far from where the bulks of the distributions in fact are. Note also how the trnormal density allows for even more charitable cases, which results from it being long-tailed.}
\label{fig:fplinesPlot}
\end{figure}

\hypertarget{higher-order-bayesian-networks}{%
\subsection{Higher-order Bayesian
Networks}\label{higher-order-bayesian-networks}}

The higher-order framework we are advocating is not only applicable to
the evaluation of individual pieces of evidence. Complex bodies of
evidence---for example, those represented by Bayesian networks---can
also be assessed using higher-order probabilities. We use here as an
illustration a simplified Bayesian network developed by Fenton \& Neil
(2018). The network is reproduced in Figure \ref{fig:scBNplot} and
represents the evidence in the infamous British case R. v. Clark (EWCA
Crim 54, 2000).\footnote{ Sally Clark's first son died in 1996 soon
  after birth, and her second son died in similar circumstances a few
  years later in 1998. At trial, the paediatrician Roy Meadow testified
  that the probability that a child from such a family would die of
  Sudden Infant Death Syndrome (SIDS) was 1 in 8,543. Meadow calculated
  that therefore the probability of both children dying of SIDS was
  approximately 1 in 73 million. Sally Clark was convicted of murdering
  her infant sons. The conviction was reversed on appeal. The case of
  appeal was based on new evidence: signs of a potentially lethal
  disease were found in one of the bodies.} We refer to the reader to
Chapter \textbf{XYZ} for a more general discussion of Bayesian networks.
Here it suffices to note that the arrows depict relationships of
influence between variables. \textsf{Amurder} and \textsf{Bmurder} are
binary nodes corresponding to whether Sally Clark's sons, call them A
and B, were murdered. These nodes influence whether signs of disease
(\textsf{Adisease} and \textsf{Bdisease}) and bruising
(\textsf{Abruising} and \textsf{Bbruising}) were present. Also, since
A's death preceded in time B's death, whether A was murdered casts some
light on the probability that B was also murdered.

The choice of the probabilities in the network is quite specific, and it
is not clear where such precise values come from. The standard response
invokes \emph{sensitivity analysis}: a range of plausible values is
tested. As already discussed, this approach ignores the shape of the
underlying distributions. Sensitivity analysis does not make any
different between probability measures (or point estimates) in terms of
their plausibility, but some will be more plausible than others.
Moreover, if the sensitivity analysis is guided by extreme values, these
might play an undeservedly strong role. These concerns can be addressed,
at least in part, by adding higher order probabilities. In a precise
Bayesian network, each node is associated with a probability table
determined by a finite list of numbers (precise probabilities). But
suppose that, instead of precise numbers, we have densities over
parameter values for the numbers in the probability tables.\footnote{The
  densities of interests can then be approximated by (1) sampling
  parameter values from the specified distributions, (2) plugging them
  into the construction of the BN, and (3) evaluating the probability of
  interest in that precise BN. The list of the probabilities thus
  obtained will approximate the density of interest. In what follows we
  will work with sample sizes of 10k.} An example of a higher-order
Bayesian network for the Sally Clark case is given in Figure
\ref{fig:SCwithHOP}.

\begin{figure}[H]

\begin{center}\includegraphics[width=0.5\linewidth]{chapter-outline_files/figure-latex/scBNplot2-1} \end{center}
\caption{Bayesian network for the Sally Clark case, with marginal prior probabilities.}
\label{fig:scBNplot}
\end{figure}

\begin{figure}[H]

\begin{center}\includegraphics[width=1.1\linewidth,height=2\textheight,angle=90]{chapter-outline_files/figure-latex/SCwithHOP-1} \end{center}

\caption{Example of a higher-order Bayesian network for the Sally Clark Case.}
\label{fig:SCwithHOP}
\end{figure}

With the help of the higher-order Bayesian network, we can investigate
the impact of different items of evidence on the Sally Clark's
probability of guilt (Figure \ref{fig:SCwithHOP2}). The starting point
is the prior density for the \s{Guilt} node (first graph). Next, the
network is updated with evidence showing signs of bruising on both
children (second graph). Next, the assumption that both children lacked
signs of potentially lethal disease is added (third graph). Finally, we
consider the state of the evidence at the time of the appellate case:
signs of bruising existed on both children, but signs of lethal disease
were discovered only on the first child. Interestingly, in the strongest
scenario against Sally Clark (third visualization), the median of the
posterior distribution is above .95, but the uncertainty around that
median is still too wide to warrant a conviction.\footnote{The lower
  limit of the 89\% Highest Posterior Density Intervals (HPDI) is at
  .83.} This underscores the fact that relying on point estimates can
lead to overconfidence. Paying attention to the higher-order uncertainty
about the first-order probability can make a difference to trial
decisions.

\begin{figure}[H]

\begin{center}\includegraphics[width=0.9\linewidth]{chapter-outline_files/figure-latex/SCwithHOP2-1} \end{center}


\caption{Impact of incoming evidence in the Sally Clark case.}
\label{fig:SCwithHOP2}
\end{figure}

\hypertarget{weight-of-evidence}{%
\section{Weight of evidence}\label{weight-of-evidence}}

We now illustrate one of the theoretical payoffs of higher-order
probabilism. It allows to develop an elegant theory of the weight of
evidence that outperforms the existing proposals. We will start with an
informal sketch of the idea of the weight of evidence, as opposed to the
balance of the evidence. We will then explore a few attempts at modeling
this idea, first based on precise probabilism and then based on
imprecise probabilism. Finally, we will show how higher-order
probabilism can offer a better theory.

\hypertarget{examples-and-desiderata}{%
\subsection{Examples and Desiderata}\label{examples-and-desiderata}}

In the 1872 manuscript \emph{The Fixation of Belief} (W3 295), C. S.
Peirce makes the following observation about sampling from a bag of
beans that are either black or white:

\begin{quote} When we have drawn a thousand times, if about half have been white, we have great confidence in this result ... a confidence which would be entirely wanting if, instead of sampling the bag by 1000 drawings, we had done so by only two.
\end{quote}

\noindent In both cases, our best assessment of the probability that the
next draw will be a black bean is .5, but how sure we should be of that
assessment is quite different depending on whether it is based on two or
one thousands draws. In other words, the \emph{weight} of the evidence
seems much greater after drawing a thousand beans and finding out that
half are black, compared to drawing just two beans and again finding out
that half are black. The weight of the evidence is different in the two
cases, but its \emph{balance}---understood here as the empirical
proportion of black-to-white beans---is the same.\footnote{Similar
  remarks can be found in Peirce's 1878 \emph{Probability of Induction}.
  There, he also proposes to represent uncertainty by at least two
  numbers, the first depending on the inferred probability, and the
  second measuring the amount of knowledge obtained. For the latter,
  Peirce proposes to use some dispersion-related measure of error (but
  then suggests that an error of that estimate should also be estimated
  and so on, so that ideally more numbers representing errors would be
  needed).}

Peirce did not use the expression the weight of evidence (and, in fact,
he used the phrase to refer to the balance of evidence, W3 294) (Kasser,
2016). However, his remarks anticipated what came to be called weight of
evidence by Keynes in his 1921 \emph{A Treatise on Probability}:

\begin{quote}
As the relevant evidence at our disposal increases, the magnitude of the
probability of the argument may either increase or decrease, according as the new knowledge strengthens the unfavourable or the favourable evidence; but something seems to have increased in either case---we have a more substantial basis upon which to rest our conclusion. I express this by saying that an accession of new evidence increases the weight of an argument. (p. 71)
\end{quote}

\noindent The key point is the same (Levi, 2011): since the balance of
probability alone cannot characterize all important aspects of
evidential appraisal, another dimension---the weight of an
argument---must be deployed to quantify uncertainty.\footnote{Keynes
  entertained the possibility of measuring weight of evidence in terms
  of the variance of the posterior distribution of a certain parameter,
  but was quite attached to the idea that weight should increase with
  new information, even if the dispersion may increase with new evidence
  {[}TP 80-82{]}. So he proposed only a very rough sketch of a positive
  proposal. Moreover, he was unclear how a measure of weight could be
  part of decision-making. He was ultimately skeptical about the
  practical significance of the notion {[}TP 83{]}.}

It is instructive to examine more closely Keynes' claim that the weight
of evidence, unlike its balance, always increases as more evidence is
taken into account. While balance can oscillate one direction or the
other, weight would seem to always increase. We can state this
requirement as follows:

\vspace{1mm}
\begin{tabular}{lp{9cm}}
(Monotonicity) & If $E$ is relevant to $X$ given $K$, where $K$ is background knowledge, $V(X\vert K \wedge E) > V(X\vert K)$, where $V$ is the weight of evidence.
\end{tabular}
\vspace{1mm}

\noindent Monotonicity is consistent with Peirce's example about drawing
from a bag of beans. As the sample size increases and the relative
proportion of black-to-white beans remains constant, the weight of the
evidence increases. That is:

\vspace{1mm}
\begin{tabular}{lp{10cm}}
(Weak increase) & In urn-like cases, the evidential weight obtained by a larger sample is greater, if the relative frequencies in the samples remain the same.
\end{tabular}
\vspace{1mm}

\noindent This formulation can be strengthened by dropping the
assumption of equal relative frequencies:

\vspace{1mm}
\begin{tabular}{lp{11cm}}
(Strong increase) & In urn-like cases, the evidential weight obtained by a larger sample is higher.
\end{tabular}

\noindent We think there are good reasons to reject Monotonicity and
Strong Increase, but we agree with Weak Increase. Monotonicity and
Strong Increase are consistent with a certain conception of the weight
of evidence, what we might call \emph{quantity} of evidence. There is no
doubt that, as more evidence is taken into account, the quantity of the
evidence must increase. But the weight of evidence need not be
identified with its quantity alone. As more evidence is taken into
account, the new evidence may speak less clearly in favor or against a
hypothesis. For consider this example:

\begin{quote}
\textbf{A (possibly) rigged lottery:} Initially, you think the lottery  is  fair. You have no reason to doubt that. So, you calculate precisely the probability that a certain ticket number will be drawn. Then, rumors begin to surface that the lottery is rigged and that only numbers that satisfy a complicated equation will be drawn. You now have more relevant evidence at your disposal, but that evidence is more confusing and muddled than before. 
\end{quote}

\noindent Arguably, this is a scenario in which the quantity of evidence
has increased, but the weight of evidence has not. If this is right,
weight and quantity can come apart.

So, we seek a theory of evidential weight that can model two intuitions.
The first is that, as the sample size increases, the weight of the
evidence must also increase (under certain conditions, along the lines
of Weak Increase). The second intuition is that that, even when the
quantity of evidence increases, the weight of evidence might not (as
illustrated by the example of the rigged lottery).

\hypertarget{weight-and-precise-probabilism}{%
\subsection{Weight and precise
probabilism}\label{weight-and-precise-probabilism}}

An obvious place to look for a theory of the weight of evidence is
within precise probabilism. But, we will argue, our best bet is to look
beyond it.

\hypertarget{absolute-value-of-the-likelihood-ratio}{%
\subsubsection{Absolute value of the likelihood
ratio}\label{absolute-value-of-the-likelihood-ratio}}

Weight of evidence can be modeled by generalizing the likelihood ratio
as a measure of the value of evidence. The likelihood ratio is a
\emph{directional} measure: the evidence may be favorable or unfavorable
to a hypothesis (compared to another). By contrast, Keynes' remarks
suggests that weight is a non-directional measure. It appears to always
increase no matter the balance (though we will ultimately reject this
claim). To strip the likelihood ratio of its directionality, it is
enough to take the absolute value of the natural log (Nance, 2016, sec.
3.5). The weight of evidence \(E\) relative to the pair of hypotheses
\(H, H'\) would be \[\vert \ln (LR_{H, H'}(E)) \vert, \] where
\(LR_{H, H'}(E)=\frac{P(E \vert H)}{P(E \vert H')}\). By the properties
of logarithms, \(ln(1/x) = -ln(x)\), so two items of evidence of
equivalent strength---but opposite directionality---would have the same
weight. So, for example, \(|ln(1/3)|= |ln(3)| = 1.61\).

This account is simple and elegant, but faces two difficulties. The
first is technical in nature. As many have pointed out, there is a
problem with decomposition. Consider two items of evidence that, taken
together, have a likelihood ratio of one, say one has a likelihood ratio
of 1/3 and the other of 3. Assuming they are probabilistically
independent given a hypothesis of interest, their combined likelihood
ratio results from multiplying the individual likelihood ratios. Thus,
their combined weight would be zero since
\(\ln (LR_H(E1 \wedge E_2))=\ln (1) = 0\). However, by adding the
weights one by one, the combined weight would be different from zero,
since \(|ln(1/3)| + |ln(3)| = 3.22\). So, do the two items of evidence
have zero weight or not? Depending on how evidence is decomposed, it
appears to have different weights.

There might be a technical fix to the problem of decomposition. But
another, conceptually deeper problem exists. Whether the log of the
likelihood has a positive or negative sign, its absolute value is always
positive. Therefore, as more evidence is added, the weight of evidence
would always increase. This shows that this account agrees with
Monotonicity and Strong Increase, and in this sense, it only captures
the notion of quantity of evidence. The problem, then, is that it cannot
capture the intuition underlying the scenario of the rigged lottery in
which, even when quantity increases, weight does not.

\hypertarget{completeness-and-resilience}{%
\subsubsection{Completeness and
resilience}\label{completeness-and-resilience}}

Another strategy, within precise probabilism, is to model notions that
are related, albeit not identical, to weight. The literature has
suggested that precise probabilism contains the conceptual resources to
model two related ideas: resilience (or stability) and completeness.
Roughly speaking, resilience tracks how additional evidence may change
the current probability assessment, while completeness tracks how the
fact that some evidence is missing may change the current probability
assessment.

Skyrms (1977) offers the following account of resilience: a precise
probability assessment of \(H\) based on some evidence \(E\), say
\(P(H \vert E)\), is resilient (or stable) whenever it does not wiggle
too much relative to a reasonable set of additional items of evidence
\(E'\) that can be conditioned on. In other words, \(P(H \vert E)\) is
resilient enough if and only if the absolute difference
\(| P(H | E) - P(H \vert E \wedge E')|\) (for any element \(E'\) in a
reasonable set of additional evidence) is within a certain acceptable
limit. For an account of completeness, Kaye's sketches the following
proposal\todo{CITE KAYE, which one?}: instead of merely assessing
\(P(H \vert E)\), one should assess \(P(H \vert E \wedge M)\), where
\(M\) is the known fact that some evidence is missing. Whenever the
evidence available is missing information that one would reasonably
expect to see in a case (in a sense to be specified), this fact must be
taken into account in assessing the probability of the hypothesis of
interest.\footnote{Interestingly, the missing evidence is not a
  higher-order fact about the currently available evidence, but itself a
  fact that counts as evidence together with other facts. This, we think
  is a weakness in the theory, because much of the heavy lifting has to
  be done by assessing the conditional probabilities involving missing
  evidence conditions and these do not fall out of the theory, but have
  to be plugged in by hand on a case-by-case basis.}

A good account of resilience and completeness should address a number of
questions. For example, in assessing resilience, which items of
additional evidence should be included in the reasonable set?\footnote{The
  set must be restricted somehow, or else no probability assessment
  would ever be resilient. Another problem is that resilience consists
  in evaluating a probability based on evidence not yet available. The
  evidence could take different values---for example, in a criminal
  case, the additional evidence could be exculpatory or incriminating.
  How can additional evidence whose value is unknown affect one's
  assessment of the probability of a hypothesis? Another question is,
  which items of evidence should be included is the reasonable set?} Or,
in assessing completeness, when should an item of evidence count as
missing?\footnote{Admittedly, there are cases in which it is clear that
  evidence is missing, say because it was destroyed, lost or could not
  be retrieved or collected for one reason or another. But evidence
  could be missing in a more speculative sense. For example, if it is
  routine to see breathalyzer evidence in drunk driving cases, what
  should one make of the fact that such evidence is missing in a given
  case? The more general worry here is that evidence is always
  incomplete. Peirce's example of drawing from a bag of beans makes this
  clear. In principle, more draws could be made and more evidence could
  be gathered. Another difficulty concerns the impact that the missing
  evidence should have on the other evidence. For example, in a case of
  tax fraud, suppose the recording of phone conversation was deleted by
  accident. There is no dispute about that. Now, should this fact
  increase or decrease the probability that the defendant committed
  fraud?} But, it is clear that resilience and completeness both play a
role in trial proceedings. On appeal, a defendant might argue that, had
additional evidence be taken into account, the verdict should have been
different. Questions of missing evidence are also routinely brought up
in litigation. And the rules of evidence often deal with questions of
missing evidence and offer a variety of remedies, from jury instructions
to sanctions.

So, a good theory of how evidence is assessed at trial should include an
account of completeness and resilience. A question remains, however. Are
resilience and completeness the theory of weight we are looking for?
This is far from clear. We think that weight, completeness and
resilience are conceptually distinct notions. And, they are likely to
play distinct roles in legal fact-finding. Or perhaps completeness and
resilience, together with precise first-order probabilities, are all we
need to assess evidence at trial, and the notion of weight is redundant.
These are difficult questions that require a close examination of each
notion. We focus here on weight and leave the discussion of how weight
compares to completeness and resilience to another time.

\hypertarget{weight-and-imprecise-probabilism}{%
\subsection{Weight and imprecise
probabilism}\label{weight-and-imprecise-probabilism}}

A theory of weight can also be formulated using the conceptual resources
of imprecise probabilism.
\todo{CITE WEATHERSON ON WEIGHT AND IMPRECISE PROBABILISM} The idea is
this: a body of evidence is weightier whenever the representor set of
probability measures compatible with the evidence is in some sensible
sense smaller. Consider a case of complete ignorance about the bias of a
coin. The representor set will contain \emph{any} probability measure.
This corresponds to complete lack of evidence and null weight, as
expected. At the other extreme, consider a case in which the fairness of
the coin is known for sure. The supporting evidence here would have
maximal weight and the representor set would only contain the precise
probability measure that assigns .5 to the two outcomes. All other
intermediate cases would fall somewhere in between.\footnote{More
  formally, let \(x_1\) and \(x_2\) the two extreme probability
  assignments in the representator set compatible with the available
  evidence \(E\). Then, the weight of \(E\) would be 1-\(x_1-x_2\). As
  expected, if there is only one probability measure, the weight would
  be one.}

This account accommodates the intuition underlying the rigged lottery
example. As more evidenced is accumulated, the representor set can
become larger and include more probability measures than before. Thus,
weight of evidence can decrease even when the quantity of evidence
increases. For this reason, a theory of weight based on imprecise
probabilism is promising. The problem, however, is that this theory
inherits the difficulties of imprecise probabilism, which we have
already mentioned. For example, recall that imprecise probabilism cannot
model a situation in which an epistemic agent believes that the coin
could have bias .8 or .4, but thinks that one bias is more likely than
the other. Arguably, evidence compatible with the coin having two
equally likely biases should possess different weight than evidence
compatible with the coin having two biases one more likely than the
other. The latter situation is closer to full weight---the situation in
which the exact bias of the coin is known for sure. But a theory of
weight based on imprecise probabilism is ill-equipped to accommodate
these nuances.

\hypertarget{weight-and-higher-order-probabilism}{%
\subsection{Weight and higher order
probabilism}\label{weight-and-higher-order-probabilism}}

We are finally ready to present our own account of the weight of
evidence. This accout has two distinctive features. First, it is based
on higher-order probabilism. Second, it is information-theoretic. To
develop this account, we will begin with a very short introduction to
Shannon's theory of information.

\hypertarget{entropy-of-a-distribution}{%
\subsubsection{Entropy of a
distribution}\label{entropy-of-a-distribution}}

\todo{I think you crippled the explanation too much, as now there is no connection to expected surprise, more needs to be said here. If not today, at some later point.}

Let \(X\) be a random variable and \(P\) a probability distribution over
its values. Shannon's measure of information, \(H(X)\), reads:
\begin{align*}
H(X)  & =
- \sum \mathsf{P}(x_i) \log_2 \mathsf{P}(x_i)
\end{align*} \noindent  Consider the simple case in which \(X\) can take
two values---outcome 1 and outcome 0---whose probabilities are \(p\) and
\(1-p\). Figure (\textbf{REFER TO PLOT of H(X)})\todo{WHAT PLOT?} shows
\(H(X)\) as a function of \(p\). As is apparent from the graph, \(H(X)\)
is greatest when the two outcomes have an equal probability of .5. The
more the probabilities deviate from .5, the more \(H(X)\) approaches
zero.

To make sense of this, \(H(X)\) can be thought as the \textit{entropy}
(i.e.~the lack of information) contained in the distribution associated
with random variable \(X\). When the two outcomes have equal
probability, entropy is greatest. When they have different
probabilities, one outcome will be more probable than the other. The
more probable one of the outcomes (and thus the less likely the other),
the lower the entropy. Intuitively, \(H(X)\) captures the idea that
entropy is greatest when the indecision about which outcome will occur
is maximal, and the entropy decreases when such indecision decreases.

If \(H(X)\) is a measure of entropy---that is, a measure of lack of
information---why call it a measure of information? \(H(X)\) is also a
measure of information in the following sense: it describes the
\textit{expected amount of information} one would receive upon learning
the actual value of \(X\). After all, the higher the entropy, the less
informative a distribution, the more you expect to learn upon finding
out the actual value of \(X\). Conversely, the lower the entropy of the
distribution, the more informative the distribution, the less one
expects to learn upon finding out the actual value of \(X\).\footnote{Since
  working with continuous distributions is not straightforward, we will
  be using \emph{grid approximations} of continuous distributions: we
  will split \(X\) into a 1000 bins and use the normalized densities for
  their centers to obtain their corresponding probabilities. As long as
  we do not change our level of precision (which would inevitably lead
  to changes in entropy) in our comparisons, this is not a problem. An
  additional advantage is that now we do not have to deal with the
  intricacies of explicit analytic calculations for continuous
  variables.}

\hypertarget{informativeness-of-a-distribution}{%
\subsubsection{Informativeness of a
distribution}\label{informativeness-of-a-distribution}}

Since \(H(X)\) can be thought as the entropy of a distribution, we will
switch to the notation \(H(P)\). This notation emphasizes the
distribution \(P\) rather than the random variable \(X\). The entropy of
a distribution is to be contrasted with its informativeness, which we
will denote by \(W(P)\), the weight of the distribution. How should we
measure the informativeness (or weight) of a distribution?

Since informativeness is the opposite of entropy, it is tempting to take
the short route and define \(W(P)\) as \(1-H(P)\). But, for reasons that
will be become clear (\textbf{WHEN?}), the weight (or informativeness)
of a distribution is more aptly modeled by comparing it to the least
informative distribution, the uniform distribution which expresses
complete uncertainty. The idea is this: the more informative a
distribution, the more it departs from the uniform distribution, the
more weight it has, on a scale from 0 to 1. If the drop from uncertainty
is complete, the entropy drops to zero, and thus the weight should be 1;
if the drop is null, the entropy remains the same, and thus the weight
should be zero; if the drop is half, the weight should be .5; and so on
for other intermediate cases. This pattern can be captured by the
following definition of informativeness (or weight) of a distribution:
\begin{align*}
\mathsf{w(P_i)} & = 1 - \left( \frac{H(\mathsf{P})}{H(\mathsf{uniform})}\right)
\end{align*} \noindent where \(\mathsf{P}\) is the probability
distribution of interest and \(\mathsf{uniform}\) is the baseline
uniform distribution.\footnote{Since we are using grip approximation,
  \(\mathsf{P}\) is the discrete probability distribution for a given
  number of bins \(n\), and uniform is the discrete uniform distribution
  for the same number of bins. In some contexts it might make sense to
  measure improvement with respect to a non-uniform prior. In such
  cases, \(H(\mathsf{uniform})\) is to be replaced by
  \(H(\mathsf{prior})\). Note that the entropy of a uniform distribution
  is pretty straightforward, so we can simplify: \begin{align*}
  H(\mathsf{uniform}) & = \sum_{i=1}^n \nicefrac{1}{n} \log_2 \frac{1}{\nicefrac{1}{n}} \\
  & = \log_2(n) \\
  \mathsf{w(P_i)} & = 1 - \left( \frac{H(\mathsf{P})}{\log_2(n)}\right)
  \end{align*}}

The behavior of the proposed measure of weight can be illustrated using
distributions of various of shapes, displayed in Figure
\ref{fig:weightsWeird}). A bimodal normal distribution ``glued'' from
two normal distributions carries less weight than a unimodal normal
distribution with the same standard deviation centered around the mean
of the two modes. If multiple points have non-zero probability, the
weight depends on how uneven the distribution is. If the distribution
falls entirely on a single point, the weight is maximal (=1), as
expected.

\begin{figure}[H]

\begin{center}\includegraphics[width=1\linewidth]{chapter-outline_files/figure-latex/fig:weightsWeird-1} \end{center}
\caption{Examples of various distributions with their entropies and weights, ordered by weights. (1) beta(4,4), (2) uniform starting from .5 to 1, (3), uniform strating from .6 to 1, (4) two normal distributions centered around .4 and .6 with standard deviation .05, glued at .5. (5) normal centered around .5 with the same standard deviation, (6) one that assigns .5 to each of .4  and .6, (7) One that assigns .3 to .4 and .7 to .6., (8) one that assigns all weight to .5, and (9) one that assigns all weight to .7.}

\label{fig:weightsWeird}
\end{figure}

\hypertarget{weight-of-evidence-1}{%
\subsubsection{Weight of evidence}\label{weight-of-evidence-1}}

Suppose a distribution \(P\) depicts what an agent thinks, at some point
in time, about the probability of the possible outcomes a random
variable \(H\). In this sense, the informativeness (or weight) of a
distribution \(W(P)\) measures how informed an agent is about \(H\). But
it measures the information level of an agent only relative to the state
of full uncertainty represented by the uniform distribution. There are
two things missing from this account: first, not every agent starts with
a uniform prior; second, how informed an agent is must depend on the
evidence available to that agent. What we need, then, is an account of
how informed agents are \textit{on the basis of the evidence} they have,
or in other words, an account of the weight (or informativeness) of the
evidence they have.

If the agent starts with a uniform prior over the values of a random
variable of interest, \(W(P)\) would be a good enough approximation of
how informed the evidence made them. In general, however, how much more
information is obtained is context-dependent. The weight of evidence,
then, must depend on what the agent already knows. Here is a general
recipe. In a given context, consider the prior distribution \(P_0\) for
the target hypothesis \(H\) given what the agent already knows. Then,
the agent updates by a body of evidence \(E\). Call this posterior
distribution \(P_E\), where the updating is done by standard Bayesian
conditionalization. Take the difference between the weight of the prior
distribution, \(W(P_0)\), and the weight of the posterior distribution,
\(W(P_E)\). The difference between the two--\(W(P_E)-W(P_0)\) or more
succinctly \(\Delta W\)---masures the impact that evidence \(E\) has on
the information level of the evidence. The difference \(\Delta W\),
then, is our proposed measure of the weight of the evidence.\footnote{If
  you prefer to think that weight of evidence should be always positive
  as a result of adding evidence, you might prefer the absolute value of
  the difference. We, however, prefer to keep track of whether the
  evidence makes the agent more or less informed about an issue.}

More precisely, the calculation follows the following schema:

\begin{enumerate}
\item Start with a prior distribution over the parameter space of interest and with distributions expressing the agent's uncertainty about other probabilities involved in the calculation of the posterior, say the likelihood of the evidence under different possible outcomes.
\item Sample from these distributions.
\item For each sample, treat it as a selection of precise probabilities, apply Bayes' theorem to calculate the posterior.
\item The set of the results is the sampling distribution expressing your posterior uncertainty.
\end{enumerate}

Higher order probabilism is then put to use to deliver a theory of
weight. What is now in \textbf{section 11} (``Weight of a
distribution'') and \textbf{sections 13} and \textbf{14} ('' Weight of
evidence'' and ``Weights in Bayesian Networks'') forms the bulk of the
theory.

We should also demonstrate that the proposed theory of weight does meet
the intuitive desiderata and can handle the motivating examples. To
better appreciset the novelty of the proposal, It might be interesting
to raise the following questions:

\begin{itemize}

\item[q1] what does a theory of weight based on precise probabilism look like? (maybe it consists of something like Skyrms' resilience or Kaye's completeness, the problem being that these are not measures of weight, but of something else, more on these later)

\item[q2] what does a theory of weight based on imprecise probabilism look like? (is Joyce's theory essentially an attempt to use imprecise probabilism to construct a theory of weight? )\todo{Well, it's a bit funny as Joyce's weight uses precise chance hypotheses instead of IP, so hard to say}

\item[q3] what does a theory of weight based on higher order probabilism look like?

\end{itemize}

Here we are defending a theory fo weight based on higher order
probabilism, but it is interesting to contrast it with a theory of
weight based on the other version of legal probabilism. Here we can also
show why Joyce's theory of weight does not work (either in the main text
or a footnote).

\textbf{Comment:} The current exposition in chapter 11, 13 and 14,
however, is complicated---perhaps overly so. The move from ``weight of a
distribution'' to ``weight of evidence'' is not intuitive and can
confuse the reader. Is there a simpler story to be told here? I think
so. See below.
\todo{Brilliant, I think I can start talking about conditional probabilities to begin with}

\textbf{Suggestion:} There seems to be a nice symmetry. Start with
precise probabilism. We can use sharp probability theory to offer a
theory of the value of the evidence (i.e.~likelihood ratio). Actually, I
think that the likelihood ratio model the idea of balance of the
evidence. What Keynes distinction weight/balance shows is that
likelihood ratio are not, by themselves, enough to model the value of
the evidence. The straightforward move here seems to just have
\textbf{higher order likelihood ratios}. Wouldn't higher order
likelihood ratio be essentially your formal model of the weight of the
evidence? Your measure of weight tracks the difference between (the
weight of the) prior distribution (and the weight of the) posterior
distribution. But higher order likelihood ratios essentially do the same
thing, just like precise likelihood ratios track the difference between
prior and posterior. Is this right? \todo{Yup, more or less}

\textbf{Comment:} If weight if measured by higher order likelihood
ratios, then this can be seen as a generalization of thoughts that many
other had -- say that the absolute value of the likelihood ratio is a
measure of weight (Nance, Glenn Shafer) or that likelihood ratio must be
a measure of weight (Good; see current \textbf{section 4}). So I think
using ````higher order likelihood ratio'' could be a more appealing way
to sell the idea of weight of evidence since most people are already
familiar with likelihood ratios.

\todo{For some reason you dropped Good, so my comment won't make much sense}

\todo{Here comes an explanation about first-order}

Notice that the notion of weight, in principle, is a separate package
from our proposal to go second-order. In principle, even if you just
consider two hypotheses and plain old likelihoods and likelihood ratios,
you could deploy the notion of weight. But this would bring us back to
Good's notion of weight, which does not capture the intuitions about
weight of evidence that we wanted to capture.

\todo{I think going through a BN example with Sally Clark with weights is important here, perhaps also with expected values and so on, if you have the time to copy and clean up those bits!}

\hypertarget{conclusion}{%
\section*{Conclusion}\label{conclusion}}
\addcontentsline{toc}{section}{Conclusion}

\hypertarget{refs}{}
\begin{CSLReferences}{1}{0}
\leavevmode\vadjust pre{\hypertarget{ref-bradley2015critical}{}}%
Bradley, D. (2015). \emph{A critical introduction to formal
epistemology}. Bloomsbury Publishing.

\leavevmode\vadjust pre{\hypertarget{ref-bradley2012scientific}{}}%
Bradley, S. (2012). \emph{Scientific uncertainty and decision making}
(PhD thesis). London School of Economics; Political Science (University
of London).

\leavevmode\vadjust pre{\hypertarget{ref-bradley2019imprecise}{}}%
Bradley, S. (2019). {Imprecise Probabilities}. In E. N. Zalta (Ed.),
\emph{The {Stanford} encyclopedia of philosophy} ({S}pring 2019).
\url{https://plato.stanford.edu/archives/spr2019/entries/imprecise-probabilities/};
Metaphysics Research Lab, Stanford University.

\leavevmode\vadjust pre{\hypertarget{ref-CampbellMoore2020accuracy}{}}%
Campbell-Moore, C. (2020). \emph{Accuracy and imprecise probabilities}.

\leavevmode\vadjust pre{\hypertarget{ref-Carr2020impreciseEvidence}{}}%
Carr, J. R. (2020). Imprecise evidence without imprecise credences.
\emph{Philosophical Studies}, \emph{177}(9), 2735--2758.
\url{https://doi.org/10.1007/s11098-019-01336-7}

\leavevmode\vadjust pre{\hypertarget{ref-Dahlman2022Information}{}}%
Dahlman, C., \& Nordgaard, A. (2022). \emph{Information economics in the
criminal standard of proof}.

\leavevmode\vadjust pre{\hypertarget{ref-deadman1984fiber2}{}}%
Deadman, H. A. (1984a). Fiber evidence and the wayne williams trial
(conclusion). \emph{FBI L. Enforcement Bull.}, \emph{53}, 10--19.

\leavevmode\vadjust pre{\hypertarget{ref-deadman1984fiber1}{}}%
Deadman, H. A. (1984b). Fiber evidence and the wayne williams trial
(part i). \emph{FBI L. Enforcement Bull.}, \emph{53}, 12--20.

\leavevmode\vadjust pre{\hypertarget{ref-Dietrich2016pooling}{}}%
Dietrich, F., \& List, C. (2016). Probabilistic opinion pooling. In A.
Hajek \& C. Hitchcock (Eds.), \emph{Oxford handbook of philosophy and
probability}. Oxford: Oxford University Press.

\leavevmode\vadjust pre{\hypertarget{ref-Lee2017impreciseEpistemology}{}}%
Elkin, L. (2017). \emph{Imprecise probability in epistemology} (PhD
thesis). Ludwig-Maximilians-Universit{ä}t;
Ludwig-Maximilians-Universität München.

\leavevmode\vadjust pre{\hypertarget{ref-Elkin2018resolving}{}}%
Elkin, L., \& Wheeler, G. (2018). Resolving peer disagreements through
imprecise probabilities. \emph{Noûs}, \emph{52}(2), 260--278.
\url{https://doi.org/10.1111/nous.12143}

\leavevmode\vadjust pre{\hypertarget{ref-Fenton2018Risk}{}}%
Fenton, N., \& Neil, M. (2018). \emph{Risk assessment and decision
analysis with bayesian networks}. Chapman; Hall.

\leavevmode\vadjust pre{\hypertarget{ref-VanFraassen2006vague}{}}%
Fraassen, B. C. V. (2006). Vague expectation value loss.
\emph{Philosophical Studies}, \emph{127}(3), 483--491.
\url{https://doi.org/10.1007/s11098-004-7821-2}

\leavevmode\vadjust pre{\hypertarget{ref-Gardenfors1982unreliable}{}}%
Gärdenfors, P., \& Sahlin, N.-E. (1982). Unreliable probabilities, risk
taking, and decision making. \emph{Synthese}, \emph{53}(3), 361--386.
\url{https://doi.org/10.1007/bf00486156}

\leavevmode\vadjust pre{\hypertarget{ref-joyce2005probabilities}{}}%
Joyce, J. M. (2005). How probabilities reflect evidence.
\emph{Philosophical Perspectives}, \emph{19}(1), 153--178.

\leavevmode\vadjust pre{\hypertarget{ref-Kaplan1968decision}{}}%
Kaplan, J. (1968). Decision theory and the fact-finding process.
\emph{Stanford Law Review}, \emph{20}(6), 1065--1092.

\leavevmode\vadjust pre{\hypertarget{ref-kasser2016two}{}}%
Kasser, J. (2016). Two conceptions of weight of evidence in peirce's
illustrations of the logic of science. \emph{Erkenntnis}, \emph{81}(3),
629--648.

\leavevmode\vadjust pre{\hypertarget{ref-keynes1921treatise}{}}%
Keynes, J. M. (1921). \emph{A treatise on probability, 1921}. London:
Macmillan.

\leavevmode\vadjust pre{\hypertarget{ref-konek2013foundations}{}}%
Konek, J. (2013). \emph{New foundations for imprecise bayesianism} (PhD
thesis). University of Michigan.

\leavevmode\vadjust pre{\hypertarget{ref-Kyburg1961}{}}%
Kyburg, H. E. (1961). \emph{Probability and the logic of rational
belief}. Wesleyan University Press.

\leavevmode\vadjust pre{\hypertarget{ref-kyburg2001uncertain}{}}%
Kyburg Jr, H. E., \& Teng, C. M. (2001). \emph{Uncertain inference}.
Cambridge University Press.

\leavevmode\vadjust pre{\hypertarget{ref-Levi1974ideterminate}{}}%
Levi, I. (1974). On indeterminate probabilities. \emph{The Journal of
Philosophy}, \emph{71}(13), 391. \url{https://doi.org/10.2307/2025161}

\leavevmode\vadjust pre{\hypertarget{ref-Levi1980enterprise}{}}%
Levi, I. (1980). \emph{The enterprise of knowledge: An essay on
knowledge, credal probability, and chance}. MIT Press.

\leavevmode\vadjust pre{\hypertarget{ref-levi2011weight}{}}%
Levi, I. (2011). The weight of argument. In \emph{Fundamental
uncertainty} (pp. 39--58). Springer.

\leavevmode\vadjust pre{\hypertarget{ref-Mayo-Wilson2016scoring}{}}%
Mayo-Wilson, C., \& Wheeler, G. (2016). Scoring imprecise credences: A
mildly immodest proposal. \emph{Philosophy and Phenomenological
Research}, \emph{92}(1), 55--78.
\url{https://doi.org/10.1111/phpr.12256}

\leavevmode\vadjust pre{\hypertarget{ref-nance2016}{}}%
Nance, D. A. (2016). \emph{The burdens of proof: Discriminatory power,
weight of evidence, and tenacity of belief}. Cambridge University Press.

\leavevmode\vadjust pre{\hypertarget{ref-o2006uncertain}{}}%
O'Hagan, A., Buck, C. E., Daneshkhah, A., Eiser, J. R., Garthwaite, P.
H., Jenkinson, D. J., \ldots{} Rakow, T. (2006). \emph{Uncertain
judgements: Eliciting experts' probabilities}.

\leavevmode\vadjust pre{\hypertarget{ref-Rinard2013against}{}}%
Rinard, S. (2013). Against radical credal imprecision. \emph{Thought: A
Journal of Philosophy}, \emph{2}(1), 157--165.
\url{https://doi.org/10.1002/tht3.84}

\leavevmode\vadjust pre{\hypertarget{ref-Schoenfield2017accuracy}{}}%
Schoenfield, M. (2017). The accuracy and rationality of imprecise
credences. \emph{Noûs}, \emph{51}(4), 667--685.
\url{https://doi.org/10.1111/nous.12105}

\leavevmode\vadjust pre{\hypertarget{ref-seidenfeld2012forecasting}{}}%
Seidenfeld, T., Schervish, M., \& Kadane, J. (2012). Forecasting with
imprecise probabilities. \emph{International Journal of Approximate
Reasoning}, \emph{53}, 1248--1261.
\url{https://doi.org/10.1016/j.ijar.2012.06.018}

\leavevmode\vadjust pre{\hypertarget{ref-Sjerps2015Uncertainty}{}}%
Sjerps, M. J., Alberink, I., Bolck, A., Stoel, R. D., Vergeer, P., \&
Zanten, J. H. van. (2015). {Uncertainty and LR: to integrate or not to
integrate, that's the question}. \emph{Law, Probability and Risk},
\emph{15}(1), 23--29. \url{https://doi.org/10.1093/lpr/mgv005}

\leavevmode\vadjust pre{\hypertarget{ref-Skyrms1977Resiliency-Prop}{}}%
Skyrms, B. (1977). Resiliency, propensity, and causal necessity.
\emph{Journal of Philosophy}, \emph{74}(11), 704--713.

\leavevmode\vadjust pre{\hypertarget{ref-Stewart2018pooling}{}}%
Stewart, R. T., \& Quintana, I. O. (2018). Learning and pooling, pooling
and learning. \emph{Erkenntnis}, \emph{83}(3), 1--21.
\url{https://doi.org/10.1007/s10670-017-9894-2}

\leavevmode\vadjust pre{\hypertarget{ref-Sturgeon2008grain}{}}%
Sturgeon, S. (2008). Reason and the grain of belief. \emph{No{û}s},
\emph{42}(1), 139--165. Retrieved from
\url{http://www.jstor.org/stable/25177157}

\leavevmode\vadjust pre{\hypertarget{ref-Taroni2015Dismissal}{}}%
Taroni, F., Bozza, S., Biedermann, A., \& Aitken, C. (2015). {Dismissal
of the illusion of uncertainty in the assessment of a likelihood ratio}.
\emph{Law, Probability and Risk}, \emph{15}(1), 1--16.
\url{https://doi.org/10.1093/lpr/mgv008}

\leavevmode\vadjust pre{\hypertarget{ref-Thomason2003How-the-Probabi}{}}%
Thompson, W. C., Taroni, F., \& Aitken, C. G. G. (2003). How the
probability of a false positive affects the value of {DNA} evidence.
\emph{Journal of Forensic Science}, \emph{48}(1), 47--54.

\leavevmode\vadjust pre{\hypertarget{ref-Titelbaum2020Fundamentals-of}{}}%
Titelbaum, M. G. (2020). \emph{Fundamentals of bayesian epistemology}.

\leavevmode\vadjust pre{\hypertarget{ref-walley1991statistical}{}}%
Walley, P. (1991). \emph{Statistical reasoning with imprecise
probabilities}. Chapman; Hall London.

\end{CSLReferences}

\end{document}
