% Options for packages loaded elsewhere
\PassOptionsToPackage{unicode}{hyperref}
\PassOptionsToPackage{hyphens}{url}
\PassOptionsToPackage{dvipsnames,svgnames*,x11names*}{xcolor}
%
\documentclass[
  11pt,
  dvipsnames,enabledeprecatedfontcommands]{scrartcl}
\usepackage{amsmath,amssymb}
\usepackage{lmodern}
\usepackage{ifxetex,ifluatex}
\ifnum 0\ifxetex 1\fi\ifluatex 1\fi=0 % if pdftex
  \usepackage[T1]{fontenc}
  \usepackage[utf8]{inputenc}
  \usepackage{textcomp} % provide euro and other symbols
\else % if luatex or xetex
  \usepackage{unicode-math}
  \defaultfontfeatures{Scale=MatchLowercase}
  \defaultfontfeatures[\rmfamily]{Ligatures=TeX,Scale=1}
\fi
% Use upquote if available, for straight quotes in verbatim environments
\IfFileExists{upquote.sty}{\usepackage{upquote}}{}
\IfFileExists{microtype.sty}{% use microtype if available
  \usepackage[]{microtype}
  \UseMicrotypeSet[protrusion]{basicmath} % disable protrusion for tt fonts
}{}
\usepackage{xcolor}
\IfFileExists{xurl.sty}{\usepackage{xurl}}{} % add URL line breaks if available
\IfFileExists{bookmark.sty}{\usepackage{bookmark}}{\usepackage{hyperref}}
\hypersetup{
  pdftitle={Corroboration},
  pdfauthor={Marcello/Rafal},
  colorlinks=true,
  linkcolor=Maroon,
  filecolor=Maroon,
  citecolor=Blue,
  urlcolor=blue,
  pdfcreator={LaTeX via pandoc}}
\urlstyle{same} % disable monospaced font for URLs
\usepackage{graphicx}
\makeatletter
\def\maxwidth{\ifdim\Gin@nat@width>\linewidth\linewidth\else\Gin@nat@width\fi}
\def\maxheight{\ifdim\Gin@nat@height>\textheight\textheight\else\Gin@nat@height\fi}
\makeatother
% Scale images if necessary, so that they will not overflow the page
% margins by default, and it is still possible to overwrite the defaults
% using explicit options in \includegraphics[width, height, ...]{}
\setkeys{Gin}{width=\maxwidth,height=\maxheight,keepaspectratio}
% Set default figure placement to htbp
\makeatletter
\def\fps@figure{htbp}
\makeatother
\setlength{\emergencystretch}{3em} % prevent overfull lines
\providecommand{\tightlist}{%
  \setlength{\itemsep}{0pt}\setlength{\parskip}{0pt}}
\setcounter{secnumdepth}{5}
%\documentclass{article}

% %packages
\usepackage{booktabs}
%\usepackage[left]{showlabels}
\usepackage{multirow}
\usepackage{subcaption}
\usepackage{wrapfig}
\usepackage{graphicx}
\usepackage{longtable}
\usepackage{ragged2e}
\usepackage{etex}
%\usepackage{yfonts}
\usepackage{marvosym}
\usepackage[notextcomp]{kpfonts}
\usepackage{nicefrac}
\newcommand*{\QED}{\hfill \footnotesize {\sc Q.e.d.}}
\usepackage{floatrow}
\usepackage{multicol}

\usepackage[textsize=footnotesize]{todonotes}
\newcommand{\ali}[1]{\todo[color=gray!40]{\textbf{Alicja:} #1}}
\newcommand{\mar}[1]{\todo[color=blue!40]{#1}}
\newcommand{\raf}[1]{\todo[color=olive!40]{#1}}

%\linespread{1.5}
\newcommand{\indep}{\!\perp \!\!\! \perp\!}


\setlength{\parindent}{10pt}
\setlength{\parskip}{1pt}


%language
%\usepackage{times}
\usepackage{mathptmx}
\usepackage[scaled=0.86]{helvet}
\usepackage{t1enc}
%\usepackage[utf8x]{inputenc}
%\usepackage[polish]{babel}
%\usepackage{polski}




%AMS
\usepackage{amsfonts}
\usepackage{amssymb}
\usepackage{amsthm}
\usepackage{amsmath}
\usepackage{mathtools}

\usepackage{geometry}
 \geometry{a4paper,left=35mm,top=20mm,}


%environments
\newtheorem{fact}{Fact}


% allow page breaks in equations
\allowdisplaybreaks


%abbreviations
\newcommand{\ra}{\rangle}
\newcommand{\la}{\langle}
\newcommand{\n}{\neg}
\newcommand{\et}{\wedge}
\newcommand{\jt}{\rightarrow}
\newcommand{\ko}[1]{\forall  #1\,}
\newcommand{\ro}{\leftrightarrow}
\newcommand{\exi}[1]{\exists\, {_{#1}}}
\newcommand{\pr}[1]{\ensuremath{\mathsf{P}(#1)}}
\newcommand{\cost}{\mathsf{cost}}
\newcommand{\benefit}{\mathsf{benefit}}
\newcommand{\ut}{\mathsf{ut}}

\newcommand{\odds}{\mathsf{Odds}}
\newcommand{\ind}{\mathsf{Ind}}
\newcommand{\nf}[2]{\nicefrac{#1\,}{#2}}
\newcommand{\R}[1]{\texttt{#1}}
\newcommand{\prr}[1]{\mbox{$\mathtt{P}_{prior}(#1)$}}
\newcommand{\prp}[1]{\mbox{$\mathtt{P}_{posterior}(#1)$}}



\newtheorem{q}{\color{blue}Question}
\newtheorem{lemma}{Lemma}
\newtheorem{theorem}{Theorem}
\newtheorem{corollary}{Corollary}[fact]


%technical intermezzo
%---------------------

\newcommand{\intermezzoa}{
	\begin{minipage}[c]{13cm}
	\begin{center}\rule{10cm}{0.4pt}



	\tiny{\sc Optional Content Starts}
	
	\vspace{-1mm}
	
	\rule{10cm}{0.4pt}\end{center}
	\end{minipage}\nopagebreak 
	}


\newcommand{\intermezzob}{\nopagebreak 
	\begin{minipage}[c]{13cm}
	\begin{center}\rule{10cm}{0.4pt}

	\tiny{\sc Optional Content Ends}
	
	\vspace{-1mm}
	
	\rule{10cm}{0.4pt}\end{center}
	\end{minipage}
	}
	
	
%--------------------






















\newtheorem*{reply*}{Reply}
\usepackage{enumitem}
\newcommand{\question}[1]{\begin{enumerate}[resume,leftmargin=0cm,labelsep=0cm,align=left]
\item #1
\end{enumerate}}

\usepackage{float}

% \setbeamertemplate{blocks}[rounded][shadow=true]
% \setbeamertemplate{itemize items}[ball]
% \AtBeginPart{}
% \AtBeginSection{}
% \AtBeginSubsection{}
% \AtBeginSubsubsection{}
% \setlength{\emergencystretch}{0em}
% \setlength{\parskip}{0pt}






\usepackage[authoryear]{natbib}

%\bibliographystyle{apalike}



\usepackage{tikz}
\usetikzlibrary{positioning,shapes,arrows}

\ifluatex
  \usepackage{selnolig}  % disable illegal ligatures
\fi

\title{Corroboration}
\author{Marcello/Rafal}
\date{}

\begin{document}
\maketitle

\hypertarget{on-gardiners-paper-on-corroboration}{%
\section{On Gardiner's paper on
corroboration}\label{on-gardiners-paper-on-corroboration}}

The claim is that `probabilistic balance' cannot account for
corroboration because:

\begin{enumerate}
\def\labelenumi{\arabic{enumi}.}
\item
  corroborating evidence increases the probability of a hypothesis
\item
  but also rules out relevant error possibilities
\end{enumerate}

Suppose W1 claims p and also W2 claims p.~They corroborate one another.
So the probability of p based on W1 alone is lower than the probability
of p based on W1 and W2. But there is something else. W1 could be wrong
in various wasy: was lying, got confused, forgot, etc. The convergence
of W1 and W2 helps to eliminate these error possibilities.

Why cannot `probabilistic balance' account for that? There could be
corroborating evidence that has marginal effect on probabilistic
balance---the probability of the hypothesis moves from very high to
slightly higher. Yet, despite the slight change in probability,
corroboration has a strong credibility boosting effect that probability
cannot capture. This can be explained by noting that corroborating
evidence eliminate salient error possibilities, even though it does not
change dramatically the probability of the hypothesis of interest.

This is the motivating example Gardiner uses:

\begin{quote}
A cold-hit DNA search---that is, trawling through DNA databases for
similar allele patterns-identifies a match with Brett. The allele
profile is rare and so the chance of somebody unrelated to Brett having
a similar profile is extremely low.1 But the database does not include
the entire population. This DNA match makes Brett the lead suspect and,
indeed, renders it probable Brett is guilty. Later Corey, a known fence,
is apprehended selling the watch. When questioned, Corey says `I bought
the watch from someone called Brett. He stole it.'
\end{quote}

I wonder whether this example is correct:

\begin{itemize}
\item
  Is the guilt probability based on DNA match that high? If we take
  error probabilities into account, such as false positives, the
  probability isn't that high, as shown in the likelihood ratio chapter.
\item
  Not sure if corroboration based Corey's assertion is negligible. If we
  measure the strength of evidence in terms of LR or BF, there is going
  to be a stronger difference, right?
\end{itemize}

Other questions:

\begin{itemize}
\item
  How does Rafal's probabilistic account of corroboration cope with this
  kind of example?
\item
  Does relying on a specific story, rather than the generic claim `the
  defendant is guilt,' address Gardiner's challenge?
\end{itemize}

Comments I sent to Gardiner about her paper (on April 1, 2022):

\begin{enumerate}
\def\labelenumi{\arabic{enumi})}
\item
  First, a comment about the DNA case you start out with. A lot hinges
  on how the case is defined. If all the facts are settled in sufficient
  detail and the only question open is who did it, a DNA match can be
  powerful in addressing this identity question. Many courts said,
  perhaps mistakenly, that it is sufficient for a conviction under such
  circumstances. See the cases cited by Andrea Roth. But, even under
  such circumstances, I am not sure it is uncontroversial to say that
  the probability of guilt is 99.5\% and it increases only marginally by
  corroborating evidence. It might actually be lower and increase by
  quite a bit because the corroborating evidence bears on the claim the
  defendant did it (`he stole it'), while the match bears on the fact
  that the defendant is the source of the crime traces. So there are
  many details up for discussion in that example.
\item
  I kept asking myself, couldn't the relevance of error possibilities be
  a matter of how probable they are and whether they meet a
  probabilistic threshold of relevance? Maybe this thought is wrong, but
  I wanted you to say why it is wrong. In a DNA evidence case, the
  probability that aliens did it (a distant, irrelevant error
  possibility) seems lower than the probability that a twin with the
  same genetic profile did it (a less distant, possibly relevant, error
  probability).

  2b) A related comment. You say there are error possibilities which are
  equally probable, yet one is relevant and the other is not. Do you
  have in mind lottery/newspaper scenarios? But what about legal cases?
  On pp.11-12, you suggest that two items of evidence may affect the
  posterior probability of guilt to the same extent, even though one
  addresses a relevant error possibility while the other does not. The
  example consists of the evidence ``Holly was in the US when the crime
  occurred'' and ``police never framed or conspired against Holly in the
  past.'' It is not clear to me which of these two items of evidence you
  think addresses a relevant error possibility. Maybe the fact that the
  police conspire against people isn't relevant because we tend to think
  that police do not do such things. But again, isn't this because we
  think this is less likely than say the possibility that Holly would
  have been in another country?
\item
  To set up the question of corroboration, you need an account of what
  it means for two items of evidence to be independent. Two witnesses
  who are looking at the same scene aren't independent because, if they
  are truthful and accurate, they must report the same thing. The notion
  of independence at work here is one of conditional probabilistic
  independence given the target hypothesis. I do not know how a relevant
  alternative theory can provide an account of independence, or perhaps
  needs to borrow the probabilistic one.
\item
  Towards the end, pp.~13 onwards, you examine whether a single piece of
  evidence may suffice for a conviction or whether a corroboration
  requirement applies. Some have proposed a ban against single-evidence
  convictions, but virtually any legal system allows a conviction based
  on a single piece of evidence. (Interestingly, Roman and medieval law
  did not allow that, but that was before the rationalistic principles
  of ``free proof'' took hold.) Saying that a single item suffices for a
  conviction is misleading, though, because a trial is adversarial and
  so, even if there is a single item of evidence, there is always other
  ancillary information, say, what the witness said in response to
  questions.

  4b) Relatedly, it seems to me, what counts as a relevant error
  possibility is not defined before the trial. It depends on what the
  two sides disagree about. That an animal killed the victim might not
  be relevant at first, but might become relevant as more information
  comes in. So, I am not sure I'd reject your claim 2 on p.~13, but I'd
  say it needs to be stated more clearly, in a way that takes into
  account how adversarial scrutiny affects the formation of error
  possibilities.
\end{enumerate}

\end{document}
