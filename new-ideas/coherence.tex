% Options for packages loaded elsewhere
\PassOptionsToPackage{unicode}{hyperref}
\PassOptionsToPackage{hyphens}{url}
\PassOptionsToPackage{dvipsnames,svgnames*,x11names*}{xcolor}
%
\documentclass[
  11pt,
  dvipsnames,enabledeprecatedfontcommands]{scrartcl}
\usepackage{amsmath,amssymb}
\usepackage{lmodern}
\usepackage{ifxetex,ifluatex}
\ifnum 0\ifxetex 1\fi\ifluatex 1\fi=0 % if pdftex
  \usepackage[T1]{fontenc}
  \usepackage[utf8]{inputenc}
  \usepackage{textcomp} % provide euro and other symbols
\else % if luatex or xetex
  \usepackage{unicode-math}
  \defaultfontfeatures{Scale=MatchLowercase}
  \defaultfontfeatures[\rmfamily]{Ligatures=TeX,Scale=1}
\fi
% Use upquote if available, for straight quotes in verbatim environments
\IfFileExists{upquote.sty}{\usepackage{upquote}}{}
\IfFileExists{microtype.sty}{% use microtype if available
  \usepackage[]{microtype}
  \UseMicrotypeSet[protrusion]{basicmath} % disable protrusion for tt fonts
}{}
\usepackage{xcolor}
\IfFileExists{xurl.sty}{\usepackage{xurl}}{} % add URL line breaks if available
\IfFileExists{bookmark.sty}{\usepackage{bookmark}}{\usepackage{hyperref}}
\hypersetup{
  pdftitle={Coherence},
  pdfauthor={Marcello/Rafal},
  colorlinks=true,
  linkcolor=Maroon,
  filecolor=Maroon,
  citecolor=Blue,
  urlcolor=blue,
  pdfcreator={LaTeX via pandoc}}
\urlstyle{same} % disable monospaced font for URLs
\usepackage{graphicx}
\makeatletter
\def\maxwidth{\ifdim\Gin@nat@width>\linewidth\linewidth\else\Gin@nat@width\fi}
\def\maxheight{\ifdim\Gin@nat@height>\textheight\textheight\else\Gin@nat@height\fi}
\makeatother
% Scale images if necessary, so that they will not overflow the page
% margins by default, and it is still possible to overwrite the defaults
% using explicit options in \includegraphics[width, height, ...]{}
\setkeys{Gin}{width=\maxwidth,height=\maxheight,keepaspectratio}
% Set default figure placement to htbp
\makeatletter
\def\fps@figure{htbp}
\makeatother
\setlength{\emergencystretch}{3em} % prevent overfull lines
\providecommand{\tightlist}{%
  \setlength{\itemsep}{0pt}\setlength{\parskip}{0pt}}
\setcounter{secnumdepth}{5}
%\documentclass{article}

% %packages
\usepackage{booktabs}
%\usepackage[left]{showlabels}
\usepackage{multirow}
\usepackage{subcaption}
\usepackage{wrapfig}
\usepackage{graphicx}
\usepackage{longtable}
\usepackage{ragged2e}
\usepackage{etex}
%\usepackage{yfonts}
\usepackage{marvosym}
\usepackage[notextcomp]{kpfonts}
\usepackage{nicefrac}
\newcommand*{\QED}{\hfill \footnotesize {\sc Q.e.d.}}
\usepackage{floatrow}
\usepackage{multicol}

\usepackage[textsize=footnotesize]{todonotes}
\newcommand{\ali}[1]{\todo[color=gray!40]{\textbf{Alicja:} #1}}
\newcommand{\mar}[1]{\todo[color=blue!40]{#1}}
\newcommand{\raf}[1]{\todo[color=olive!40]{#1}}

%\linespread{1.5}
\newcommand{\indep}{\!\perp \!\!\! \perp\!}


\setlength{\parindent}{10pt}
\setlength{\parskip}{1pt}


%language
%\usepackage{times}
\usepackage{mathptmx}
\usepackage[scaled=0.86]{helvet}
\usepackage{t1enc}
%\usepackage[utf8x]{inputenc}
%\usepackage[polish]{babel}
%\usepackage{polski}




%AMS
\usepackage{amsfonts}
\usepackage{amssymb}
\usepackage{amsthm}
\usepackage{amsmath}
\usepackage{mathtools}

\usepackage{geometry}
 \geometry{a4paper,left=35mm,top=20mm,}


%environments
\newtheorem{fact}{Fact}


% allow page breaks in equations
\allowdisplaybreaks


%abbreviations
\newcommand{\ra}{\rangle}
\newcommand{\la}{\langle}
\newcommand{\n}{\neg}
\newcommand{\et}{\wedge}
\newcommand{\jt}{\rightarrow}
\newcommand{\ko}[1]{\forall  #1\,}
\newcommand{\ro}{\leftrightarrow}
\newcommand{\exi}[1]{\exists\, {_{#1}}}
\newcommand{\pr}[1]{\ensuremath{\mathsf{P}(#1)}}
\newcommand{\cost}{\mathsf{cost}}
\newcommand{\benefit}{\mathsf{benefit}}
\newcommand{\ut}{\mathsf{ut}}

\newcommand{\odds}{\mathsf{Odds}}
\newcommand{\ind}{\mathsf{Ind}}
\newcommand{\nf}[2]{\nicefrac{#1\,}{#2}}
\newcommand{\R}[1]{\texttt{#1}}
\newcommand{\prr}[1]{\mbox{$\mathtt{P}_{prior}(#1)$}}
\newcommand{\prp}[1]{\mbox{$\mathtt{P}_{posterior}(#1)$}}



\newtheorem{q}{\color{blue}Question}
\newtheorem{lemma}{Lemma}
\newtheorem{theorem}{Theorem}
\newtheorem{corollary}{Corollary}[fact]


%technical intermezzo
%---------------------

\newcommand{\intermezzoa}{
	\begin{minipage}[c]{13cm}
	\begin{center}\rule{10cm}{0.4pt}



	\tiny{\sc Optional Content Starts}
	
	\vspace{-1mm}
	
	\rule{10cm}{0.4pt}\end{center}
	\end{minipage}\nopagebreak 
	}


\newcommand{\intermezzob}{\nopagebreak 
	\begin{minipage}[c]{13cm}
	\begin{center}\rule{10cm}{0.4pt}

	\tiny{\sc Optional Content Ends}
	
	\vspace{-1mm}
	
	\rule{10cm}{0.4pt}\end{center}
	\end{minipage}
	}
	
	
%--------------------






















\newtheorem*{reply*}{Reply}
\usepackage{enumitem}
\newcommand{\question}[1]{\begin{enumerate}[resume,leftmargin=0cm,labelsep=0cm,align=left]
\item #1
\end{enumerate}}

\usepackage{float}

% \setbeamertemplate{blocks}[rounded][shadow=true]
% \setbeamertemplate{itemize items}[ball]
% \AtBeginPart{}
% \AtBeginSection{}
% \AtBeginSubsection{}
% \AtBeginSubsubsection{}
% \setlength{\emergencystretch}{0em}
% \setlength{\parskip}{0pt}






\usepackage[authoryear]{natbib}

%\bibliographystyle{apalike}



\usepackage{tikz}
\usetikzlibrary{positioning,shapes,arrows}

\ifluatex
  \usepackage{selnolig}  % disable illegal ligatures
\fi

\title{Coherence}
\author{Marcello/Rafal}
\date{}

\begin{document}
\maketitle

\hypertarget{questions}{%
\section{Questions}\label{questions}}

\begin{itemize}
\item
  How do we define coherence of a story (narrative, etc.)?
\item
  Coherence of what? Is just the coherence of a set of statements
  (making up a story, narrative, etc.)? Or does coherence also involve
  the statements describing the supporting evidence?
\item
  If coherence also takes into account the supporting evidence, should
  evidence be given a more entrenched status (a weak form of
  foundationalism)? Or do statements about facts (elements of a story)
  and statements about evidence play the same role in an account of
  coherence?
\item
  What is the relationship between coherence and corroboration
  (congruence of different, independent witness reports)? is
  corroboration a specific case of coherence?
\item
  What is the connection between accuracy, coherence and probability? Is
  coherence a good guide to truth?
\item
  Why is coherence important? How can it be---should it be---used as a
  decision criterion in trial proceedings?
\item
  Is coherence a single notion or multilevel notion, a combination of
  various theoretical virtues such as specificity, explanatory power,
  plausibility, consistency with background beliefs, etc.?
\end{itemize}

\hypertarget{philosophical-proposals-from-sep-entry-on-coherence}{%
\section{Philosophical proposals (from SEP entry on
Coherence)}\label{philosophical-proposals-from-sep-entry-on-coherence}}

\hypertarget{ewing-1934-consistency-plus-mutual-logical-support}{%
\subsection{Ewing (1934): consistency plus mutual logical
support}\label{ewing-1934-consistency-plus-mutual-logical-support}}

``According to Ewing, a coherent set is characterized partly by
consistency and partly by the property that every belief in the set
follows logically from the others taken together.''

\hypertarget{lewis-mutual-probabilistic-support}{%
\subsection{Lewis: mutual probabilistic
support}\label{lewis-mutual-probabilistic-support}}

``As Lewis defines the term, a set of `supposed facts asserted' is
coherent (congruent) just in case every element in the set is supported
by all the other elements taken together, whereby `support' is
understood not in logical terms but in a probabilistic sense''

\hypertarget{bovens-and-olsson-2000-mutual-probabilistic-support-of-subsets}{%
\subsection{Bovens and Olsson 2000: mutual probabilistic support (of
subsets)}\label{bovens-and-olsson-2000-mutual-probabilistic-support-of-subsets}}

``Against Lewis's proposal one could hold that it seems arbitrary to
focus merely on the support single elements of a set receive from the
rest of the set (cf.~Bovens and Olsson 2000). Why not consider the
support any subset, not just singletons, receives from the rest?''

\hypertarget{laurence-bonjour-1985-multilevel-coherence}{%
\subsection{Laurence BonJour (1985): multilevel
coherence}\label{laurence-bonjour-1985-multilevel-coherence}}

Coherence of a system of beliefs depends on:

\begin{enumerate}
\def\labelenumi{\arabic{enumi}.}
\item
  logical consistency
\item
  degree of probabilistic consistency
\item
  presence of inferential connections between its component beliefs and
  the number and strength of such connections (or proportional density
  of such connections)\footnote{``BonJour's third criterion, taken at
    face value, entails therefore that a bigger system will generally
    have a higher degree of coherence due to its sheer size. But this is
    at least not obviously correct. A possible modified coherence
    criterion could state that what is correlated with higher coherence
    is not the number of inferential connections but rather the
    inferential density of the system, where the latter is obtained by
    dividing the number of inferential connections by the number of
    beliefs in the system.''}
\item
  coherence is diminished if system of belief is divided into subsystems
  of beliefs which are relatively unconnected to each other by
  inferential connections.
\item
  coherence is diminished by presence of unexplained anomalies in the
  believed content of the system.
\end{enumerate}

\hypertarget{rescher-1973-maximal-consistency-plus-plausibility}{%
\subsection{Rescher 1973: maximal consistency plus
plausibility}\label{rescher-1973-maximal-consistency-plus-plausibility}}

``the purpose of Rescher's investigation is \ldots{} to find a truth
criterion \ldots{} a systematic procedure for selecting from a set of
conflicting and even contradictory truth-candidates those elements which
it is rational to accept as bona fide truths.''

``His solution amounts to first identifying the maximal consistent
subsets of the original set, i.e., the subsets that are consistent but
would become inconsistent if extended by further elements of the
original set, and then choosing the most `plausible' among these
subsets. Plausibility is characterized in way that reveals no obvious
relation to the traditional concept of coherence.''

\hypertarget{thagard-explanatory-coherence}{%
\subsection{Thagard: explanatory
coherence}\label{thagard-explanatory-coherence}}

\hypertarget{lewis-bonjour-debate-abut-congruent-reports-corroboration}{%
\subsection{Lewis Bonjour debate abut congruent reports
(corroboration?)}\label{lewis-bonjour-debate-abut-congruent-reports-corroboration}}

Take two witnesses who say the same thing about a certain matter
(congruence of reports). So there is coherence (congruence,
corroboration?) between the content of their assertions.

\begin{itemize}
\item
  Lewis position: ``if the beliefs in a set have no initial credibility,
  then no justification will ensue from observing the coherence of that
  set. Thus, Lewis is advocating weak foundationalism rather than a pure
  coherence theory.''
\item
  Bonjour position: coherence triggers a confidence boost even without
  an antecedent credibility of the individual beliefs, contrast Lewis
\end{itemize}

Lewis seems right because the following are inconsistent in probability
theory:

\begin{enumerate}
\def\labelenumi{(\alph{enumi})}
\item
  conditional independence of items of evidence \(E_1\) and \(E_2\)
  relative to a statement \(A\) (the statement about which the reports
  are congruent),
\item
  non foundationalism (i.e.~individual reports \(E_1\) and \(E_2\) do
  not raise, taken separately, the probability of \(A\)) and
\item
  coherence justification (i.e.~reports taken together do raise the
  probability of \(A\)).
\end{enumerate}

Unlike Lewis, Bonjour believes that (c) can exist without (b).

\hypertarget{probabilistic-measure-of-coherence}{%
\subsection{Probabilistic measure of
coherence}\label{probabilistic-measure-of-coherence}}

SEP has a quick discussion of these measures, includign a mesure by
Fitelson. This is better summarized in Urbaniak and Kowalewska's paper
on coherence. Check!

\hypertarget{coherence-and-truth-conduciveness}{%
\subsection{Coherence and truth
conduciveness}\label{coherence-and-truth-conduciveness}}

There is an ``Analysis debate'' about whether coherence is truth
conducive.

\begin{itemize}
\item
  One might say the more statements, the less probable, but the more
  coherent. Thus, coherence is not truth conducive (does not lead to
  higher probability) simply because
  \(\pr{A\wedge B}>\pr{A\wedge B\wedge C}\) even though \(A\wedge B\)
  may be less coherent than \(A\wedge B\wedge C\).
\item
  But what if we consider the supporting evidence? Is it the case that
  also
  \(\pr{A\wedge B \vert E_a, E_b}>\pr{A\wedge B\wedge C \vert E_a, E_b, E_c}\).
\item
  Incidentally, this seems to connect with the conjunction problem as
  well as the discussion between accuracy and specificity.
\item
  Also, is it correct to define truth-conduciveness in terms of
  probability?
\end{itemize}

\hypertarget{impossibility-results-bovens-and-hartmann-2003-and-others}{%
\subsection{Impossibility results: Bovens and Hartmann (2003) and
others}\label{impossibility-results-bovens-and-hartmann-2003-and-others}}

This draws from the Lewis/Bonjour debate, probabilistic measures of
coherence and truth conduciveness debate.

\begin{itemize}
\item
  ``The question of interest, then, is whether more coherence implies a
  higher probability (given independence and individual credibility)
  everything else being equal. We are now finally in a position to state
  the impossibility theorems. What they show is that no measure of
  coherence is truth conducive even in a weak ceteris paribus sense,
  under the favorable conditions of (conditional) independence and
  individual credibility.''
\item
  Key idea of the proof: ``They show that there are sets S and S', each
  containing three propositions, such that which set is more likely to
  be true will depend on the level at which the individual credibility
  (reliability) is held fixed. Thus for lower degrees of reliability,
  one set, say S, will be more probable than the other set, S'; for
  higher degrees of reliability, the situation will be reversed. One can
  now find a counterexample to the truth conduciveness of any measure C
  through a strategic choice of the level at which the reliability is
  held fixed.''
\item
  This gives rise to a paradox: ``These impossibility results give rise
  to a thought-provoking paradox. It can hardly be doubted that we trust
  and rely on coherence reasoning when judging the believability of
  information, in everyday life and in science \ldots{} But how can this
  be when in fact coherence is not truth conducive?''
\end{itemize}

Some thoughts and questions:

\begin{itemize}
\item
  Don't Urbaniak and Kowalewska show that their measure of coherence is
  truth conducive? How does their notion of structured coherence fit
  into the debate about these impossibility results?
\item
  The other notes on specificity and accuracy claims that a more
  specific story is better for accuracy even though a more specific
  narrative is less probable. So the issue here might that high
  probability is not the right guide to accuracy or truth, however
  strange that may sound. So the denate is based on that false
  assumption. No one seems to question that assumption. To be explored
  more.
\end{itemize}

\end{document}
