\documentclass[11pt,dvipsnames,enabledeprecatedfontcommands]{scrartcl}
\usepackage{lmodern}
\usepackage{amssymb,amsmath}
\usepackage{ifxetex,ifluatex}
\usepackage{fixltx2e} % provides \textsubscript
\ifnum 0\ifxetex 1\fi\ifluatex 1\fi=0 % if pdftex
  \usepackage[T1]{fontenc}
  \usepackage[utf8]{inputenc}
\else % if luatex or xelatex
  \ifxetex
    \usepackage{mathspec}
  \else
    \usepackage{fontspec}
  \fi
  \defaultfontfeatures{Ligatures=TeX,Scale=MatchLowercase}
\fi
% use upquote if available, for straight quotes in verbatim environments
\IfFileExists{upquote.sty}{\usepackage{upquote}}{}
% use microtype if available
\IfFileExists{microtype.sty}{%
\usepackage[]{microtype}
\UseMicrotypeSet[protrusion]{basicmath} % disable protrusion for tt fonts
}{}
\PassOptionsToPackage{hyphens}{url} % url is loaded by hyperref
\usepackage[unicode=true]{hyperref}
\PassOptionsToPackage{usenames,dvipsnames}{color} % color is loaded by hyperref
\hypersetup{
            colorlinks=true,
            linkcolor=Maroon,
            citecolor=Blue,
            urlcolor=blue,
            breaklinks=true}
\urlstyle{same}  % don't use monospace font for urls
\usepackage{graphicx,grffile}
\makeatletter
\def\maxwidth{\ifdim\Gin@nat@width>\linewidth\linewidth\else\Gin@nat@width\fi}
\def\maxheight{\ifdim\Gin@nat@height>\textheight\textheight\else\Gin@nat@height\fi}
\makeatother
% Scale images if necessary, so that they will not overflow the page
% margins by default, and it is still possible to overwrite the defaults
% using explicit options in \includegraphics[width, height, ...]{}
\setkeys{Gin}{width=\maxwidth,height=\maxheight,keepaspectratio}
\IfFileExists{parskip.sty}{%
\usepackage{parskip}
}{% else
\setlength{\parindent}{0pt}
\setlength{\parskip}{6pt plus 2pt minus 1pt}
}
\setlength{\emergencystretch}{3em}  % prevent overfull lines
\providecommand{\tightlist}{%
  \setlength{\itemsep}{0pt}\setlength{\parskip}{0pt}}
\setcounter{secnumdepth}{5}
% Redefines (sub)paragraphs to behave more like sections
\ifx\paragraph\undefined\else
\let\oldparagraph\paragraph
\renewcommand{\paragraph}[1]{\oldparagraph{#1}\mbox{}}
\fi
\ifx\subparagraph\undefined\else
\let\oldsubparagraph\subparagraph
\renewcommand{\subparagraph}[1]{\oldsubparagraph{#1}\mbox{}}
\fi

% set default figure placement to htbp
\makeatletter
\def\fps@figure{htbp}
\makeatother

%\documentclass{article}

% %packages
 \usepackage{booktabs}

\usepackage{multirow}
\usepackage{multicol}

\usepackage{graphicx}
\usepackage{longtable}
\usepackage{ragged2e}
\usepackage{etex}
%\usepackage{yfonts}
\usepackage{marvosym}
\usepackage[notextcomp]{kpfonts}
\usepackage{nicefrac}
\newcommand*{\QED}{\hfill \footnotesize {\sc Q.e.d.}}
\usepackage{floatrow}



\usepackage[textsize=scriptsize, textwidth = 1.5cm]{todonotes}
%\linespread{1.5}


\setlength{\parindent}{10pt}
\setlength{\parskip}{1pt}


%language
\usepackage{times}
\usepackage{t1enc}
%\usepackage[utf8x]{inputenc}
%\usepackage[polish]{babel}
%\usepackage{polski}

\usepackage{mathptmx}
\usepackage[scaled=0.88]{helvet}


%AMS
\usepackage{amsfonts}
\usepackage{amssymb}
\usepackage{amsthm}
\usepackage{amsmath}
\usepackage{mathtools}

\usepackage{geometry}
 \geometry{a4paper,left=20mm,top=15mm,bottom = 20mm, right = 20mm}


%environments
\newtheorem{fact}{Fact}



%abbreviations
\newcommand{\ra}{\rangle}
\newcommand{\la}{\langle}
\newcommand{\n}{\neg}
\newcommand{\et}{\wedge}
\newcommand{\jt}{\rightarrow}
\newcommand{\ko}[1]{\forall  #1\,}
\newcommand{\ro}{\leftrightarrow}
\newcommand{\exi}[1]{\exists\, {_{#1}}}
\newcommand{\pr}[1]{\mathsf{P}(#1)}
\newcommand{\cost}{\mathsf{cost}}


\newcommand{\odds}{\mathsf{Odds}}
\newcommand{\ind}{\mathsf{Ind}}
\newcommand{\nf}[2]{\nicefrac{#1\,}{#2}}
\newcommand{\R}[1]{\texttt{#1}}
\newcommand{\prr}[1]{\mbox{$\mathtt{P}_{prior}(#1)$}}
\newcommand{\prp}[1]{\mbox{$\mathtt{P}_{posterior}(#1)$}}



\newtheorem{q}{\color{blue}Question}
\newtheorem{lemma}{Lemma}
\newtheorem{theorem}{Theorem}



%technical intermezzo
%---------------------

\newcommand{\intermezzoa}{
	\begin{minipage}[c]{13cm}
	\begin{center}\rule{10cm}{0.4pt}



	\tiny{\sc Optional Content Starts}
	
	\vspace{-1mm}
	
	\rule{10cm}{0.4pt}\end{center}
	\end{minipage}\nopagebreak 
	}


\newcommand{\intermezzob}{\nopagebreak 
	\begin{minipage}[c]{13cm}
	\begin{center}\rule{10cm}{0.4pt}

	\tiny{\sc Optional Content Ends}
	
	\vspace{-1mm}
	
	\rule{10cm}{0.4pt}\end{center}
	\end{minipage}
	}
%--------------------

\DeclareUnicodeCharacter{0301}{*************************************}




















\newtheorem*{reply*}{Reply}
\usepackage{enumitem}
\newcommand{\question}[1]{\begin{enumerate}[resume,leftmargin=0cm,labelsep=0cm,align=left]
\item #1
\end{enumerate}}

\usepackage{float}

% \setbeamertemplate{blocks}[rounded][shadow=true]
% \setbeamertemplate{itemize items}[ball]
% \AtBeginPart{}
% \AtBeginSection{}
% \AtBeginSubsection{}
% \AtBeginSubsubsection{}
% \setlength{\emergencystretch}{0em}
% \setlength{\parskip}{0pt}






\usepackage[authoryear]{natbib}

%\bibliographystyle{apalike}

\author{}
\date{\vspace{-2.5em}}

\begin{document}

\begin{center}
\large {\sc \textbf{Rethinking Legal Probabilism}}

\large Rafa\l\, Urbaniak
\end{center}

\vspace{1mm}

\thispagestyle{empty}

\noindent \large \textbf{1. Scientific goal} \normalsize

The \textbf{goal of this project} is to
\textbf{develop  a probabilistic  modelling method of handling the multiplicity of items of evidence,  hypotheses and theories of what happened, and the resulting decisions in the court of law}.
In light of the current criticism of the probabilistic approach to such
issues, such a method should (1) be
\textbf{sensitive to the argumentative structure} involved, and (2)
capture the idea that in a legal context we are dealing with a
\textbf{class of competing narrations}.

The point of departure is to represent narrations as bayesian networks
enriched with additional layer of information as to which nodes
correspond to evidence, and which are binary narration nodes.
\textbf{The key idea is that with such bayesian networks as building material, various features requested by the critics (such as coherence, resiliency, missing evidence,  explaining evidence, or ways to handle a multiplicity of proposed narrations) can be explicated in terms of corresponding properties of,  operations on, and relations between bayesian networks.}

The output will be a
\textbf{unifying extended probabilistic model embracing key aspects of the narrative and argumentative approaches, with  implementation in the programming language \textbf{\textsf{R}}.}
What the project will uniquely bring to the table is joining the
familiarity with epistemological debates, familiarity with the details
of evidence assessment in legal cases and technical skill to
programmatically implement, simulate and test various theoretical moves.

\vspace{2mm}

\noindent \large \textbf{2. Significance} \normalsize

\noindent \large \textbf{2.1. State of the art.}
\normalsize \textbf{Legal probabilism (LP)} comprises two core tenets:
(1) the evidence presented at trial can be assessed, weighed and
combined by means of probability theory; and (2) legal decision rules,
such as proof beyond a reasonable doubt in criminal cases, can be
explicated in probabilistic terms {[}21,35,45{]}. Skepticism about wider
mathematical and quantitative models of legal evidence is still
widespread among prominent legal scholars and practitioners, partially
in light of conceptual difficulties with probabilitistic decision
criteria and arising puzzles of naked statistical evidence
{[}1,7,33,34,40,50{]}. Relatedly, various informal notions have been
claimed to be essential for a proper explication of judiciary decision
standards {[}17,27,30,48,58{]}. Recently, the No Plausible Alternative
Story theory (NPAS) {[}2--4,4,6,42--44{]}, and argumentation theory
{[}5,29,57{]} suggest that LP cannot capture the multiple-scenario based
and argumentatively structured nature of legal evidence evaluation.
However, LP has been defended against this objection by developing
appropriate Bayesian Networks methods.

A Bayesian network comprises a directed acyclic graph of relations of
dependence between variables and conditional probability tables
corresponding to these relations. Simple graphical patterns (called
\emph{idioms}) often appear while modeling the relationships between
evidence and hypotheses. Complex graphical models can be constructed by
combining these in a modular way {[}15,20,31,38,49{]}. BNs have been
used to reconstruct legal cases {[}18,32{]}, and to develop a general
approach {[}55{]}. Once all the pieces of evidence and claims are
represented as nodes, one should use the \textbf{scenario idiom}. A BN
that uses the scenario idiom would consist of the nodes for the states
and events in the scenario (with each node linked to the supporting
evidence), a separate scenario node that has all states and events as
its children, and a node corresponding to the ultimate hypothesis as a
child of the scenario node. The scenario node in a sense unifies the
different events and states: changing the probability of one part of the
scenario will also usually change the probability of the other parts.
This strategy is supposed to make sense of the notion of the
\textbf{coherence of a scenario} as different from its probability given
the evidence. On this approach {[}53--56{]}, coherence is identified
with the prior probability of the scenario node. The approach is also
supposed to model reasoning with multiple scenarios. Given a class of
narrations, all the nodes used in some of the separate BNs that
correspond to them are to be used to build one large BN, and all
separate scenario nodes are to be included in the final large BN. An
alternative approach to using BNs in legal context {[}39{]} starts with
a criticizm a higher-order probabilistic approach to narrations I have
developed in {[}51{]}. My approach so far has no connection to BNs and
so it ``fails to offer a convincing and operational means to structure
and compare competing narratives.'' The authors of {[}39{]} represent
separate narrations in terms of separate BNs, and deploy
\textbf{bayesian model comparison and averaging} as a tool for reasoning
with multiple scenarios.

\vspace{1mm}

\noindent \large \textbf{2.2. Pioneering nature of the project.}
\normalsize Here are the key reasons why I am convinced
\textbf{the scenario node approach is not satisfactory}. (A) Adding a
parent node by \emph{fiat} without any good reasons to think the nodes
are connected other than being a part of a single story, introduces
probabilistic dependencies between the elements of a narration. (B)
Another problem is the identification of prior probability with
coherence. This does not add up intuitively: there are coherent but
unlikely stories. (C) In general, the legal probabilistic approach to
coherence is very simple and fails to engage with rich philosophical
literature exactly on this topic {[}14,22,23,28,37,41,47{]}. (D) The
merging procedure with scenario nodes assumes that for the nodes that
are common to the networks to be merged, both the directions of the
arrows in the DAGs and the conditional probability tables are the same
across different narrations. This is unrealistic.

Here are the key
\textbf{limitations of the model selection and averaging approach}. (E)
The use equal priors is highly debatable. This approach would render
prior probabilities quite sensitive to the choice of hypotheses and thus
potentially arbitrary. Also, this approach seems particularly unsuitable
for criminal cases. If the only two hypotheses/models on the table
ultimately say ``the defendant is guilty'' and ``the defendant is
innocent'', the prior probability of each would be 50\%. But defendants
in criminal cases, however, should be presumed innocent until proven
guilty. A 50\% prior probability of guilt seems excessive. Some take the
presumption to mean that the prior probability of guilt should be set to
a small value {[}24,25{]}, but it is not clear whether this
interpretation can be justified on epistemological or decision-theoretic
grounds. (F) More recent models rely on relevant background information
about people's opportunities to commit crimes {[}19{]}. But even if
these models are successful in giving well-informed assessments of prior
probabilities any evidence-based assessment of prior probabilities, they
are likely to violate existing normative requirements of the trial
system {[}8,16,46{]}. People who belong to certain demographic groups
will be regarded as having a higher prior probability of committing a
wrong than others, and this outcome can be seen as unfair {[}9{]}. (G)
Model selection based on likelihood (given equal priors) or posterior
model probabilities in general (if priors are not assumed to be equal)
boils down to a variant of the threshold view, and so all the
difficulties with the threshold view apply. (H) There is a rich
literature on the difficulties that linear pooling runs into
{[}11--13{]}. One problem is that the method satisfies the unanimity
assumption: whenever all models share a degree of belief in a claim,
this is exactly the output degree for that belief. Another problem is
that linear pooling does not preserve probabilistic independence
{[}36{]}: even if all models agree that certain nodes are independent,
they might end up being dependent in the output. There is also a variety
of impossibility theorems in the neighborhood {[}26{]}.

\vspace{1mm}

\noindent \textbf{The key elements in my planned work are as follows.}
\textbf{(1) Representation.} Use BNs taken separately without scenario
nodes to represent various narrations. without assuming the conditional
probability tables or directions of edges are the same across the BNs,
adding another layer of information, dividing nodes into evidence and
narration nodes. \textbf{(2) Dynamic BNs.} Averaging does not seem to be
the right way to model cross-examination. To take the argumentative
approach seriously and to be able to model relations such as
``undercutting'' and ``rebutting'' I will consider another dimension:
BNs changing through time in light of other BNs.
\textbf{(3) BN-based coherence.} with Alicja Kowalewska I have
formulated a BN-based coherence measure that is not a function of a
probabilistic measure and a set of propositions alone, because it is
also sensitive to the selection and direction of arrows in a Bayesian
Network representing an agent's credal state. Now, it needs to be
deployed (implemented in \textbf{\textsf{R}} for BNs) and properly
tested on real-life cases discussed in the LP literature.
\textbf{(4) Divide and conquer.} I propose that ensemble methods should
be deployed for multiple narration variants available from one side (as
in when, say, the prosecution story comes with uncertainty about the
direction of an arrow or about a particular probability table), but
selection methods should be used when final decision is to be made
between narrations proposed by the opposing sides.
\textbf{(5) Ensemble methods.} One question that arises is whether the
general concerns about linear pooling arise for such limited
applications. If not, the remaining concern is what priors should be
used. In light of the controversial nature of equal priors, I plan to
study the consequences of rescaling coherence scores (already mentioned)
to constitute model priors. The idea is that given that narrations are
to be developed by the sides themselves, taking coherence of their
narration as determining the prior might be more fair than using equal
priors or relying on geographical or population statistics. If yes,
perhaps some other methods boiling down to a variant of sensitivity
analysis can be deployed: look at all BNs corresponding to some variant
of the narration of one of the sides, find the ones that give strongest
(and the weakest) support to the final conclusion, and these give you a
range of possible outcomes. \textbf{(6) Selection criteria.} The
so-called New Legal Probabilism (NLP) is an attempt to improve on the
underspecificity of NPAS {[}10{]}. While still being at most
semi-formal, the approach is more specific about the conditions that a
successful accusing narration is to satisfy for the conviction beyond
reasonable doubt to be justified. Di Bello works out the philosophical
details of requirements such as evidential support and completeness,
resiliency, and narrativity. It is far from obvious that such conditions
are susceptible to a Bayesian networks explication. I plan to rely on a
more expressive probabilistic framework {[}51,52{]}, that is capable of
expressing such features within a formalized higher-order language. The
key hypothesis is that they can be recast in terms of properties of BNs
and that the existing BN programming tools can be extended to implement
testing for these criteria. This will make them susceptible to
programmatic implementation and further study by means of computational
methods. The hope is that on one hand, they will do better than the
existing proposals, and where they fail, further insights can be gained
by studying the reasons behind such failures.

\vspace{1mm}

\noindent \large \textbf{3. Work plan}

\vspace{1mm} \normalsize

Throughout the whole project I plan to cooperate with Marcello Di Bello
(Arizona State University). Over the last year we co-authored the
Stanford Encyclopedia of Philosophy entry on Legal Probabilism. We
decided to continue our fruitful cooperation. For over two months we
have been working on a book proposal to be submitted to Oxford
University Press exactly on the issues to be studied in this research
project. Marcello Di Bello is an excellent philsopher with extensive
research experience in the philosophy of legal evidence, and he would
bring his expertise to the table when working both on the book and on
the papers, whereas I would be focused on the technical aspects and the
underlying formal philosophy. During the last year of the research
\textbf{I plan a six months' stay at Arizona State University}, to work
in person with Di Bello on finalizing the book that presents the results
of the project.

Apart from publications, the results will be presented at various
conferences devoted to legal reasoning. These include the yearly
conferences of the
\emph{International Association for Artificial Intelligence and Law} and
of the \emph{Foundation for Legal Knowledge Based Systems (JURIX)}, and
more general conferences gathering formal philosophers, so that the
research is inspired by interaction not only with legal evidence
scholars, computer scientists, but also philosophers. I am also already
an invited speaker at the upcoming ``Probability and Proof'' conference
that will be part of an international conference on the philosophy of
legal evidence (``The Michele Taruffo Girona Evidence Week'') in Girona
(Spain) May 23-27 2022. See the table at the end for the planned stages
and output of research.

\noindent \large \textbf{4. Methodology}

\vspace{1mm} \normalsize

I will be using four methods: (a) informal conceptual analysis; (b)
formal conceptual analysis; (c) computational metods (R simulations,
etc.); and (d) case studies. I will rethink and model the existing
impliementations of whole-case-scale-BNs in legal evidence from the
perspective of the new framework and reconstruct cases which extensively
use probabilistic reasoning, but for which BNs have not been yet
proposed. Both the literature already listed and many textbooks on
quantitative evidence in forensics are great sources of such cases.

What the project will uniquely bring to the table is joining the
familiarity with epistemological debates on the nature of coherence
(which legal probabilists ignore or are unaware of), familiarity with
the details of evidence assessment in legal cases (which formal
epistemologists seem to be unaware of) and technical skill to
programmatically implement, simulate and test various theoretical moves.

A larger initiative involving reconstructing various cases using
different representation methods and comparing the representations,
called \emph{Probability and statistics in forensic science}, took place
at the Isaac Newton Institute for Mathematical Sciences. My approach
will be in the same vein.

I have extensive experience in analytic philosophy, conceptual analysis,
philosophical logic and probabilistic and decision-theoretic methods as
deployed in philosophical contexts. I also have teaching and publishing
record that involves statistical programming in the \textsf{\textbf{R}}
language (my sample programming projects can be visited at
\url{https://rfl-urbaniak.github.io/menu/projects.html}), and so am also
competent to develop AI implementations and tests of the ideas to be
developed.

One research risk is that it will turn out that some of the informal
requirements cannot be spelled out in probabilistic terms, or expressed
in terms of properties of Bayesian networks. In such an event, I will
study the reasons for this negative result. It might happen that there
are independent reasons to abandon a given condition, or it might be the
case that probabilistic inexplicability of a given condition is an
argument against the probabilistic approach. Either way, finding out
which option holds and why would also lead to a deeper understanding of
the framework and lead to a publication in academic journals. Another
research risk is that the case studies will show that other methods are
more efficient or transparent. This would itself constitute a result
that could be used to further modify the framework so that its best
aspects could be preserved, while the disadvantages discovered during
case studies avoided.

\vspace{1mm}

\begin{center}
\begin{tabular}{p{1.6cm}|p{14.8cm}}
\footnotesize \textbf{$\mathtt{Stage  \,\, 1}$} \newline  \tiny Philosophical \&  formal   unification & 
Obtain a unifying extended  probabilistic framework by incorporating further insights  from philosophical and psychological accounts of legal narrations, and from the argumentation approach. Defend its philosophical plausibility. \scriptsize (6 months)
\\ 
& \footnotesize A philosophical paper published in an  academic journal such as \emph{Synthese}, \emph{Mind} or \emph{Ratio Juris}. Working title: \emph{Why care about narration selection principles?}\\ \hline
\footnotesize \textbf{$\mathtt{Stage \,\, 2}$} \newline  \tiny AI implementation 
 & Develop Bayesian Network Methods for the obtained formal framework, so that the insights from the argumentation approach and informal epistemology, mediated through it, can be incorporated in AI tools. \scriptsize (12 months)
\\ & \footnotesize A   technical paper  published in a journal such as \emph{IfCoLog Journal of Logics and their Applications}, \emph{Law, Probability and Risk} or \emph{Artificial Intelligence and Law}. Working title: \emph{Implementation of narration assessment criteria in Bayesian Networks with \textbf{\textsf{R}}}.
\\ \hline
\footnotesize \textbf{$\mathtt{Stage  \,\, 3}$} \newline  \tiny    Case studies & 
Evaluate the developed framework and AI tools  by conducting case studies from its perspective.    \scriptsize (6 months)  \\
& \footnotesize One paper  on  how the formal framework handles case studies and  one paper  on how the developed AI tools handle real-life situations in   journals such as \emph{Artificial Intelligence} or \emph{Argument \& Computation}. Working titles: \emph{Rethinking the famous BN-modeled cases within the narration framework} and \emph{BNarr, an \textbf{\textsf{R}} package to model narrations with Bayesian Networks}.
\\ \hline
\footnotesize \textbf{$\mathtt{Stage  \,\, 4}$} \newline  \tiny    Back to challenges  & 
Investigate the extent to which the new framework helps to handle the issues raised in points A-H. \scriptsize (12 months) \\ 
& \footnotesize Completion of the planned book.
\end{tabular}
\end{center}

\vspace{-2mm}

\section*{References}\label{references}
\addcontentsline{toc}{section}{References}

\vspace{-4mm}

\footnotesize 

\hypertarget{refs}{}
\hypertarget{ref-allen2007problematic}{}
{[}1{]} Ronald Allen and Michael Pardo. 2007. The problematic value of
mathematical models of evidence. \emph{The Journal of Legal Studies} 36,
1 (2007), 107--140. DOI:\url{https://doi.org/10.1086/508269}

\hypertarget{ref-allen2001naturalized}{}
{[}2{]} Ronald J Allen and Brian Leiter. 2001. Naturalized epistemology
and the law of evidence. \emph{Virginia Law Review} 87, 8 (2001),
1491--1550.

\hypertarget{ref-Allen1986A-Reconceptuali}{}
{[}3{]} Ronald J. Allen. 1986. A reconceptualization of civil trials.
\emph{Boston University Law Review} 66, (1986), 401--437.

\hypertarget{ref-Allen2010No-Plausible-Al}{}
{[}4{]} Ronald J. Allen. 2010. No plausible alternative to a plausible
story of guilt as the rule of decision in criminal cases. In
\emph{Prueba y esandares de prueba en el derecho}, Juan Cruz and Larry
Laudan (eds.). Instituto de Investigaciones Filosoficas-UNAM.

\hypertarget{ref-bex2011ArgumentsStoriesCriminal}{}
{[}5{]} Floris J. Bex. 2011. \emph{Arguments, stories and criminal
evidence: A formal hybrid theory}. Springer, Dordrecht ; New York.

\hypertarget{ref-clermont2015TrialTraditionalProbability}{}
{[}6{]} Kevin M. Clermont. 2015. Trial by Traditional Probability,
Relative Plausibility, or Belief Function? \emph{Case Western Reserve
Law Review} 66, 2 (2015), 353--391.

\hypertarget{ref-Cohen81}{}
{[}7{]} Jonathan L. Cohen. 1981. Subjective probability and the paradox
of the Gatecrasher. \emph{Arizona State Law Journal} (1981), 627--634.

\hypertarget{ref-dahlman2017}{}
{[}8{]} Christian Dahlman. 2017. Determining the base rate for guilt.
\emph{Law, Probability and Risk} 17, 1 (2017), 15--28.

\hypertarget{ref-DiBelloONeil2020}{}
{[}9{]} Marcello Di Bello and Collin O'Neil. 2020. Profile evidence,
fairness and the risk of mistaken convictions. \emph{Ethics} 130, 2
(2020), 147--178.

\hypertarget{ref-di2013statistics}{}
{[}10{]} Marcello Di Bello. 2013. Statistics and probability in criminal
trials. PhD thesis. University of Stanford.

\hypertarget{ref-Dietrich2016Probabilistic}{}
{[}11{]} F. Dietrich and C. List. 2016. Probabilistic opinion pooling.
In \emph{Oxford handbook of philosophy and probability}, A. Hajek and C.
Hitchcock (eds.). Oxford University Press.

\hypertarget{ref-dietrich2017probabilistic1}{}
{[}12{]} Franz Dietrich and Christian List. 2017. Probabilistic opinion
pooling generalized. part one: General agendas. \emph{Social Choice and
Welfare} 48, 4 (2017), 747--786.

\hypertarget{ref-dietrich2017probabilistic2}{}
{[}13{]} Franz Dietrich and Christian List. 2017. Probabilistic opinion
pooling generalized. part two: The premise-based approach. \emph{Social
Choice and Welfare} 48, 4 (2017), 787--814.

\hypertarget{ref-Douven2007measuring}{}
{[}14{]} Igor Douven and Wouter Meijs. 2007. Measuring coherence.
\emph{Synthese} 156, 3 (April 2007), 405--425.
DOI:\url{https://doi.org/10.1007/s11229-006-9131-z}

\hypertarget{ref-Edwards1991Influence-diagr}{}
{[}15{]} W. Edwards. 1991. Influence diagrams, bayesian imperialism, and
the collins case: An appeal to reason. \emph{Cardozo Law Review} 13,
(1991), 1025--1074.

\hypertarget{ref-engel2012NeglectBaseRate}{}
{[}16{]} Christoph Engel. 2012. Neglect the Base Rate: It's the Law!
\emph{Preprints of the Max Planck Institute for Research on Collective
Goods} 23, (2012). DOI:\url{https://doi.org/10.2139/ssrn.2192423}

\hypertarget{ref-enoch2015sense}{}
{[}17{]} David Enoch and Talia Fisher. 2015. Sense and sensitivity:
Epistemic and instrumental approaches to statistical evidence.
\emph{Stan. L. Rev.} 67, (2015), 557--611.

\hypertarget{ref-Fenton2018Risk}{}
{[}18{]} Norman Fenton and Martin Neil. 2018. \emph{Risk assessment and
decision analysis with bayesian networks}. Chapman; Hall.

\hypertarget{ref-fenton2019OpportunityPriorProofbased}{}
{[}19{]} Norman Fenton, David Lagnado, Christian Dahlman, and Martin
Neil. 2019. The opportunity prior: A proof-based prior for criminal
cases. \emph{Law, Probability and Risk} (May 2019), {[}online first{]}.
DOI:\url{https://doi.org/10.1093/lpr/mgz007}

\hypertarget{ref-fenton2013GeneralStructureLegal}{}
{[}20{]} Norman Fenton, Martin Neil, and David A. Lagnado. 2013. A
General Structure for Legal Arguments About Evidence Using Bayesian
Networks. \emph{Cognitive Science} 37, 1 (January 2013), 61--102.
DOI:\url{https://doi.org/10.1111/cogs.12004}

\hypertarget{ref-Finkelstein1970A}{}
{[}21{]} Michael O. Finkelstein and William B. Fairley. 1970. A Bayesian
approach to identification evidence. \emph{Harvard Law Review} 83, 3
(1970), 489--517.

\hypertarget{ref-fitelson2003ProbabilisticTheoryCoherence}{}
{[}22{]} Branden Fitelson. 2003. A Probabilistic Theory of Coherence.
\emph{Analysis} 63, 3 (2003), 194--199.

\hypertarget{ref-fitelson2003comments}{}
{[}23{]} Branden Fitelson. 2003. Comments on jim franklin's the
representation of context: Ideas from artificial intelligence (or, more
remarks on the contextuality of probability). \emph{Law, Probability and
Risk} 2, 3 (2003), 201--204.

\hypertarget{ref-Friedman2000}{}
{[}24{]} Richard D. Friedman. 2000. A presumption of innocence, not of
even odds. \emph{Stanford Law Review} 52, 4 (2000), 873--887.

\hypertarget{ref-friedmanEtAl1995}{}
{[}25{]} Richard D. Friedman, Ronald J. Allen, David J. Balding, Peter
Donnelly, and David H. Kaye. 1995. Probability and proof in State v.
Skipper: An internet exchange. \emph{Jurimetrics} 35, 3 (1995),
277--310.

\hypertarget{ref-Gallow2018No}{}
{[}26{]} J. Gallow. 2018. No one can serve two epistemic masters.
\emph{Philosophical Studies} 175, 10 (2018), 2389--2398.
DOI:\url{https://doi.org/10.1007/s11098-017-0964-8}

\hypertarget{ref-gardiner2018}{}
{[}27{]} Georgi Gardiner. 2018. Legal burdens of proof and statistical
evidence. In \emph{Routledge handbook of applied epistemology}, David
Coady and James Chase (eds.). Routledge.

\hypertarget{ref-glass2002}{}
{[}28{]} David H. Glass. 2002. Coherence, Explanation, and Bayesian
Networks. In \emph{Artificial Intelligence and Cognitive Science}, G.
Goos, J. Hartmanis, J. van Leeuwen, Michael O'Neill, Richard F. E.
Sutcliffe, Conor Ryan, Malachy Eaton and Niall J. L. Griffith (eds.).
Springer Berlin Heidelberg, Berlin, Heidelberg, 177--182.
DOI:\url{https://doi.org/10.1007/3-540-45750-X_23}

\hypertarget{ref-gordon2007}{}
{[}29{]} Thomas F. Gordon, Henry Prakken, and Douglas Walton. 2007. The
Carneades model of argument and burden of proof. \emph{Artificial
Intelligence} 171, 10-15 (2007), 875--896.

\hypertarget{ref-haack2011legal}{}
{[}30{]} Susan Haack. 2014. Legal probabilism: An epistemological
dissent. In \emph{Haack2014-HAAEMS}. 47--77.

\hypertarget{ref-hepler2007ObjectorientedGraphicalRepresentations}{}
{[}31{]} A. B. Hepler, A. P. Dawid, and V. Leucari. 2007.
Object-oriented graphical representations of complex patterns of
evidence. \emph{Law, Probability and Risk} 6, 1-4 (October 2007),
275--293. DOI:\url{https://doi.org/10.1093/lpr/mgm005}

\hypertarget{ref-kadane2011probabilistic}{}
{[}32{]} Joseph B Kadane and David A Schum. 2011. \emph{A probabilistic
analysis of the sacco and vanzetti evidence}. John Wiley \& Sons.

\hypertarget{ref-kaye79}{}
{[}33{]} David H. Kaye. 1979. The laws of probability and the law of the
land. \emph{The University of Chicago Law Review} 47, 1 (1979), 34--56.

\hypertarget{ref-laudan2006truth}{}
{[}34{]} Larry Laudan. 2006. \emph{Truth, error, and criminal law: An
essay in legal epistemology}. Cambridge University Press.

\hypertarget{ref-Lempert1986}{}
{[}35{]} Richard O. Lempert. 1986. The new evidence scholarship:
Analysing the process of proof. \emph{Boston University Law Review} 66,
(1986), 439--477.

\hypertarget{ref-List2011Group}{}
{[}36{]} Christian List and Philip Pettit. 2011. \emph{Group agency: The
possibility, design, and status of corporate agents}. Oxford University
Press.

\hypertarget{ref-meijs2007}{}
{[}37{]} Wouter Meijs and Igor Douven. 2007. On the alleged
impossibility of coherence. \emph{Synthese} 157, 3 (July 2007),
347--360. DOI:\url{https://doi.org/10.1007/s11229-006-9060-x}

\hypertarget{ref-neil2000BuildingLargescaleBayesian}{}
{[}38{]} Martin Neil, Norman Fenton, and Lars Nielson. 2000. Building
large-scale Bayesian Networks. \emph{The Knowledge Engineering Review}
15, 3 (2000), 257--284.
DOI:\url{https://doi.org/10.1017/S0269888900003039}

\hypertarget{ref-Fenton2019Modelling}{}
{[}39{]} Martin Neil, Norman Fenton, David Lagnado, and Richard David
Gill. 2019. Modelling competing legal arguments using bayesian model
comparison and averaging. \emph{Artificial Intelligence and Law} (March
2019). DOI:\url{https://doi.org/10.1007/s10506-019-09250-3}

\hypertarget{ref-Nesson1979Reasonable-doub}{}
{[}40{]} Charles R. Nesson. 1979. Reasonable doubt and permissive
inferences: The value of complexity. \emph{Harvard Law Review} 92, 6
(1979), 1187--1225. DOI:\url{https://doi.org/10.2307/1340444}

\hypertarget{ref-olsson2001}{}
{[}41{]} Erik J. Olsson. 2001. Why Coherence Is Not Truth-Conducive.
\emph{Analysis} 61, 3 (2001), 236--241.

\hypertarget{ref-pardo2018}{}
{[}42{]} Michael S. Pardo. 2018. Safety vs.~Sensitivity: Possible worlds
and the law of evidence. \emph{Legal Theory} 24, 1 (2018), 50--75.

\hypertarget{ref-Pardo2008judicial}{}
{[}43{]} Michael S. Pardo and Ronald J. Allen. 2008. Judicial proof and
the best explanation. \emph{Law and Philosophy} 27, 3 (2008), 223--268.

\hypertarget{ref-Pennington1991}{}
{[}44{]} Nancy Pennington and Reid Hastie. 1991. A cognitive theory of
juror decision making: The story model. \emph{Cardozo Law Review} 13,
(1991), 519--557.

\hypertarget{ref-Posner1973}{}
{[}45{]} Richard Posner. 1973. \emph{The economic analysis of law}.
Brown \& Company.

\hypertarget{ref-schweizer2013LawDoesnSay}{}
{[}46{]} Mark Schweizer. 2013. The Law Doesn't Say Much About Base
Rates. \emph{SSRN Electronic Journal} (2013).
DOI:\url{https://doi.org/10.2139/ssrn.2329387}

\hypertarget{ref-shogenji1999}{}
{[}47{]} Tomoji Shogenji. 1999. Is Coherence Truth Conducive?
\emph{Analysis} 59, 4 (1999), 338--345.

\hypertarget{ref-Smith_conviction_mind_2017}{}
{[}48{]} Martin Smith. 2017. When does evidence suffice for conviction?
\emph{Mind} (2017). DOI:\url{https://doi.org/10.1093/mind/fzx026}

\hypertarget{ref-taroni2006bayesian}{}
{[}49{]} Franco Taroni, Alex Biedermann, Silvia Bozza, Paolo Garbolino,
and Colin Aitken. 2014. \emph{Bayesian networks for probabilistic
inference and decision analysis in forensic science} (2nd ed.). John
Wiley \& Sons.

\hypertarget{ref-Thomson86}{}
{[}50{]} Judith J. Thomson. 1986. Liability and individualized evidence.
\emph{Law and Contemporary Problems} 49, 3 (1986), 199--219.

\hypertarget{ref-urbaniak2018narration}{}
{[}51{]} Rafal Urbaniak. 2018. Narration in judiciary fact-finding: A
probabilistic explication. \emph{Artificial Intelligence and Law}
(2018), 1--32. DOI:\url{https://doi.org/10.1007/s10506-018-9219-z}

\hypertarget{ref-Urbaniak2017Narration-in-ju}{}
{[}52{]} Rafal Urbaniak. 2018. Narration in judiciary fact-finding: A
probabilistic explication. \emph{Artificial Intelligence and Law}
(2018), 1--32. DOI:\url{https://doi.org/10.1007/s10506-018-9219-z}

\hypertarget{ref-vlek2015}{}
{[}53{]} Charlotte S. Vlek, Henry Prakken, Silja Renooij, and and Bart
Verheij. 2015. Representing the quality of crime scenarios in a bayesian
network. In \emph{Legal knowledge and information systems}, IOS Press,
133--140.

\hypertarget{ref-vlek2013modeling}{}
{[}54{]} Charlotte Vlek, Henry Prakken, Silja Renooij, and Bart Verheij.
2013. Modeling crime scenarios in a bayesian network. In
\emph{Proceedings of the fourteenth international conference on
artificial intelligence and law}, ACM, 150--159.

\hypertarget{ref-vlek2014building}{}
{[}55{]} Charlotte Vlek, Henry Prakken, Silja Renooij, and Bart Verheij.
2014. Building bayesian networks for legal evidence with narratives: A
case study evaluation. \emph{Artificial Intelligence and Law} 22,
(2014), 375--421.

\hypertarget{ref-vlek2016stories}{}
{[}56{]} Charlotte Vlek. 2016. \emph{When stories and numbers meet in
court: Constructing and explaining bayesian networks for criminal cases
with scenarios}. Rijksuniversiteit Groningen.

\hypertarget{ref-Walton2002}{}
{[}57{]} Douglas N. Walton. 2002. \emph{Legal argumentation and
evidence}. Penn State University Press.

\hypertarget{ref-wells1992naked}{}
{[}58{]} Gary L. Wells. 1992. Naked statistical evidence of liability:
Is subjective probability enough? \emph{Journal of Personality and
Social Psychology} 62, 5 (1992), 739--752.
DOI:\url{https://doi.org/10.1037/0022-3514.62.5.739}

\end{document}
