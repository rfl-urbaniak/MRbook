\documentclass[11pt,dvipsnames,enabledeprecatedfontcommands]{scrartcl}
\usepackage{lmodern}
\usepackage{amssymb,amsmath}
\usepackage{ifxetex,ifluatex}
\usepackage{fixltx2e} % provides \textsubscript
\ifnum 0\ifxetex 1\fi\ifluatex 1\fi=0 % if pdftex
  \usepackage[T1]{fontenc}
  \usepackage[utf8]{inputenc}
\else % if luatex or xelatex
  \ifxetex
    \usepackage{mathspec}
  \else
    \usepackage{fontspec}
  \fi
  \defaultfontfeatures{Ligatures=TeX,Scale=MatchLowercase}
\fi
% use upquote if available, for straight quotes in verbatim environments
\IfFileExists{upquote.sty}{\usepackage{upquote}}{}
% use microtype if available
\IfFileExists{microtype.sty}{%
\usepackage[]{microtype}
\UseMicrotypeSet[protrusion]{basicmath} % disable protrusion for tt fonts
}{}
\PassOptionsToPackage{hyphens}{url} % url is loaded by hyperref
\usepackage[unicode=true]{hyperref}
\PassOptionsToPackage{usenames,dvipsnames}{color} % color is loaded by hyperref
\hypersetup{
            pdftitle={Rethinking legal probabilism},
            pdfauthor={Rafał Urbaniak},
            colorlinks=true,
            linkcolor=Maroon,
            citecolor=Blue,
            urlcolor=blue,
            breaklinks=true}
\urlstyle{same}  % don't use monospace font for urls
\usepackage{graphicx,grffile}
\makeatletter
\def\maxwidth{\ifdim\Gin@nat@width>\linewidth\linewidth\else\Gin@nat@width\fi}
\def\maxheight{\ifdim\Gin@nat@height>\textheight\textheight\else\Gin@nat@height\fi}
\makeatother
% Scale images if necessary, so that they will not overflow the page
% margins by default, and it is still possible to overwrite the defaults
% using explicit options in \includegraphics[width, height, ...]{}
\setkeys{Gin}{width=\maxwidth,height=\maxheight,keepaspectratio}
\IfFileExists{parskip.sty}{%
\usepackage{parskip}
}{% else
\setlength{\parindent}{0pt}
\setlength{\parskip}{6pt plus 2pt minus 1pt}
}
\setlength{\emergencystretch}{3em}  % prevent overfull lines
\providecommand{\tightlist}{%
  \setlength{\itemsep}{0pt}\setlength{\parskip}{0pt}}
\setcounter{secnumdepth}{5}
% Redefines (sub)paragraphs to behave more like sections
\ifx\paragraph\undefined\else
\let\oldparagraph\paragraph
\renewcommand{\paragraph}[1]{\oldparagraph{#1}\mbox{}}
\fi
\ifx\subparagraph\undefined\else
\let\oldsubparagraph\subparagraph
\renewcommand{\subparagraph}[1]{\oldsubparagraph{#1}\mbox{}}
\fi

% set default figure placement to htbp
\makeatletter
\def\fps@figure{htbp}
\makeatother

%\documentclass{article}

% %packages
 \usepackage{booktabs}

\usepackage{multirow}

\usepackage{graphicx}
\usepackage{longtable}
\usepackage{ragged2e}
\usepackage{etex}
%\usepackage{yfonts}
\usepackage{marvosym}
\usepackage[notextcomp]{kpfonts}
\usepackage{nicefrac}
\newcommand*{\QED}{\hfill \footnotesize {\sc Q.e.d.}}
\usepackage{floatrow}

\usepackage[textsize=footnotesize]{todonotes}
%\linespread{1.5}


\setlength{\parindent}{10pt}
\setlength{\parskip}{1pt}


%language
\usepackage{times}
\usepackage{t1enc}
%\usepackage[utf8x]{inputenc}
%\usepackage[polish]{babel}
%\usepackage{polski}

\usepackage{mathptmx}
\usepackage[scaled=0.88]{helvet}


%AMS
\usepackage{amsfonts}
\usepackage{amssymb}
\usepackage{amsthm}
\usepackage{amsmath}
\usepackage{mathtools}

\usepackage{geometry}
 \geometry{a4paper,left=20mm,top=15mm,bottom = 15mm, right = 20mm}


%environments
\newtheorem{fact}{Fact}



%abbreviations
\newcommand{\ra}{\rangle}
\newcommand{\la}{\langle}
\newcommand{\n}{\neg}
\newcommand{\et}{\wedge}
\newcommand{\jt}{\rightarrow}
\newcommand{\ko}[1]{\forall  #1\,}
\newcommand{\ro}{\leftrightarrow}
\newcommand{\exi}[1]{\exists\, {_{#1}}}
\newcommand{\pr}[1]{\mathsf{P}(#1)}
\newcommand{\cost}{\mathsf{cost}}


\newcommand{\odds}{\mathsf{Odds}}
\newcommand{\ind}{\mathsf{Ind}}
\newcommand{\nf}[2]{\nicefrac{#1\,}{#2}}
\newcommand{\R}[1]{\texttt{#1}}
\newcommand{\prr}[1]{\mbox{$\mathtt{P}_{prior}(#1)$}}
\newcommand{\prp}[1]{\mbox{$\mathtt{P}_{posterior}(#1)$}}



\newtheorem{q}{\color{blue}Question}
\newtheorem{lemma}{Lemma}
\newtheorem{theorem}{Theorem}



%technical intermezzo
%---------------------

\newcommand{\intermezzoa}{
	\begin{minipage}[c]{13cm}
	\begin{center}\rule{10cm}{0.4pt}



	\tiny{\sc Optional Content Starts}
	
	\vspace{-1mm}
	
	\rule{10cm}{0.4pt}\end{center}
	\end{minipage}\nopagebreak 
	}


\newcommand{\intermezzob}{\nopagebreak 
	\begin{minipage}[c]{13cm}
	\begin{center}\rule{10cm}{0.4pt}

	\tiny{\sc Optional Content Ends}
	
	\vspace{-1mm}
	
	\rule{10cm}{0.4pt}\end{center}
	\end{minipage}
	}
%--------------------






















\newtheorem*{reply*}{Reply}
\usepackage{enumitem}
\newcommand{\question}[1]{\begin{enumerate}[resume,leftmargin=0cm,labelsep=0cm,align=left]
\item #1
\end{enumerate}}

\usepackage{float}

% \setbeamertemplate{blocks}[rounded][shadow=true]
% \setbeamertemplate{itemize items}[ball]
% \AtBeginPart{}
% \AtBeginSection{}
% \AtBeginSubsection{}
% \AtBeginSubsubsection{}
% \setlength{\emergencystretch}{0em}
% \setlength{\parskip}{0pt}






\usepackage[authoryear]{natbib}

%\bibliographystyle{apalike}

\title{Rethinking legal probabilism}
\author{Rafał Urbaniak}
\date{}

\begin{document}
\maketitle

\tableofcontents

\thispagestyle{empty}

\section{Scientific goal}\label{scientific-goal}

\todo{define legal probabilism, talk about piecemeal evaluation vs incorporation?}

As many miscarriages of justice indicate, scientific evidence is easily
misinterpreted in court. This happens partially due to miscommunication
between the parties involved, but also because incorporating scientific
evidence in the context of a whole case can be really hard.\\
While probabilistic tools for piecemeal evaluation of scientific
evidence in legal contexts are quite well
developed,\todo{say sth about replicability crisis in forensic sciences at some point?}
the construction of a more general probabilistic model of incorporating
such evidence in a wider context of a whole case, and probabilistic
explication of legal decision standards, remain a challenge. Methods are
needed to support proper assessment of evidence in such contexts. This
project intends to contribute to further developments on such methods in
a philosophically motivated manner.

The assessment of evidence in the court of law can be viewed from at
least three perspectives: as an interplay of arguments, as an assessment
of probabilities involved, or as an interaction of competing narrations.
Each perspective presents an account of legal reasoning (Di Bello \&
Verheij, 2018; van Eemeren \& Verheij, 2017). Individually, each of
these strains has been investigated. The probablistic approach is the
most developed one, but a probabilistic approach to the incorporation of
scientific evidence in the context of a whole case is still
underdeveloped. This is partially so in light of various lines of
criticism developed by the representatives of the other strains, as such
challenges have not been satisfactorily adressed by the proponents of
legal probablism.

The \textbf{goal of this project} is to contribute to the
\textbf{development of legal probabilism} by formulating its variant
that accomodates important \textbf{insights provided by its critics}. A
crucial point of criticism is that the fact-finding process should be
conceptualized as \textbf{a competition of narrations}. I plan to
develop methods that allow the probabilist to take this perspective, and
explain how such methods allow the legal probabilist to address various
other objections present in the literature. The key idea is that once
\textbf{narrations are represented as bayesian networks, various criteria on,  features of  and operations on narrations can be explicated in terms of corresponding properties of and operations on bayesian networks}.
The conceptual developments are accompanied by technical accounts.
\textbf{\textsf{R}} code capturing to the technical features developed
is made available to the reader. Thus, the output will be a
\textbf{unifying extended probabilistic model embracing key aspects of the narrative and argumentative approaches, susceptible to AI implementation.}
The methods employed include: Bayesian statistical methods (including
Bayesian approach to higher-order probability), imprecise probabilities
and Bayesian networks.

\section{Significance}\label{significance}

(state of the art, justification for tackling a specific scientific
problem, justification for the pioneering nature of the project, the
impact of the project results on the development of the research field
and scientific discipline);

\subsection{State of the art}\label{state-of-the-art}

From among the three perspectives mentioned in the beginning, I focus on
the probablistic approach and take it as my point of departure, for
various reasons:

\begin{itemize}
\item
  The project is to be informed by and reflect on the actual practice of
  legal evidence evaluation, and much of scientific evidence in such
  contexts has probabilistic form.
\item
  Probabilistic tools are fairly well-developed both for applications
  and within formal epistemology, reaching a state of fruition which I
  think should inspire deeper reflection.
\item
  Statistical computing tools for such methods are avaialable, which
  makes programming development and preliminary evaluation of the ideas
  to be defended a viable enterprise.
\end{itemize}

One important difficulty is that there are various thought experiments
in which the probability of guilt is very high and yet conviciton or
finding of liability is intuitively unjustified---these are known as
proof paradoxes (Cohen, 1977; Redmayne, 2008). \todo{Add an example}

At least \emph{prima facie}, then, it seems that some conditions other
than high posterior probability of liability have to be satisfied for
the decision to penalize (or find liable) to be justified. Accordingly,
various informal notions have been claimed to be essential for a proper
explication of judiciary decision standards (Haack, 2014; Wells, 1992).
For instance, evidence is claimed to be insufficient for conviction if
it is not \emph{sensitive} to the issue at hand: if it remained the same
even if the accused was innocent (Enoch \& Fisher, 2015). Or, to look at
another approach, evidence is claimed to be insufficient for conviction
if it doesn't \emph{normically support} it: if---given the same
evidence---no explanation would be needed even if the accused was
innocent (Smith, 2017). A legal probabilist needs either to show that
these notions are unnecessary or inadequate for the purpose at hand, or
to indicate how they can be explicated in probabilistic terms.

\todo{describe what the probabilistic model is before we get deeper?}

Another point of criticism of the wider proabilist model, legal
proceedings are back-and-forth between opposing parties in which
cross-examination is of crucial importance, reasoning goes not only
evidence-to-hypothesis, but also hypotheses-to-evidence {[}Wells (1992);
allen2007problematic{]} in a way that seems analogous to inference to
the best explanation (Dant, 1988), which notoriously is claimed to not
be susceptible to probabilistic analysis (Lipton, 2004). An informal
philosophical account inspired by such considerations---The
\textbf{No Plausible Alternative Story (NPAS)} theory (Allen, 2010)---is
that the courtroom is a confrontation of competing narrations (Ho, 2008;
Wagenaar, Van Koppen, \& Crombag, 1993) offered by the sides, and the
narrative to be selected should be the most plausible one. The view is
conceptually plausible (Di Bello, 2013), and finds support in
psychological evidence (Pennington \& Hastie, 1991, 1992).

It would be a great advantage of legal probabilism if it could to model
phenomena captured by the narrative approach, but how is the legal
probabilist to make sense of them? From her perspective, the key
disadvantage of NPAS is that it abandons the rich toolbox of
probabilistic methods and takes the key notion of plausibility to be a
primitive notion which should be understood only intuitively.

Initial philosophical analysis of the approach has been performed, (Di
Bello, 2013) pioneering a probabilistic understanding of narrations.

\todo{add more about Marcello}

\subsection{Choice of problem}\label{choice-of-problem}

\subsection{Pioneering nature \& impact}\label{pioneering-nature-impact}

\section{Concept and work plan}\label{concept-and-work-plan}

(general work plan, specific research goals, results of preliminary
research, risk analysis);

\section{Research methodology}\label{research-methodology}

(underlying scientific methodology, methods, techniques and research
tools, methods of results analysis, equipment and devices to be used in
research);

\section*{References}\label{references}
\addcontentsline{toc}{section}{References}

\hypertarget{refs}{}
\hypertarget{ref-Allen2010No-Plausible-Al}{}
Allen, R. J. (2010). No plausible alternative to a plausible story of
guilt as the rule of decision in criminal cases. In J. Cruz \& L. Laudan
(Eds.), \emph{Prueba y esandares de prueba en el derecho}. Instituto de
Investigaciones Filosoficas-UNAM.

\hypertarget{ref-Cohen1977The-probable-an}{}
Cohen, J. (1977). \emph{The probable and the provable}. Oxford
University Press.

\hypertarget{ref-dant1988gambling}{}
Dant, M. (1988). Gambling on the truth: The use of purely statistical
evidence as a basis for civil liability. \emph{Columbia Journal of Law
and Social Problems}, \emph{22}, 31--70. HeinOnline.

\hypertarget{ref-di2013statistics}{}
Di Bello, M. (2013). \emph{Statistics and probability in criminal
trials} (PhD thesis). University of Stanford.

\hypertarget{ref-di2018evidential}{}
Di Bello, M., \& Verheij, B. (2018). Evidential reasoning. In
\emph{Handbook of legal reasoning and argumentation} (pp. 447--493).
Springer.

\hypertarget{ref-enoch2015sense}{}
Enoch, D., \& Fisher, T. (2015). Sense and sensitivity: Epistemic and
instrumental approaches to statistical evidence. \emph{Stan. L. Rev.},
\emph{67}, 557--611. HeinOnline.

\hypertarget{ref-haack2011legal}{}
Haack, S. (2014). Legal probabilism: An epistemological dissent. In
\emph{Haack2014-HAAEMS} (pp. 47--77).

\hypertarget{ref-ho2008philosophy}{}
Ho, H. L. (2008). \emph{A philosophy of evidence law: Justice in the
search for truth}. Oxford University Press.

\hypertarget{ref-Lipton2004-LIPITT}{}
Lipton, P. (2004). \emph{Inference to the best explanation}.
Routledge/Taylor; Francis Group.

\hypertarget{ref-pennington1991cognitive}{}
Pennington, N., \& Hastie, R. (1991). A cognitive theory of juror
decision making: The story model. \emph{Cardozo Law Review}, \emph{13},
519--557. HeinOnline.

\hypertarget{ref-pennington1992explaining}{}
Pennington, N., \& Hastie, R. (1992). Explaining the evidence: Tests of
the story model for juror decision making. \emph{Journal of personality
and social psychology}, \emph{62}(2), 189--204. American Psychological
Association.

\hypertarget{ref-redmayne2008exploring}{}
Redmayne, M. (2008). Exploring the proof paradoxes. \emph{Legal Theory},
\emph{14}(4), 281--309. Cambridge University Press.

\hypertarget{ref-Smith_conviction_mind_2017}{}
Smith, M. (2017). When does evidence suffice for conviction?
\emph{Mind}.

\hypertarget{ref-vanEemeren2017}{}
van Eemeren, F., \& Verheij, B. (2017). Argumentation theory in formal
and computational perspective. \emph{IFCoLog Journal of Logics and Their
Applications}, \emph{4}(8), 2099--2181.

\hypertarget{ref-wagenaar1993anchored}{}
Wagenaar, W., Van Koppen, P., \& Crombag, H. (1993). \emph{Anchored
narratives: The psychology of criminal evidence.} St Martin's Press.

\hypertarget{ref-wells1992naked}{}
Wells, G. (1992). Naked statistical evidence of liability: Is subjective
probability enough? \emph{Journal of Personality and Social Psychology},
\emph{62}(5), 739--752. American Psychological Association.

\end{document}
