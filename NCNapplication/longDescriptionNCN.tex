\documentclass[11pt,dvipsnames,enabledeprecatedfontcommands]{scrartcl}
\usepackage{lmodern}
\usepackage{amssymb,amsmath}
\usepackage{ifxetex,ifluatex}
\usepackage{fixltx2e} % provides \textsubscript
\ifnum 0\ifxetex 1\fi\ifluatex 1\fi=0 % if pdftex
  \usepackage[T1]{fontenc}
  \usepackage[utf8]{inputenc}
\else % if luatex or xelatex
  \ifxetex
    \usepackage{mathspec}
  \else
    \usepackage{fontspec}
  \fi
  \defaultfontfeatures{Ligatures=TeX,Scale=MatchLowercase}
\fi
% use upquote if available, for straight quotes in verbatim environments
\IfFileExists{upquote.sty}{\usepackage{upquote}}{}
% use microtype if available
\IfFileExists{microtype.sty}{%
\usepackage[]{microtype}
\UseMicrotypeSet[protrusion]{basicmath} % disable protrusion for tt fonts
}{}
\PassOptionsToPackage{hyphens}{url} % url is loaded by hyperref
\usepackage[unicode=true]{hyperref}
\PassOptionsToPackage{usenames,dvipsnames}{color} % color is loaded by hyperref
\hypersetup{
            pdftitle={Rethinking legal probabilism},
            pdfauthor={Rafał Urbaniak},
            colorlinks=true,
            linkcolor=Maroon,
            citecolor=Blue,
            urlcolor=blue,
            breaklinks=true}
\urlstyle{same}  % don't use monospace font for urls
\usepackage{graphicx,grffile}
\makeatletter
\def\maxwidth{\ifdim\Gin@nat@width>\linewidth\linewidth\else\Gin@nat@width\fi}
\def\maxheight{\ifdim\Gin@nat@height>\textheight\textheight\else\Gin@nat@height\fi}
\makeatother
% Scale images if necessary, so that they will not overflow the page
% margins by default, and it is still possible to overwrite the defaults
% using explicit options in \includegraphics[width, height, ...]{}
\setkeys{Gin}{width=\maxwidth,height=\maxheight,keepaspectratio}
\IfFileExists{parskip.sty}{%
\usepackage{parskip}
}{% else
\setlength{\parindent}{0pt}
\setlength{\parskip}{6pt plus 2pt minus 1pt}
}
\setlength{\emergencystretch}{3em}  % prevent overfull lines
\providecommand{\tightlist}{%
  \setlength{\itemsep}{0pt}\setlength{\parskip}{0pt}}
\setcounter{secnumdepth}{5}
% Redefines (sub)paragraphs to behave more like sections
\ifx\paragraph\undefined\else
\let\oldparagraph\paragraph
\renewcommand{\paragraph}[1]{\oldparagraph{#1}\mbox{}}
\fi
\ifx\subparagraph\undefined\else
\let\oldsubparagraph\subparagraph
\renewcommand{\subparagraph}[1]{\oldsubparagraph{#1}\mbox{}}
\fi

% set default figure placement to htbp
\makeatletter
\def\fps@figure{htbp}
\makeatother

%\documentclass{article}

% %packages
 \usepackage{booktabs}

\usepackage{multirow}

\usepackage{graphicx}
\usepackage{longtable}
\usepackage{ragged2e}
\usepackage{etex}
%\usepackage{yfonts}
\usepackage{marvosym}
\usepackage[notextcomp]{kpfonts}
\usepackage{nicefrac}
\newcommand*{\QED}{\hfill \footnotesize {\sc Q.e.d.}}
\usepackage{floatrow}

\usepackage[textsize=footnotesize]{todonotes}
%\linespread{1.5}


\setlength{\parindent}{10pt}
\setlength{\parskip}{1pt}


%language
\usepackage{times}
\usepackage{t1enc}
%\usepackage[utf8x]{inputenc}
%\usepackage[polish]{babel}
%\usepackage{polski}

\usepackage{mathptmx}
\usepackage[scaled=0.88]{helvet}


%AMS
\usepackage{amsfonts}
\usepackage{amssymb}
\usepackage{amsthm}
\usepackage{amsmath}
\usepackage{mathtools}

\usepackage{geometry}
 \geometry{a4paper,left=20mm,top=15mm,bottom = 15mm, right = 20mm}


%environments
\newtheorem{fact}{Fact}



%abbreviations
\newcommand{\ra}{\rangle}
\newcommand{\la}{\langle}
\newcommand{\n}{\neg}
\newcommand{\et}{\wedge}
\newcommand{\jt}{\rightarrow}
\newcommand{\ko}[1]{\forall  #1\,}
\newcommand{\ro}{\leftrightarrow}
\newcommand{\exi}[1]{\exists\, {_{#1}}}
\newcommand{\pr}[1]{\mathsf{P}(#1)}
\newcommand{\cost}{\mathsf{cost}}


\newcommand{\odds}{\mathsf{Odds}}
\newcommand{\ind}{\mathsf{Ind}}
\newcommand{\nf}[2]{\nicefrac{#1\,}{#2}}
\newcommand{\R}[1]{\texttt{#1}}
\newcommand{\prr}[1]{\mbox{$\mathtt{P}_{prior}(#1)$}}
\newcommand{\prp}[1]{\mbox{$\mathtt{P}_{posterior}(#1)$}}



\newtheorem{q}{\color{blue}Question}
\newtheorem{lemma}{Lemma}
\newtheorem{theorem}{Theorem}



%technical intermezzo
%---------------------

\newcommand{\intermezzoa}{
	\begin{minipage}[c]{13cm}
	\begin{center}\rule{10cm}{0.4pt}



	\tiny{\sc Optional Content Starts}
	
	\vspace{-1mm}
	
	\rule{10cm}{0.4pt}\end{center}
	\end{minipage}\nopagebreak 
	}


\newcommand{\intermezzob}{\nopagebreak 
	\begin{minipage}[c]{13cm}
	\begin{center}\rule{10cm}{0.4pt}

	\tiny{\sc Optional Content Ends}
	
	\vspace{-1mm}
	
	\rule{10cm}{0.4pt}\end{center}
	\end{minipage}
	}
%--------------------






















\newtheorem*{reply*}{Reply}
\usepackage{enumitem}
\newcommand{\question}[1]{\begin{enumerate}[resume,leftmargin=0cm,labelsep=0cm,align=left]
\item #1
\end{enumerate}}

\usepackage{float}

% \setbeamertemplate{blocks}[rounded][shadow=true]
% \setbeamertemplate{itemize items}[ball]
% \AtBeginPart{}
% \AtBeginSection{}
% \AtBeginSubsection{}
% \AtBeginSubsubsection{}
% \setlength{\emergencystretch}{0em}
% \setlength{\parskip}{0pt}






\usepackage[authoryear]{natbib}

%\bibliographystyle{apalike}

\title{Rethinking legal probabilism}
\author{Rafał Urbaniak}
\date{}

\begin{document}
\maketitle

\tableofcontents

\thispagestyle{empty}

\section{Scientific goal}\label{scientific-goal}

As many miscarriages of justice indicate, scientific evidence is easily
misinterpreted in court. This happens mostly because of the
communication gap between the parties involved. For this reason, methods
are needed to support proper assessment of evidence in such contexts.
The assessment of evidence in the court of law can be viewed from at
least three perspectives: as an interplay of arguments, as an assessment
of probabilities involved, or as an interaction of competing narrations.
Each perspective has been developed into a full-blown account of legal
reasoning (Di Bello \& Verheij, 2018; van Eemeren \& Verheij, 2017).
While individually each of these strains has been investigated,
exploration of the relations between them is in a rather early stage.
Crucially, the criticism of the probabilistic approach present in the
literature have not been satisfactorily adressed by the proponents of
legal probablism.

The \textbf{goal of this project} is to contribute to the
\textbf{development of legal probabilism} by formulating its variant
that accomodates important \textbf{insights provided by its critics}. A
crucial point of criticism is that the fact-finding process should be
conceptualized as \textbf{a competition of narrations}. I plan to
develop methods that allow the probabilist to take this perspective, and
explain how such methods allow the legal probabilist to address various
other objections present in the literature. The key idea is that once
\textbf{narrations are represented as bayesian networks, various criteria on,  features of  and operations on narrations can be explicated in terms of corresponding properties of and operations on bayesian networks}.
The conceptual developments are accompanied by technical accounts.
\textbf{\textsf{R}} code capturing to the technical features developed
is made available to the reader. Thus, the output will be a
\textbf{unifying extended probabilistic model embracing key aspects of the narrative and argumentative approaches, susceptible to AI implementation.}
The methods employed include: Bayesian statistical methods (including
Bayesian approach to higher-order probability), imprecise probabilities
and Bayesian networks.

\section{Significance}\label{significance}

(state of the art, justification for tackling a specific scientific
problem, justification for the pioneering nature of the project, the
impact of the project results on the development of the research field
and scientific discipline);

\section{Concept and work plan}\label{concept-and-work-plan}

(general work plan, specific research goals, results of preliminary
research, risk analysis);

\section{Research methodology}\label{research-methodology}

(underlying scientific methodology, methods, techniques and research
tools, methods of results analysis, equipment and devices to be used in
research);

\section*{References}\label{references}
\addcontentsline{toc}{section}{References}

\hypertarget{refs}{}
\hypertarget{ref-di2018evidential}{}
Di Bello, M., \& Verheij, B. (2018). Evidential reasoning. In
\emph{Handbook of legal reasoning and argumentation} (pp. 447--493).
Springer.

\hypertarget{ref-vanEemeren2017}{}
van Eemeren, F., \& Verheij, B. (2017). Argumentation theory in formal
and computational perspective. \emph{IFCoLog Journal of Logics and Their
Applications}, \emph{4}(8), 2099--2181.

\end{document}
