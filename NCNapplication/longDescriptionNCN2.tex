\documentclass[11pt,dvipsnames,enabledeprecatedfontcommands]{scrartcl}
\usepackage{lmodern}
\usepackage{amssymb,amsmath}
\usepackage{ifxetex,ifluatex}
\usepackage{fixltx2e} % provides \textsubscript
\ifnum 0\ifxetex 1\fi\ifluatex 1\fi=0 % if pdftex
  \usepackage[T1]{fontenc}
  \usepackage[utf8]{inputenc}
\else % if luatex or xelatex
  \ifxetex
    \usepackage{mathspec}
  \else
    \usepackage{fontspec}
  \fi
  \defaultfontfeatures{Ligatures=TeX,Scale=MatchLowercase}
\fi
% use upquote if available, for straight quotes in verbatim environments
\IfFileExists{upquote.sty}{\usepackage{upquote}}{}
% use microtype if available
\IfFileExists{microtype.sty}{%
\usepackage[]{microtype}
\UseMicrotypeSet[protrusion]{basicmath} % disable protrusion for tt fonts
}{}
\PassOptionsToPackage{hyphens}{url} % url is loaded by hyperref
\usepackage[unicode=true]{hyperref}
\PassOptionsToPackage{usenames,dvipsnames}{color} % color is loaded by hyperref
\hypersetup{
            pdftitle={Rethinking legal probabilism},
            pdfauthor={Rafał Urbaniak},
            colorlinks=true,
            linkcolor=Maroon,
            citecolor=Blue,
            urlcolor=blue,
            breaklinks=true}
\urlstyle{same}  % don't use monospace font for urls
\usepackage{graphicx,grffile}
\makeatletter
\def\maxwidth{\ifdim\Gin@nat@width>\linewidth\linewidth\else\Gin@nat@width\fi}
\def\maxheight{\ifdim\Gin@nat@height>\textheight\textheight\else\Gin@nat@height\fi}
\makeatother
% Scale images if necessary, so that they will not overflow the page
% margins by default, and it is still possible to overwrite the defaults
% using explicit options in \includegraphics[width, height, ...]{}
\setkeys{Gin}{width=\maxwidth,height=\maxheight,keepaspectratio}
\IfFileExists{parskip.sty}{%
\usepackage{parskip}
}{% else
\setlength{\parindent}{0pt}
\setlength{\parskip}{6pt plus 2pt minus 1pt}
}
\setlength{\emergencystretch}{3em}  % prevent overfull lines
\providecommand{\tightlist}{%
  \setlength{\itemsep}{0pt}\setlength{\parskip}{0pt}}
\setcounter{secnumdepth}{5}
% Redefines (sub)paragraphs to behave more like sections
\ifx\paragraph\undefined\else
\let\oldparagraph\paragraph
\renewcommand{\paragraph}[1]{\oldparagraph{#1}\mbox{}}
\fi
\ifx\subparagraph\undefined\else
\let\oldsubparagraph\subparagraph
\renewcommand{\subparagraph}[1]{\oldsubparagraph{#1}\mbox{}}
\fi

% set default figure placement to htbp
\makeatletter
\def\fps@figure{htbp}
\makeatother

%\documentclass{article}

% %packages
 \usepackage{booktabs}

\usepackage{multirow}

\usepackage{graphicx}
\usepackage{longtable}
\usepackage{ragged2e}
\usepackage{etex}
%\usepackage{yfonts}
\usepackage{marvosym}
\usepackage[notextcomp]{kpfonts}
\usepackage{nicefrac}
\newcommand*{\QED}{\hfill \footnotesize {\sc Q.e.d.}}
\usepackage{floatrow}

\usepackage[textsize=scriptsize, textwidth = 1.5cm]{todonotes}
%\linespread{1.5}


\setlength{\parindent}{10pt}
\setlength{\parskip}{1pt}


%language
\usepackage{times}
\usepackage{t1enc}
%\usepackage[utf8x]{inputenc}
%\usepackage[polish]{babel}
%\usepackage{polski}

\usepackage{mathptmx}
\usepackage[scaled=0.88]{helvet}


%AMS
\usepackage{amsfonts}
\usepackage{amssymb}
\usepackage{amsthm}
\usepackage{amsmath}
\usepackage{mathtools}

\usepackage{geometry}
 \geometry{a4paper,left=20mm,top=15mm,bottom = 15mm, right = 20mm}


%environments
\newtheorem{fact}{Fact}



%abbreviations
\newcommand{\ra}{\rangle}
\newcommand{\la}{\langle}
\newcommand{\n}{\neg}
\newcommand{\et}{\wedge}
\newcommand{\jt}{\rightarrow}
\newcommand{\ko}[1]{\forall  #1\,}
\newcommand{\ro}{\leftrightarrow}
\newcommand{\exi}[1]{\exists\, {_{#1}}}
\newcommand{\pr}[1]{\mathsf{P}(#1)}
\newcommand{\cost}{\mathsf{cost}}


\newcommand{\odds}{\mathsf{Odds}}
\newcommand{\ind}{\mathsf{Ind}}
\newcommand{\nf}[2]{\nicefrac{#1\,}{#2}}
\newcommand{\R}[1]{\texttt{#1}}
\newcommand{\prr}[1]{\mbox{$\mathtt{P}_{prior}(#1)$}}
\newcommand{\prp}[1]{\mbox{$\mathtt{P}_{posterior}(#1)$}}



\newtheorem{q}{\color{blue}Question}
\newtheorem{lemma}{Lemma}
\newtheorem{theorem}{Theorem}



%technical intermezzo
%---------------------

\newcommand{\intermezzoa}{
	\begin{minipage}[c]{13cm}
	\begin{center}\rule{10cm}{0.4pt}



	\tiny{\sc Optional Content Starts}
	
	\vspace{-1mm}
	
	\rule{10cm}{0.4pt}\end{center}
	\end{minipage}\nopagebreak 
	}


\newcommand{\intermezzob}{\nopagebreak 
	\begin{minipage}[c]{13cm}
	\begin{center}\rule{10cm}{0.4pt}

	\tiny{\sc Optional Content Ends}
	
	\vspace{-1mm}
	
	\rule{10cm}{0.4pt}\end{center}
	\end{minipage}
	}
%--------------------






















\newtheorem*{reply*}{Reply}
\usepackage{enumitem}
\newcommand{\question}[1]{\begin{enumerate}[resume,leftmargin=0cm,labelsep=0cm,align=left]
\item #1
\end{enumerate}}

\usepackage{float}

% \setbeamertemplate{blocks}[rounded][shadow=true]
% \setbeamertemplate{itemize items}[ball]
% \AtBeginPart{}
% \AtBeginSection{}
% \AtBeginSubsection{}
% \AtBeginSubsubsection{}
% \setlength{\emergencystretch}{0em}
% \setlength{\parskip}{0pt}






\usepackage[authoryear]{natbib}

%\bibliographystyle{apalike}

\title{Rethinking legal probabilism}
\author{Rafał Urbaniak}
\date{}

\begin{document}
\maketitle

\tableofcontents

\thispagestyle{empty}

\section{Scientific goal}\label{scientific-goal}

As many miscarriages of justice indicate, scientific evidence is easily
misinterpreted in court. This happens partially due to miscommunication
between the parties involved, but also because incorporating scientific
evidence in the context of a whole case can be really hard. While
probabilistic tools for piecemeal evaluation of scientific evidence and
spotting probablistic fallacies in legal contexts are quite well
developed,\todo{say sth about replicability crisis in forensic sciences at some point?}
the construction of a more general probabilistic model of incorporating
such evidence in a wider context of a whole case, probabilistic
explication of and theorizing about evidence evaluation and legal
decision standards, remain a challenge. Legal probabilism, for our
purpose, is the view that this challenge can and should be met. This
project intends to contribute to further development of this enterprise
in a philosophically motivated manner.

The assessment of evidence in the court of law can be viewed from at
least three perspectives: as an interplay of arguments, as an assessment
of probabilities involved, or as an interaction of competing narrations.
Each perspective presents an account of legal reasoning (Di Bello \&
Verheij, 2018; van Eemeren \& Verheij, 2017). Individually, each of
these strains has been investigated. The probablistic approach is the
most developed one but legal probabilism is still underdeveloped ---to a
large extent this is so in light of various lines of criticism developed
by the representatives of the other strains, as such challenges have not
been satisfactorily adressed by the proponents of legal probablism.

The \textbf{goal of this project} is to contribute to the
\textbf{development of legal probabilism} by formulating its variant
that accomodates important \textbf{insights provided by its critics}. A
crucial point of criticism is that the fact-finding process should be
conceptualized as \textbf{a competition of narrations}. I plan to
develop methods that allow the probabilist to take this perspective, and
explain how such methods allow the legal probabilist to address various
other objections present in the literature. The key idea is that once
\textbf{narrations are represented as bayesian networks, various criteria on,  features of  and operations on narrations can be explicated in terms of corresponding properties of and operations on bayesian networks}.
Further, the hypothesis is that such an improved framework will
facilitate addressing key objections raised against legal probabilism in
the literature.

The conceptual developments are accompanied by technical accounts.
\textbf{\textsf{R}} code capturing to the technical features developed
is made available to the reader. Thus, the output will be a
\textbf{unifying extended probabilistic model embracing key aspects of the narrative and argumentative approaches, susceptible to AI implementation.}
The methods employed include: Bayesian statistical methods (including
Bayesian approach to higher-order probability), imprecise probabilities
and Bayesian networks.

\section{Significance}\label{significance}

(state of the art, justification for tackling a specific scientific
problem, justification for the pioneering nature of the project, the
impact of the project results on the development of the research field
and scientific discipline);

\subsection{State of the art}\label{state-of-the-art}

One of the functions of the trial is to resolve disputes about questions
of facts. Did the defendant rob the bank? Who is the father of the
child? Did this drug cause birth defects? To answer these questions, the
litigants will present evidence of different kinds: eyewitness
testimonies, DNA matches, epidemiological studies, etc. The evidence
presented will often be in conflict with other evidence. In a bank
robbery case, for example, the prosecution may present eyewitness
testimony that the defendant was seen driving a truck near the bank a
few minutes after the robbery took place. The defense may respond that
no traces were found at the crime scene that would match the defendant.
The fact-finders, judges or lay jurors, should address these conflicts
by assessing and weighing the evidence, and on that basis reach a final
decision. This is a difficult task. The evidence presented at trial can
be complex and open to multiple interpretations, and even when it is
assessed carefully, it may still lead to an incorrect verdict. How
should judges and jurors respond to this uncertainty?

From among the three perspectives mentioned in the beginning, I will
focus on the probablistic approach and take it as my point of departure,
for various reasons:

\begin{itemize}
\item
  The project is to be informed by and reflect on the actual practice of
  legal evidence evaluation, and much of scientific evidence in such
  contexts has probabilistic form.
\item
  Probabilistic tools are fairly well-developed both for applications
  and within formal epistemology, reaching a state of fruition which I
  think should inspire deeper reflection.
\item
  Statistical computing tools for such methods are avaialable, which
  makes programming development and preliminary computational and
  data-driven evaluation of the ideas to be defended a viable
  enterprise.
\end{itemize}

Accoringly, the view in focus of this research is legal probabilism,
according to which probability theory helps to understand the
fallibility of trial decision-making, theorize about and improve upon
the assessment of the evidence presented in court. It is an ongoing
research program that comprises a variety of claims about evidence
assessment and decision-making at trial. At its simplest, it comprises
two core tenets: first, that the evidence presented at trial can be
assessed, weighed and combined by means of probability theory; and
second, that legal decision rules, such as proof beyond a reasonable
doubt in criminal cases, can be explicated in probabilistic terms.

In the Middle Ages, before the advent of probability theory, there
existed an informal mathematics of legal evidence (Wigmore, 1901).
Formalistic procedures fixed the number of witnesses required to
establish a claim. Lawyers would list ways in which items of evidence
could be added or subtracted to weaken or strengthen one's case. This
formalistic system fell into disrepute as the Enlightenment principle of
`free proof' gained wide acceptance (Damaška, 1995). Concurrently, the
development of probability theory brought forth a new approach to
weighing evidence and making decisions under uncertainty. The early
theorists of probability in the 17th and 18th century were as much
interested in games of chance as they were interested in the uncertainty
of trial decisions (Hacking, 1975, Daston (1988), Franklin (2001)).
Bernoulli (1713) was one of the first to formulate probabilistic rules
for combining different pieces of evidence in legal cases and assessing
to what extent they supported a claim of interest. He was also one of
the first to suggest that decision rules at trial could be understood as
probability thresholds.

Bernoulli's prescient insights attained greater popularity in the 20th
century amidst the law and economics movement (Calabresi, 1961, Becker
(1968), Posner (1973)). In a seminal article, Finkelstein \& Fairley
(1970) gave one of the first systematic analyses of how probability
theory, and Bayes' theorem in particular, can help to weigh evidence at
trial. Lempert (1977) was one of the first to rely on probability
theory, specifically likelihood ratios, for assessing the relevance of
evidence. Such contributions fueled what has been called the New
Evidence Scholarship, a rigorous way of studying the process of legal
proof at trial (Lempert, 1986). After the discovery of DNA
fingerprinting in the eighties, many legal probabilists focused on how
probability theory could be used to quantify the strength of a DNA match
under various circumstances {[}Kaye (1986); National Research Council
(1992), Robertson \& Vignaux (1995); Koehler (1996);
Kaye2010The-Double-Heli{]}.

In response to these developments, Tribe (1971) attacked what he called
`trial by mathematics'. His critique ranged from listing well-known
cases of misuse or probabilities in legal contexts and practical
difficulties in assessing the probability of someone's criminal or civil
liability to the dehumanization of trial decisions. After Tribe, many
have criticized legal probabilism on a variety of grounds, both
theoretical and practical, arguing that probabilistic models are either
inadequate or unhelpful (Brilmayer, 1986; Dant, 1988, Allen (1986); see,
for instance, Underwood, 1977,cohen86).

More recently, alternative frameworks for modeling evidential reasoning
and decision-making at trial have been proposed. They are based on
inference to the best explanation (Allen, 2010; Pardo \& Allen, 2008),
narratives and stories (Allen, 1986, 2010; Allen \& Leiter, 2001;
Clermont, 2015,Pardo (2018); Pennington \& Hastie, 1991a), and
argumentation theory (Bex, 2011; Gordon, Prakken, \& Walton, 2007;
Walton, 2002). Those who favor a conciliatory stance have combined legal
probabilism with other frameworks, offering preliminary sketches of
hybrid theories (Verheij, 2014, Urbaniak (2018)).

Some legal scholars and practitioners have voiced their support for
legal probabilism explicitly (Tillers \& Gottfried, 2007). Yet
skepticism about mathematical and quantitative models of legal evidence
is still widespread among prominent legal scholars and practitioners
(see, for example, Allen \& Pardo, 2007). Even among legal probabilists,
few would think it possible to quantify precisely the probability of
someone's guilt or civil liability. The probabilistic
\href{mailto:formalism---@taroni2006bayesian}{\nolinkurl{formalism-\/-\/-@taroni2006bayesian}}
write---`should primarily be considered as an aid to structure and guide
one's inferences under uncertainty, rather than a way to reach precise
numerical assessments' (p.~xv).

Perhaps the most difficult challenge for legal probabilism---at least,
one that has galvanized philosophical attention in recent years---comes
from the paradoxes of legal proof or puzzles of naked statistical
evidence. In a number of seminal papers, (Cohen, 1981; Nesson, 1979; and
Thomson, 1986) formulated scenarios in which, even if the probability of
guilt or civil liability, based on the available evidence, is
particularly high, a verdict against the defendant seems unwarranted.
Arguably, these scenarios underscore a theoretical difficulty with
probabilistic accounts of legal standards of proof. Many articles have
been written on the topic, initially by legal scholars. In the last
decade, philosophers have also joined the debate, while interest among
legal scholars has waned.\footnote{For critical surveys see (Redmayne,
  2008) and (Gardiner, 2018,pardo2019).} Other conceptual challenges for
legal probabilism include the so-called problem of conjunction and the
reference class problem.

One important difficulty is that there are various thought experiments
in which the probability of guilt is very high and yet conviciton or
finding of liability is intuitively unjustified---these are known as
proof paradoxes (Cohen, 1977; Redmayne, 2008). \todo{Add an example}

At least \emph{prima facie}, then, it seems that some conditions other
than high posterior probability of liability have to be satisfied for
the decision to penalize (or find liable) to be justified. Accordingly,
various informal notions have been claimed to be essential for a proper
explication of judiciary decision standards (Haack, 2014; Wells, 1992).
For instance, evidence is claimed to be insufficient for conviction if
it is not \emph{sensitive} to the issue at hand: if it remained the same
even if the accused was innocent (Enoch \& Fisher, 2015). Or, to look at
another approach, evidence is claimed to be insufficient for conviction
if it doesn't \emph{normically support} it: if---given the same
evidence---no explanation would be needed even if the accused was
innocent (Smith, 2017). A legal probabilist needs either to show that
these notions are unnecessary or inadequate for the purpose at hand, or
to indicate how they can be explicated in probabilistic terms.

\todo{describe what the probabilistic model is before we get deeper?}

Another point of criticism of the wider proabilist model, legal
proceedings are back-and-forth between opposing parties in which
cross-examination is of crucial importance, reasoning goes not only
evidence-to-hypothesis, but also hypotheses-to-evidence (Allen \& Pardo,
2007; Wells, 1992) in a way that seems analogous to inference to the
best explanation (Dant, 1988), which notoriously is claimed to not be
susceptible to probabilistic analysis (Lipton, 2004). An informal
philosophical account inspired by such considerations---The
\textbf{No Plausible Alternative Story (NPAS)} theory (Allen, 2010)---is
that the courtroom is a confrontation of competing narrations (Ho, 2008;
Wagenaar, Van Koppen, \& Crombag, 1993) offered by the sides, and the
narrative to be selected should be the most plausible one. The view is
conceptually plausible (Di Bello, 2013), and finds support in
psychological evidence (Pennington \& Hastie, 1991b, 1992).

It would be a great advantage of legal probabilism if it could to model
phenomena captured by the narrative approach, but how is the legal
probabilist to make sense of them? From her perspective, the key
disadvantage of NPAS is that it abandons the rich toolbox of
probabilistic methods and takes the key notion of plausibility to be a
primitive notion which should be understood only intuitively.

\#\# Pioneering nature of the project

Recent work in Artificial Intelligence made it possible to use
probability theory---in the form of Bayesian networks---to weigh and
assess complex bodies of evidence consisting of multiple components.

Initial philosophical analysis of the approach has been performed, (Di
Bello, 2013) pioneering a probabilistic understanding of narrations.

\todo{add more about Marcello}

\subsection{Choice of problem}\label{choice-of-problem}

\subsection{Pioneering nature \& impact}\label{pioneering-nature-impact}

\section{Concept and work plan}\label{concept-and-work-plan}

(general work plan, specific research goals, results of preliminary
research, risk analysis);

\section{Research methodology}\label{research-methodology}

(underlying scientific methodology, methods, techniques and research
tools, methods of results analysis, equipment and devices to be used in
research);

\section*{References}\label{references}
\addcontentsline{toc}{section}{References}

\hypertarget{refs}{}
\hypertarget{ref-Allen1986A-Reconceptuali}{}
Allen, R. J. (1986). A reconceptualization of civil trials. \emph{Boston
University Law Review}, \emph{66}, 401--437.

\hypertarget{ref-Allen2010No-Plausible-Al}{}
Allen, R. J. (2010). No plausible alternative to a plausible story of
guilt as the rule of decision in criminal cases. In J. Cruz \& L. Laudan
(Eds.), \emph{Prueba y esandares de prueba en el derecho}. Instituto de
Investigaciones Filosoficas-UNAM.

\hypertarget{ref-allen2001naturalized}{}
Allen, R. J., \& Leiter, B. (2001). Naturalized epistemology and the law
of evidence. \emph{Virginia Law Review}, \emph{87}(8), 1491--1550.
JSTOR.

\hypertarget{ref-allen2007problematic}{}
Allen, R., \& Pardo, M. (2007). The problematic value of mathematical
models of evidence. \emph{The Journal of Legal Studies}, \emph{36}(1),
107--140. JSTOR.

\hypertarget{ref-becker1968crime}{}
Becker, G. S. (1968). Crime and punishment: An economic approach.
\emph{Journal of Political Economy}, \emph{76}, 169--217. Springer.

\hypertarget{ref-Bernoulli1713Ars-conjectandi}{}
Bernoulli, J. (1713). \emph{Ars conjectandi}.

\hypertarget{ref-bex2011ArgumentsStoriesCriminal}{}
Bex, F. J. (2011). \emph{Arguments, stories and criminal evidence: A
formal hybrid theory}. Law and philosophy library. Dordrecht ; New York:
Springer.

\hypertarget{ref-brilmayer1986}{}
Brilmayer, L. (1986). Second-order evidence and bayesian logic.
\emph{Boston University Law Review}, \emph{66}, 673--691.

\hypertarget{ref-Calabresi1961}{}
Calabresi, G. (1961). Some thoughts on risk distribution and the law of
torts. \emph{Yale Law Journal}, \emph{70}, 499--553.

\hypertarget{ref-clermont2015TrialTraditionalProbability}{}
Clermont, K. M. (2015). Trial by Traditional Probability, Relative
Plausibility, or Belief Function? \emph{Case Western Reserve Law
Review}, \emph{66}(2), 353--391.

\hypertarget{ref-Cohen1977The-probable-an}{}
Cohen, J. (1977). \emph{The probable and the provable}. Oxford
University Press.

\hypertarget{ref-Cohen81}{}
Cohen, J. L. (1981). Subjective probability and the paradox of the
Gatecrasher. \emph{Arizona State Law Journal}, 627--634.

\hypertarget{ref-damaska1996free}{}
Damaška, M. R. (1995). Free proof and its detractors. \emph{The American
Journal of Comparative Law}, \emph{43}(3), 343--357.

\hypertarget{ref-dant1988gambling}{}
Dant, M. (1988). Gambling on the truth: The use of purely statistical
evidence as a basis for civil liability. \emph{Columbia Journal of Law
and Social Problems}, \emph{22}, 31--70. HeinOnline.

\hypertarget{ref-daston1988}{}
Daston, L. (1988). \emph{Classical probability in the enlightenment}.
Princeton University Press.

\hypertarget{ref-di2013statistics}{}
Di Bello, M. (2013). \emph{Statistics and probability in criminal
trials} (PhD thesis). University of Stanford.

\hypertarget{ref-di2018evidential}{}
Di Bello, M., \& Verheij, B. (2018). Evidential reasoning. In
\emph{Handbook of legal reasoning and argumentation} (pp. 447--493).
Springer.

\hypertarget{ref-enoch2015sense}{}
Enoch, D., \& Fisher, T. (2015). Sense and sensitivity: Epistemic and
instrumental approaches to statistical evidence. \emph{Stan. L. Rev.},
\emph{67}, 557--611. HeinOnline.

\hypertarget{ref-Finkelstein1970A}{}
Finkelstein, M. O., \& Fairley, W. B. (1970). A Bayesian approach to
identification evidence. \emph{Harvard Law Review}, \emph{83}(3),
489--517.

\hypertarget{ref-Franklin2001}{}
Franklin, J. (2001). \emph{The science of conjecture: Evidence and
probability before pascal}. John Hopkins University Press.

\hypertarget{ref-gardiner2018}{}
Gardiner, G. (2018). Legal burdens of proof and statistical evidence. In
D. Coady \& J. Chase (Eds.), \emph{Routledge handbook of applied
epistemology}. Routledge.

\hypertarget{ref-gordon2007}{}
Gordon, T. F., Prakken, H., \& Walton, D. (2007). The Carneades model of
argument and burden of proof. \emph{Artificial Intelligence},
\emph{171}(10-15), 875--896.

\hypertarget{ref-haack2011legal}{}
Haack, S. (2014). Legal probabilism: An epistemological dissent. In
\emph{Haack2014-HAAEMS} (pp. 47--77).

\hypertarget{ref-Hacking1984}{}
Hacking, I. (1975). \emph{The emergence of probability: A philosophical
study of early ideas about probability, induction and statistical
inference}. Cambridge University Press.

\hypertarget{ref-ho2008philosophy}{}
Ho, H. L. (2008). \emph{A philosophy of evidence law: Justice in the
search for truth}. Oxford University Press.

\hypertarget{ref-kaye1986admissibility}{}
Kaye, D. H. (1986). The admissibility of ``probability evidence'' in
criminal trials---part I. \emph{Jurimetrics Journal}, 343--346.

\hypertarget{ref-Koehler1996On-Conveying-th}{}
Koehler, J. J. (1996). On conveying the probative value of DNA evidence:
Frequencies, likelihood ratios, and error rates. \emph{University of
Colorado law Review}, \emph{67}, 859--886.

\hypertarget{ref-lempert1977modeling}{}
Lempert, R. O. (1977). Modeling relevance. \emph{Michigan Law Review},
\emph{75}, 1021--1057. JSTOR.

\hypertarget{ref-Lempert1986}{}
Lempert, R. O. (1986). The new evidence scholarship: Analysing the
process of proof. \emph{Boston University Law Review}, \emph{66},
439--477.

\hypertarget{ref-Lipton2004-LIPITT}{}
Lipton, P. (2004). \emph{Inference to the best explanation}.
Routledge/Taylor; Francis Group.

\hypertarget{ref-NRCI1992}{}
National Research Council. (1992). \emph{DNA technology in forensic
science \textup{{[}NRC I{]}}}. Committee on DNA technology in Forensic
Science, National Research Council.

\hypertarget{ref-Nesson1979Reasonable-doub}{}
Nesson, C. R. (1979). Reasonable doubt and permissive inferences: The
value of complexity. \emph{Harvard Law Review}, \emph{92}(6),
1187--1225.

\hypertarget{ref-pardo2018}{}
Pardo, M. S. (2018). Safety vs.~Sensitivity: Possible worlds and the law
of evidence. \emph{Legal Theory}, \emph{24}(1), 50--75.

\hypertarget{ref-Pardo2008judicial}{}
Pardo, M. S., \& Allen, R. J. (2008). Judicial proof and the best
explanation. \emph{Law and Philosophy}, \emph{27}(3), 223--268.

\hypertarget{ref-Pennington1991}{}
Pennington, N., \& Hastie, R. (1991a). A cognitive theory of juror
decision making: The story model. \emph{Cardozo Law Review}, \emph{13},
519--557.

\hypertarget{ref-pennington1991cognitive}{}
Pennington, N., \& Hastie, R. (1991b). A cognitive theory of juror
decision making: The story model. \emph{Cardozo Law Review}, \emph{13},
519--557. HeinOnline.

\hypertarget{ref-pennington1992explaining}{}
Pennington, N., \& Hastie, R. (1992). Explaining the evidence: Tests of
the story model for juror decision making. \emph{Journal of personality
and social psychology}, \emph{62}(2), 189--204. American Psychological
Association.

\hypertarget{ref-Posner1973}{}
Posner, R. (1973). \emph{The economic analysis of law}. Brown \&
Company.

\hypertarget{ref-redmayne2008exploring}{}
Redmayne, M. (2008). Exploring the proof paradoxes. \emph{Legal Theory},
\emph{14}(4), 281--309. Cambridge University Press.

\hypertarget{ref-Robertson1995evidence}{}
Robertson, B., \& Vignaux, G. A. (1995). DNA evidence: Wrong answers or
wrong questions? \emph{Genetica}, \emph{96}, 145--152.

\hypertarget{ref-Smith_conviction_mind_2017}{}
Smith, M. (2017). When does evidence suffice for conviction?
\emph{Mind}.

\hypertarget{ref-Thomson86}{}
Thomson, J. J. (1986). Liability and individualized evidence. \emph{Law
and Contemporary Problems}, \emph{49}(3), 199--219.

\hypertarget{ref-Tillers2007}{}
Tillers, P., \& Gottfried, J. (2007). Case comment--United States v.
Copeland, 369 F. Supp. 2d 275 (E.D.N.Y. 2005): A collateral attack on
the legal maxim that proof beyond a reasonable doubt is unquantifiable?
\emph{Law, Probability and Risk}, \emph{5}(2), 135--157.

\hypertarget{ref-tribe71}{}
Tribe, L. H. (1971). Trial by mathematics: Precision and ritual in the
legal process. \emph{Harvard Law Review}, \emph{84}(6), 1329--1393.

\hypertarget{ref-Underwood1977The-thumb-on-th}{}
Underwood, B. D. (1977). The thumb on the scale of justice: Burdens of
persuasion in criminal cases. \emph{Yale Law Journal}, \emph{86(7)},
1299--1348.

\hypertarget{ref-urbaniak2018narration}{}
Urbaniak, R. (2018). Narration in judiciary fact-finding: A
probabilistic explication. \emph{Artificial Intelligence and Law},
1--32.

\hypertarget{ref-vanEemeren2017}{}
van Eemeren, F., \& Verheij, B. (2017). Argumentation theory in formal
and computational perspective. \emph{IFCoLog Journal of Logics and Their
Applications}, \emph{4}(8), 2099--2181.

\hypertarget{ref-verheij2014catch}{}
Verheij, B. (2014). To catch a thief with and without numbers:
Arguments, scenarios and probabilities in evidential reasoning.
\emph{Law, Probability and Risk}, \emph{13}(3-4), 307--325. Citeseer.

\hypertarget{ref-wagenaar1993anchored}{}
Wagenaar, W., Van Koppen, P., \& Crombag, H. (1993). \emph{Anchored
narratives: The psychology of criminal evidence.} St Martin's Press.

\hypertarget{ref-Walton2002}{}
Walton, D. N. (2002). \emph{Legal argumentation and evidence}. Penn
State University Press.

\hypertarget{ref-wells1992naked}{}
Wells, G. (1992). Naked statistical evidence of liability: Is subjective
probability enough? \emph{Journal of Personality and Social Psychology},
\emph{62}(5), 739--752. American Psychological Association.

\hypertarget{ref-wigmore1901number}{}
Wigmore, J. H. (1901). Required numbers of witnesses; a brief history of
the numerical system in england. \emph{Harvard Law Review},
\emph{15}(2), 83--108.

\end{document}
