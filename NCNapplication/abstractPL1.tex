\PassOptionsToPackage{unicode=true}{hyperref} % options for packages loaded elsewhere
\PassOptionsToPackage{hyphens}{url}
\PassOptionsToPackage{dvipsnames,svgnames*,x11names*}{xcolor}
%
\documentclass[11pt,dvipsnames,enabledeprecatedfontcommands]{scrartcl}
\usepackage{lmodern}
\usepackage{amssymb,amsmath}
\usepackage{ifxetex,ifluatex}
\usepackage{fixltx2e} % provides \textsubscript
\ifnum 0\ifxetex 1\fi\ifluatex 1\fi=0 % if pdftex
  \usepackage[T1]{fontenc}
  \usepackage[utf8]{inputenc}
  \usepackage{textcomp} % provides euro and other symbols
\else % if luatex or xelatex
  \usepackage{unicode-math}
  \defaultfontfeatures{Ligatures=TeX,Scale=MatchLowercase}
\fi
% use upquote if available, for straight quotes in verbatim environments
\IfFileExists{upquote.sty}{\usepackage{upquote}}{}
% use microtype if available
\IfFileExists{microtype.sty}{%
\usepackage[]{microtype}
\UseMicrotypeSet[protrusion]{basicmath} % disable protrusion for tt fonts
}{}
\IfFileExists{parskip.sty}{%
\usepackage{parskip}
}{% else
\setlength{\parindent}{0pt}
\setlength{\parskip}{6pt plus 2pt minus 1pt}
}
\usepackage{xcolor}
\usepackage{hyperref}
\hypersetup{
            colorlinks=true,
            linkcolor=Maroon,
            filecolor=Maroon,
            citecolor=Blue,
            urlcolor=blue,
            breaklinks=true}
\urlstyle{same}  % don't use monospace font for urls
\usepackage{graphicx,grffile}
\makeatletter
\def\maxwidth{\ifdim\Gin@nat@width>\linewidth\linewidth\else\Gin@nat@width\fi}
\def\maxheight{\ifdim\Gin@nat@height>\textheight\textheight\else\Gin@nat@height\fi}
\makeatother
% Scale images if necessary, so that they will not overflow the page
% margins by default, and it is still possible to overwrite the defaults
% using explicit options in \includegraphics[width, height, ...]{}
\setkeys{Gin}{width=\maxwidth,height=\maxheight,keepaspectratio}
\setlength{\emergencystretch}{3em}  % prevent overfull lines
\providecommand{\tightlist}{%
  \setlength{\itemsep}{0pt}\setlength{\parskip}{0pt}}
\setcounter{secnumdepth}{5}
% Redefines (sub)paragraphs to behave more like sections
\ifx\paragraph\undefined\else
\let\oldparagraph\paragraph
\renewcommand{\paragraph}[1]{\oldparagraph{#1}\mbox{}}
\fi
\ifx\subparagraph\undefined\else
\let\oldsubparagraph\subparagraph
\renewcommand{\subparagraph}[1]{\oldsubparagraph{#1}\mbox{}}
\fi

% set default figure placement to htbp
\makeatletter
\def\fps@figure{htbp}
\makeatother

%\documentclass{article}

% %packages
 \usepackage{booktabs}

\usepackage{multirow}
\usepackage{multicol}

\usepackage{graphicx}
\usepackage{longtable}
\usepackage{ragged2e}
\usepackage{etex}
%\usepackage{yfonts}
\usepackage{marvosym}
\usepackage[notextcomp]{kpfonts}
\usepackage{nicefrac}
\newcommand*{\QED}{\hfill \footnotesize {\sc Q.e.d.}}
\usepackage{floatrow}



\usepackage[textsize=scriptsize, textwidth = 1.5cm]{todonotes}
%\linespread{1.5}


\setlength{\parindent}{10pt}
\setlength{\parskip}{1pt}


%language
\usepackage{times}
\usepackage{t1enc}
%\usepackage[utf8x]{inputenc}
%\usepackage[polish]{babel}
%\usepackage{polski}

\usepackage{mathptmx}
\usepackage[scaled=0.88]{helvet}


%AMS
\usepackage{amsfonts}
\usepackage{amssymb}
\usepackage{amsthm}
\usepackage{amsmath}
\usepackage{mathtools}

\usepackage{geometry}
 \geometry{a4paper,left=20mm,top=15mm,bottom = 20mm, right = 20mm}


%environments
\newtheorem{fact}{Fact}



%abbreviations
\newcommand{\ra}{\rangle}
\newcommand{\la}{\langle}
\newcommand{\n}{\neg}
\newcommand{\et}{\wedge}
\newcommand{\jt}{\rightarrow}
\newcommand{\ko}[1]{\forall  #1\,}
\newcommand{\ro}{\leftrightarrow}
\newcommand{\exi}[1]{\exists\, {_{#1}}}
\newcommand{\pr}[1]{\mathsf{P}(#1)}
\newcommand{\cost}{\mathsf{cost}}


\newcommand{\odds}{\mathsf{Odds}}
\newcommand{\ind}{\mathsf{Ind}}
\newcommand{\nf}[2]{\nicefrac{#1\,}{#2}}
\newcommand{\R}[1]{\texttt{#1}}
\newcommand{\prr}[1]{\mbox{$\mathtt{P}_{prior}(#1)$}}
\newcommand{\prp}[1]{\mbox{$\mathtt{P}_{posterior}(#1)$}}



\newtheorem{q}{\color{blue}Question}
\newtheorem{lemma}{Lemma}
\newtheorem{theorem}{Theorem}



%technical intermezzo
%---------------------

\newcommand{\intermezzoa}{
	\begin{minipage}[c]{13cm}
	\begin{center}\rule{10cm}{0.4pt}



	\tiny{\sc Optional Content Starts}
	
	\vspace{-1mm}
	
	\rule{10cm}{0.4pt}\end{center}
	\end{minipage}\nopagebreak 
	}


\newcommand{\intermezzob}{\nopagebreak 
	\begin{minipage}[c]{13cm}
	\begin{center}\rule{10cm}{0.4pt}

	\tiny{\sc Optional Content Ends}
	
	\vspace{-1mm}
	
	\rule{10cm}{0.4pt}\end{center}
	\end{minipage}
	}
%--------------------

\DeclareUnicodeCharacter{0301}{*************************************}




















\newtheorem*{reply*}{Reply}
\usepackage{enumitem}
\newcommand{\question}[1]{\begin{enumerate}[resume,leftmargin=0cm,labelsep=0cm,align=left]
\item #1
\end{enumerate}}

\usepackage{float}

% \setbeamertemplate{blocks}[rounded][shadow=true]
% \setbeamertemplate{itemize items}[ball]
% \AtBeginPart{}
% \AtBeginSection{}
% \AtBeginSubsection{}
% \AtBeginSubsubsection{}
% \setlength{\emergencystretch}{0em}
% \setlength{\parskip}{0pt}






\usepackage[authoryear]{natbib}

%\bibliographystyle{apalike}

\author{}
\date{\vspace{-2.5em}}

\begin{document}

\begin{center}
\large {\sc \textbf{Rekonceptualizacja probabilizmu w kontekstach prawnych}}

\large Rafa\l\, Urbaniak
\end{center}

\vspace{1mm}

\thispagestyle{empty}

Probabilism prawny to projekt badawczy, który posługuje się teorią
prawdopodobieństwa do analizy, modelowania i poprawy oceny dowodów i
procesów decyzyjnych w sprawach sądowych. Probabilizm prawny jest
podejściem reprezentowanym przez mniejszość, ale zyskał na popularności
w drugiej połowie wieku dwudziestego w powiązaniu z ruchem ``prawo i
ekonomia.''

W sprawach prawnych, różne linie dowodowe mogą się zbiegać, jak na
przykłady gdy dwoje świadków zeznaje, że widziano oskarżonego na miejscu
przestępstwa, albo mogą iść w przeciwne strony, jak gdy świadkowie
zeznają, iż widzieli oskarżonego na miejscu przestępstwa, ale test DNA
pokazuje że nie ma zgodności między oskarżonym a materiałem z miejsca
przestępstwa. Innym źródłem złożoności jest to, że hipotezy wysuwane
przez różne strony mają nieraz dość złożoną strukturę. W jaki sposób
rozmaite stwierdzenia i dowody je wspierające powinny być składane w
jedność i oceniane?

Probabilistyczne narzędzia fragmentarycznej oceny dowodów naukowych i
narzędzia pomagające dostrzegać błędy probabilistyczne są całkiem dobrze
rozwinięte. Jednakże, konstrukcja bardziej ogólnego modelu
umieszczającego takie dowody w szerszym kontekście całej sprawy,
przydatnego dla teoretyzowania o ocenie dowodów i stadnardach
dowodowych, pozostaje wyzwaniem. Projekt zamierza przyczynić się do
dalszego rozwoju tego przedsięwzięcia polegając na motywacjach
pochodzących z różnych źródeł: epistemologii formalnej, rozwoju sieci
bayeisańskich, praktyki oceny dowodów probabilistycznych i zastrzeżeń
sformułowanych przez krytyków probabilizmu prawnego.

Punktem wyjścia będzie reprezentacja narracji za pomocą sieci
bayesiańskich z dodatkowymi informacjami na temat tego, które elementy
odpowiadają dowodom, a które narracjom. Kluczowy pomysł polega na tym,
że budując na takich sieciach można wyeksplikować różne własności
narracji wymagane prez krytyków (takie jak koherencja, odporność,
brakujące dowody, wyjaśnianie dowodów, czy sposoby radzenia sobie z
mnogością proponowanych narracji).

\end{document}
