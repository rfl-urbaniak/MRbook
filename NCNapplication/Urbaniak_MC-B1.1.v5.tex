

\documentclass[11pt, a4paper]{article}
%\usepackage{fontspec}
%\linespread{1.2}


%This is for comments on margins
\usepackage[textsize=tiny]{todonotes}


\usepackage{pgfgantt}
\definecolor{barblue}{RGB}{153,204,254}
\definecolor{groupblue}{RGB}{51,102,254}
\definecolor{linkred}{RGB}{165,0,33}
%bibstuff

%\usepackage{natbib}


% DOCUMENT LAYOUT
\usepackage{geometry}
\geometry{top=20mm, bottom=20mm, left=2.5cm, right=2cm,marginparsep=2pt, marginparwidth=.8in}
%\setlength\parindent{10mm}
\usepackage{fancyhdr}
\usepackage{lastpage}
\pagestyle{fancy}
\fancyhf{}
 \renewcommand{\headrulewidth}{0pt}
\rfoot{Part B1 -  Page \thepage \hspace{1pt} of \pageref{LastPage}}
\fancyhead[C]{PhAIee, Standard EF}
\fancyhead[R]{Rafal Urbaniak}


\usepackage{longtable}
% FONTS
%\usepackage[usenames,dvipsnames]{color}
%\usepackage{xunicode}
%\usepackage{xltxtra}
%\defaultfontfeatures{Mapping=tex-text}
%\setromanfont [Ligatures={Common}, Numbers={OldStyle}, Variant=01]{Linux Libertine O}
%\setmonofont[Scale=0.8]{Monaco}

% ---- CUSTOM COMMANDS
%\chardef\&="E050
%\newcommand{\html}[1]{\href{#1}{\scriptsize\textsc{[html]}}}
%\newcommand{\pdf}[1]{\href{#1}{\scriptsize\textsc{[pdf]}}}
%\newcommand{\doi}[1]{\href{#1}{\scriptsize\textsc{[doi]}}}
% ---- MARGIN YEARS

\usepackage{marginnote}
%\newcommand{\amper{}}{\chardef\amper="E0BD }
\newcommand{\years}[1]{\marginnote{\normalsize #1}}
\renewcommand*{\raggedleftmarginnote}{}
\setlength{\marginparsep}{7pt}
\reversemarginpar

\usepackage{lmodern}

\usepackage{booktabs}
\usepackage{multirow}
\usepackage{multicol}
\usepackage{fancybox}



% PDF SETUP % ---- FILL IN HERE THE DOC TITLE AND AUTHOR
%\usepackage[dvipdfm, bookmarks, colorlinks, breaklinks,
%pdftitle={Rafal Urbaniak - cv}, 	pdfauthor={Rafal Urbaniak},
%	pdfproducer={http://nitens.org/taraborelli/cvtex}]{hyperref}
%\hypersetup{linkcolor=blue,citecolor=blue,filecolor=black}

\usepackage{sectsty}
\sectionfont{\vspace{-2mm} \normalsize  \textbf}
\subsectionfont{\normalsize  \itshape \textbf} 	


\usepackage{enumitem}

\usepackage{kpfonts}


%\usepackage{hyperref}
%\usepackage{bibentry}
%\usepackage[style=verbose]{biblatex}
%\addbibresource{/Users/rafal/Documents/DOCS/works/Filmat9,/Users/rafal/Documents/Papers/Papers.bib}


%\usepackage{natbib}
%\usepackage{csquotes}
%\usepackage[style=verbose-ibid,backend=bibtex]{biblatex}
%\usepackage[style=footnote-dw]{biblatex}
%\bibliographystyle{apalike}
\usepackage[backend=biber,style=verbose-trad1]{biblatex}
%\usepackage[backend=biber,autocite=footnote,style=authortitle-ibid]{biblatex}
\bibliography{/home/rafal/GoogleDrive/Papers/papers11}



\usepackage[pdftex,bookmarks=true,bookmarksnumbered=true,bookmarksopen=true,bookmarksopenlevel=3,citecolor=black,filecolor=black,linkcolor=blue,urlcolor=black]{hyperref}


% DOCUMENT

\usepackage{url}




\begin{document}

\begin{center}
\Large{Philosophically enriched AI tools for evidence evaluation}
\end{center}

\vspace{-6mm}



\section{Excellence}

\vspace{-4mm}




\subsection{Quality and credibility of the research/innovation project; level of novelty, appropriate consideration of inter/multidisciplinary and gender aspects}


\vspace{-2mm}

\paragraph{Introduction}

As many miscarriages of justice indicate, scientific evidence is easily misinterpreted in court. This happens mostly because of the communication gap between the parties involved.  For this reason, methods are needed to  support   proper assessment of evidence in such contexts.

The \textbf{assessment of evidence} in the \textbf{court of law} can be viewed from at least three  perspectives: as an interplay of \textbf{arguments}, as an assessment of  \textbf{probabilities} involved, or as an interaction of competing \textbf{narrations}. Each  perspective has been developed into a full-blown account of  \textbf{ legal reasoning}.\footcite{di2018evidential,vanEemeren2017}

While individually each of these strains has been investigated, exploration of  the \textbf{relations between them} is in a rather early stage.
The goal of this project is to combine ideas from informal and  Bayesian epistemology and AI to contribute to this development  by obtaining a \textbf{unifying extended probabilistic model embracing key aspects of the narrative and argumentative approaches, susceptible to AI implementation.}  



%The approaches have been subjected to various degrees of formalization. The argumentative approach employs techniques from \textbf{non-classical logics} and Artificial Intelligence. The probabilistic approach not only uses the \textbf{standard probabilistic toolkit}, but also makes the practical use of the framework more viable by employing the so-called \textbf{Bayesian networks} (BN). The proponents of the narrative approach are split between those who abandon formal methods and find \emph{informal epistemological considerations} more fruitful, and those who try to make sense of \textbf{narrations in a formal setting}, especially in terms of BN. 


Because (i) scientific evidence has probabilistic form, (ii) the probabilistic tools are so well-developed, and (iii) Bayesian epistemology is such a fruitful field, I  take the probabilistic approach as the point of departure. I have already developed 
 a conciliatory  probabilistic framework accommodating some elements of the  narration approach,\footcite{Urbaniak2017Narration-in-ju} and argued\footcite{Urbaniak2017} that the framework handles at least some of the philosophical difficulties involved.\footcite{Urbaniak2018} The next step is to pursue the goals from the first bundle:


\vspace{-3mm}

\begin{center}
\begin{tabular}{p{2.3cm}|p{13.2cm}}
\footnotesize \textbf{$\mathtt{Bundle \,\, 1}$} \newline  \tiny Philosophical \&  formal \newline  unification & 
Obtain a unifying extended  probabilistic framework by incorporating further insights  from philosophical and psychological accounts of legal narrations, and from the argumentation approach. Defend its philosophical plausibility.
\end{tabular}
\end{center}

\vspace{-3mm}


Existing AI approaches to narrations (developed mostly in Groningen at the host institution ) resulted in a practically useful  \textbf{Bayesian network modeling method for legal narrations (BNM)},\footcite{ bex2010hybrid,bex2013legal,vlek2013modeling,vlek2014building,verheij2014catch,verheij2015arguments,vlek2016stories,verheijproof2017} but -- from the informal perspective -- this is achieved at the price of not capturing certain essential aspects of legal narrations.  My framework does capture some of these aspects, but it is rather non-standard, is not related to existing AI methods in argumentation approach, and no BNM  for it exists. Hence, the second bundle of goals:


\vspace{-1mm}


\begin{center}
\begin{tabular}{p{2.3cm}|p{13.2cm}}
\footnotesize \textbf{$\mathtt{Bundle \,\, 2}$} \newline  \tiny AI implementation 
 & Develop BNM for the obtained formal framework, so that the insights from the argumentation approach and informal epistemology, mediated through it, can be incorporated in AI tools.
\end{tabular}
\end{center}


\vspace{-3mm}


\noindent Finally, the obtained tools should be tested against real cases:

\vspace{-3mm}


\begin{center}
\begin{tabular}{p{2.3cm}|p{13.2cm}}
\footnotesize \textbf{$\mathtt{Bundle \,\, 3}$} \newline  \tiny    Case studies & 
Evaluate the developed framework and AI tools  by conducting case studies from its perspective.  
\end{tabular}
\end{center}

\vspace{-3mm}


First, I'll give a few examples of philosophical problems with the standard probabilistic approach to legal evidence evaluation. Next, I'll describe the informal narration view and its \linebreak probabilistic analysis as pioneered by  Di Bello. This will allow me to roughly describe the current shape of the extended formal framework that I developed. Then I will explain how narrations (understood differently) are modeled by means of \textbf{Bayesian Networks (BN)} and why this method doesn't work for narrations as understood in my framework. This will allow me to present the overview of the action.







%Provide an introduction, discuss the state-of-the-art, specific objectives and give an
%overview of the action.
%Discuss the research methodology and approach, highlighting the type of research /
%innovation activities proposed.
%Explain the originality and innovative aspects of the planned research as well as the
%contribution that the action is expected to make to advancements within the research
%field. Describe any novel concepts, approaches or methods that will be implemented

%Discuss the interdisciplinary aspects of the action (if relevant).
%Discuss the gender dimension in the research content (if relevant).In research activities
%where human beings are involved as subjects or end-users, gender differences may
%exist. In these cases the gender dimension in the research content has to be addressed
%as an integral part of the proposal to ensure the highest level of scientific quality.



\vspace{-5mm}


\paragraph{Legal probabilism \& its philosophical difficulties}


Degrees of belief, from the perspective of \linebreak Bayesian epistemology,\footcite{bovens2004bayesian,bradley2015critical}  are to be  identified with, or at least modeled as functions from propositions into the interval $[0,1]$ satisfying the usual axioms of probability theory. This allows Bayesian philosophers to help themselves to a rich array of established mathematical results to approach various philosophical problems.
But is Bayesian epistemology indeed adequate and useful? 


An important context in which this issue is painfully relevant is the philosophical reflection on legal fact-finding. After all, in the court of law, fact-finders have to rely on uncertain evidence in reaching a binary decision of serious consequences.  Clearly, as far as (1) scientific  or statistical evidence (such as DNA identification or epidemiological studies) and its strength assessment and (2) avoiding errors of judgment in explicit reasoning about probabilities are involved, the use of Bayesian me\-thods and proper reflection on them have been extremely useful.\footcite{aitken2004statistics,aitken2008fundamentals,finkelstein2009basic,robertson2016interpreting} 

The question remains  whether Bayesian tools can be used for the construction of a more general model of judiciary fact-finding and whether the judiciary decision standards can be explicated in Bayesian terms.   Ideally, the court convicts  if and only if  the defendant is guilty. In reality,  the best we can hope for is that the conviction is justified by the available evidence.  Can this notion of justification be adequately explicated within the Bayesian framework? Are such explications fruitful? \textbf{Legal probabilism (LP)}  has it that a conviction is justified just in case the probability of guilt given total available evidence is sufficiently high (this is LP in its simplest version;\footcite{Bernoulli1713Ars-conjectandi} other probabilistic explications of the decision standard are available,\footcite{cheng2012reconceptualizing,kaplow2014likelihood,Miller2018} but \emph{mutatis mutandis} the difficulties apply to all of them,\footfullcite{Urbaniak2018} so I stick to this one in what follows).

Quite a few philosophical difficulties with  LP arise.\footcite{Cohen1977The-probable-an} 
 I'll just give two examples.  Imagine that we know that  99 out of 100 people in a certain group participated in   a prison riot, and the sole innocent person can't be identified. Should a random member of the group be penalized for participation based on this knowledge? What if it was 9999 out of 10000? 99999 out of 100000? Many people have the intuition that conviction based only on such statistical evidence would not be just. But if high probability of guilt is not enough, what is? And how can this factor be accounted for in terms of Bayesian epistemology? 

The issue is less speculative than it might seem. Ultimately, all  forensic evidence that has to do with suspect identification, such as DNA identification, is  of statistical nature. Yet, DNA evidence is sometimes considered sufficient for conviction. What is the difference between   statistical evidence sufficient for conviction and otherwise? Can this difference be explained (or explained away) in a principled manner from the Bayesian perspective?


Here's another example of a philosophical difficulty, called the difficulty about conjunction. Say we decide to convict just in case the probability of guilt given total evidence is above a certain threshold $t$. Given that such decisions are to be made with uncertainty, $t$ should be close to 1, but not identical to 1. Now, say the claim to be proven is a conjunction of (probabilistically) independent claims $A$ and $B$, as it well may be. Do we require $\mathtt{P}(A $\&$ B)>t$, or do we require $\mathtt{P}(A)>t$ and $\mathtt{P}(B)>t$ (these conditions are not equivalent (because $\mathtt{P}(A $\&$ B)=\mathtt{P}(A)\times \mathtt{P}(B)$)? 



% \paragraph{The goals of the project} 

% \begin{enumerate}[nosep]
% %	  \setlength\itemsep{-0.1em} 
% \item Systematically evaluate known objections to Bayesian modeling of judiciary fact-finding
% \item Assess the  existing attempts to address them from the Bayesian perspective.
% \item Evaluate the plausibility of  approaches critical to the Bayesian perspective, focusing in particular on the main competitor, the so-called \emph{narrative theory} (investigated extensively at the host institution).

% \item Develop a probabilistic approach to judiciary fact-finding expressive enough to  incorporate various insights provided by these critical approaches (this requires extending the formal probabilistic language with propositional operators, such as \emph{is part of narration} or \emph{is part of evidence}).
% \item Study the expressibility of this extended framework and formulate explications of appopriate conditions on narrations normally formulated only informally (such as resilience, lack of gaps, explaining evidence, etc.).
% \item Deploy this extended probabilistic framework to handle the objections and difficulties put forward by the critics of legal probabilism.
% \item Implement the results to contribute to the ongoing research program at the host institution: development of techniques that  employ  Bayesian networks  to model legal argumentation and scenario construction in judicary fact-finding (this will require identifying the features of Bayesian networks that those new conditions on narrations correspond to) 
%  correspond to when considered.
%  \item Draw the connections between the obtained framework and techniqued used in argumentation diagrams. \end{enumerate}



\vspace{-4mm}



\paragraph{Informal philosophical approaches \& the narration view}






 In light of difficulties that the existing probabilistic approaches  run into, various informal notions  have been claimed to be essential for  a proper explication of judiciary decision standards.\footcite{wells1992naked,haack2011legal,redmayne2008exploring} For instance,  evidence is claimed to be insufficient for conviction if it isn't sensitive to the issue at hand: if it remained the same even if the accused was innocent.\footcite{enoch2015sense} Or, to look at another approach,  evidence is claimed to be insufficient for conviction if it doesn't normically support it: if -- given the same evidence -- no explanation would be needed even if the accused was innocent.\footcite{Smith_conviction_mind_2017}   A Bayesian epistemologist needs either to show that these notions are  unnecessary or inadequate for the purpose at hand, or to indicate how they can be explicated in probabilistic terms. 

Moreover, legal proceedings are back-and-forth between opposing parties in which  cross-examination is of crucial importance,\footcite{Stein2005Foundations-of-}  reasoning goes not only evidence-to-hypothesis, but also hypotheses-to-evidence \footcite{wells1992naked,allen2007problematic} in a way that seems analogous to inference to the best explanation, \footcite{dant1988gambling} which notoriously is claimed to not be susceptible to probabilistic analysis.\footcite{Lipton2004-LIPITT} How is a Bayesian epistemologist to make sense of these phenomena? 



 An informal account inspired by these considerations --    The \textbf{No Plausible Alternative Story (NPAS)} theory\footcite{allen2010no}  -- is that the courtroom is a confrontation of competing narrations\footcite{wagenaar1993anchored,ho2008philosophy} offered by the sides, and  the narrative to be selected should be the most plausible one. The view is conceptually plausible\footcite{di2013statistics} and finds  support  in psychological evidence.\footcite{pennington1991cognitive,pennington1992explaining}  NPAS  is argued to be immune to the philosophical difficulties faced by  the probabilistic approach -- for instance, the gatecrasher evidence provides no narration of what happened, and so, from this perspective, conviction is not justified.

 
This, however, doesn't happen without a cost. From the perspective of a formal epistemologist,  the key disadvantage of NPAS is that it abandons the rich toolbox of probabilistic methods and takes the key notion of plausibility to be a primitive notion which should be understood only intuitively. 

\vspace{-6mm}

\paragraph{New Legal Probabilism  (NLP) and its formal counterpart}
 NLP  is an attempt to improve on this underspecificity of NPAS.\footfullcite{di2013statistics} While still being at most semi-formal, the approach is more specific about the conditions that a successful accusing narration is to satisfy for the conviction beyond reasonable doubt to be justified. Di Bello identifies four key requirements that a successful convicting narration should satisfy:

\vspace{-2mm}

\begin{center}
\begin{tabular}{@{}lp{11.5cm}@{}}
\toprule
 (Evidential support) &The defendant's guilt probability on the evidence should be sufficiently supported by the evidence, and a successful accusing narration should explain the relevant evidence. \\
(Evidential completeness) &  The evidence available at trial should be complete as far as a reasonable fact-finders' expectations are concerned. \\
(Resiliency)&  The prosecutor's narrative, based on the available evidence, should not be susceptible to revision given reasonably possible future arguments and evidence. \\
(Narrativity) & The narrative offered by the prosecutor should answer all 
the natural or reasonable questions one may have about what happened, given the content of the prosecutor's narration and the available evidence. \\
\bottomrule
\end{tabular}
\end{center}

\vspace{-2mm}


 \emph{Prima facie,} it is far from obvious that such conditions are susceptible  to a Bayesian explication. However, I have already developed  a more expressive probabilistic framework (call it \textbf{Narration-Friendly Probabilism (NFP)}) capable of expressing such features within a formalized language.\footfullcite{Urbaniak2017Narration-in-ju} On NFP, \textbf{the notion of narration is quite wide}: narrations not only contain factual statements about what happened, but also claims about evidence, about narrations,  about relations between evidence and various parts of various narrations etc. I extend the basic \textbf{propositional language  with propositional operators} $N_i$ and $E$ corresponding to ``\emph{\dots is part of narration $i$}'' and ``\emph{\dots is part of the evidence},'' and  model  narrations as finite sets of sentences from this language. Due to this intuitive move, many important aspects of narrations normally discussed only informally, similar to those discussed by Di Bello, become expressible in terms of probabilistic measures for such a formal language.  Let's very briefly gesture towards a few examples:
 
\vspace{-4mm}

 \begin{itemize}\setlength\itemsep{-1.5mm}
 \item A defending narration explains  a piece of evidence  $e$ just in case  if there is an  attacking narration whose posterior is raised conditional on $e$, the probability of $e$ conditional on this defending narration is above the negligibility threshold.
 \item An attacking narration misses some evidence just in case there are some statements not in the evidence set such that the probability of the claim that at least one of them is part of evidence conditional on the existing evidence ($\{\varphi\vert \varphi \in \mbox{ Evidence}\}$), its description ($\{E\varphi\vert \varphi \in \mbox{ Evidence}\}$), and on this attacking narration is above the strong plausibility threshold. 
 \item A narration contains gaps just in case there are some claims which are not part of it, but conditional on the content and the description of this narration and the evidence available, their disjunction is strongly plausible, and it is strongly plausible conditional on the content (but not on the description) of this narration and on evidence that at least one of these claims is part of the narration.
 \item  A narration is dominant just in case it doesn't miss any evidence, it doesn’t contain any gap, and in light of all available information and evidence it is at least as likely any other accusing narration, and is strongly plausible.
 \end{itemize}

\vspace{-4mm}

 While the relation of NFP to the gatecrasher paradox has been briefly discussed,\footfullcite{Urbaniak2017} a full assessment of how it relates to the philosophical difficulties with LP is still missing. It is also not clear how NFP can be captured in terms of BN models, and how it relates to the argumentative approach.


\vspace{-5mm}

\paragraph{Narrations \& Bayesian Networks} The AI strain of research on legal narrations employs a \textbf{different notion of a narration} (compared to the one used in NFP), on which it is  a description of events as they unfolded.   \textbf{Research performed mostly  at the proposed host institution (University of Groningen)}   resulted in  formal ways of modeling reasoning with narrations in this  sense, including a way of representing narrations by means of fairly specific BNs called \textbf{legal idioms} (see footnote 5 for references). %\footcite{hepler2007object,bex2010hybrid,bex2013legal,vlek2013modeling,vlek2014building,verheij2014catch,vlek2016stories,verheijproof2017}
   Let's take a brief look at how this proceeds. 


A BN represents a joint probability distribution over a collection of variables and consists of a graph expressing the connections between variables in the domain and probabilistic dependencies between them. Roughly, it's a directed acyclic graph in which nodes correspond to variables,  each parentless node comes with an assignment of prior probabilities to various values of a given variable, and each connection corresponds to a certain assignment of conditional probabilities to combinations of values of the connected variables.  


Of course, the class of relevant variables to be included in considerations of a legal case is not easily predictable. However, certain types of events and dependencies are quite common and researchers have developed a list of those, called legal idioms,\footcite{fenton2013general} bringing  some consistency and repeatability to the table. On this approach, \textbf{narrations are instantiations legal idioms} and consist of elements connected in  certain at least partially predictable fashion.

 To give an extremely simple example, take a very general  idiom  \verb|killing|  (see the illustration on page \pageref{obrazki}). Usually, a killing has a \verb|motive|, is done using a \verb|weapon|, involve the \verb|act| of using it against someone, and leads to the victim's \verb|death|. In an idiom, these elements are all connected to the scenario node (in this case, \verb|killing|), and their conditional probability given the scenario node is 1. Moreover, a scenario provides a decision about the guilt of the accused. Quite importantly, idioms can also include information about what types of evidence should be expected -- but this is beyond the scope of this example. 

 Such an idiom can be instantiated in multiple ways --  for clarity I treat particular values of variables as individual nodes. On one narration,  David is attacked by Goliath, David has a sling and shoots the attacking Goliath in front of  the head.
On another, David wants to  kill Goliath because he sees him as a competitor in a twisted love affair, and sneakily shoots him in the back of his head. These two narrations result in two separate instantiations of the \verb|killing| \, idiom. 


Now, one beautiful aspect of treating scenarios in terms of Bayesian networks is that once evidence, say witness' testimonies and coroner's report, becomes available, these narrations can be considered in a single Bayesian network that includes the evidence (highlighted in the second illustration on the next page). When it comes to probabilistic treatment of the notion of narration, this is a clear step forward.  The unity of a narration, its relation to evidence, and  dependencies between its elements are represented in a way that's not only fairly intuitive, but also technically correct,  and susceptible to further probabilistic analysis. 


However, from the perspective of NPAS and NLP, various desirable  properties of  narrations  can be defined only with explicit reference to evidence, to narrations, and to arguments they contain. The fact that the AI approach employs a more event-oriented notion of narration in BNM, while it makes modeling easier,  also  makes it quite difficult to see how the evaluation of such aspects and key elements of NFP are to be implemented in AI tools for modeling legal narrations. 

 Another issue is that BNM doesn't square too easily with the argumentative approach, in which diagrammatic techniques are used for the \mbox{analysis} of the constellation of evidence, reasons and hypotheses involved in a case.\footcite{dung1995acceptability,prakken2010abstract}  While certain translations between Bayesian networks and argumentation diagrams exist, the sets of features that can be naturally modeled in these frameworks are quite different.

\vspace{-1mm}

\hspace{-16mm} \label{obrazki}
\includegraphics[width=18.5cm,height=11cm]{Idiom10.png}

\vspace{-5mm}

\hspace{-16mm}
\includegraphics[width=18.5cm,height=11cm]{total3.png}

\vspace{-3mm}

\noindent 

% Another important feature of a narration that's hard to represent  by means of a Bayesian network is completeness, called also \emph{coherence}. On this approach,\footcite[18]{verheij2015arguments} a narration, as represented by a Bayesian network, is considered complete if it has all the nodes predicted by its idiom. This, however, means that most of the heavy lifting here is done informally: the construction and selection of an appropriate idiom is something that clearly isn't to be judged from within a Bayesian network, but rather from reflecting on it informally, and whatever knowledge is needed to reach such decisions. 

% Yet another key notion that needs to be modeled in such contexts is that of reasonable doubt. On the approach, \footcite[28]{verheij2015arguments} reasonable doubt arises as soon as the probability of a convicting scenario given the evidence is less than 1.  In a later paper,\footcite{verheijproof2017} lack of reasonable doubt regarding a claim ultimately boils down to all scenarios entailing it. This means that narrations that could have raised a reasonable doubt have been somehow excluded from the model.  But this, again, makes judgment about reasonable doubt something that is not modeled \emph{within} the framework, but rather achieved by informal reflection on a model. Moreover, the approach seems a bit too stringent: after all, real convictions almost never result from the probability of a convicting narrative given evidence being simply 1. Some possibility of error always remains, and it should not be identified with  the existence of reasonable doubt about this narrative. 



%The approach, however, has its weaknesses.


 % For instance, \emph{undercutting}, where one argument attacks the inferential step of an argument (as opposed to denying its conjunction) is hard to represent in a Bayesian network. One might say that it is similar (at best) to the situation in which conditionalizing on an additional variable decreases a given conditional probability, but even this feature is rather complex and cannot be easily seen when one looks at a BN.









\vspace{-5mm}


\paragraph{Overview of the action}\label{section:packages}\label{overview}
 
Work on the project will be divided into the following partially overlapping bundles (see the gantt chart in section 3.1 for timing). 


\begin{itemize}\setlength\itemsep{-1.5mm}

\vspace{-2mm}

\item \verb|Bundle 1| (10 months) will be devoted to philosophical issues and developing further formalization techniques. %\footfullcite{Urbaniak2017Narration-in-ju} 
This involves investigating informal approaches to legal narrations other than NPAS and NLP,\footfullcite{wagenaar1993anchored,ho2008philosophy} and extending NFP to accommodate the philosophical insights they provide. The second part  of the bundle requires an in-depth and complete study of philosophical difficulties that LP runs into, and  addressing them from the perspective of NFP.  The third part of the bundle will require incorporating insights from argumentation theory in the framework.\footcite{FransvanEemeren2014,PietroBaroni2018} %\footcite{prakken1997logical,prakken2008formalising,prakken2001logics,twining2007argumentation}  
In particular, it will be essential whether the key notions employed in the argumentative framework can be explicated in NFP, and whether these explications can be further implemented in BNM.   

%Accordingly, the planned outcome is two papers devoted to these issues respectively, published in top academic journals such as \todo{add}
% The first 7 months will not only allow me to develop a complete philosophical story about NFP, but also to acquire AI skills necessary for the work on \verb|Bundle 2|. 

\item \verb|Bundle 2| (12 months) is the core of the project. Here, together with prof. Verheij and a few BA and MA students,  I will work on implementing the insights obtained from \verb|Bundle 1| in the AI setting. This will require expanding the theoretical approach to BNM to accommodate the extensions required by NFP, and working out the details of how the result is to be implemented (my language of preference is \textsf{R}, but it is quite likely that I will need to learn and use the tools already used in Groningen).


%I estimate that the first part of the bundle will result in two papers published in journals such as \todo{aaaa}, while the last part will lead to a more programming-oriented publication in a journal such as \todo{aaa}.
\item \verb|Bundle 3| (11 months) tests the obtained framework against case studies. A larger initiative involving reconstructing various cases using different representation methods and comparing the representations, called \emph{Probability and statistics in forensic science}, took place at the Isaac Newton Institute for Mathematical Sciences, and  researchers in Groningen have experience in using the methodology.   
\end{itemize}
\vspace{-2mm}



%specific objectives and give an
%overview of the action.

%miscommunication?

\vspace{-8mm}

\paragraph{Methodology}  
%Standard arguments for the legitimacy of Ba\-ye\-sia\-nism\footnote{See for example \cite{earman1992bayes,Urbach1993-HOWSRT} for an early yet fairly comprehensive survey, or \cite{Pettigrew2011Epistemic-Utili} for a discussion of more recent contributions. See also \cite{Swinburne2001-SWIEJ,bovens2004bayesian,bradley2015critical}.}   deploy usually rather  abstract pieces of reasoning to the effect that if one's degrees of beliefs satisfy certain  conditions, they also have to satisfy the probabilistic requirements.  My approach to thinking about the plausibility of Bayesian epistemology is rather unlike such approaches. Instead, I prefer the \emph{proof-of-the-pudding}  methodology.  I am convinced that an important part of the philosophical assessment of the Bayesian research program has to do with  its achievements or failures in contributing to debates in philosophy which are not themselves debates about the status of Bayesianism itself. In particular, it would be great news if insights from Bayesian epistemology could be used to further development of forensic AI and deepening our understanding of judiciary decision making.

In this research I plan not only to use  methods used in formal philosophy (conceptual analysis and the  use of probabilistic tools), but also to help myself to tools made available by forensic AI (Data Analysis, Bayesian Networks and Probabilistic Graphical Models in general, methods developed in Groningen in particular), and to include an empirical factor by using case studies to  test how well the developed framework can  model real cases, and how the obtained representation is related to the ones obtainable by means of existing representation methods. 




\vspace{-6mm}

\paragraph{Novelty \& interdisciplinarity} 

 It is unusual to try to evaluate Bayesian epistemology relying on factors other than the intuitions of formal philosophers. From the perspective of philosophy, it is only recently that researchers attempt  to connect considerations from formal philosophy with methods in AI as applied to forensics.  The underlying idea of the project is that somewhat abstract considerations, when taken seriously, would lead to further improvement of  the existing methods in forensic AI developed in Groningen, and that more direct ``contact with reality'' and immersion in AI would have positive impact on those abstract considerations as well. 



\vspace{-4mm}

\subsection{Quality and appropriateness of the training and of the two way transfer of
knowledge between the researcher and the host}


\vspace{-3mm}

\paragraph{Import of the researcher}


%Outline how a two-way transfer of knowledge will occur between the researcher and
%the host institution(s):
%o Explain how the experienced researcher will gain new knowledge during the
%fellowship at the hosting organisation(s).
%o Outline the previously acquired knowledge and skills that the researcher will
%transfer to the host organisation(s).


I have extensive experience in analytic philosophy, conceptual analysis, philosophical logic (including non-monotonic logics, such as adaptive logics which have been developed at  Ghent University, where I spent quite a few years),  and probabilistic and decision-theoretic methods as deployed in philosophical contexts. Since I  would be hosted by the department of Artificial Intelligence at the Bernoulli Institute for Mathematics, Computer Science and Artificial Intelligence at the University of Groningen, where researchers come mostly from mathematical and engineering background, these skills  by no means would be common in the environment. The idea is that interacting with a more philosophically-minded and yet mathematically competent researcher might provide researchers at the hosting institution with fresh perspective on some of the problems they are working with. The host, prof. Verheij, is very open to interaction and cooperation (he already successfully cooperates with philosophers, such as Marcello Di Bello, or Jan-Willem Romeijn).




 


\paragraph{Training of the researcher}

 I would be training through research under the guidance of prof. Bart Verheij and other members of the research staff. Prof. Verheij is a key authority in formal argumentation theory,  which is the framework that is least familiar to me, and the plan is to learn from him about it.

  However, what would particularly make it a learning experience for me is how different this environment would be from my usual academic framework. Surrounded by engineers and computer scientists I will be quickly forced to get real about the strengths and limitations of the framework under development. Being asked tricky questions that only a computer scientist or engineer would ask is an excellent way of preserving philosophical mental health and a superb motivation for caring about the import of the results obtained to domains other than philosophy itself. 


Moreover, my experience in computer science is rather limited. I know the basics of \verb|PROLOG| and can work with  statistics and basics of Bayesian networks in \textsf{R} (mostly, \verb|bnlearn|), but that's it. Groningen offers excellent undergraduate and master's level programs in AI (for years it's been ranked 1st or 2nd in the country) and I intend to take  courses relevant to my research project in preparation for the contribution to the development of the AI tools based on NFP (this is part of the reason why \verb|Bundle 2| starts only a few months into the project -- at the time of working on \verb|Bundle 1| I would also be acquiring AI skills necessary to proceed further). These courses  include (but are not limited to, the choice depends on availability): 
\emph{Programming in C/C++, Introduction to Information Systems, Introduction to Intelligent Systems, Introduction to AI, Artificial Intelligence, Cognitive Modelling}, and \emph{Arguing Agents}. Having obtained some understanding of the AI aspect, my learning experience would be further improved by interaction with BA and MA students cooperating with me and prof. Verheij. 




%Describe the training that will be offered. Typical training activities in Individual
%Fellowships may include:
%o Primarily, training-through-research by the means of an individual
%personalised project, under the guidance of the supervisor and other members
%of the research staff of the host organisation(s)
%o Hands-on training activities for developing scientific skills (new techniques,
%instruments, research integrity, 'big data'/'open science') and transferable skills
%(entrepreneurship, proposal preparation, patent applications, management of
%IPR, project management, task coordination, supervising and monitoring, take
%up and exploitation of research results)
%Inter-sectoral or interdisciplinary transfer of knowledge (e.g. through
%secondments)
%Participation in the research and financial management of the action
%Organisation of scientific/training/dissemination events
%Communication, outreach activities and horizontal skills
%Training dedicated to gender issues

\vspace{-5mm}

\subsection{Quality of the supervision and of the integration in the team/institution}

\vspace{-2mm}

My host will be prof. Bart Verheij, who holds the Chair of Artificial Intelligence and Argumentation and is the head of the Department of Artificial Intelligence at the University of Groninen. He is also the president of the International Association for Artificial Intelligence and Law and affiliated faculty at CODEX, The Stanford Center for Legal Informatics. 

In multiple publications (which I have already cited, his \emph{h}-index is 30), prof. Verheij studies the logic of argumentation, with special focus on connecting qualitative and quantitative methods in the analysis and representations of reasoning in forensics and in law. 

In years 2012-2017 (this period included a 2013-2014 research fellowship at the Stanford Center for Legal Informatics) prof. Verheij led an NWO (Netherlands Organisation for Scientific Research) forensic science project \emph{Designing and Understanding Forensic Bayesian Networks with Arguments and Scenarios}.  He and his team studied the already described connections between narrations and Bayesian networks.  It is precisely their results that my research intends to focus on, either by developing them further, or by modifying the proposed framework itself in light of the insights of the members of prof. Verheij's group.  

%1.3 Quality of the supervision and of the integration in the team/institution
%Describe the qualifications and experience of the supervisor(s). Provide information
%regarding the supervisors' level of experience on the research topic proposed and their
%track record of work, including main international collaborations, as well as the level
%of experience in supervising/training especially at advanced level (PhD, postdoctoral
%researchers). Information provided should include participation in projects,
%publications, patents and any other relevant results.


%Describe the hosting arrangements. 2 The application must show that the experienced
%researcher will be well-integrated within the team/institution so that all parties gain
%maximum knowledge and skills from the fellowship. The nature and the quality of the
%research group/environment as a whole should be outlined, together with the measures
%taken to integrate the researcher in the different areas of expertise, disciplines, and
%international networking opportunities that the host could offer.


%https://www.rug.nl/staff/l.c.verbrugge/research/publications.html
%http://www.staff.science.uu.nl/~renoo101/Prof/Edu/index.html

\vspace{-5mm}

\subsection{Potential of the researcher to reachor re-inforce professional maturity/independence during the fellowship}

\vspace{-2mm}

%1.4
%Potential of the researcher to reach
%maturity/independence during the fellowship

In my career so far I have published on multiple topics. I started with my PhD  on non-standard foundations of mathematics and the philosophy of mathematics. After finishing my PhD I published papers on mathematical nominalism, philosophy of language, non-classical logics, formal theories of truth, history of logic, philosophy of science and formal epistemology. Over the last couple of years my interests turned to a topic where I hope my results can make a  real difference in applying philosophy and formal methods to real-life problems.

%Please keep in mind that the fellowships will be awarded to the most talented
%researchers as shown by the proposed research and their track record (Curriculum
%Vitae, section 4), in relation to their level of experience.

\vspace{-3mm}


\section{Impact}


\vspace{-4mm}

%2. Impact


\subsection{Enhancing the future career prospects of the researcher after the fellowship}


\vspace{-3mm}

Obtaining real AI skills and experiences resulting from immersion in quite a different setting (as contrasted with a department of philosophy) would widen my perspective and allow me to competently navigate my future research around topics on the border of AI and philosophy. 

I am currently a professor at University of Gda\' nsk (Poland) and I  spent the last ten years -- on and off -- doing my research at the Centre for Logic and Philosophy of Science at the Ghent University in Belgium. By the time of the fellowship by funding in Ghent will have ended. In Poland, for the period of the fellowship I plan to take an unpaid leave while continuing my   research cooperation with colleagues in Gda\' nsk from a distance. The fellowship would allow me to develop a wider collaboration network between Dutch, Belgian (such as Frederik Van De Putte) and  Polish scholars,  to continue cooperation between Gda\' nsk and Groningen after I return to Poland. 

During my stay in Groningen I would not only have the opportunity to work with prof. Bart Verheij, but also   prof. Jan-Willem Romeijn, who is an expert in bayesian epistemology and  works on philosophy of statistics and philosophy of science, and prof. Anne Ruth Mackor at the Faculty of Law.  I will also be able  to interact with other Dutch researchers based in Utrecht who cooperate with prof. Verheij and either visit Groningen quite systematically or are easy to visit once I am in Groningen: prof. Henry Prakken (Utrecht \& Groningen) who works on formal modeling of legal arguments,  dr Silja Renooij (Utrecht), who specializes in Bayesian networks, and dr Floris Bex (Utrecht), whose research focuses on the relation between Bayesian networks and legal narrations and arguments. 

I will also be able to interact with the  ongoing doctoral students of prof. Verheij (especially, Atefeh Keshavarzi) and cooperate with BA and MA students in AI, working on the implementations of the formal framework. 

Also, the plan is to apply for a workshop grant with the Lorentz Center to gather researchers working on related topics. This would allow my results obtained during the fellowship face criticism, but more importantly will make discussion regarding potential joint research grants that would make future collaboration possible and would improve the position of  Gda\'nsk on the academic map of research in AI \& Law. 


%2.1 Enhancing the future career prospects of the researcher after the fellowship
%Explain the expected impact of the planned research and training (i.e. the added value
%of the fellowship) on the future career prospects of the experienced researcher after
%the fellowship. Focus on how the new competences and skills (as explained in 1.4)
%can make the researcher more successful in their long-term career.


\vspace{-5mm}

\subsection{Quality of the proposed measures to exploit and disseminate the project results}\label{sec:papers}


\vspace{-2mm}

The main result would be publications in leading scientific peer-reviewed journals in the domain (some publications, of course, will be co-authored with Groningen researchers). Given the description of the bundles and their timing, I predict that:

\vspace{-4mm}

\begin{itemize}\setlength\itemsep{-1.5mm}
\item \verb|Bundle 1| will result in  two more philosophical papers (\verb|PB1.1| and \verb|PB1.2|), published in  academic journals such as \emph{Synthese}, \emph{Mind} or \emph{Ratio Juris}. 

\item \verb|Bundle 2| will lead to two   technical papers (\verb|PB2.1| and \verb|PB2.2|)  published in journals such as \emph{IfCoLog Journal of Logics and their Applications}, \emph{Law, Probability and Risk} or \emph{Artificial Intelligence and Law}.

\item \verb|Bundle 3|  will lead both to a publication of two papers (\verb|PB3.1| and \verb|PB3.2|) on  how the framework handles case studies in  journals such as \emph{Artificial Intelligence} or \emph{Argument \& Computation}.
\end{itemize}

\vspace{-4mm}


\noindent Apart from publications, the results will be presented at conferences devoted to legal reasoning. These include the yearly conferences of the \emph{International Association for Artificial Intelligence and Law
} and of the \emph{Foundation for Legal Knowledge Based Systems (JURIX)}, and more general conferences gathering formal philosophers, so that the research is inspired by interaction not only by legal evidence scholars, computer scientists, but also philosophers. 
Finally, near the end of the period, I plan to organize a conference devoted to the topic of the research.


\vspace{-5mm}



\subsection{Quality of the proposed measures to communicate the project activities to different target audiences}

\vspace{-3mm}

While in the Netherlands, I would be happy to engage with the North Sea Group of evidence scholars, founded earlier this year by scholars from the Netherlands, Norway, Sweden and the UK with a common interest in theoretical approaches to legal evidence. The organizing committee comprises: Anne Ruth Mackor (Groningen), Eivind Kolflaath (Bergen), Christian Dahlman (Lund) and David Lagnado (University College London). 

Since 2008 I have been running a blog, \verb|Entia| \verb|et| \verb|Nomina| devoted to formal philosophy. For such a narrow field it's quite often visited (at the time of  writing, 1388 pageviews last month). I have been slowly reshaping it to fit my most recent interests, but the plan is to proceed at full speed once the fellowship starts, releasing popularizing posts on the topic of evidence scholarship once every two weeks. If opportunity arises, I also plan to arrange a couple of pub quiz nights concerned with a mixture of mathematics, epistemology and forensic science. Over the last few years I have quite systematically visited Polish high schools  teaching philosophy classes on formal philosophy and engaging youth in formal philosophy. In the Netherlands, if possible, I would love the opportunity to engage youth at Dutch high schools as well and will directly contact local high schools pursuing this idea.  

%2.3. Quality of the proposed measures to communicate the project activities to
%different target audiences
%Demonstrate how the planned public engagement activities contribute to creating
%awareness of the performed research. Demonstrate how both the research and results
%will be made known to the public in such a way they can be understood by non-
%specialists.
%The type of outreach activities could range from an Internet presence, press articles
%and participating in European Researchers' Night events to presenting science,
%research and innovation activities to students from primary and secondary schools or
%universities in order to develop their interest in research careers.



%!!
%Concrete planning for communication activities must be included in the Gantt chart.

\vspace{-3mm}


\section{Quality and Efficiency of the Implementation}

%3.
%Quality and Efficiency of the Implementation

\vspace{-3mm}

\subsection{Coherence and effectiveness of the work plan, including appropriateness of
the allocation of tasks and resources}

\vspace{-3mm}

Work packages and their timing has already been described and justified in \textbf{Overview of the action}, the  papers planned have been  described in \emph{Section \ref{sec:papers}}.  The assumption is that the project begins with September 2019 (but this plan is flexible).

\vspace{-5mm}


\hspace{-7mm}
\begin{ganttchart}[
y unit chart=0.6cm,
x unit =0.45cm,
    canvas/.append style={fill=none, draw=black!15, line width=.75pt},
    hgrid style/.style={draw=black!25, line width=.75pt},
    vgrid={*1{draw=black!40, line width=.75pt}},
 %   today=7,
 %   today rule/.style={
 %     draw=black!64,
 %     dash pattern=on 3.5pt off 4.5pt,
 %     line width=1.5pt
 %   },
%    today label font=\small\bfseries,
    title/.style={draw=none, fill=none},
    title label font=\bfseries\footnotesize,
    title label node/.append style={below=7pt},
    include title in canvas=false,
    bar label font=\mdseries\small\color{black!70},
    bar label node/.append style={left=2cm},
    bar/.append style={draw=none, fill=black!63},
    bar incomplete/.append style={fill=barblue},
    bar progress label font=\mdseries\footnotesize\color{black!70},
    group incomplete/.append style={fill=groupblue},
    group left shift=0,
    group right shift=0,
    group height=.5,
    group peaks tip position=0,
    group label node/.append style={left=.6cm},
    group progress label font=\bfseries\small,
    link/.style={-latex, line width=1.5pt, linkred},
    link label font=\scriptsize\bfseries,
    link label node/.append style={below left=-2pt and 0pt}
  ]{1}{24}

\gantttitle[title label node/.append style={below left=7pt and -3pt}
  ]{Months:\quad1}{1}
  \gantttitlelist{2,...,24}{1} \\
\ganttgroup[
%progress=57
]{\tiny B1: Philosophical \&  formal   unification}{1}{10} \\

%\ganttbar[]{\tiny Philosophical papers}{5}{6}
%\\

\ganttmilestone{\tiny Paper PB1.1}{5} \\ \ganttmilestone{\tiny Paper PB1.2}{9}
\\

		% \ganttbar[
  % %  progress=75,
  %   name=WBS1A
  % ]{\textbf{WBS 1.1} Activity A}{1}{10} \\


\ganttbar[]{\tiny Workshop in Groningen}{7}{7}\\

\ganttgroup[
%progress=57
]{\tiny B2: AI implementation }{8}{20} \\


\ganttmilestone{\tiny Paper PB2.1}{14} \\ \ganttmilestone{\tiny Paper PB2.2}{19}
\\



\ganttbar[]{\tiny Workshop at the Lorentz Center}{20}{20}\\


\ganttgroup[
%progress=57
]{\tiny B3: Case studies}{7}{10} \ganttgroup[]{}{16}{24}\\


\ganttmilestone{\tiny Paper PB3.1}{10} \\ \ganttmilestone{\tiny Paper PB3.2}{22}
\end{ganttchart}

\vspace{3cm}

%\hspace{-4mm}
%\includegraphics[width=16.5cm,height=6cm]{MCSBUNDLE.png}

%3.1
%Coherence and effectiveness of the work plan, including appropriateness of
%the allocation of tasks and resources
%Describe how the work planning and the resources mobilised will ensure that the
%research and training objectives will be reached. Explain why the number of person-
%months planned and requested for the project is appropriate in relation to the proposed
%activities.
%Additionally, a Gantt chart must be included in the text listing the following:
%o Work Packages titles (there should be at least 1 WP);
%o Indication of major deliverables, if applicable;
%o Indication of major milestones, if applicable;
%o Secondments, if applicable.
%co
%The schedule should be in terms of number of months elapsed from the start of the
%action.


%A deliverable is a distinct output of the action, meaningful in terms of the action’s overall objectives and may be a report, a document, a
%technical diagram, a software, etc. Deliverable numbers should be ordered according to delivery dates. Use the numbering convention <WP
%number>.<number of deliverable within that WP>. For example, deliverable 4.2 would be the second deliverable from work package 4.
%Milestones are control points in the action that help to chart progress. Milestones may correspond to the completion of a key deliverable,
%allowing the next phase of the work to begin. They may also be needed at intermediary points so that, if problems have arisen, corrective
%measures can be taken. A milestone may be a critical decision point in the action where, for example, the researcher must decide which of several
%technologies to adopt for further development.

\vspace{-3.5cm}


\subsection{Appropriateness of the management structure and procedures, including risk
 management}

%3.2
%Appropriateness of the management structure and procedures, including risk
%management
%Describe the organisation and management structure, as well as the progress monitoring
%mechanisms put in place, to ensure that objectives are reached.

\vspace{-2mm}

Progress monitoring will be performed via systematic work with the host and assessment of the results by journal referees and peer academics attending specialized conferences.

One research risk is that it will turn out that some of the informal requirements cannot be spelled out in probabilistic terms, or expressed in terms of properties of Bayesian networks. In such an event, I will turn to studying the reasons for this negative result. It might turn out that there are independent reasons to abandoned a given condition, or it might be the case that probabilistic inexplicability of a given condition is an argument against the probabilistic approach. Either way, finding out which option holds and why would also lead to a deeper understanding of the framework and lead to a publication in academic journals. Another research risk is that the case studies will show that other methods are more efficient or transparent. This would itself constitute a result that could be used to further modify the framework so that its best aspects could be preserved, while the disadvantages discovered during case studies avoided. 


% Discuss the research
%and/or administrative risks that might endanger reaching the action objectives and the
%contingency plans to be put in place should risk occur.
%If applicable, discuss any involvement of an entity with a capital or legal link to the
%beneficiary (in particular, the name of the entity, type of link with the beneficiary and
%tasks to be carried out).

%If needed, please indicate here information on the support services provided by the host
%institution (European offices, HR services...).

\vspace{-3mm}

\subsection{Appropriateness of the institutional environment (infrastructure)}

\vspace{-2mm}

This research doesn't require any special facilities, aside from access to a good library and regular meetings with my host and other researchers, all of which will be provided by the hosting institution. 

\vspace{-2mm}

\paragraph{Gender issues} Both the host and the PI are male. The gender balance of the host institution is good. Out of 15 tenured researchers in AI  at the Bernoulli Institute for Mathematics, Computer Science and Artificial Intelligence, 8 are female, and my research stay will not bring gender imbalance to the Institute. Among the  researchers that I intend to interact with at the University of Groningen  is dr Charlotte Vlek, who has done research on the topic in the past (she is now employed by the University of Groningen, but not at a research position), and prof. Anne Ruth Mackor at the Faculty of Law. Another female researcher I plan to consult is dr  Silja Renooij at the University of Utrecht, specializing in Bayesian Networks. 


%Appropriateness of the institutional environment (infrastructure)
%The active contribution of the beneficiary to the research and training activities should be
%described. For Global Fellowships the role of partner organisations in Third Countries for
%the outgoing phase should also appear.
%co
%Give a description of the main tasks and commitments of the beneficiary and all partner
%organisations (if applicable).
%Describe the infrastructure, logistics, facilities offered insofar as they are necessary for
%the good implementation of the action.

%STOP PAGE COUNT – MAX 10 PAGES



%\bibliographystyle{apalike}
%\bibliography{/Users/rafal/GoogleDrive/Papers/Papers9}














%TOPRINT:
%SHEN
%KEPPENS 2012
%Timmer Meyer
%Crupi
%pardo juridicial proof
%Dawid 2010 beware of the DAG








%toread verheji
%- broeders 2009 - communication gap - decision making in the forensic arena
% - Kaptein 2009 - argumentative
% - dawid 2011 - narrative   Evidence, Inference and Enquiry book et al
% - Anderson 2005 - probabilistic  - Analysis of Evidence book
% pollock 1995, cognitive carpentry
%Dung 1995, On the acceptability of arguments and its fundamental role
%Prakken 2010, An abstract framework for argumentation with structured arguments
% debate in Science and Justice vols 51 52
%bex 2011 connection with argumentative and narrations  - arguments, stories... book


%Bex 2010
%- Aitken Taroni 2004 textbook
% jensen nielsen 2007 for BN
% - Shen 2006  connection of narratives with prob - A scenariodriven decision support system for serious crime investigation, Law, Probability and Risk   %JEST PDF

% Keppens 2012 BN and  argument diagram extraction
% Fenton 2013 BN risk assessment
%Hepler 2007 BN and scenarios  Object-oriented grapphical representations of complex patterns of evidence, Law Probability and Risk
%Timmer, Meyer, Inference and attack in Bayesian Networks
%Wagenaar 1993 nromative theory of reasoning with scenarios: Anchored Narratives. The psychology of Criminal Evidence



%Josephson and Josephson 1996 - best explanation: abductive inference
%Josephson 2002
%Pardo juridical proof and best explanation
%Timmer 2014 inspiration for extraction of arguments from bn
%CRupi On Bayesian Measures of evidential support PDF JEST
% Evett The impact of the principles of evidence interpretation PDF JEST



%Bennet feldmoan 1981 reconstructing reality in the courtroom, book
%Dawid 2010 misinterpretation of bn
%Pearl 2009 misinterpretation of bn
%Hunter 2013 A probabilistic approach to modelling uncertain logical arguments
%


%NWO project Designing nad Understanding Forensic Bayesian Networks with Arguments and Scenarios

%NWO ToKeN project: Making sense of evidence













\end{document}