\PassOptionsToPackage{unicode=true}{hyperref} % options for packages loaded elsewhere
\PassOptionsToPackage{hyphens}{url}
\PassOptionsToPackage{dvipsnames,svgnames*,x11names*}{xcolor}
%
\documentclass[11pt,dvipsnames,enabledeprecatedfontcommands]{scrartcl}
\usepackage{lmodern}
\usepackage{amssymb,amsmath}
\usepackage{ifxetex,ifluatex}
\usepackage{fixltx2e} % provides \textsubscript
\ifnum 0\ifxetex 1\fi\ifluatex 1\fi=0 % if pdftex
  \usepackage[T1]{fontenc}
  \usepackage[utf8]{inputenc}
  \usepackage{textcomp} % provides euro and other symbols
\else % if luatex or xelatex
  \usepackage{unicode-math}
  \defaultfontfeatures{Ligatures=TeX,Scale=MatchLowercase}
\fi
% use upquote if available, for straight quotes in verbatim environments
\IfFileExists{upquote.sty}{\usepackage{upquote}}{}
% use microtype if available
\IfFileExists{microtype.sty}{%
\usepackage[]{microtype}
\UseMicrotypeSet[protrusion]{basicmath} % disable protrusion for tt fonts
}{}
\IfFileExists{parskip.sty}{%
\usepackage{parskip}
}{% else
\setlength{\parindent}{0pt}
\setlength{\parskip}{6pt plus 2pt minus 1pt}
}
\usepackage{xcolor}
\usepackage{hyperref}
\hypersetup{
            colorlinks=true,
            linkcolor=Maroon,
            filecolor=Maroon,
            citecolor=Blue,
            urlcolor=blue,
            breaklinks=true}
\urlstyle{same}  % don't use monospace font for urls
\usepackage{graphicx,grffile}
\makeatletter
\def\maxwidth{\ifdim\Gin@nat@width>\linewidth\linewidth\else\Gin@nat@width\fi}
\def\maxheight{\ifdim\Gin@nat@height>\textheight\textheight\else\Gin@nat@height\fi}
\makeatother
% Scale images if necessary, so that they will not overflow the page
% margins by default, and it is still possible to overwrite the defaults
% using explicit options in \includegraphics[width, height, ...]{}
\setkeys{Gin}{width=\maxwidth,height=\maxheight,keepaspectratio}
\setlength{\emergencystretch}{3em}  % prevent overfull lines
\providecommand{\tightlist}{%
  \setlength{\itemsep}{0pt}\setlength{\parskip}{0pt}}
\setcounter{secnumdepth}{5}
% Redefines (sub)paragraphs to behave more like sections
\ifx\paragraph\undefined\else
\let\oldparagraph\paragraph
\renewcommand{\paragraph}[1]{\oldparagraph{#1}\mbox{}}
\fi
\ifx\subparagraph\undefined\else
\let\oldsubparagraph\subparagraph
\renewcommand{\subparagraph}[1]{\oldsubparagraph{#1}\mbox{}}
\fi

% set default figure placement to htbp
\makeatletter
\def\fps@figure{htbp}
\makeatother

%\documentclass{article}

% %packages
 \usepackage{booktabs}

\usepackage{multirow}
\usepackage{multicol}

\usepackage{graphicx}
\usepackage{longtable}
\usepackage{ragged2e}
\usepackage{etex}
%\usepackage{yfonts}
\usepackage{marvosym}
\usepackage[notextcomp]{kpfonts}
\usepackage{nicefrac}
\newcommand*{\QED}{\hfill \footnotesize {\sc Q.e.d.}}
\usepackage{floatrow}



\usepackage[textsize=scriptsize, textwidth = 1.5cm]{todonotes}
%\linespread{1.5}


\setlength{\parindent}{10pt}
\setlength{\parskip}{1pt}


%language
\usepackage{times}
\usepackage{t1enc}
%\usepackage[utf8x]{inputenc}
%\usepackage[polish]{babel}
%\usepackage{polski}

\usepackage{mathptmx}
\usepackage[scaled=0.88]{helvet}


%AMS
\usepackage{amsfonts}
\usepackage{amssymb}
\usepackage{amsthm}
\usepackage{amsmath}
\usepackage{mathtools}

\usepackage{geometry}
 \geometry{a4paper,left=20mm,top=15mm,bottom = 20mm, right = 20mm}


%environments
\newtheorem{fact}{Fact}



%abbreviations
\newcommand{\ra}{\rangle}
\newcommand{\la}{\langle}
\newcommand{\n}{\neg}
\newcommand{\et}{\wedge}
\newcommand{\jt}{\rightarrow}
\newcommand{\ko}[1]{\forall  #1\,}
\newcommand{\ro}{\leftrightarrow}
\newcommand{\exi}[1]{\exists\, {_{#1}}}
\newcommand{\pr}[1]{\mathsf{P}(#1)}
\newcommand{\cost}{\mathsf{cost}}


\newcommand{\odds}{\mathsf{Odds}}
\newcommand{\ind}{\mathsf{Ind}}
\newcommand{\nf}[2]{\nicefrac{#1\,}{#2}}
\newcommand{\R}[1]{\texttt{#1}}
\newcommand{\prr}[1]{\mbox{$\mathtt{P}_{prior}(#1)$}}
\newcommand{\prp}[1]{\mbox{$\mathtt{P}_{posterior}(#1)$}}



\newtheorem{q}{\color{blue}Question}
\newtheorem{lemma}{Lemma}
\newtheorem{theorem}{Theorem}



%technical intermezzo
%---------------------

\newcommand{\intermezzoa}{
	\begin{minipage}[c]{13cm}
	\begin{center}\rule{10cm}{0.4pt}



	\tiny{\sc Optional Content Starts}
	
	\vspace{-1mm}
	
	\rule{10cm}{0.4pt}\end{center}
	\end{minipage}\nopagebreak 
	}


\newcommand{\intermezzob}{\nopagebreak 
	\begin{minipage}[c]{13cm}
	\begin{center}\rule{10cm}{0.4pt}

	\tiny{\sc Optional Content Ends}
	
	\vspace{-1mm}
	
	\rule{10cm}{0.4pt}\end{center}
	\end{minipage}
	}
%--------------------

\DeclareUnicodeCharacter{0301}{*************************************}




















\newtheorem*{reply*}{Reply}
\usepackage{enumitem}
\newcommand{\question}[1]{\begin{enumerate}[resume,leftmargin=0cm,labelsep=0cm,align=left]
\item #1
\end{enumerate}}

\usepackage{float}

% \setbeamertemplate{blocks}[rounded][shadow=true]
% \setbeamertemplate{itemize items}[ball]
% \AtBeginPart{}
% \AtBeginSection{}
% \AtBeginSubsection{}
% \AtBeginSubsubsection{}
% \setlength{\emergencystretch}{0em}
% \setlength{\parskip}{0pt}






\usepackage[authoryear]{natbib}

%\bibliographystyle{apalike}

\author{}
\date{\vspace{-2.5em}}

\begin{document}

\begin{center}
\large {\sc \textbf{Rethinking Legal Probabilism}}

\large Rafa\l\, Urbaniak
\end{center}

\vspace{1mm}

\thispagestyle{empty}

Legal probabilism is a research program that relies on probability
theory to analyze, model and improve the evaluation of evidence and the
process of decision-making in trial proceedings. Legal probabilism
remains a minority view among legal scholars, but attained greater
popularity in the second half of the 20th century in conjunction with
the law and economics movement.

In legal cases, different lines of evidence may converge, such as two
witnesses who testify that the defendant was seen at the crime scene, or
they may diverge, such as a witness who asserts the defendant was seen
at the crime scene while DNA testing shows no genetic match between the
defendant and the scene. Another source of complexity is that the
hypotheses put forward by the parties in a trial are often complex
structures of statements. How can different statements, and their
supporting evidence, be combined and the overall prosecutor's case (or
the defense's case) be put together and evaluated?

Probabilistic tools for piecemeal evaluation of scientific evidence and
spotting probabilistic fallacies in legal contexts are quite well
developed. Yet, the construction of a more general probabilistic model
of incorporating such evidence in a wider context of a whole case,
useful for theorizing about evidence evaluation and legal decision
standards, remains a challenge. This project intends to contribute to
further development of this enterprise relying on motivations that come
from different sources: formal epistemology, the development of bayesian
networks, the practice of probabilistic evidence evaluation and the
points raised by the critics of legal probabilism.

The point of departure is to represent narrations as bayesian networks
enriched with additional layer of information as to which nodes
correspond to evidence, and which are binary narration nodes. The key
idea is that with such bayesian networks as building material, various
features requested by the critics (such as coherence, resiliency,
missing evidence, explaining evidence, or ways to handle a multiplicity
of proposed narrations) can be explicated in terms of corresponding
properties of, operations on, and relations between bayesian networks.

\end{document}
