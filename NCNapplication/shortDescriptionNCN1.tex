\documentclass[11pt,dvipsnames,enabledeprecatedfontcommands]{scrartcl}
\usepackage{lmodern}
\usepackage{amssymb,amsmath}
\usepackage{ifxetex,ifluatex}
\usepackage{fixltx2e} % provides \textsubscript
\ifnum 0\ifxetex 1\fi\ifluatex 1\fi=0 % if pdftex
  \usepackage[T1]{fontenc}
  \usepackage[utf8]{inputenc}
\else % if luatex or xelatex
  \ifxetex
    \usepackage{mathspec}
  \else
    \usepackage{fontspec}
  \fi
  \defaultfontfeatures{Ligatures=TeX,Scale=MatchLowercase}
\fi
% use upquote if available, for straight quotes in verbatim environments
\IfFileExists{upquote.sty}{\usepackage{upquote}}{}
% use microtype if available
\IfFileExists{microtype.sty}{%
\usepackage[]{microtype}
\UseMicrotypeSet[protrusion]{basicmath} % disable protrusion for tt fonts
}{}
\PassOptionsToPackage{hyphens}{url} % url is loaded by hyperref
\usepackage[unicode=true]{hyperref}
\PassOptionsToPackage{usenames,dvipsnames}{color} % color is loaded by hyperref
\hypersetup{
            colorlinks=true,
            linkcolor=Maroon,
            citecolor=Blue,
            urlcolor=blue,
            breaklinks=true}
\urlstyle{same}  % don't use monospace font for urls
\usepackage{graphicx,grffile}
\makeatletter
\def\maxwidth{\ifdim\Gin@nat@width>\linewidth\linewidth\else\Gin@nat@width\fi}
\def\maxheight{\ifdim\Gin@nat@height>\textheight\textheight\else\Gin@nat@height\fi}
\makeatother
% Scale images if necessary, so that they will not overflow the page
% margins by default, and it is still possible to overwrite the defaults
% using explicit options in \includegraphics[width, height, ...]{}
\setkeys{Gin}{width=\maxwidth,height=\maxheight,keepaspectratio}
\IfFileExists{parskip.sty}{%
\usepackage{parskip}
}{% else
\setlength{\parindent}{0pt}
\setlength{\parskip}{6pt plus 2pt minus 1pt}
}
\setlength{\emergencystretch}{3em}  % prevent overfull lines
\providecommand{\tightlist}{%
  \setlength{\itemsep}{0pt}\setlength{\parskip}{0pt}}
\setcounter{secnumdepth}{5}
% Redefines (sub)paragraphs to behave more like sections
\ifx\paragraph\undefined\else
\let\oldparagraph\paragraph
\renewcommand{\paragraph}[1]{\oldparagraph{#1}\mbox{}}
\fi
\ifx\subparagraph\undefined\else
\let\oldsubparagraph\subparagraph
\renewcommand{\subparagraph}[1]{\oldsubparagraph{#1}\mbox{}}
\fi

% set default figure placement to htbp
\makeatletter
\def\fps@figure{htbp}
\makeatother

%\documentclass{article}

% %packages
 \usepackage{booktabs}

\usepackage{multirow}
\usepackage{multicol}

\usepackage{graphicx}
\usepackage{longtable}
\usepackage{ragged2e}
\usepackage{etex}
%\usepackage{yfonts}
\usepackage{marvosym}
\usepackage[notextcomp]{kpfonts}
\usepackage{nicefrac}
\newcommand*{\QED}{\hfill \footnotesize {\sc Q.e.d.}}
\usepackage{floatrow}



\usepackage[textsize=scriptsize, textwidth = 1.5cm]{todonotes}
%\linespread{1.5}


\setlength{\parindent}{10pt}
\setlength{\parskip}{1pt}


%language
\usepackage{times}
\usepackage{t1enc}
%\usepackage[utf8x]{inputenc}
%\usepackage[polish]{babel}
%\usepackage{polski}

\usepackage{mathptmx}
\usepackage[scaled=0.88]{helvet}


%AMS
\usepackage{amsfonts}
\usepackage{amssymb}
\usepackage{amsthm}
\usepackage{amsmath}
\usepackage{mathtools}

\usepackage{geometry}
 \geometry{a4paper,left=20mm,top=15mm,bottom = 20mm, right = 20mm}


%environments
\newtheorem{fact}{Fact}



%abbreviations
\newcommand{\ra}{\rangle}
\newcommand{\la}{\langle}
\newcommand{\n}{\neg}
\newcommand{\et}{\wedge}
\newcommand{\jt}{\rightarrow}
\newcommand{\ko}[1]{\forall  #1\,}
\newcommand{\ro}{\leftrightarrow}
\newcommand{\exi}[1]{\exists\, {_{#1}}}
\newcommand{\pr}[1]{\mathsf{P}(#1)}
\newcommand{\cost}{\mathsf{cost}}


\newcommand{\odds}{\mathsf{Odds}}
\newcommand{\ind}{\mathsf{Ind}}
\newcommand{\nf}[2]{\nicefrac{#1\,}{#2}}
\newcommand{\R}[1]{\texttt{#1}}
\newcommand{\prr}[1]{\mbox{$\mathtt{P}_{prior}(#1)$}}
\newcommand{\prp}[1]{\mbox{$\mathtt{P}_{posterior}(#1)$}}



\newtheorem{q}{\color{blue}Question}
\newtheorem{lemma}{Lemma}
\newtheorem{theorem}{Theorem}



%technical intermezzo
%---------------------

\newcommand{\intermezzoa}{
	\begin{minipage}[c]{13cm}
	\begin{center}\rule{10cm}{0.4pt}



	\tiny{\sc Optional Content Starts}
	
	\vspace{-1mm}
	
	\rule{10cm}{0.4pt}\end{center}
	\end{minipage}\nopagebreak 
	}


\newcommand{\intermezzob}{\nopagebreak 
	\begin{minipage}[c]{13cm}
	\begin{center}\rule{10cm}{0.4pt}

	\tiny{\sc Optional Content Ends}
	
	\vspace{-1mm}
	
	\rule{10cm}{0.4pt}\end{center}
	\end{minipage}
	}
%--------------------

\DeclareUnicodeCharacter{0301}{*************************************}




















\newtheorem*{reply*}{Reply}
\usepackage{enumitem}
\newcommand{\question}[1]{\begin{enumerate}[resume,leftmargin=0cm,labelsep=0cm,align=left]
\item #1
\end{enumerate}}

\usepackage{float}

% \setbeamertemplate{blocks}[rounded][shadow=true]
% \setbeamertemplate{itemize items}[ball]
% \AtBeginPart{}
% \AtBeginSection{}
% \AtBeginSubsection{}
% \AtBeginSubsubsection{}
% \setlength{\emergencystretch}{0em}
% \setlength{\parskip}{0pt}






\usepackage[authoryear]{natbib}

%\bibliographystyle{apalike}

\author{}
\date{\vspace{-2.5em}}

\begin{document}

\begin{center}
\Large {\sc \textbf{Rethinking Legal Probabilism}}

\large Rafa\l\, Urbaniak
\end{center}

\vspace{2mm}

\thispagestyle{empty}

\noindent \Large \textbf{1. Scientific goal} \normalsize

\vspace{1mm}

The \textbf{goal of this project} is to
\textbf{develop of a probabilistic  modelling method of the  interaction of various items of evidence and hypotheses and the resulting decisions in the court of law}.
In light of the current criticism of the probabilistic approach, such a
model should (1) be \textbf{sensitive to the argumentative structure}
involved, and (2) capture the idea that in a legal context we are
dealing with a \textbf{class of competing narrations}. At this point in
the literature, it is not clear how this is to be achieved within the
probabilistic approach.

The key idea is to represent \textbf{narrations  as bayesian networks}.
Then, the argumentative structure becomes clear, and various features
requested by the critics can be explicated in terms of corresponding
properties of, operations on, and relations between bayesian networks.

The output will be a
\textbf{unifying extended probabilistic model embracing key aspects of the narrative and argumentative approaches, with  implementation in the programming language \textbf{\textsf{R}}.}
What the project will uniquely bring to the table is joining the
familiarity with epistemological debates, familiarity with the details
of evidence assessment in legal cases and technical skill to
programmatically implement, simulate and test various theoretical moves.

\vspace{2mm}

\noindent \Large \textbf{2. Significance} \normalsize

\vspace{1mm}

\noindent \large \textbf{2.1. State of the art}

\normalsize 

\vspace{1mm}

\textbf{Legal probabilism (LP)} comprises two core tenets: (1) the
evidence presented at trial can be assessed, weighed and combined by
means of probability theory; and (2) legal decision rules, such as proof
beyond a reasonable doubt in criminal cases, can be explicated in
probabilistic terms. LP gained popularity amidst the law and economics
movement (Becker, 1968; Calabresi, 1961; Finkelstein \& Fairley, 1970;
Lempert, 1977, 1986; Posner, 1973).

These developments faced \textbf{criticism} (Allen, 1986; Brilmayer,
1986; Cohen, 1986; Dant, 1988; Tribe, 1971; Underwood, 1977). The
negative trend has been somewhat mitigated by the discovery of DNA
fingerprinting in the eighties and progress in forensic science in
general, with the increasing role of quantitative evidence in the court
of law (Kaye, 1986, 2010; Koehler, 1996; National Research Council,
1992; Robertson \& Vignaux, 1995).

Skepticism about wider mathematical and quantitative models of legal
evidence is still widespread among prominent legal scholars and
practitioners Allen \& Pardo (2007), partially in light of
\textbf{conceptual difficulties} extensively discussed in the
literature, which arise when one wants to formulate a probabilistic
decision criterion for the court of law. For instance, the threshold
view has it that sufficiently high probability of guilt is sufficient
for conviction (Dekay, 1996; Kaye, 1979; Laudan, 2006). The threshold
view is blocked by the so-called paradoxes of legal proof or puzzles of
naked statistical evidence (Cohen, 1981; Nesson, 1979; Thomson, 1986).
Another conceptual problem is the so-called difficulty about
conjunction. It arises, because intuitively there should be no
difference between the trier's acceptance of \(A\) and \(B\) separately,
and her acceptance of their conjunction, \(A\wedge B\). On the threshold
view, such a difference might arise.

How to define the \textbf{standards of proof}, or whether they should be
even defined in the first place, remains contentious (Diamond, 1990;
Gardiner, 2018; Horowitz \& Kirkpatrick, 1996; Laudan, 2006; Newman,
1993; Redmayne, 2008; Walen, 2015) Various informal notions have been
claimed to be essential for a proper explication of judiciary decision
standards (Enoch \& Fisher, 2015; Haack, 2014; Smith, 2017; Wells,
1992). A legal probabilist needs either to show that these notions are
unnecessary or inadequate for the purpose at hand, or to indicate how
they can be explicated in probabilistic terms.

\textbf{Alternative frameworks} for modeling evidential reasoning and
decision-making at trial have been proposed. They are based on inference
to the best explanation (Allen, 2010; Hastie, 2019; Ho, 2019; Nance,
2019; Pardo \& Allen, 2008; Schwartz \& Sober, 2019), narratives and
stories (Allen, 1986, 2010; Allen \& Leiter, 2001; Clermont, 2015;
Pardo, 2018; Pennington \& Hastie, 1991a), and argumentation theory
(Bex, 2011; Gordon, Prakken, \& Walton, 2007; Walton, 2002). Preliminary
sketches of hybrid theories (Urbaniak, 2018a; Verheij, 2014) are also
available.

The main point of criticism is that legal proceedings are back-and-forth
between opposing parties in which cross-examination and the
argumentative structure are of crucial importance, reasoning goes not
only evidence-to-hypothesis, but also hypotheses-to-evidence (Allen \&
Pardo, 2007; Wells, 1992) in a way that seems analogous to inference to
the best explanation (Dant, 1988), which notoriously is claimed to not
be susceptible to probabilistic analysis (Lipton, 2004). An informal
philosophical account inspired by such considerations---The
\textbf{No Plausible Alternative Story (NPAS)} theory (Allen, 2010)---is
that the courtroom is a confrontation of competing narrations (Ho, 2008;
Wagenaar, Van Koppen, \& Crombag, 1993) offered by the sides, and the
narrative to be selected should be the most plausible one. The view is
conceptually plausible (Di Bello, 2013), and finds support in
psychological evidence (Pennington \& Hastie, 1991b, 1992). It would be
a great advantage of LP if it could model such phenomena and by doing so
resolve the already mentioned conceptual challenges.

The idea that \textbf{Bayesian networks} can be used for probabilistic
reasoning in legal fact-finding started gaining traction in late
eighties and early nineties (Edwards, 1991). A Bayesian network
comprises two components: a directed acyclic graph of relations of
dependence (represented by arrows) between variables (represented by
nodes) and conditional probability tables corresponding to these
relations. Fairly simple graphical patterns (called \emph{idioms}) often
appear while modeling the relationships between evidence and hypotheses.
Complex graphical models can be constructed by combining these basic
patters in a modular way (Bovens \& Hartmann, 2004; Fenton \& Neil,
2018a; Fenton, Neil, \& Lagnado, 2013; Friedman, 1974; Hepler, Dawid, \&
Leucari, 2007; Neil, Fenton, \& Nielson, 2000; Taroni, Biedermann,
Bozza, Garbolino, \& Aitken, 2014).

Some attempts have been made to use Bayesian networks to weigh and
assess complex bodies of evidence consisting of multiple components.
Kadane \& Schum (2011) made one the first attempts to model an
\textbf{entire criminal case}, Sacco \& Vanzetti from 1920, using
probabilistic graphs. Similarly, Fenton \& Neil (2018b) constructed a
Bayesian network for the famous Sally Clark case. The literature
contains also a general approach the use of BNs for modeling whole
cases. The main idea is that once all the pieces of evidence and claims
are represented as nodes, one should use the \textbf{scenario idiom} to
model complex hypotheses, consisting of a sequence of events organized
in space and time: a scenario (Vlek, Prakken, Renooij, \& Verheij,
2014).\footnote{A discussion of modelling crime scenarios by means of
  graphical devices mixed with probabilities can be also found in the
  work of Shen, Keppens, Aitken, Schafer, \& Lee (2007)\}, Bex (2011),
  Bex (2015), Dawid \& Mortera (2018) and Verheij (2017). See also the
  survey by Di Bello \& Verheij (2018).} A BN that uses the scenario
idiom would consist of the following components: first, nodes for the
states and events in the scenario, with each node linked to the
supporting evidence; second, a separate scenario node that has all
states and events as its children; finally, a node corresponding to the
ultimate hypothesis as a child of the scenario node. The scenario node
in a sense unifies the different events and states. Because of this
unifying role, increasing the probability of one part of the scenario
will also increase the probability of the other parts. This is intended
to capture the idea that the different components of a scenario form an
interconnected sequence of events.

This strategy is supposed to make sense of the notion of the
\textbf{coherence of a scenario} as different from its probability given
the evidence. On this approach (Vlek, 2016; Vlek, Prakken, Renooij, \&
Bart Verheij, 2015; Vlek, Prakken, Renooij, \& Verheij, 2013; Vlek et
al., 2014), coherence is identified with the prior probability of the
scenario node. The approach formally represents reasoning with
\textbf{multiple scenarios}. Given a class of narrations, all the nodes
used in some of the separate BNs are to be used to build one large BN,
and separate scenario nodes are to be added to it, so that one BN
supposedly represents multiple scenarios at once.

An \textbf{alternative approach} to representation of and reasoning with
multiple scenarios has been developed by Neil, Fenton, Lagnado, \& Gill
(2019). They criticize (Urbaniak, 2018a) by pointing out that it makes
no connection to BNs and so it ``fails to offer a convincing and
operational means to structure and compare competing narratives.'' This
is a fair assessment of the limits of what I have achieved so far. They
represent separate narrations in terms of separate BNs, and deploy
bayesian model comparison and averaging (linear pooling) as a tool for
reasoning with multiple scenarios.

\vspace{1mm}

\noindent \large \textbf{2.2. Pioneering nature of the project}

\vspace{1mm} \normalsize

Here are the key reasons why I am convinced
\textbf{the scenario node approach is not satisfactory}. (A) Adding a
parent node by \emph{fiat} without any good reasons to think the nodes
are connected other than being a part of a single story, introduces
probabilistic dependencies between the elements of a narration. (B)
Another problem results from the identification of prior probability
with coherence. This does not add up intuitively. (C) In general, the
legal probabilistic approach to coherence is very simple and fails to
engage with rich philosophical literature exactly on this topic (Douven
\& Meijs, 2007; Fitelson, 2003a, 2003b; Glass, 2002; Meijs \& Douven,
2007; Olsson, 2001; Shogenji, 1999).\footnote{This appraoch also fails
  to include a long list of counterexamples to the existing proposals
  and desiderata that a probabilistic coherence measure should satisfy
  (Akiba, 2000; Bovens \& Hartmann, 2004; Crupi, Tentori, \& Gonzalez,
  2007; Koscholke, 2016; Merricks, 1995; Schippers \& Koscholke, 2019;
  Shogenji, 1999, 2001, 2006; Siebel, 2004, 2006).} (D) The merging
procedure with scenario nodes assumes that for the nodes that are common
to the networks to be merged, both the directions of the arrows in the
DAGs and the conditional probability tables are the same across
different narrations. This is unrealistic.

Here are the key
\textbf{limitations of the model selection and averaging approach}. (E)
The assumption of equal priors is highly debatable. This approach would
render prior probabilities quite sensitive to the choice of hypotheses
and thus potentially arbitrary. Also, this approach seems particularly
unsuitable for criminal cases. If the only two hypotheses/models on the
table ultimately say ``the defendant is guilty'' and ``the defendant is
innocent'', the prior probability of each would be 50\%. But defendants
in criminal cases, however, should be presumed innocent until proven
guilty. A 50\% prior probability of guilt seems excessive.\footnote{Some
  (Williamson, 2010) try to defend a variant of the principle of
  indifference by reference to informational entropy, and a proposal
  along this line has been used in practical recommendation by expert
  committees (ENFSI Expert Working Group Marks Conclusion Scale
  Committee, 2006). However, this attempt has been sensibly criticized
  by Biedermann, Taroni, \& Garbolino (2007). The question remains, what
  should the proper application of informational entropy in the context
  of BN selection and averaging look like, given that information
  entropy considerations independently are in epistemologically decent
  standing?} Some take the presumption to mean that the prior
probability of guilt should be set to a small value (Friedman, 2000;
Friedman, Allen, Balding, Donnelly, \& Kaye, 1995), but it is not clear
whether this interpretation can be justified on epistemological or
decision-theoretic grounds. (F) More recent models rely on relevant
background information about people's opportunities to commit crimes
(Fenton, Lagnado, Dahlman, \& Neil, 2019). But even if these models are
successful in giving well-informed assessments of prior probabilities
any evidence-based assessment of prior probabilities, they are likely to
violate existing normative requirements of the trial system (Dahlman,
2017; Engel, 2012; Schweizer, 2013). People who belong to certain
demographic groups will be regarded as having a higher prior probability
of committing a wrong than others, and this outcome can be seen as
unfair (Di Bello \& O'Neil, 2020). (G) Model selection based on
likelihood (given equal priors) or posterior model probabilities in
general (if priors are not assumed to be equal) boils down to a variant
of the threshold view, and so all the difficulties with the threshold
view apply. (H) There is a rich literature on the difficulties that
linear pooling runs into (see the surveys in Dietrich \& List, 2016;
Franz Dietrich \& List, 2017a, 2017b). One problem is that the method
satisfies the unanimity assumption: whenever all models share a degree
of belief in a claim, this is exactly the output degree for that belief.
Another problem is that linear pooling does not preserve probabilistic
independence (List \& Pettit, 2011): even if all models agree that
certain nodes are independent, they might end up being dependent in the
output. There is also a variety of impossibility theorems in the
neighborhood (Gallow, 2018).

\vspace{2mm}

\noindent \textbf{The key steps in my proposal are as follows}:

\vspace{1mm}

\noindent
\textbf{Representation.} Use BNs taken separately without scenario nodes
to represent various narrations. without assuming the conditional
probability tables or directions of edges are the same across the BNs,
adding another layer of information, dividing nodes into evidence and
narration nodes.

\noindent \textbf{Dynamic BNs.} Averaging does not seem to be the right
way to model cross-examination. To take the argumentative approach
seriously and to be able to model relations such as ``undercutting'' and
``rebutting'' I will consider another dimension: BNs changing through
time in light of other BNs.

\noindent
 \textbf{BN-based coherence.} I have formulated a BN-based coherence
measure that is not a function of a probabilistic measure and a set of
propositions alone, because it is also sensitive to the selection and
direction of arrows in a Bayesian Network representing an agent's credal
state. Now, it needs to be deployed (implemented in \textbf{\textsf{R}}
for BNs) and properly tested on real-life cases discussed in the LP
literature.

\noindent
 \textbf{Divide and conquer.} I propose that ensemble methods should be
deployed for multiple narration variants available from one side (as in
when, say, the prosecution story comes with uncertainty about the
direction of an arrow or about a particular probability table), but
selection methods should be used when final decision is to be made
between narrations proposed by the opposing sides.

\noindent
\textbf{Ensemble methods.} One question that arises is whether the
general concerns about linear pooling arise for such limited
applications. If not, the remaining concern is what priors should be
used. In light of the controversial nature of equal priors, I plan to
study the consequences of rescaling coherence scores (already mentioned)
to constitute model priors. The idea is that given that narrations are
to be developed by the sides themselves, taking coherence of their
narration as determining the prior might be more fair than using equal
priors or relying on geographical or population statistics. If yes,
perhaps some other methods boiling down to a variant of sensitivity
analysis can be deployed: look at all BNs corresponding to some variant
of the narration of one of the sides, find the strongest and the weakest
one, and these give you a range of possible outcomes.

\noindent
 \textbf{Selection criteria.} The so-called New Legal Probabilism (NLP)
is an attempt to improve on the underspecificity of NPAS (Di Bello,
2013). While still being at most semi-formal, the approach is more
specific about the conditions that a successful accusing narration is to
satisfy for the conviction beyond reasonable doubt to be justified. Di
Bello works out the philosophical details of requirements such as
evidential support and completeness, resiliency, and narrativity. It is
far from obvious that such conditions are susceptible to a Bayesian
networks explication. I plan to rely on a more expressive probabilistic
framework (Urbaniak, 2018b, 2018a), that is capable of expressing such
features within a formalized higher-order language. The key hypothesis
is that they can be recast in terms of properties of BNs and that the
existing BN programming tools can be extended to implement testing for
these criteria. This will make them susceptible to programmatic
implementation and further study by means of computational methods. The
hope is that on one hand, they will do better than the existing
proposals, and where they fail, further insights can be gained by
studying the reasons behind such failures.

\vspace{1mm}

\noindent \Large \textbf{3. Work plan}

\vspace{1mm} \normalsize

Throughout the whole project I plan to cooperate with Marcello Di Bello
(Arizona State University). Over the last year we co-authored the
Stanford Encyclopedia of Philosophy entry on Legal Probabilism. We
decided to continue our fruitful cooperation. For over two months we
have been working on a book proposal to be submitted to Oxford
University Press exactly on the issues to be studied in this research
project. Marcello Di Bello is an excellent philsopher with extensive
research experience in the philosophy of legal evidence, and he would
bring his expertise to the table when working both on the book and on
the papers, whereas I would be focused on the technical aspects and the
underlying formal philosophy. During the last year of the research
\textbf{I plan a six months' stay at Arizona University}, to work in
person with Di Bello on finalizing the book that presents the results of
the project.

Apart from publications, the results will be presented at various
conferences devoted to legal reasoning. These include the yearly
conferences of the
\emph{International Association for Artificial Intelligence and Law} and
of the \emph{Foundation for Legal Knowledge Based Systems (JURIX)}, and
more general conferences gathering formal philosophers, so that the
research is inspired by interaction not only with legal evidence
scholars, computer scientists, but also philosophers. I am also already
an invited speaker at the upcoming ``Probability and Proof'' conference
that will be part of an international conference on the philosophy of
legal evidence (``The Michele Taruffo Girona Evidence Week'') in Girona
(Spain) May 23-27 2022.

\vspace{1mm}

\begin{center}
\begin{tabular}{p{2.3cm}|p{13.8cm}}
\footnotesize \textbf{$\mathtt{Stage  \,\, 1}$} \newline  \tiny Philosophical \&  formal \newline  unification & 
Obtain a unifying extended  probabilistic framework by incorporating further insights  from philosophical and psychological accounts of legal narrations, and from the argumentation approach. Defend its philosophical plausibility. \scriptsize (6 months)
\\ 
& \footnotesize A philosophical paper published in an  academic journal such as \emph{Synthese}, \emph{Mind} or \emph{Ratio Juris}. Working title: \emph{Why care about narration selection principles?}\\ \\
\footnotesize \textbf{$\mathtt{Stage \,\, 2}$} \newline  \tiny AI implementation 
 & Develop Bayesian Network Methods for the obtained formal framework, so that the insights from the argumentation approach and informal epistemology, mediated through it, can be incorporated in AI tools. \scriptsize (12 months)
\\ & \footnotesize A   technical paper  published in a journal such as \emph{IfCoLog Journal of Logics and their Applications}, \emph{Law, Probability and Risk} or \emph{Artificial Intelligence and Law}. Working title: \emph{Implementation of narration assessment criteria in Bayesian Networks with \textbf{\textsf{R}}}.
\\ \\
\footnotesize \textbf{$\mathtt{Stage  \,\, 3}$} \newline  \tiny    Case studies & 
Evaluate the developed framework and AI tools  by conducting case studies from its perspective.    \scriptsize (6 months)  \\
& \footnotesize One paper  on  how the formal framework handles case studies and  one paper  on how the developed AI tools handle real-life situations in   journals such as \emph{Artificial Intelligence} or \emph{Argument \& Computation}. Working titles: \emph{Rethinking the famous BN-modeled cases within the narration framework} and \emph{BNarr, an \textbf{\textsf{R}} package to model narrations with Bayesian Networks}.
\\ \\
\footnotesize \textbf{$\mathtt{Stage  \,\, 4}$} \newline  \tiny    Back to challenges \& output & 
Investigate the extent to which the new framework helps to handle the issues raised in points A-H. \scriptsize (12 months) \\ 
& \footnotesize Completion of the planned book.
\end{tabular}
\end{center}

\vspace{2mm}

\pagebreak 

\noindent \Large \textbf{4. Methodology}

\vspace{1mm} \normalsize

Standard arguments for the legitimacy of Bayesianism\footnote{See for
  example (Earman, 1992; Urbach \& Howson, 1993) for an early yet fairly
  comprehensive survey, or (Pettigrew, 2011) for a discussion of more
  recent contributions. See also (Bovens \& Hartmann, 2004; Bradley,
  2015; Swinburne, 2001).} deploy usually rather abstract pieces of
reasoning to the effect that if one's degrees of beliefs satisfy certain
conditions, they also have to satisfy the probabilistic requirements. My
approach to thinking about the plausibility of Bayesian epistemology is
rather unlike such approaches. Instead, I prefer the
\emph{proof-of-the-pudding} methodology. I am convinced that an
important part of the philosophical assessment of the Bayesian research
program has to do with its achievements or failures in contributing to
debates in philosophy which are not themselves debates about the status
of Bayesianism itself. In particular, it would be great news if insights
from Bayesian epistemology could be used to further development of
forensic AI and deepening our understanding of judiciary decision
making.

I will be using four methods: (a) informal conceptual analysis; (b)
formal conceptual analysis; (c) computational metods (R simulations,
etc.); and (d) case studies. I will rethink and model the existing
impliementations of whole-case-scale-BNs in legal evidence from the
perspective of the new framework and reconstruct cases which extensively
use probabilistic reasoning, but for which BNs have not been yet
proposed. Both the literature already listed and many textbooks on
quantitative evidence in forensics are great sources of such cases.

What the project will uniquely bring to the table is joining the
familiarity with epistemological debates on the nature of coherence
(which legal probabilists like Vlek or Fenton ignore or are unaware of),
familiarity with the details of evidence assessment in legal cases
(which formal epistemologists such as Fitelson ignore) and technical
skill to programmatically implement, simulate and test various
theoretical moves.

A larger initiative involving reconstructing various cases using
different representation methods and comparing the representations,
called \emph{Probability and statistics in forensic science}, took place
at the Isaac Newton Institute for Mathematical Sciences. My approach
will be in the same vein.

I have extensive experience in analytic philosophy, conceptual analysis,
philosophical logic and probabilistic and decision-theoretic methods as
deployed in philosophical contexts. I also have teaching and publishing
record that involves statistical programming in the \textsf{\textbf{R}}
language (my sample programming projects can be visited at
\url{https://rfl-urbaniak.github.io/menu/projects.html}), and so am also
competent to develop AI implementations and tests of the ideas to be
developed.

One research risk is that it will turn out that some of the informal
requirements cannot be spelled out in probabilistic terms, or expressed
in terms of properties of Bayesian networks. In such an event, I will
study the reasons for this negative result. It might happen that there
are independent reasons to abandon a given condition, or it might be the
case that probabilistic inexplicability of a given condition is an
argument against the probabilistic approach. Either way, finding out
which option holds and why would also lead to a deeper understanding of
the framework and lead to a publication in academic journals. Another
research risk is that the case studies will show that other methods are
more efficient or transparent. This would itself constitute a result
that could be used to further modify the framework so that its best
aspects could be preserved, while the disadvantages discovered during
case studies avoided.

\pagebreak 

\section*{References}\label{references}
\addcontentsline{toc}{section}{References}

\footnotesize 

\hypertarget{refs}{}
\hypertarget{ref-Akiba2000Shogenjis}{}
Akiba, K. (2000). Shogenji's probabilistic measure of coherence is
incoherent. \emph{Analysis}, \emph{60}(4), 356--359. Oxford University
Press (OUP). Retrieved from
\url{https://doi.org/10.1093/analys/60.4.356}

\hypertarget{ref-Allen1986A-Reconceptuali}{}
Allen, R. J. (1986). A reconceptualization of civil trials. \emph{Boston
University Law Review}, \emph{66}, 401--437.

\hypertarget{ref-Allen2010No-Plausible-Al}{}
Allen, R. J. (2010). No plausible alternative to a plausible story of
guilt as the rule of decision in criminal cases. In J. Cruz \& L. Laudan
(Eds.), \emph{Prueba y esandares de prueba en el derecho}. Instituto de
Investigaciones Filosoficas-UNAM.

\hypertarget{ref-allen2001naturalized}{}
Allen, R. J., \& Leiter, B. (2001). Naturalized epistemology and the law
of evidence. \emph{Virginia Law Review}, \emph{87}(8), 1491--1550.
JSTOR.

\hypertarget{ref-allen2007problematic}{}
Allen, R., \& Pardo, M. (2007). The problematic value of mathematical
models of evidence. \emph{The Journal of Legal Studies}, \emph{36}(1),
107--140. JSTOR.

\hypertarget{ref-becker1968crime}{}
Becker, G. S. (1968). Crime and punishment: An economic approach.
\emph{Journal of Political Economy}, \emph{76}, 169--217. Springer.

\hypertarget{ref-bex2015IntegratedTheoryCausal}{}
Bex, F. (2015). An integrated theory of causal stories and evidential
arguments. In \emph{Proceedings of the 15th International Conference on
Artificial Intelligence and Law - ICAIL '15} (pp. 13--22). San Diego,
California: ACM Press.

\hypertarget{ref-bex2011ArgumentsStoriesCriminal}{}
Bex, F. J. (2011). \emph{Arguments, stories and criminal evidence: A
formal hybrid theory}. Law and philosophy library. Dordrecht ; New York:
Springer.

\hypertarget{ref-Biedermann2007equal}{}
Biedermann, A., Taroni, F., \& Garbolino, P. (2007). Equal prior
probabilities: Can one do any better? \emph{Forensic Science
International}, \emph{172}(2-3), 85--93. Elsevier BV. Retrieved from
\url{https://doi.org/10.1016/j.forsciint.2006.12.008}

\hypertarget{ref-bovens2004bayesian}{}
Bovens, L., \& Hartmann, S. (2004). \emph{Bayesian epistemology}. Oxford
University Press.

\hypertarget{ref-bradley2015critical}{}
Bradley, D. (2015). \emph{A critical introduction to formal
epistemology}. Bloomsbury Publishing.

\hypertarget{ref-brilmayer1986}{}
Brilmayer, L. (1986). Second-order evidence and bayesian logic.
\emph{Boston University Law Review}, \emph{66}, 673--691.

\hypertarget{ref-Calabresi1961}{}
Calabresi, G. (1961). Some thoughts on risk distribution and the law of
torts. \emph{Yale Law Journal}, \emph{70}, 499--553.

\hypertarget{ref-clermont2015TrialTraditionalProbability}{}
Clermont, K. M. (2015). Trial by Traditional Probability, Relative
Plausibility, or Belief Function? \emph{Case Western Reserve Law
Review}, \emph{66}(2), 353--391.

\hypertarget{ref-Cohen81}{}
Cohen, J. L. (1981). Subjective probability and the paradox of the
Gatecrasher. \emph{Arizona State Law Journal}, 627--634.

\hypertarget{ref-cohen86}{}
Cohen, J. L. (1986). Twelve questions about Keynes's concept of weight.
\emph{British Journal for the Philosophy of Science}, \emph{37}(3),
263--278.

\hypertarget{ref-crupi2007BayesianMeasuresEvidential}{}
Crupi, V., Tentori, K., \& Gonzalez, M. (2007). On Bayesian Measures of
Evidential Support: Theoretical and Empirical Issues. \emph{Philosophy
of Science}, \emph{74}(2), 229--252.

\hypertarget{ref-dahlman2017}{}
Dahlman, C. (2017). Determining the base rate for guilt. \emph{Law,
Probability and Risk}, \emph{17}(1), 15--28.

\hypertarget{ref-dant1988gambling}{}
Dant, M. (1988). Gambling on the truth: The use of purely statistical
evidence as a basis for civil liability. \emph{Columbia Journal of Law
and Social Problems}, \emph{22}, 31--70. HeinOnline.

\hypertarget{ref-dawid2018graphical}{}
Dawid, A. P., \& Mortera, J. (2018). Graphical models for forensic
analysis. In \emph{Handbook of graphical models} (pp. 491--514). CRC
Press.

\hypertarget{ref-Dekay1996}{}
Dekay, M. L. (1996). The difference between Blackstone-like error ratios
and probabilistic standards of proof. \emph{Law and Social Inquiry},
\emph{21}, 95--132.

\hypertarget{ref-di2013statistics}{}
Di Bello, M. (2013). \emph{Statistics and probability in criminal
trials} (PhD thesis). University of Stanford.

\hypertarget{ref-DiBelloONeil2020}{}
Di Bello, M., \& O'Neil, C. (2020). Profile evidence, fairness and the
risk of mistaken convictions. \emph{Ethics}, \emph{130}(2), 147--178.

\hypertarget{ref-di2018evidential}{}
Di Bello, M., \& Verheij, B. (2018). Evidential reasoning. In
\emph{Handbook of legal reasoning and argumentation} (pp. 447--493).
Springer.

\hypertarget{ref-diamond90}{}
Diamond, H. A. (1990). Reasonable doubt: To define, or not to define.
\emph{Columbia Law Review}, \emph{90}(6), 1716--1736.

\hypertarget{ref-Dietrich2016Probabilistic}{}
Dietrich, F., \& List, C. (2016). Probabilistic opinion pooling. In A.
Hajek \& C. Hitchcock (Eds.), \emph{Oxford handbook of philosophy and
probability}. Oxford University Press.

\hypertarget{ref-dietrich2017probabilistic1}{}
Dietrich, F., \& List, C. (2017a). Probabilistic opinion pooling
generalized. part one: General agendas. \emph{Social Choice and
Welfare}, \emph{48}(4), 747--786. Springer.

\hypertarget{ref-dietrich2017probabilistic2}{}
Dietrich, F., \& List, C. (2017b). Probabilistic opinion pooling
generalized. part two: The premise-based approach. \emph{Social Choice
and Welfare}, \emph{48}(4), 787--814. Springer.

\hypertarget{ref-Douven2007measuring}{}
Douven, I., \& Meijs, W. (2007). Measuring coherence. \emph{Synthese},
\emph{156}(3), 405--425. Springer Science; Business Media LLC. Retrieved
from \url{https://doi.org/10.1007/s11229-006-9131-z}

\hypertarget{ref-earman1992bayes}{}
Earman, J. (1992). \emph{Bayes or bust? A critical examination of
bayesian confirmation theory}. Cambridge: MIT Press.

\hypertarget{ref-Edwards1991Influence-diagr}{}
Edwards, W. (1991). Influence diagrams, bayesian imperialism, and the
collins case: An appeal to reason. \emph{Cardozo Law Review}, \emph{13},
1025--1074.

\hypertarget{ref-ENFSI2006entropy}{}
ENFSI Expert Working Group Marks Conclusion Scale Committee. (2006).
Conclusion scale for shoeprint and toolmarks examinations. \emph{Journal
of Forensic Identification}, \emph{56}, 255--280.

\hypertarget{ref-engel2012NeglectBaseRate}{}
Engel, C. (2012). Neglect the Base Rate: It's the Law! \emph{Preprints
of the Max Planck Institute for Research on Collective Goods},
\emph{23}.

\hypertarget{ref-enoch2015sense}{}
Enoch, D., \& Fisher, T. (2015). Sense and sensitivity: Epistemic and
instrumental approaches to statistical evidence. \emph{Stan. L. Rev.},
\emph{67}, 557--611. HeinOnline.

\hypertarget{ref-Fenton2018risk}{}
Fenton, N., \& Neil, M. (2018a). \emph{Risk assessment and decision
analysis with Bayesian networks}. Chapman; Hall.

\hypertarget{ref-Fenton2018Risk}{}
Fenton, N., \& Neil, M. (2018b). \emph{Risk assessment and decision
analysis with bayesian networks}. Chapman; Hall.

\hypertarget{ref-fenton2019OpportunityPriorProofbased}{}
Fenton, N., Lagnado, D., Dahlman, C., \& Neil, M. (2019). The
opportunity prior: A proof-based prior for criminal cases. \emph{Law,
Probability and Risk}, {[}online first{]}.

\hypertarget{ref-fenton2013GeneralStructureLegal}{}
Fenton, N., Neil, M., \& Lagnado, D. A. (2013). A General Structure for
Legal Arguments About Evidence Using Bayesian Networks. \emph{Cognitive
Science}, \emph{37}(1), 61--102.

\hypertarget{ref-Finkelstein1970A}{}
Finkelstein, M. O., \& Fairley, W. B. (1970). A Bayesian approach to
identification evidence. \emph{Harvard Law Review}, \emph{83}(3),
489--517.

\hypertarget{ref-fitelson2003ProbabilisticTheoryCoherence}{}
Fitelson, B. (2003a). A Probabilistic Theory of Coherence.
\emph{Analysis}, \emph{63}(3), 194--199.

\hypertarget{ref-fitelson2003comments}{}
Fitelson, B. (2003b). Comments on jim franklin's the representation of
context: Ideas from artificial intelligence (or, more remarks on the
contextuality of probability). \emph{Law, Probability and Risk},
\emph{2}(3), 201--204. Oxford Univ Press.

\hypertarget{ref-friedman1974}{}
Friedman, M. (1974). Explanation and scientific understanding.
\emph{Journal of Philosophy}, \emph{71}, 5--19.

\hypertarget{ref-Friedman2000}{}
Friedman, R. D. (2000). A presumption of innocence, not of even odds.
\emph{Stanford Law Review}, \emph{52}(4), 873--887.

\hypertarget{ref-friedmanEtAl1995}{}
Friedman, R. D., Allen, R. J., Balding, D. J., Donnelly, P., \& Kaye, D.
H. (1995). Probability and proof in State v. Skipper: An internet
exchange. \emph{Jurimetrics}, \emph{35}(3), 277--310.

\hypertarget{ref-Gallow2018No}{}
Gallow, J. (2018). No one can serve two epistemic masters.
\emph{Philosophical Studies}, \emph{175}(10), 2389--2398. Springer
Verlag.

\hypertarget{ref-gardiner2018}{}
Gardiner, G. (2018). Legal burdens of proof and statistical evidence. In
D. Coady \& J. Chase (Eds.), \emph{Routledge handbook of applied
epistemology}. Routledge.

\hypertarget{ref-glass2002}{}
Glass, D. H. (2002). Coherence, Explanation, and Bayesian Networks. In
G. Goos, J. Hartmanis, J. van Leeuwen, M. O'Neill, R. F. E. Sutcliffe,
C. Ryan, M. Eaton, et al. (Eds.), \emph{Artificial Intelligence and
Cognitive Science} (Vol. 2464, pp. 177--182). Berlin, Heidelberg:
Springer Berlin Heidelberg.

\hypertarget{ref-gordon2007}{}
Gordon, T. F., Prakken, H., \& Walton, D. (2007). The Carneades model of
argument and burden of proof. \emph{Artificial Intelligence},
\emph{171}(10-15), 875--896.

\hypertarget{ref-haack2011legal}{}
Haack, S. (2014). Legal probabilism: An epistemological dissent. In
\emph{Haack2014-HAAEMS} (pp. 47--77).

\hypertarget{ref-hastie2019CaseRelativePlausibilitya}{}
Hastie, R. (2019). The case for relative plausibility theory: Promising,
but insufficient. \emph{The International Journal of Evidence \& Proof},
\emph{23}(1-2), 134--140.

\hypertarget{ref-hepler2007ObjectorientedGraphicalRepresentations}{}
Hepler, A. B., Dawid, A. P., \& Leucari, V. (2007). Object-oriented
graphical representations of complex patterns of evidence. \emph{Law,
Probability and Risk}, \emph{6}(1-4), 275--293.

\hypertarget{ref-ho2008philosophy}{}
Ho, H. L. (2008). \emph{A philosophy of evidence law: Justice in the
search for truth}. Oxford University Press.

\hypertarget{ref-lai2019HowPlausibleRelative}{}
Ho, H. L. (2019). How plausible is the relative plausibility theory of
proof? \emph{The International Journal of Evidence \& Proof},
\emph{23}(1-2), 191--197.

\hypertarget{ref-Horowitz1996}{}
Horowitz, I. A., \& Kirkpatrick, L. C. (1996). A concept in search of a
definition: The effect of reasonable doubt instrcutions on certainty of
guilt standards and jury verdicts. \emph{Law and Human Behaviour},
\emph{20}(6), 655--670.

\hypertarget{ref-kadane2011probabilistic}{}
Kadane, J. B., \& Schum, D. A. (2011). \emph{A probabilistic analysis of
the sacco and vanzetti evidence}. John Wiley \& Sons.

\hypertarget{ref-kaye79}{}
Kaye, D. H. (1979). The laws of probability and the law of the land.
\emph{The University of Chicago Law Review}, \emph{47}(1), 34--56.

\hypertarget{ref-kaye1986admissibility}{}
Kaye, D. H. (1986). The admissibility of ``probability evidence'' in
criminal trials---part I. \emph{Jurimetrics Journal}, 343--346.

\hypertarget{ref-Kaye2010The-Double-Heli}{}
Kaye, D. H. (2010). \emph{The double helix and the law of evidence}.
Harvard University Press.

\hypertarget{ref-Koehler1996On-Conveying-th}{}
Koehler, J. J. (1996). On conveying the probative value of DNA evidence:
Frequencies, likelihood ratios, and error rates. \emph{University of
Colorado law Review}, \emph{67}, 859--886.

\hypertarget{ref-koscholke2016evaluating}{}
Koscholke, J. (2016). Evaluating Test Cases for Probabilistic Measures
of Coherence. \emph{Erkenntnis}, \emph{81}(1), 155--181.

\hypertarget{ref-laudan2006truth}{}
Laudan, L. (2006). \emph{Truth, error, and criminal law: An essay in
legal epistemology}. Cambridge University Press.

\hypertarget{ref-lempert1977modeling}{}
Lempert, R. O. (1977). Modeling relevance. \emph{Michigan Law Review},
\emph{75}, 1021--1057. JSTOR.

\hypertarget{ref-Lempert1986}{}
Lempert, R. O. (1986). The new evidence scholarship: Analysing the
process of proof. \emph{Boston University Law Review}, \emph{66},
439--477.

\hypertarget{ref-Lipton2004-LIPITT}{}
Lipton, P. (2004). \emph{Inference to the best explanation}.
Routledge/Taylor; Francis Group.

\hypertarget{ref-List2011Group}{}
List, C., \& Pettit, P. (2011). \emph{Group agency: The possibility,
design, and status of corporate agents}. Oxford University Press.

\hypertarget{ref-meijs2007}{}
Meijs, W., \& Douven, I. (2007). On the alleged impossibility of
coherence. \emph{Synthese}, \emph{157}(3), 347--360.

\hypertarget{ref-Merricks1995}{}
Merricks, T. (1995). Warrant entails truth. \emph{Philosophy and
Phenomenological Research}, \emph{55}, 841--855.

\hypertarget{ref-nance2019LimitationsRelativePlausibility}{}
Nance, D. A. (2019). The limitations of relative plausibility theory.
\emph{The International Journal of Evidence \& Proof}, \emph{23}(1-2),
154--160.

\hypertarget{ref-NRCI1992}{}
National Research Council. (1992). \emph{DNA technology in forensic
science \textup{{[}NRC I{]}}}. Committee on DNA technology in Forensic
Science, National Research Council.

\hypertarget{ref-neil2000BuildingLargescaleBayesian}{}
Neil, M., Fenton, N., \& Nielson, L. (2000). Building large-scale
Bayesian Networks. \emph{The Knowledge Engineering Review},
\emph{15}(3), 257--284.

\hypertarget{ref-Fenton2019Modelling}{}
Neil, M., Fenton, N., Lagnado, D., \& Gill, R. D. (2019). Modelling
competing legal arguments using bayesian model comparison and averaging.
\emph{Artificial Intelligence and Law}. Retrieved from
\url{https://doi.org/10.1007/s10506-019-09250-3}

\hypertarget{ref-Nesson1979Reasonable-doub}{}
Nesson, C. R. (1979). Reasonable doubt and permissive inferences: The
value of complexity. \emph{Harvard Law Review}, \emph{92}(6),
1187--1225.

\hypertarget{ref-newman1993}{}
Newman, J. O. (1993). Beyon ``reasonable doub''. \emph{New York
University Law Review}, \emph{68}(5), 979--1002.

\hypertarget{ref-olsson2001}{}
Olsson, E. J. (2001). Why Coherence Is Not Truth-Conducive.
\emph{Analysis}, \emph{61}(3), 236--241.

\hypertarget{ref-pardo2018}{}
Pardo, M. S. (2018). Safety vs.~Sensitivity: Possible worlds and the law
of evidence. \emph{Legal Theory}, \emph{24}(1), 50--75.

\hypertarget{ref-Pardo2008judicial}{}
Pardo, M. S., \& Allen, R. J. (2008). Judicial proof and the best
explanation. \emph{Law and Philosophy}, \emph{27}(3), 223--268.

\hypertarget{ref-Pennington1991}{}
Pennington, N., \& Hastie, R. (1991a). A cognitive theory of juror
decision making: The story model. \emph{Cardozo Law Review}, \emph{13},
519--557.

\hypertarget{ref-pennington1991cognitive}{}
Pennington, N., \& Hastie, R. (1991b). A cognitive theory of juror
decision making: The story model. \emph{Cardozo Law Review}, \emph{13},
519--557. HeinOnline.

\hypertarget{ref-pennington1992explaining}{}
Pennington, N., \& Hastie, R. (1992). Explaining the evidence: Tests of
the story model for juror decision making. \emph{Journal of personality
and social psychology}, \emph{62}(2), 189--204. American Psychological
Association.

\hypertarget{ref-Pettigrew2011Epistemic-Utili}{}
Pettigrew, R. (2011). Epistemic utility arguments for probabilism. In
\emph{Stanford encyclopedia of philosophy}.

\hypertarget{ref-Posner1973}{}
Posner, R. (1973). \emph{The economic analysis of law}. Brown \&
Company.

\hypertarget{ref-redmayne2008exploring}{}
Redmayne, M. (2008). Exploring the proof paradoxes. \emph{Legal Theory},
\emph{14}(4), 281--309. Cambridge University Press.

\hypertarget{ref-Robertson1995evidence}{}
Robertson, B., \& Vignaux, G. A. (1995). DNA evidence: Wrong answers or
wrong questions? \emph{Genetica}, \emph{96}, 145--152.

\hypertarget{ref-Schippers2019General}{}
Schippers, M., \& Koscholke, J. (2019). A General Framework for
Probabilistic Measures of Coherence. \emph{Studia Logica}.

\hypertarget{ref-schwartz2019WhatRelativePlausibility}{}
Schwartz, D. S., \& Sober, E. (2019). What is relative plausibility?
\emph{The International Journal of Evidence \& Proof}, \emph{23}(1-2),
198--204.

\hypertarget{ref-schweizer2013LawDoesnSay}{}
Schweizer, M. (2013). The Law Doesn't Say Much About Base Rates.
\emph{SSRN Electronic Journal}.

\hypertarget{ref-shen2007ScenariodrivenDecisionSupporta}{}
Shen, Q., Keppens, J., Aitken, C., Schafer, B., \& Lee, M. (2007). A
scenario-driven decision support system for serious crime investigation.
\emph{Law, Probability and Risk}, \emph{5}(2), 87--117.

\hypertarget{ref-shogenji1999}{}
Shogenji, T. (1999). Is Coherence Truth Conducive? \emph{Analysis},
\emph{59}(4), 338--345.

\hypertarget{ref-Shogenji2001Reply}{}
Shogenji, T. (2001). Reply to akiba on the probabilistic measure of
coherence. \emph{Analysis}, \emph{61}(2), 147--150. Oxford University
Press (OUP). Retrieved from
\url{https://doi.org/10.1093/analys/61.2.147}

\hypertarget{ref-Shogenji2006Why}{}
Shogenji, T. (2006). Why does coherence appear truth-conducive?
\emph{Synthese}, \emph{157}(3), 361--372. Springer Science; Business
Media LLC. Retrieved from
\url{https://doi.org/10.1007/s11229-006-9062-8}

\hypertarget{ref-Siebel2004On-Fitelsons-me}{}
Siebel, M. (2004). On Fitelson's measure of coherence. \emph{Analysis},
\emph{64}, 189--190.

\hypertarget{ref-siebel2006against}{}
Siebel, M. (2006). Against probabilistic measures of coherence. In
\emph{Coherence, truth and testimony} (pp. 43--68). Springer.

\hypertarget{ref-Smith_conviction_mind_2017}{}
Smith, M. (2017). When does evidence suffice for conviction?
\emph{Mind}.

\hypertarget{ref-Swinburne2001-SWIEJ}{}
Swinburne, R. (2001). \emph{Epistemic justification}. Oxford University
Press.

\hypertarget{ref-taroni2006bayesian}{}
Taroni, F., Biedermann, A., Bozza, S., Garbolino, P., \& Aitken, C.
(2014). \emph{Bayesian networks for probabilistic inference and decision
analysis in forensic science} (2nd ed.). John Wiley \& Sons.

\hypertarget{ref-Thomson86}{}
Thomson, J. J. (1986). Liability and individualized evidence. \emph{Law
and Contemporary Problems}, \emph{49}(3), 199--219.

\hypertarget{ref-tribe71}{}
Tribe, L. H. (1971). Trial by mathematics: Precision and ritual in the
legal process. \emph{Harvard Law Review}, \emph{84}(6), 1329--1393.

\hypertarget{ref-Underwood1977The-thumb-on-th}{}
Underwood, B. D. (1977). The thumb on the scale of justice: Burdens of
persuasion in criminal cases. \emph{Yale Law Journal}, \emph{86(7)},
1299--1348.

\hypertarget{ref-Urbach1993-HOWSRT}{}
Urbach, P., \& Howson, C. (1993). \emph{Scientific reasoning: The
bayesian approach}. Open Court.

\hypertarget{ref-urbaniak2018narration}{}
Urbaniak, R. (2018a). Narration in judiciary fact-finding: A
probabilistic explication. \emph{Artificial Intelligence and Law},
1--32.

\hypertarget{ref-Urbaniak2017Narration-in-ju}{}
Urbaniak, R. (2018b). Narration in judiciary fact-finding: A
probabilistic explication. \emph{Artificial Intelligence and Law},
1--32.

\hypertarget{ref-verheij2014catch}{}
Verheij, B. (2014). To catch a thief with and without numbers:
Arguments, scenarios and probabilities in evidential reasoning.
\emph{Law, Probability and Risk}, \emph{13}(3-4), 307--325. Citeseer.

\hypertarget{ref-verheijproof2017}{}
Verheij, B. (2017). Proof with and without probabilities. correct
evidential reasoning with presumptive arguments, coherent hypotheses and
degrees of uncertainty. \emph{Artificial Intelligence and Law}, 1--28.
Springer.

\hypertarget{ref-vlek2016stories}{}
Vlek, C. (2016). \emph{When stories and numbers meet in court:
Constructing and explaining bayesian networks for criminal cases with
scenarios}. Rijksuniversiteit Groningen.

\hypertarget{ref-vlek2015}{}
Vlek, C. S., Prakken, H., Renooij, S., \& Bart Verheij. (2015).
Representing the quality of crime scenarios in a bayesian network. In A.
Rotolo (Ed.), \emph{Legal knowledge and information systems} (pp.
133--140). IOS Press.

\hypertarget{ref-vlek2013modeling}{}
Vlek, C., Prakken, H., Renooij, S., \& Verheij, B. (2013). Modeling
crime scenarios in a bayesian network. In \emph{Proceedings of the
fourteenth international conference on artificial intelligence and law}
(pp. 150--159). ACM.

\hypertarget{ref-vlek2014building}{}
Vlek, C., Prakken, H., Renooij, S., \& Verheij, B. (2014). Building
bayesian networks for legal evidence with narratives: A case study
evaluation. \emph{Artificial Intelligence and Law}, \emph{22}, 375--421.
Springer.

\hypertarget{ref-wagenaar1993anchored}{}
Wagenaar, W., Van Koppen, P., \& Crombag, H. (1993). \emph{Anchored
narratives: The psychology of criminal evidence.} St Martin's Press.

\hypertarget{ref-walen2015}{}
Walen, A. (2015). Proof beyond a reasonable doubt: A balanced
retributive account. \emph{Louisiana Law Review}, \emph{76}(2),
355--446.

\hypertarget{ref-Walton2002}{}
Walton, D. N. (2002). \emph{Legal argumentation and evidence}. Penn
State University Press.

\hypertarget{ref-wells1992naked}{}
Wells, G. (1992). Naked statistical evidence of liability: Is subjective
probability enough? \emph{Journal of Personality and Social Psychology},
\emph{62}(5), 739--752. American Psychological Association.

\hypertarget{ref-williamson2010defence}{}
Williamson, J. (2010). \emph{In defence of objective bayesianism}.
Oxford University Press Oxford.

\end{document}
