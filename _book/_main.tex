\PassOptionsToPackage{unicode=true}{hyperref} % options for packages loaded elsewhere
\PassOptionsToPackage{hyphens}{url}
%
\documentclass[]{book}
\usepackage{lmodern}
\usepackage{amssymb,amsmath}
\usepackage{ifxetex,ifluatex}
\usepackage{fixltx2e} % provides \textsubscript
\ifnum 0\ifxetex 1\fi\ifluatex 1\fi=0 % if pdftex
  \usepackage[T1]{fontenc}
  \usepackage[utf8]{inputenc}
  \usepackage{textcomp} % provides euro and other symbols
\else % if luatex or xelatex
  \usepackage{unicode-math}
  \defaultfontfeatures{Ligatures=TeX,Scale=MatchLowercase}
\fi
% use upquote if available, for straight quotes in verbatim environments
\IfFileExists{upquote.sty}{\usepackage{upquote}}{}
% use microtype if available
\IfFileExists{microtype.sty}{%
\usepackage[]{microtype}
\UseMicrotypeSet[protrusion]{basicmath} % disable protrusion for tt fonts
}{}
\IfFileExists{parskip.sty}{%
\usepackage{parskip}
}{% else
\setlength{\parindent}{0pt}
\setlength{\parskip}{6pt plus 2pt minus 1pt}
}
\usepackage{hyperref}
\hypersetup{
            pdftitle={Title},
            pdfauthor={Marcello Di Bello and Rafal Urbaniak},
            pdfborder={0 0 0},
            breaklinks=true}
\urlstyle{same}  % don't use monospace font for urls
\usepackage{color}
\usepackage{fancyvrb}
\newcommand{\VerbBar}{|}
\newcommand{\VERB}{\Verb[commandchars=\\\{\}]}
\DefineVerbatimEnvironment{Highlighting}{Verbatim}{commandchars=\\\{\}}
% Add ',fontsize=\small' for more characters per line
\usepackage{framed}
\definecolor{shadecolor}{RGB}{248,248,248}
\newenvironment{Shaded}{\begin{snugshade}}{\end{snugshade}}
\newcommand{\AlertTok}[1]{\textcolor[rgb]{0.94,0.16,0.16}{#1}}
\newcommand{\AnnotationTok}[1]{\textcolor[rgb]{0.56,0.35,0.01}{\textbf{\textit{#1}}}}
\newcommand{\AttributeTok}[1]{\textcolor[rgb]{0.77,0.63,0.00}{#1}}
\newcommand{\BaseNTok}[1]{\textcolor[rgb]{0.00,0.00,0.81}{#1}}
\newcommand{\BuiltInTok}[1]{#1}
\newcommand{\CharTok}[1]{\textcolor[rgb]{0.31,0.60,0.02}{#1}}
\newcommand{\CommentTok}[1]{\textcolor[rgb]{0.56,0.35,0.01}{\textit{#1}}}
\newcommand{\CommentVarTok}[1]{\textcolor[rgb]{0.56,0.35,0.01}{\textbf{\textit{#1}}}}
\newcommand{\ConstantTok}[1]{\textcolor[rgb]{0.00,0.00,0.00}{#1}}
\newcommand{\ControlFlowTok}[1]{\textcolor[rgb]{0.13,0.29,0.53}{\textbf{#1}}}
\newcommand{\DataTypeTok}[1]{\textcolor[rgb]{0.13,0.29,0.53}{#1}}
\newcommand{\DecValTok}[1]{\textcolor[rgb]{0.00,0.00,0.81}{#1}}
\newcommand{\DocumentationTok}[1]{\textcolor[rgb]{0.56,0.35,0.01}{\textbf{\textit{#1}}}}
\newcommand{\ErrorTok}[1]{\textcolor[rgb]{0.64,0.00,0.00}{\textbf{#1}}}
\newcommand{\ExtensionTok}[1]{#1}
\newcommand{\FloatTok}[1]{\textcolor[rgb]{0.00,0.00,0.81}{#1}}
\newcommand{\FunctionTok}[1]{\textcolor[rgb]{0.00,0.00,0.00}{#1}}
\newcommand{\ImportTok}[1]{#1}
\newcommand{\InformationTok}[1]{\textcolor[rgb]{0.56,0.35,0.01}{\textbf{\textit{#1}}}}
\newcommand{\KeywordTok}[1]{\textcolor[rgb]{0.13,0.29,0.53}{\textbf{#1}}}
\newcommand{\NormalTok}[1]{#1}
\newcommand{\OperatorTok}[1]{\textcolor[rgb]{0.81,0.36,0.00}{\textbf{#1}}}
\newcommand{\OtherTok}[1]{\textcolor[rgb]{0.56,0.35,0.01}{#1}}
\newcommand{\PreprocessorTok}[1]{\textcolor[rgb]{0.56,0.35,0.01}{\textit{#1}}}
\newcommand{\RegionMarkerTok}[1]{#1}
\newcommand{\SpecialCharTok}[1]{\textcolor[rgb]{0.00,0.00,0.00}{#1}}
\newcommand{\SpecialStringTok}[1]{\textcolor[rgb]{0.31,0.60,0.02}{#1}}
\newcommand{\StringTok}[1]{\textcolor[rgb]{0.31,0.60,0.02}{#1}}
\newcommand{\VariableTok}[1]{\textcolor[rgb]{0.00,0.00,0.00}{#1}}
\newcommand{\VerbatimStringTok}[1]{\textcolor[rgb]{0.31,0.60,0.02}{#1}}
\newcommand{\WarningTok}[1]{\textcolor[rgb]{0.56,0.35,0.01}{\textbf{\textit{#1}}}}
\usepackage{longtable,booktabs}
% Fix footnotes in tables (requires footnote package)
\IfFileExists{footnote.sty}{\usepackage{footnote}\makesavenoteenv{longtable}}{}
\usepackage{graphicx,grffile}
\makeatletter
\def\maxwidth{\ifdim\Gin@nat@width>\linewidth\linewidth\else\Gin@nat@width\fi}
\def\maxheight{\ifdim\Gin@nat@height>\textheight\textheight\else\Gin@nat@height\fi}
\makeatother
% Scale images if necessary, so that they will not overflow the page
% margins by default, and it is still possible to overwrite the defaults
% using explicit options in \includegraphics[width, height, ...]{}
\setkeys{Gin}{width=\maxwidth,height=\maxheight,keepaspectratio}
\setlength{\emergencystretch}{3em}  % prevent overfull lines
\providecommand{\tightlist}{%
  \setlength{\itemsep}{0pt}\setlength{\parskip}{0pt}}
\setcounter{secnumdepth}{5}
% Redefines (sub)paragraphs to behave more like sections
\ifx\paragraph\undefined\else
\let\oldparagraph\paragraph
\renewcommand{\paragraph}[1]{\oldparagraph{#1}\mbox{}}
\fi
\ifx\subparagraph\undefined\else
\let\oldsubparagraph\subparagraph
\renewcommand{\subparagraph}[1]{\oldsubparagraph{#1}\mbox{}}
\fi

% set default figure placement to htbp
\makeatletter
\def\fps@figure{htbp}
\makeatother

\usepackage{todonotes}

\title{Title}
\author{Marcello Di Bello and Rafal Urbaniak}
\date{2021-02-02}

\begin{document}
\maketitle

{
\setcounter{tocdepth}{1}
\tableofcontents
}
\part{What is legal probabilism?}

\chapter{The emergence of legal probabilism}

This chapter will introduce legal probabilism and contain an account of early\\
discussions of legal probabilism, how
it came about, when, major contributions, etc.
I see essentially two moments in
the history of legal probabilism: the early days when probability
theory was invented (Bernoulli, Laplace, Condorcet, etc.), and then
the second half of the 20th century with the emergence
of the New Evidence Scholarship (Lempert)
and law and economics. But the history
might be more complicated.

\chapter{A skeptical perspective}

This chapter would discuss puzzles and
hypothetical scenarios, mostly the debate
about naked statistical evidence. This
is a discussion of the most compelling objections that have
been raised against legal probabilism.
Most of these objections
trace back to Cohen, although
other pivotal players
are Laurence Tribe and Ronald Allen.
I think this chapter could
also have a historical flavor or perhaps it could be more
systematic. Not sure about
the best presentation format.

\section{The difficulty about conjunction}

\section{The complexity objection}

\section{The problem of corroboration}

\section{The problem of artificial precision}

\section{Naked statistical evidence}\label{sec:naked}

\section{The problem of priors}

\section{The reference class problem}

\section{Non-probabilistic perspectives}

\part{Evidence assessment}

After the first part, the rest of the book will
be a deep dive into what probability theory can do for us
when it is applied to trial proceedings.

Instead of addressing the common objections
upfront, the strategy of the book would
be to set the objections
aside -- keep them on the back burner
as it were -- and return to them once we have
a clearer sense of legal probabilism
and its limits.

\todo{We need to clearly set the limit of discussion of objections in the first part}

This part of the book is devoted to how probability
theory can help---or not help---in assessing
trial evidence. I think it is important that
we start very simple and then we progressively
get more complex.

\chapter{Bayes' Theorem and the usual fallacies}

This chapter shows how we can use probability theory
and Bayes' theory to spot common probabilistic fallacies,
prosecutor's fallacy, base rate fallacy, etc.
This is the simple stuff.

I think this chapter should also show the limitation
of this approach. That is, we should make clear that these
are probabilistic fallacies. They are fallacies only insofar as the trier of facts
aim to determine the posterior probability of guilt. Which they might not.
\todo{careful here, some come up without explicit calculations}

The chapter will also be accompanied
by case studies.

\section{Assuming independence}

\section{The prosecutor's fallacy}

\section{Base rate fallacy}

\section{Defense attorney's fallacy}

\section{Uniqueness fallacy}

\section{Case studies}

\subsection{Collins}

\subsection{Sally Clark}

\chapter{Complications and caveats}

\todo{not sure if this isn't too early}

Here we examine a number of complications that
emerge from the simple Bayes' theorem approach
described in the earlier chapter. Here are some of the common difficulties:

\begin{itemize}
 
 \item How do we determine the priors?
 
 \item More generally, how do we determine the numerical 
 values of any of the probabilities involved? 
 It might work for DNA matches, but what about non0numerical evidence 
 such as eyewitnesses? 
 
 \item How do we combine different pieces of evidence?  
 
 \item How we we formulate complex hypotheses, 
 say narratives, stories or explanations? 
 
 \item How do we take into account things 
 like the coherence of one's story or 
 the explanatory power of one's hypothesis?
 (evidence-to-hypothesis reasoning 
 versus hypothesis-to-evidence reasoning).
 
 \item Ronald Allen's objections 
 and Susan Haack's objections. 
 
 \end{itemize}

\section{Complex hypotheses and complex bodies of evidence}

\section{Source, activity and offense level hypotheses}

\section{Where do the numbers come from?}

\section{Modeling corroboration}

\section{Stories, explanations and coherence}

\chapter{Likelihood Ratios and Relevance}

Here we present likelihood ratios as a possible
answer to some of the complications. Pros and cons
of this approach. It addresses
the problems of priors to some extent,
but it leaves a lot of the other complications essentially unresolved.
The likelihood approach raises
complication of its own.

\section{Odds version of Bayes' theorem}

\section{Bayesian factor v. likelihood ratio}

\section{Choosing competing hypotheses}

\section{The two-stain problem}

\section{Case study: cold-hit DNA match evaluation}

\chapter{Bayesian Networks}

Here we present Bayesian networks
as the best answer that legal probabilists
can offer. We illustrate Bayesian networks
with examples and show how they can answer
some of the complications. We try to be
as honest as possible. We want to be a reliable and trustworthy
source of discussion, not partisan. We also discuss
how Bayesian networks can help address certain
puzzles about relevance.

\section{Multiple pieces of evidence and complex hypotheses}

\section{Bayesian networks to the rescue}

\section{Legal evidence idioms}

\section{Scenario idioms}

\section{Modeling relevance}

\section{Case study: Sally Clark}

\todo{Do we really want to get into BNs for DNA evidence evaluation?}

\section{DNA evidence}

\todo{We already mention corroboration at two places, we should clean this up.}

\chapter{Corroboration}

Here we zoom into a particular topic. This should be a
place to review the literature on corroboration and
for Rafal to present his own probabilistic solution to the corroboration puzzle.

Use BNs in the exposition!

\section{Boole's formula and Cohen's challenge}

\section{Modeling substantial rise in case of agreement}

\section{Ekel\"of's corroboration measure and evidentiary mechanisms}

\section{General approach  with multiple false stories and multiple witnesses}

\chapter{Coherence}

Looks like coherence (cohesiveness and related ideas)
plays an important role in assessing evidence at trial.
Here it would be place to review the literature on coherence
and for Rafal to preset his
own probabilistic solution to the coherence puzzle,
emphasizing legal applications.

\section{Philosophical motivations}

\section{Existing probabilistic coherence measures}

\section{An array of counterexamples}

\section{Coherence of structured narrations with bayesian networks}

\section{Application to legal cases}

\chapter{Probabilistic approach to narrations}

\section{New legal probabilism}

\section{A probabilistic framework for narrations}

\section{Probabilistic constraints and desiderata for narrations}

\section{Bayesian network deployment}

\section{Comparison to existing approaches}

\todo{perhaps Allen, Haack \& Moss here? }

\begin{itemize}
\item
  The Dutch school and its challenges
\item
  Merging/aggregation/selection issues
\item
  Conditions on narration
\item
  Formal representation and programmatic deployment
\end{itemize}

\part{Trial Decisions}

We turn from assessing
evidence to trial decisions.
The question is this, when is the evidence strong
enough to meet the governing burden of proof?

\chapter{Standards of proof}

Discuss various probabilistic explications and their challenges (Rafal's paper on this?)

\section{A probabilistic take on the narrative approach}

Rafal's stuff

\chapter{Expected utility}

This chapter reviews the literature
that describes how
expected utility can be used
to define rules for trial decisions.

\chapter{The risk of error}

This chapter introduces different
ways to think about the risk
of error at trial. One dimension of the risk of error
flows from the posterior probabilities \(P(Guilt | Evidence)\).
The other dimension flows from the conditional probabilities
\(P(Conviction | Innocence)\).
This an opportunity for Marcello to present the arguments in his Mind paper,
which however Rafal has criticized. So hopefully this chapter
will be a very balanced account of the topic!

\chapter{Fairness}

This chapter discusses how decisions can be fair and to what extent probability
theory can help us think about the fairness of decisions.
One important notion of fairness that probability theory
can capture is that of equal distribution of the risk of error.
This draws on some of Marcello's argument in the Ethics paper.

\todo{talk about incompatibility of definitions, Hedden's argument against measures of fairness etc.}

\part{Trial Institutions}

Finally, this part of the book should
assess some institutions of the trial system
using probability theory and
cognate theories. I am not sure
if this is too much, but I am
putting it here just in case.

\todo{Are we competent to discuss this?}

\chapter{Rules of Evidence}

\chapter{Cross-examination}

\chapter{Conclusion}

I'd like this conclusion
to be a very
careful and nuanced discussion of the
good and bad things about
legal probabilism. What difficulties can
in principle be overcome and what other difficulties are instead
inherent to legal probabilism and thus inescapable?

\hypertarget{preface}{%
\chapter*{Preface}\label{preface}}
\addcontentsline{toc}{chapter}{Preface}

testing again

\begin{align} 
  f\left(k\right) = \binom{n}{k} p^k\left(1-p\right)^{n-k}
  \label{eq:binom}
\end{align}

This is a citation (Diamond, \protect\hyperlink{ref-diamond90}{1990}) which uses keys from the bib file listed in the preamble.

Equation \eqref{eq:binom}\footnote{This is a footnote containing a double citation (Dahlman, \protect\hyperlink{ref-dahlmanNakedStat2020}{2020}; Diamond, \protect\hyperlink{ref-diamond90}{1990}).}

Note that chapter files are found and compiled automatically, but the file names have to contain chapter numbers first. For instance, we used \texttt{01-intro.Rmd}, placed in the same folder. Observe how we included r code inline.

\begin{Shaded}
\begin{Highlighting}[]
\KeywordTok{plot}\NormalTok{(cars)}
\end{Highlighting}
\end{Shaded}

\includegraphics{_main_files/figure-latex/fig-margin-1.pdf}

\hypertarget{ch:intro}{%
\chapter{Introduction}\label{ch:intro}}

You can label chapter and section titles using \texttt{\{\#label\}} after them, e.g., we can reference Chapter \ref{ch:intro}.
Let's use chapter labels starting with ``ch:''.

Figures and tables with captions will be placed in \texttt{figure} and \texttt{table} environments, respectively.

\begin{Shaded}
\begin{Highlighting}[]
\KeywordTok{par}\NormalTok{(}\DataTypeTok{mar =} \KeywordTok{c}\NormalTok{(}\DecValTok{4}\NormalTok{, }\DecValTok{4}\NormalTok{, }\FloatTok{.1}\NormalTok{, }\FloatTok{.1}\NormalTok{))}
\KeywordTok{plot}\NormalTok{(pressure, }\DataTypeTok{type =} \StringTok{'b'}\NormalTok{, }\DataTypeTok{pch =} \DecValTok{19}\NormalTok{)}
\end{Highlighting}
\end{Shaded}

\begin{figure}

{\centering \includegraphics[width=0.8\linewidth]{_main_files/figure-latex/nice-fig-1} 

}

\caption{Here is a nice figure!}\label{fig:nice-fig}
\end{figure}

Reference a figure by its code chunk label with the \texttt{fig:} prefix, e.g., see Figure \ref{fig:nice-fig}. Similarly, you can reference tables generated from \texttt{knitr::kable()}, e.g., see Table \ref{tab:nice-tab}.

\begin{Shaded}
\begin{Highlighting}[]
\NormalTok{knitr}\OperatorTok{::}\KeywordTok{kable}\NormalTok{(}
  \KeywordTok{head}\NormalTok{(iris, }\DecValTok{20}\NormalTok{), }\DataTypeTok{caption =} \StringTok{'Here is a nice table!'}\NormalTok{,}
  \DataTypeTok{booktabs =} \OtherTok{TRUE}
\NormalTok{)}
\end{Highlighting}
\end{Shaded}

\begin{table}

\caption{\label{tab:nice-tab}Here is a nice table!}
\centering
\begin{tabular}[t]{rrrrl}
\toprule
Sepal.Length & Sepal.Width & Petal.Length & Petal.Width & Species\\
\midrule
5.1 & 3.5 & 1.4 & 0.2 & setosa\\
4.9 & 3.0 & 1.4 & 0.2 & setosa\\
4.7 & 3.2 & 1.3 & 0.2 & setosa\\
4.6 & 3.1 & 1.5 & 0.2 & setosa\\
5.0 & 3.6 & 1.4 & 0.2 & setosa\\
\addlinespace
5.4 & 3.9 & 1.7 & 0.4 & setosa\\
4.6 & 3.4 & 1.4 & 0.3 & setosa\\
5.0 & 3.4 & 1.5 & 0.2 & setosa\\
4.4 & 2.9 & 1.4 & 0.2 & setosa\\
4.9 & 3.1 & 1.5 & 0.1 & setosa\\
\addlinespace
5.4 & 3.7 & 1.5 & 0.2 & setosa\\
4.8 & 3.4 & 1.6 & 0.2 & setosa\\
4.8 & 3.0 & 1.4 & 0.1 & setosa\\
4.3 & 3.0 & 1.1 & 0.1 & setosa\\
5.8 & 4.0 & 1.2 & 0.2 & setosa\\
\addlinespace
5.7 & 4.4 & 1.5 & 0.4 & setosa\\
5.4 & 3.9 & 1.3 & 0.4 & setosa\\
5.1 & 3.5 & 1.4 & 0.3 & setosa\\
5.7 & 3.8 & 1.7 & 0.3 & setosa\\
5.1 & 3.8 & 1.5 & 0.3 & setosa\\
\bottomrule
\end{tabular}
\end{table}

\hypertarget{refs}{}
\leavevmode\hypertarget{ref-dahlmanNakedStat2020}{}%
Dahlman, C. (2020). Naked statistical evidence and incentives for lawful conduct. \emph{International Journal of Evidence and Proof}, \emph{24}(2), 162--179.

\leavevmode\hypertarget{ref-diamond90}{}%
Diamond, H. A. (1990). Reasonable doubt: To define, or not to define. \emph{Columbia Law Review}, \emph{90}(6), 1716--1736.

\end{document}
