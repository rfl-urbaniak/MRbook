% Options for packages loaded elsewhere
\PassOptionsToPackage{unicode}{hyperref}
\PassOptionsToPackage{hyphens}{url}
\PassOptionsToPackage{dvipsnames,svgnames,x11names}{xcolor}
%
\documentclass[
  11pt,
  dvipsnames,enabledeprecatedfontcommands]{scrartcl}
\usepackage{amsmath,amssymb}
\usepackage{lmodern}
\usepackage{iftex}
\ifPDFTeX
  \usepackage[T1]{fontenc}
  \usepackage[utf8]{inputenc}
  \usepackage{textcomp} % provide euro and other symbols
\else % if luatex or xetex
  \usepackage{unicode-math}
  \defaultfontfeatures{Scale=MatchLowercase}
  \defaultfontfeatures[\rmfamily]{Ligatures=TeX,Scale=1}
\fi
% Use upquote if available, for straight quotes in verbatim environments
\IfFileExists{upquote.sty}{\usepackage{upquote}}{}
\IfFileExists{microtype.sty}{% use microtype if available
  \usepackage[]{microtype}
  \UseMicrotypeSet[protrusion]{basicmath} % disable protrusion for tt fonts
}{}
\usepackage{xcolor}
\usepackage{graphicx}
\makeatletter
\def\maxwidth{\ifdim\Gin@nat@width>\linewidth\linewidth\else\Gin@nat@width\fi}
\def\maxheight{\ifdim\Gin@nat@height>\textheight\textheight\else\Gin@nat@height\fi}
\makeatother
% Scale images if necessary, so that they will not overflow the page
% margins by default, and it is still possible to overwrite the defaults
% using explicit options in \includegraphics[width, height, ...]{}
\setkeys{Gin}{width=\maxwidth,height=\maxheight,keepaspectratio}
% Set default figure placement to htbp
\makeatletter
\def\fps@figure{htbp}
\makeatother
\setlength{\emergencystretch}{3em} % prevent overfull lines
\providecommand{\tightlist}{%
  \setlength{\itemsep}{0pt}\setlength{\parskip}{0pt}}
\setcounter{secnumdepth}{5}
%\documentclass{article}

% %packages
\usepackage{booktabs}
%\usepackage[left]{showlabels}
\usepackage{multirow}
\usepackage{subcaption}
\usepackage{wrapfig}
\usepackage{graphicx}
\usepackage{longtable}
\usepackage{ragged2e}
\usepackage{etex}
%\usepackage{yfonts}
\usepackage{marvosym}
\usepackage[notextcomp]{kpfonts}
\usepackage{nicefrac}
\newcommand*{\QED}{\hfill \footnotesize {\sc Q.e.d.}}
\usepackage{floatrow}
\usepackage{multicol}

\usepackage[textsize=footnotesize]{todonotes}
\newcommand{\ali}[1]{\todo[color=gray!40]{\textbf{Alicja:} #1}}
\newcommand{\mar}[1]{\todo[color=blue!40]{#1}}
\newcommand{\raf}[1]{\todo[color=olive!40]{#1}}

%\linespread{1.5}
\newcommand{\indep}{\!\perp \!\!\! \perp\!}


\setlength{\parindent}{10pt}
\setlength{\parskip}{1pt}


%language
%\usepackage{times}
\usepackage{mathptmx}
\usepackage[scaled=0.86]{helvet}
\usepackage{t1enc}
%\usepackage[utf8x]{inputenc}
%\usepackage[polish]{babel}
%\usepackage{polski}




%AMS
\usepackage{amsfonts}
\usepackage{amssymb}
\usepackage{amsthm}
\usepackage{amsmath}
\usepackage{mathtools}

\usepackage{geometry}
 \geometry{a4paper,left=35mm,top=20mm,}


%environments
\newtheorem{fact}{Fact}


% allow page breaks in equations
\allowdisplaybreaks


%abbreviations
\newcommand{\ra}{\rangle}
\newcommand{\la}{\langle}
\newcommand{\n}{\neg}
\newcommand{\et}{\wedge}
\newcommand{\jt}{\rightarrow}
\newcommand{\ko}[1]{\forall  #1\,}
\newcommand{\ro}{\leftrightarrow}
\newcommand{\exi}[1]{\exists\, {_{#1}}}
\newcommand{\pr}[1]{\ensuremath{\mathsf{P}(#1)}}
\newcommand{\cost}{\mathsf{cost}}
\newcommand{\benefit}{\mathsf{benefit}}
\newcommand{\ut}{\mathsf{ut}}

\newcommand{\odds}{\mathsf{Odds}}
\newcommand{\ind}{\mathsf{Ind}}
\newcommand{\nf}[2]{\nicefrac{#1\,}{#2}}
\newcommand{\R}[1]{\texttt{#1}}
\newcommand{\prr}[1]{\mbox{$\mathtt{P}_{prior}(#1)$}}
\newcommand{\prp}[1]{\mbox{$\mathtt{P}_{posterior}(#1)$}}



\newtheorem{q}{\color{blue}Question}
\newtheorem{lemma}{Lemma}
\newtheorem{theorem}{Theorem}
\newtheorem{corollary}{Corollary}[fact]


%technical intermezzo
%---------------------

\newcommand{\intermezzoa}{
	\begin{minipage}[c]{13cm}
	\begin{center}\rule{10cm}{0.4pt}



	\tiny{\sc Optional Content Starts}
	
	\vspace{-1mm}
	
	\rule{10cm}{0.4pt}\end{center}
	\end{minipage}\nopagebreak 
	}


\newcommand{\intermezzob}{\nopagebreak 
	\begin{minipage}[c]{13cm}
	\begin{center}\rule{10cm}{0.4pt}

	\tiny{\sc Optional Content Ends}
	
	\vspace{-1mm}
	
	\rule{10cm}{0.4pt}\end{center}
	\end{minipage}
	}
	
	
%--------------------






















\newtheorem*{reply*}{Reply}
\usepackage{enumitem}
\newcommand{\question}[1]{\begin{enumerate}[resume,leftmargin=0cm,labelsep=0cm,align=left]
\item #1
\end{enumerate}}

\usepackage{float}

% \setbeamertemplate{blocks}[rounded][shadow=true]
% \setbeamertemplate{itemize items}[ball]
% \AtBeginPart{}
% \AtBeginSection{}
% \AtBeginSubsection{}
% \AtBeginSubsubsection{}
% \setlength{\emergencystretch}{0em}
% \setlength{\parskip}{0pt}






\usepackage[authoryear]{natbib}

%\bibliographystyle{apalike}



\usepackage{tikz}
\usetikzlibrary{positioning,shapes,arrows}

\ifLuaTeX
  \usepackage{selnolig}  % disable illegal ligatures
\fi
\IfFileExists{bookmark.sty}{\usepackage{bookmark}}{\usepackage{hyperref}}
\IfFileExists{xurl.sty}{\usepackage{xurl}}{} % add URL line breaks if available
\urlstyle{same} % disable monospaced font for URLs
\hypersetup{
  pdftitle={Key Ideas for LP Book},
  pdfauthor={Marcello},
  colorlinks=true,
  linkcolor={Maroon},
  filecolor={Maroon},
  citecolor={Blue},
  urlcolor={blue},
  pdfcreator={LaTeX via pandoc}}

\title{Key Ideas for LP Book}
\author{Marcello}
\date{}

\begin{document}
\maketitle

\paragraph*{Not just high probability}

Legal probabilism, in the simple version, focuses on the probability of
liability as the main criterion for trial decision-making. If this
probability is sufficiently high, the decision should be against the
defendant, and otherwise it should favor the defendant.

But other dimensions (should) guide decision-making and they might not
be reducible to the probability of liability. Some of these other
dimensions are:

\begin{itemize}
\item
  How certain are we about the probability of liability? (Higher-order
  uncertainty)
\item
  How good (specific, coherent, plausible, explanatory powerful) is the
  story presented?
\item
  Did the defense challenged the other party's story? Did the story
  survive the challenges?
\item
  Is any evidence missing?
\end{itemize}

A more sophisticated version of legal probabilism, then, should be able
to do at least two things: first, formally model these additional
dimension using the language of probability (or determine to what extent
they fall outside the scope of probability theory and cognate theories);
and second, show why relying on these additional dimensions in
decision-making do foster important values, such as the accuracy and
fairness of trial decisions.

\vspace{10mm}

\noindent So we can envision four central chapters:

\paragraph*{Chapter: Higher-order probability}

See existing chapter and paper on higher-order legal probabilism.

\paragraph*{Chapter: Narratives, specificity, coherence etc.}

See Rafal's paper on coherence.

\paragraph*{Chapter: Cross-examination and arguments}

See Marcello's paper on cross-examination and Bayesian networks, and
also paper on awareness growth and Bayesian networks.

\paragraph*{Chapter: Gaps in Evidence}

See existing paper on gaps in the evidence.

\vspace{10mm}

\noindent This more sophisticated version of legal probabilism should
answer some of existing challenges to simple legal probabilism.

\paragraph*{Challenge 1: We do not have the numbers}

Many critics of legal probabilism complaint that it is difficult to find
all the numbers required by the probability tables of a Bayesiaan
network. So, then, often these numbers are inserted as guess work,
educated guesses or simply becaue they cannot be left blank.
\textit{Unclear where our book will address this challange. Use comparative probability? }

\paragraph*{Challenge 2: Evidence is evaluated holistically}

. \textit{The chapter 
on story coherence should address this challange.}

\paragraph*{Challenge 3: Learning isn't updating}

Ronald Allen complains that Bayesian updating isn't an adequate model of
what goes on in the courtroom when evidence is presented. The
decision-makers do not start from priors and update them based on the
pieces of evidence presented. What happens is more complicated and
cannot be modeled by Bayesian updating. \textit{The chapter 
on cross-examination and arguments should address this challange.}

\paragraph*{Challenge 4: Trials are adversarial}

Trials are often adversarial. Evidence is examined and cross-examined.
How can this adversarial process be modeled probabilistically?
\textit{The chapter on cross-examination and arguments should address this challange.}

\paragraph*{Challenge 5: No evidence that probability reduces errors}

It is clear that people make probabilistic mistakes in reasoning, but
does this show that mistaken convictions are caused by these
probabilistic mistakes? There is no evidence of that. In what way does
probability actually improve the accuracy of legal decisions?
\textit{Discussion about accuracy and fairness should address this challange}

\end{document}
