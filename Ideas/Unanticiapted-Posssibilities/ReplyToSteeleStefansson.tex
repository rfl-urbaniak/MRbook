% Options for packages loaded elsewhere
\PassOptionsToPackage{unicode}{hyperref}
\PassOptionsToPackage{hyphens}{url}
\PassOptionsToPackage{dvipsnames,svgnames*,x11names*}{xcolor}
%
\documentclass[
  11pt,
  dvipsnames,enabledeprecatedfontcommands]{scrartcl}
\usepackage{amsmath,amssymb}
\usepackage{lmodern}
\usepackage{ifxetex,ifluatex}
\ifnum 0\ifxetex 1\fi\ifluatex 1\fi=0 % if pdftex
  \usepackage[T1]{fontenc}
  \usepackage[utf8]{inputenc}
  \usepackage{textcomp} % provide euro and other symbols
\else % if luatex or xetex
  \usepackage{unicode-math}
  \defaultfontfeatures{Scale=MatchLowercase}
  \defaultfontfeatures[\rmfamily]{Ligatures=TeX,Scale=1}
\fi
% Use upquote if available, for straight quotes in verbatim environments
\IfFileExists{upquote.sty}{\usepackage{upquote}}{}
\IfFileExists{microtype.sty}{% use microtype if available
  \usepackage[]{microtype}
  \UseMicrotypeSet[protrusion]{basicmath} % disable protrusion for tt fonts
}{}
\usepackage{xcolor}
\IfFileExists{xurl.sty}{\usepackage{xurl}}{} % add URL line breaks if available
\IfFileExists{bookmark.sty}{\usepackage{bookmark}}{\usepackage{hyperref}}
\hypersetup{
  pdftitle={Awareness Growth in Bayesian Networks},
  pdfauthor={Marcello/Rafal},
  colorlinks=true,
  linkcolor=Maroon,
  filecolor=Maroon,
  citecolor=Blue,
  urlcolor=blue,
  pdfcreator={LaTeX via pandoc}}
\urlstyle{same} % disable monospaced font for URLs
\usepackage{graphicx}
\makeatletter
\def\maxwidth{\ifdim\Gin@nat@width>\linewidth\linewidth\else\Gin@nat@width\fi}
\def\maxheight{\ifdim\Gin@nat@height>\textheight\textheight\else\Gin@nat@height\fi}
\makeatother
% Scale images if necessary, so that they will not overflow the page
% margins by default, and it is still possible to overwrite the defaults
% using explicit options in \includegraphics[width, height, ...]{}
\setkeys{Gin}{width=\maxwidth,height=\maxheight,keepaspectratio}
% Set default figure placement to htbp
\makeatletter
\def\fps@figure{htbp}
\makeatother
\setlength{\emergencystretch}{3em} % prevent overfull lines
\providecommand{\tightlist}{%
  \setlength{\itemsep}{0pt}\setlength{\parskip}{0pt}}
\setcounter{secnumdepth}{5}
%\documentclass{article}

% %packages
\usepackage{booktabs}
%\usepackage[left]{showlabels}
\usepackage{multirow}
\usepackage{subcaption}
\usepackage{wrapfig}
\usepackage{graphicx}
\usepackage{longtable}
\usepackage{ragged2e}
\usepackage{etex}
%\usepackage{yfonts}
\usepackage{marvosym}
\usepackage[notextcomp]{kpfonts}
\usepackage{nicefrac}
\newcommand*{\QED}{\hfill \footnotesize {\sc Q.e.d.}}
\usepackage{floatrow}
\usepackage{multicol}

\usepackage[textsize=footnotesize]{todonotes}
\newcommand{\ali}[1]{\todo[color=gray!40]{\textbf{Alicja:} #1}}
\newcommand{\mar}[1]{\todo[color=blue!40]{#1}}
\newcommand{\raf}[1]{\todo[color=olive!40]{#1}}

%\linespread{1.5}
\newcommand{\indep}{\!\perp \!\!\! \perp\!}


\setlength{\parindent}{10pt}
\setlength{\parskip}{1pt}


%language
%\usepackage{times}
\usepackage{mathptmx}
\usepackage[scaled=0.86]{helvet}
\usepackage{t1enc}
%\usepackage[utf8x]{inputenc}
%\usepackage[polish]{babel}
%\usepackage{polski}




%AMS
\usepackage{amsfonts}
\usepackage{amssymb}
\usepackage{amsthm}
\usepackage{amsmath}
\usepackage{mathtools}

\usepackage{geometry}
 \geometry{a4paper,left=35mm,top=20mm,}


%environments
\newtheorem{fact}{Fact}


% allow page breaks in equations
\allowdisplaybreaks


%abbreviations
\newcommand{\ra}{\rangle}
\newcommand{\la}{\langle}
\newcommand{\n}{\neg}
\newcommand{\et}{\wedge}
\newcommand{\jt}{\rightarrow}
\newcommand{\ko}[1]{\forall  #1\,}
\newcommand{\ro}{\leftrightarrow}
\newcommand{\exi}[1]{\exists\, {_{#1}}}
\newcommand{\pr}[1]{\ensuremath{\mathsf{P}(#1)}}
\newcommand{\ppr}[2]{\ensuremath{\mathsf{P}^{#1}(#2)}}
\newcommand{\cost}{\mathsf{cost}}
\newcommand{\benefit}{\mathsf{benefit}}
\newcommand{\ut}{\mathsf{ut}}

\newcommand{\odds}{\mathsf{Odds}}
\newcommand{\ind}{\mathsf{Ind}}
\newcommand{\nf}[2]{\nicefrac{#1\,}{#2}}
\newcommand{\R}[1]{\texttt{#1}}
\newcommand{\prr}[1]{\mbox{$\mathtt{P}_{prior}(#1)$}}
\newcommand{\prp}[1]{\mbox{$\mathtt{P}_{posterior}(#1)$}}



\newtheorem{q}{\color{blue}Question}
\newtheorem{lemma}{Lemma}
\newtheorem{theorem}{Theorem}
\newtheorem{corollary}{Corollary}[fact]


%technical intermezzo
%---------------------

\newcommand{\intermezzoa}{
	\begin{minipage}[c]{13cm}
	\begin{center}\rule{10cm}{0.4pt}



	\tiny{\sc Optional Content Starts}
	
	\vspace{-1mm}
	
	\rule{10cm}{0.4pt}\end{center}
	\end{minipage}\nopagebreak 
	}


\newcommand{\intermezzob}{\nopagebreak 
	\begin{minipage}[c]{13cm}
	\begin{center}\rule{10cm}{0.4pt}

	\tiny{\sc Optional Content Ends}
	
	\vspace{-1mm}
	
	\rule{10cm}{0.4pt}\end{center}
	\end{minipage}
	}
	
	
%--------------------






















\newtheorem*{reply*}{Reply}
\usepackage{enumitem}
\newcommand{\question}[1]{\begin{enumerate}[resume,leftmargin=0cm,labelsep=0cm,align=left]
\item #1
\end{enumerate}}

\usepackage{float}

% \setbeamertemplate{blocks}[rounded][shadow=true]
% \setbeamertemplate{itemize items}[ball]
% \AtBeginPart{}
% \AtBeginSection{}
% \AtBeginSubsection{}
% \AtBeginSubsubsection{}
% \setlength{\emergencystretch}{0em}
% \setlength{\parskip}{0pt}






\usepackage[authoryear]{natbib}

%\bibliographystyle{apalike}



\usepackage{tikz}
\usetikzlibrary{positioning,shapes,arrows}

\ifluatex
  \usepackage{selnolig}  % disable illegal ligatures
\fi

\title{Awareness Growth in Bayesian Networks}
\usepackage{etoolbox}
\makeatletter
\providecommand{\subtitle}[1]{% add subtitle to \maketitle
  \apptocmd{\@title}{\par {\large #1 \par}}{}{}
}
\makeatother
\subtitle{Reply to Steele and Stefánsson}
\author{Marcello/Rafal}
\date{}

\begin{document}
\maketitle

\hypertarget{introduction}{%
\section{Introduction}\label{introduction}}

Learning is modeled in the Bayesian framework by the rule of
conditionalization. This rule posits that the agent's new degree of
belief in a proposition \(H\) after a learning experience \(E\) should
be the same as the agent's old degree of belief in \(H\) conditional on
\(E\). That is, \[\ppr{E}{H}=\pr{H \vert E},\] where \(\pr{}\)
represents the agent's old degree of belief (before the learning
experience \(E\)) and \(\ppr{E}{}\) represents the agent's new degree of
belief (after the learning experience \(E\)).

One assumption here is that \(E\) is learned with certainty. After the
agent learns about \(E\), there is no longer any doubt about the truth
of \(E\). This assumption has been the topic of extensive discussion in
the literature.\footnote{As is well-known, Jeffrey's conditionalization
  relaxes this assumption.} The other assumption---which we focus on
here---is that \(E\) and \(H\) belong to the agent's algebra of
propositions. This algebra models the agent's awareness state, the
propositions that the agent entertains as live possibilities.

Bayesian conditionalization makes it impossible for an agent to learn
something they have never thought about. The algebra is fixed once and
for all, since the learning experience does not modify it. This forces a
great deal of rigidity on the learning process. Crucially, even before
learning about \(E\), the agent already knows the degree of belief in
any proposition conditional on \(E\). This picture commits the agent to
the specification of their `total possible future experience' (Howson
1976, The Development of Logical Probability), as though learning was
confined to an `initial prison' (Lakatos, 1968, Changes in the Problem
of Inductive Logic).

But, arguably, the learning process is more complex than what
conditionalization allows. Not only do we learn that some propositions
that we were entertaining are true or false, but we may also learn new
propositions that we did not entertain before. Or we may entertain new
propositions---without necessarily learning that they are true or
false---and this change in awareness may in turn change what we already
believe. How should this more complex learning process be modeled by
Bayesianism? Call this the problem of awareness growth. This problem can
perhaps be divided into two parts: (i) how to model \textit{learning} a
new proposition not in the initial awareness state of the agent; (ii)
how to model \textit{entertaining} a new proposition not in the initial
awareness state of the agent (without yet learning it).

Critics of Bayesianism and sympathizers alike have been discussing the
problem of awareness growth under different names for quite some time,
at least since the eighties. This problem arises in a number of
different contexts, for example, new scientific theories (Glymour, 1980,
Why I am not a Bayesian; Chihara 1987, Some Problems for Bayesian
Confirmation Theory; Earmann 1992, Bayes of Bust?), language changes and
paradigm shifts (Williamson 2003, Bayesianism and Language Change), and
theories of induction (Zabell, Predicting the Unpredictable).

Now, of course, the algebra of propositions could in principle be so
rich to contain anything that could possibly be conceived, expressed,
thought of. Such an algebra would not need to change at any point in the
future. God-like agents could be associated with such rich algebra of
propositions, but this is hardly a plausible model of ordinary agents
with bounded resources such as ourselves. A fully comprehensive algebra
of propositions cannot be the answer here.

A more promising proposal that has attracted considerable scholarly
attention is Reverse Bayesianism (Karni and Viero, 2015, Probabilistic
Sophistication and Reverse Bayesianism; Wenmackers and Romeijn 2016, New
Theory About Old Evidence; Bradely 2017, Decision Theory with A Human
Face) . The idea is to model awareness growth as a change in the algebra
while ensuring that the probabilities of the propositions shared between
the old and new algebra remain fixed under suitable constraints.

Let \(\mathcal{F}\) be the initial algebra of propositions and let
\(\mathcal{F}^+\) the algebra after the agent's awareness has grown.
Both contain the contradictory proposition \(\perp\) and tautologous
proposition \(\top\) and they are closed under connectives such as
disjunction \(\vee\), conjunction \(\wedge\) and negation \(\neg\).
Denote by \(X\) and \(X^+\) the subsets of these algebras which contain
only basic propositions, those that do not contain connectives.
\textbf{Reverse Bayesianism} posits that the ratio of probabilities for
any basic propositions \(A\) and \(B\) in both \(X\) and \(X^+\)---the
basic propositions shared by the old and new algebra---remain constant
through the process of awareness growth:
\[\frac{\pr{A}}{\pr{B}} = \frac{\ppr{+}{A}}{\ppr{+}{B}},\] where
\(\pr{}\) represents the agent's degree of belief before awareness
growth and \(\ppr{+}{}\) represents the agent's degree of belief after
awareness growth.

What is the justification for Reverse Bayesianism? Perhaps the best
justification is pragmatic. As the awareness of an an agent grows, the
agent would prefer not to throw away completely the epistemic work they
have done so far. The agent may desire to retain as much of their old
degrees of beliefs as possible. Reverse Bayesianism provides a simple
recipe to do that. It also coheres with the conservative spirit of
conditionalization which preserves the old probability distribution
conditional on what is learned.

Reverse Bayesianism is an elegant theory that manages to cope with a
seemingly intractable problem. But, unfortunately, it is not without
complications. Steele and Stefánsson (2021, Belief Revision for Growing
Awareness) argue that Reverse Bayesianism, when suitably formulated, can
work in a limited class of cases, what they call
\textit{awareness expansion}, but cannot work for
\textit{awareness refinement} (more on this distinction later). Their
argument rests on a number of ingenious counterexamples. We contend,
however, that their counterexamples have limited applicability and thus
constitute an overall weak case against Reverse Bayesianism. Our
argument relies on the theory of Bayesian networks and makes two key
claims. First, besides cases of expansion, there are also cases of
refinement in which Reverse Bayesianism can be made to work. Second,
Steele and Stefánsson's counterexamples only target a circumscribed
class of refinement cases, and even for those, a conservative constraint
in the spirit of Reverse Bayesianism can be carved out.

\hypertarget{counterexamples}{%
\section{Counterexamples?}\label{counterexamples}}

We begin by rehearsing two of the ingenious counterexamples to Reverse
Bayesianism by Steele and Stefánsson. One targets awareness expansion
and the other awareness refinement. The difference is intuitively
plausible, but it can be tricky to pin down formally. A rough
characterization will suffice here. Suppose, as is customary,
propositions are interpreted as sets of possible worlds, where the set
of all possible worlds is the possibility space. An algebra of
propositions thus interpreted induces a partition of the possibility
space. Refinement occurs when the new proposition added to the algebra
induces a more fine-grained partition of the possibility space.
Expansion, instead, occurs when the new proposition shows the existing
partition of the possibility space is not exhaustive and more possible
worlds are added.

The first counterexample by Steele and Stefánsson targets cases of
awareness expansion:

\begin{quote}
\textsc{Friends}: Suppose you happen to see your partner enter your best
friend's house on an evening when your partner had told you she would
have to work late. At that point, you become convinced that your partner
and best friend are having an affair, as opposed to their being warm
friends or mere acquaintances. You discuss your suspicion with another
friend of yours, who points out that perhaps they were meeting to plan a
surprise party to celebrate your upcoming birthday---a possibility that
you had not even entertained. Becoming aware of this possible
explanation for your partner's behaviour makes you doubt that she is
having an affair with your friend, relative, for instance, to their
being warm friends. (Steele and Stefánsson, 2021, Section 5, Example 2)
\end{quote}

\noindent Why does the scenario \textsc{Friends} conflict with Reverse
Bayesianim? Even though Steele and Stefánsson do not provide the
details, it pays to be explicit here at the cost of pedantry.

Initially, the algebra only contains the hypotheses `my partner and my
best friend met to have an affair' (\textit{Affair}) and `my partner and
my best friend met as friends or acquaintances'
(\textit{Friends/acquaintances}). The other proposition in the algebra
is the evidence at your disposal, that is, the fact that your partner
and your best friend met one night secretively and without telling you
(\textit{Secretive}). There may be other propositions, but these are the
ones to focus on. Hypothesis \textit{Affair} better explains the
evidence at your disposal than hypothesis
\textit{Friends/acquaintances}. In probabilistic terms, this can be
expressed by comparing likelihoods:
\[\pr{ \textit{Secretive} \vert \textit{Affair}}> \pr{\textit{Secretive} \vert \textit{Friends/acquaintances}},\]
from which it also follows that \textit{Affair} is more probable than
\textit{Friends/acquaintances}
\[\pr{\textit{Affair} \vert  \textit{Secretive} }> \pr{\textit{Friends/acquaintances} \vert \textit{Secretive}}, \tag{>}\]
so long as the prior probabilities of the two hypotheses are not skewed
in one direction.\footnote{If you were initially nearly certain your
  partner could not possibly have an affair, even the fact they behaved
  very secretively or lied to you might not affect the probability of
  the two hypotheses.}

Next, the algebra changes. A new hypothesis is added which you had not
considered before: your partner and your best friends met to plan a
surprise party for your upcoming birthday (\textit{Surprise}). This is a
game changer. The evidence \textit{Secretive} now makes better sense in
light of this new hypothesis than the hypothesis \textit{Affair}:
\[\ppr{+}{ \textit{Secretive} \vert \textit{Surprise}}> \ppr{+}{\textit{Secretive} \vert \textit{Affair}}.\]
And, this new hypothesis should be more likely than the hypothesis
\textit{Affair}:
\[\ppr{+}{ \textit{Surprise} \vert \textit{Secretive}}> \ppr{+}{ \textit{Affair} \vert \textit{Secretive}}. \tag{*}\]
Reverse Bayesianism is not yet in trouble. Steele and Stefánsson,
however, conclude that the probability of \textit{Friends/acquaintances}
should now exceed that of \textit{Affair}. They write: `Becoming aware
of this possible explanation for your partner's behaviour makes you
doubt that she is having an affair with your friend, relative, for
instance, to their being warm friends.' So:
\[\ppr{+}{\textit{Affair} \vert  \textit{Secretive} } < \ppr{+}{\textit{Friends/acquaintances} \vert \textit{Secretive}}. \tag{<}\]
Arguably, this holds because \textit{Surprise} implies
\textit{Friends/acquaintances}. In order to prepare a surprise party,
your partner and best friend have to be at least acquaintances. And
given that one implies the other, if \textit{Surprise} is more likely
than \textit{Affair} (by \(*\)), then \textit{Friends/acquaintances}
must also be more likely than \textit{Affair}. And if both (\(>\)) and
(\(<\)) holds, the ratio of the probabilities of basic propositions is
not fixed before and after the episode of awareness growth. This is a
violation of Reverse Bayesianism.

But, as Steele and Stefánsson admits, Reverse Bayesianism is not really
in trouble here. It can still be made to work with a slightly
different---though quite similar in spirit---condition, called
\textbf{Awareness Rigidity}: \[\ppr{+}{A \vert T^*}=\pr{A},\] where
\(T^*\) corresponds to a proposition that picks out, from the vantage
point of the new awareness state, what corresponds to the entire
possibility space before the episode of awareness growth. In our running
example, the proposition \(\neg\textit{Surprise}\) picks out the entire
possibility space before the episode of awareness growth. So Awareness
Rigidity would require that:
\[\ppr{+}{\textit{Friends/acquaintances} \vert \neg\textit{Surprise}}=\pr{\textit{Friends/acquaintances}}.\]
Conditional on \(\neg\textit{Surprise}\), it is indeed true that the
probability of \textit{Friends/acquaintances} has not changed before and
after the episode of awareness growth. And it is also true that
\textit{Affair} remains the most likely hypothesis in light of the
evidence (again conditional on \(\neg\textit{Surprise}\)):
\[\ppr{+}{\textit{Affair} \vert  \textit{Secretive} \& \neg\textit{Surprise} } > \ppr{+}{\textit{Friends/acquaintances} \vert \textit{Secretive} \& \neg\textit{Surprise}}. \]
Awareness Rigidity is vindicated. Reverse Bayesianism---the spirit of
it, not the letter---stands.

This is not the end of the story, however. Steele and Stefánsson offer
another counterexample to Reverse Bayesianism (which also works against
Awareness Rigidity):

\begin{quote}
\textsc{Movies}: Suppose you are deciding whether to see a movie at your
local cinema. You know that the movie's predominant language and genre
will affect your viewing experience. The possible languages you consider
are French and German and the genres you consider are thriller and
comedy. But then you realise that, due to your poor French and German
skills, your enjoyment of the movie will also depend on the level of
difficulty of the language. Since it occurs to you that the owner of the
cinema is quite simple-minded, you are, after this realisation, much
more confident that the movie will have low-level language than
high-level language. Moreover, since you associate low-level language
with thrillers, this makes you more confident than you were before that
the movie on offer is a thriller as opposed to a comedy. (Steele and
Stefánsson, 2021, Section 5, Example 3)
\end{quote}

\noindent Initially, you did not consider language difficulty. The
algebra contained the propositions \textit{French} and \textit{German},
as well as \textit{Thriller} and \textit{Comedy}. Then, you realized
another variable might be at play, namely the level of difficulty of the
language of the movie. The realization that the owner is simple-minded
suggests that the level of linguistic difficulty of the movie will be
low. The latter in turn suggests that the movie is more likely a
thriller rather than a comedy (perhaps because thrillers are
simpler---linguistically---than comedies). So, against Reverse
Bayesianism, the scenario \textsc{Movies} violates the condition
\(\frac{\pr{\textit{Thriller}}}{\pr{\textit{Comedy}}}=\frac{\ppr{+}{\textit{Thriller}}}{\ppr{+}{\textit{Comedy}}}\).

The counterexample also works against Awareness Rigidity. It is not true
that
\(\pr{\textit{Thriller}}=\ppr{+}{\textit{Thriller} \vert \textit{Thriller}\vee \textit{Comedy}}\).
To see why that is, note that this counterexample is a case of
refinement. First, you categorize movies by just language and genre, and
then you add a further category, level of difficulty. In the scenario
\textsc{Friends}, instead, the possibility space grew by adding
situations in which your partner and best friends met neither as lovers
nor solely as friends.\footnote{The addition of the hypothesis
  \textit{Surprise} is, however, an ambiguous case. For one thing,
  \textit{Surprise} is a novel hypothesis that cannot be subsumed under
  \textit{Friends/acquaintances} or \textit{Affair}. On the other,
  \textit{Surprise} seems a refinement of
  \textit{Friends/acquaintances}, since a meeting for planning a
  surprise is a more specific way to describe a meeting of
  acquaintances. We will provide a more clear-cur example of expansion
  later in the paper.} So if \textsc{Movies} is a case of refinement,
the proposition which picks out the entire possibility space should be
the same before and after awareness growth, for example,
\(\textit{Thriller}\vee \textit{Comedy}\). In cases of awareness growth
by refinement, then, Awareness Rigidity mandates that all probability
assignments stay the same.

How strong of a counterexample is \textsc{Movies}? For the sake of
clarity, it can be split into two episodes. In the first, you entertain
a new variable besides language and genre, namely the language
difficulty of the movie. In the second episode, you learn something you
did not consider before, namely that the owner is simple-minded. The
first is a case of mere refinement: you simply entertain a new way of
categorizing movies. The second is a case of learning: you learn
something you did not consider before. The tacit assumption is that
awareness growth is both \textit{entertaining} a new proposition not in
the initial awareness state of the agent (without learning it) and
\textit{learning} a new proposition not in the initial awareness state
of the agent. We agree with Steele and Stefánsson that both these
phenomena should count as instances of awareness growth. But one
wonders. Is the second episode (learning something new) necessary for
the counterexample to work together with the first episode (mere
refinement)?

Suppose the counterexample only works in tandem with an episode of
learning something new. In that were so, defenders of Reverse
Bayesianism or Awareness Rigidity could still claim that their theory
applies to a large class of cases. It applies to cases of awareness
refinement without learning and also to cases of awareness expansion.
For recall that the first putative counterexample about awareness
expansion---\textsc{Friends}---did not challenge Reverse Bayesianism
insofar as the latter is formulated in terms of its close cousin,
Awareness Rigidity. So the force of Steele and Stefánsson's
counterexamples would be rather limited.

Or perhaps there is a more straightforward counterexample that only
depicts mere refinement without an episode of learning and that still
challenges Reverse Bayesianism and Awareness Rigidity? As we shall soon
see, the answer to this question is indeed positive.

\hypertarget{a-simpler-counterexample}{%
\section{A simpler counterexample}\label{a-simpler-counterexample}}

Steele and Stefánsson's counterexample to Reverse Bayesianism in the
case of refinement is rather complex, perhaps unnecessarily so. We now
present something simpler:

\begin{quote}
\textsc{Lighting:} You have evidence that favors a certain hypothesis,
say a witness saw the defendant around the crime scene. You give some
weight to this evidence. In your assessment, that the defendant was seen
around the crime scene raises the probability that the defendant was
actually there. But now you wonder, what if it was dark when the witness
saw the defendant? You become a bit more careful and settle on this: if
the lighting conditions were good, you should still trust the evidence,
but if they were bad, you should not. Unfortunately, you cannot learn
about the actual lighting conditions, but the mere realization that it
\textit{could} have been dark makes you change the probability that the
defendant was actually there, based on the same eveidence.
\end{quote}

\noindent This scenario is simpler because it consists of mere
refinement. You wonder about the lighting conditions but you do not
learn what they were. Still, mere refinement in this scenario challenges
Reverse Bayesianism and Awareness Rigidity. That this should be so is
not easy to see. Fortunately, the theory of Bayesian networks helps to
see why.

A Bayesian network is a formal model that consists of a graph
accompanied by a probability distribution. The nodes in the graph
represent random variables that can take different values. We will use
`nodes' and `variables' interchangeably. The nodes are connected by
arrows, but no loops are allowed, hence the name direct acyclic graph
(DAG). In this framework, awareness growth brings about a change in the
graphical network---nodes and arrows are added or erased---as well as a
change in the probability distribution from the old to the new network.

To model the scenario \textsc{Lighting} with Bayesian networks, we start
with this graph: \[H \rightarrow E,\] where \(H\) is the hypothesis node
and \(E\) the evidence node. If an arrow goes from \(H\) to \(E\), the
probability distribution associated with the Bayesian network should be
defined by conditional probabilities of the form \(\pr{E=e \vert H=h}\),
where uppercase letters represent the variables (nodes) and lower case
letters represent the values of these variables.\footnote{A major point
  of contention in the interpretation of Bayesian networks is is the
  meaning of the directed arrows. They could be interpreted
  causally---as though the direction of causality proceeds from the
  events described by the hypothesis to event described by the
  evidence---but they need not be. REFERENCES?}

Since you trust the evidence, you think that it is more likely under the
hypothesis that the defendant was present at the crime scene than under
the alternative hypothesis:
\[\pr{\textit{E=seen} \vert \textit{H=present}} > \pr{\textit{E=seen} \vert \textit{H=absent}}\]
It is not necessary to fix exact numerical values for these conditional
probabilities. The inequality is a qualitative ordering of how plausible
the evidence is in light of competing hypotheses. No matter the numbers,
by the probability calculus, it follows that the evidence raises the
probability of the hypothesis \textit{H=present}:
\[\pr{\textit{H=present}\vert \textit{E=seen}} > \pr{\textit{H=present}}\]

\noindent Now, as you wonder about the lighting conditions, the graph
should be amended: \[H \rightarrow E \leftarrow L,\] where the node
\(L\) can have two values, \textit{L=good} and \textit{L=bad}. A
plausible way to update your assessment of the evidence is as follows:
\[\ppr{+}{\textit{E=seen} \vert \textit{H=present} \wedge \textit{L=good}} > \ppr{+}{\textit{E=seen} \vert \textit{H=absent} \wedge \textit{L=good}}\]
\[\ppr{+}{\textit{E=seen} \vert \textit{H=present} \wedge \textit{bad}} = \ppr{+}{\textit{E=seen} \vert \textit{H=absent} \wedge \textit{L=bad}}\]

\noindent Note the change in the probability function from \(\pr{}\) to
\(\ppr{+}{}\). Here is what you are thinking: if the lighting conditions
were good, you should still trust the evidence like you did before. But
if the lighting conditions were bad, you should regard the evidence as
no better than chance. Again, there are no exact numerical values here.

Should you now assess the evidence at your disposal---that the witness
saw the defendant at the crime scene---any differently than you did
before? Would it be wrong to think the evidence had the same value? The
evidence would have the same value if the likelihood ratios associated
with it relative to the competing hypotheses were the same before and
after awareness growth:
\[\frac{\pr{E=e \vert H=h}}{\pr{E=e \vert H=h'}}= \frac{\ppr{+}{E=e \vert H=h}}{\ppr{+}{E=e \vert H=h'}} \tag{C}.\]
But it would be quite a coincidence if (C) were true. For concreteness,
let's use some numbers:
\[\pr{\textit{E=seen} \vert \textit{H=present}}=\ppr{+}{\textit{E=seen} \vert \textit{H=present} \wedge \textit{L=good}}=.8\]
\[\pr{\textit{E=seen} \vert \textit{H=absent}}=\ppr{+}{\textit{E=seen} \vert \textit{H=absent} \wedge \textit{L=good}}=.4\]
\[\ppr{+}{\textit{E=seen} \vert \textit{H=present} \wedge \textit{L=bad}} = \ppr{+}{\textit{E=seen} \vert \textit{H=absent} \wedge \textit{L=bad}}=.5.\]
So the ratio
\(\frac{\pr{\textit{E=seen} \vert \textit{H=present}}}{\pr{\textit{E=seen} \vert \textit{H=absent}}}=2\).
Before awareness growth, you thought the evidence favored the hypothesis
\textit{H=present} moderately strongly. That seemed reasonable. But,
after the awareness growth, the ratio
\(\frac{\ppr{+}{\textit{E=seen} \vert \textit{H=present}}}{\ppr{+}{\textit{E=seen} \vert \textit{H=absent}}}=\frac{.65}{.45}\approx 1.44\).\footnote{The
  calculations here rely on the dependency structure encoded in the
  Bayesian network (see starred step below). \begin{align*}
  \ppr{+}{\textit{E=seen} \vert \textit{H=present}} &= \ppr{+}{\textit{E=seen} \wedge \textit{L=good} \vert \textit{H=present}}+\ppr{+}{\textit{E=seen} \wedge \textit{L=bad} \vert \textit{H=present}}\\
  &= \ppr{+}{\textit{E=seen} \vert \textit{H=present} \wedge \textit{L=good}}  \times \ppr{+}{\textit{L=good} \vert  \textit{H=present} }\\ & +\ppr{+}{\textit{E=seen}  \vert \textit{H=present} \wedge \textit{L=bad}} \times \ppr{+}{\textit{L=bad} \vert  \textit{H=present}}\\
  &=^* \ppr{+}{\textit{E=seen} \vert \textit{H=present} \wedge \textit{L=good}}  \times \ppr{+}{\textit{L=good}}\\ & +\ppr{+}{\textit{E=seen}  \vert \textit{H=present} \wedge \textit{L=bad}} \times \ppr{+}{\textit{L=bad}}\\
  &= .8 \times .5 +.5 *.5 = .65 
  \end{align*} \begin{align*}
  \ppr{+}{\textit{E=seen} \vert \textit{H=absent}} &= \ppr{+}{\textit{E=seen} \wedge \textit{L=good} \vert \textit{H=absent}}+\ppr{+}{\textit{E=seen} \wedge \textit{L=bad} \vert \textit{H=absent}}\\
  &= \ppr{+}{\textit{E=seen} \vert \textit{H=absent} \wedge \textit{L=good}}  \times \ppr{+}{\textit{L=good} \vert  \textit{H=absent} }\\ & +\ppr{+}{\textit{E=seen}  \vert \textit{H=absent} \wedge \textit{L=bad}} \times \ppr{+}{\textit{L=bad} \vert  \textit{H=absent}}\\
  &=^* \ppr{+}{\textit{E=seen} \vert \textit{H=absent} \wedge \textit{L=good}}  \times \ppr{+}{\textit{L=good}}\\ & +\ppr{+}{\textit{E=seen}  \vert \textit{H=absent} \wedge \textit{L=bad}} \times \ppr{+}{\textit{L=bad}}\\
  &= .4 \times .5 +.5 *.5 = .45 
  \end{align*}} This argument can be repeated with several other
numerical
assignments.\todo{Need a more general argument here. Simulation?} So,
quite often, mere refinement can weaken the evidence, even without
learning anything new. Of course, if you did learn that the lighting
conditions were bad, the evidence would become even weaker, effectively
worthless:
\[\frac{\ppr{+, \textit{L=bad}}{\textit{E=seen} \vert \textit{H=present}}}{\ppr{+, \textit{L=bad}}{\textit{E=seen} \vert \textit{H=absent}}}=1,\]
where \(\ppr{+, \textit{L=bad}}{}\) is the new probability function
after learning that \textit{L=bad}.

Why does all this matter? We have seen that, after awareness growth, you
should regard the evidence at your disposal as one that favors
\textit{H=present} less strongly or not at all. Since the prior
probability of the hypothesis should be the same before and after
awareness growth, it follows that
\[\ppr{+}{\textit{H=present} \vert \textit{E=seen}} \neq \pr{\textit{H=present} \vert \textit{E=seen}}.\]
This outcome violates Awareness Rigidity. For recall that in cases of
refinement, Awareness Rigidity requires that the probability of basic
propositions stay fixed.

Reverse Bayesianism is also violated. For example, the ratio of the
probabilities of \textit{H=present} to \textit{E=seen}, before and after
awareness growth, has changed:
\[\frac{\ppr{\textit{E=seen}}{\textit{H=present}}}{\ppr{ \textit{E=seen}}{\textit{H=seen}}} \neq \frac{\ppr{+, \textit{E=seen}}{\textit{H=present}}}{\ppr{+, \textit{E=seen}}{\textit{H=seen}}},\]
where \(\ppr{\textit{E=seen}}{}\) and \(\ppr{+, \textit{E=seen}}{}\)
represent the agent's degrees of belief, before and after awareness
growth, updated by the evidence \(\textit{E=seen}\).

All in all, the counterexample \textsc{Lighting} works even though it
only depicts refinement without learning, and thus strengthens Steele
and Stefánsson's case against Reverse Bayesianism and Awareness
Rigidity. But despite that, their criticism still remains more
circumscribed that it might appear at first. There are cases of
refinement in which Reverse Bayesianism and Awareness Rigidity are
perfectly fine in their place. This is our next topic.

\hypertarget{another-refinement}{%
\section{Another refinement}\label{another-refinement}}

Consider this variation of the \textsc{Lighting} scenario:

\begin{quote}
\textsc{Veracity}: A witness saw that the defendant was around the crime
scene and you initially took this to be evidence that the witness was
actually there. But then you had second thoughts. Instead of worrying
about the lighting conditions, you worry that the witness might be lying
or misremembering what happened. Perhaps, the witness was never there,
made things up or mixed things up. But despite that, you do not change
anything of your initial assessment of the evidence.
\end{quote}

\noindent   The rational thing to do here is to stick to your guns and
not change your earlier assessment of the evidence. Why should that be
so? And what is the difference with \textsc{Lighting}? Once again,
Bayesian networks proves to be a good analytic tool here.

The graphical network should initially look like this:
\[H\rightarrow E\] But, as your awareness grows, the graphical network
should be updated: \[H\rightarrow E \rightarrow R\] The hypothesis node
\(H\) bears on the whereabouts of the defendant. Note the difference
between \(E\) and \(R\). The evidence node bears on what the witness
saw. The reporting node bears on what the witness reports to have seen.
The chain of transmission from `seeing' to `reporting' may fail for
various reasons, such as lying or confusion.

It pays to highlight the difference between \textsc{Lighting} and
\textsc{Veracity}. They are both cases of refinement. In one, what the
witness saw could have occurred under good or bad lighting conditions;
in the other, what the witness saw could have been reported truthfully
or untruthfully. But refinement is structurally different in the two
cases. In \textsc{Lighting}, the connection between the evidence and the
hypothesis undergoes a change, since the lighting conditions affect the
witness' ability to have reliable experiences of what happened. In
\textsc{Veracity}, instead, the connection between the evidence and the
hypothesis is not affected. At stake is the extent to which what the
witness saw, if anything, is reported truthfully or not.

So, even if \textsc{Veracity} is a case of refinement, the old and new
probability functions agree with one another completely. The conditional
probabilities, \(\pr{E=e \vert H=h}\) should be the same as
\(\ppr{+}{E=e \vert H=h}\) for any values \(e\) and \(h\) of the
variables \(H\) and \(E\) that are shared before and after awareness
growth. Given the dependency structure of the two Bayesian
networks---first, \(H\rightarrow E\) and then
\(H\rightarrow E \rightarrow R\)---the equality is easy to establish
formally.\footnote{GIVE PROOF} Thus, Reverse Bayesianism and Awareness
Rigidity are perfectly fine in scenarios like \textsc{Veracity}.

A confusion should be eliminated at this point. We do not intend to
suggest that the assessment of the probability of the hypothesis
\textit{H=present} should undergo no change at all. If you worry that
the witness could have lied, shouldn't this affect your degree of
beliefs in the veracity of what they said about the defendant's
whereabouts? Surely so. To see where the confusion might lie, note that
in \textsc{Veracity} an episode of awareness refinement may actually
take place together with a form of retraction. Initially, what is taken
to be known, after the learning episode, is that the witness
\textit{saw} the defendant around the crime scene. But after awareness
growth, you realize your learning is in fact limited to what the witness
\textit{reported} to have seen. So the previous learning episode is
retracted and replaced by a more careful statement of what you learned.
This retraction will affect the probability you assign to the hypothesis
\textit{H=seen}, but this does not conflict with Reverse Bayesianism or
Awareness Rigidity. In \textsc{Lighting}, instead, no retraction of the
evidence takes place. The evidence that is known remains the fact that
the witness saw the defendant around the crime scene, even though that
experience could have been misleading due to bad lighting conditions.

Where does this leave us? The following are now well-established: (a)
Reverse Bayesianism (or its close cousin Awareness Rigidity) handles
successfully cases of awareness expansion; (b) it also handles
successfully cases of refinement like \textsc{Veracity}; but (c) it does
fail in cases of refinement like \textsc{Lighting}. Bayesian networks
helped to distinguish these two forms of refinement, and there may be
other, more fine-grained distinctions to be made. So, ultimately, Steele
and Stefánsson's argument only targets a subclass of refinement cases.
But even in those---we shall soon see---there is room to carve out a
conservative constraint that is close in spirit to Reverse Bayesianism.

\hypertarget{a-modest-conservative-constraint}{%
\section{A modest conservative
constraint}\label{a-modest-conservative-constraint}}

Recall that, in \textsc{Lighting}, the probability functions \(\pr{}\)
and \(\ppr{+}{}\) do not assign the same weight to the evidence relative
to the competing hypotheses, except in somewhat exceptional
circumstances. Hence, Reverse Bayesianism and Awareness Rigidity fail in
this scenario. But despite that, the two probability functions agree in
one important respect:
\[\pr{E=e \vert H=h} \geq \pr{E=e \vert H=h'} \textit{ iff } \ppr{+}{E=e \vert H=h} \geq \ppr{+}{E=e \vert H=h'} \tag{$C^*$},\]
where (i) \(E\) and \(H\) are nodes that are part of the graphical
network before and after awareness growth, and (ii) there is a direct
path from \(H\) to \(E\) before and after awareness
growth.\todo{Is the condition of direct path necessary?} In other words,
the plausibility ordering between hypotheses and evidence is preserved.
Condition \((C^*)\) can serve as a conservative constraint that governs
the relationship between \(\pr{}\) and \(\ppr{+}{}\). It is satisfied in
the scenario \textsc{Lighting}, but how general is this condition?

We now show that \((C^*)\) holds generally in a class of Bayesian
networks, under minimal, and entirely reasonable, assumptions. Assume
the Bayesian network has a node \(E\) with an incoming arrow from node
\(H\), before and after awareness growth. After awareness growth,
besides \(E\) and \(H\), another variable \(Y\) is under consideration.
The new graph looks like this: \[H\rightarrow E \leftarrow Y.\] For
simplicity, we assume that variables are binaries. All we need is the
following assumption: \[\pr{E=e \vert H=h} \geq \pr{E=e \vert H=h'} \]
\[\textit{ iff }\]
\[\ppr{+}{E=e \vert H=h \wedge Y=y} \geq \ppr{+}{E=e \vert H=h' \wedge Y=y} \tag{EQUAL}\]
\[\textit{ iff }\]
\[\ppr{+}{E=e \vert H=h \wedge Y=y'} \geq \ppr{+}{E=e \vert H=h' \wedge Y=y'}\]
This assumption says that the plausibility ordering remains the same
before and after awareness growth \textit{all else being the same}. It
is a minimal assumption, but enough to establish (\(C^*\)).\footnote{From
  (EQUAL) and via this chain of equivalences:
  \[[a \geq a' \& b\geq b'] \textit{ iff } [ak \geq a'k \& b(1-k)\geq b'(1-k) (\textit{with }k>0)] \textit{ iff }  [ak+b(1-k) \geq a'k+b'(1-k)],\]
  it follows that \[\pr{E=e \vert H=h} \geq \pr{E=e \vert H=h'} \]
  \[\textit{ iff }\]
  \[\ppr{+}{E=e \vert H=h \wedge Y=y}\times \ppr{+}{Y=y} + \ppr{+}{E=e \vert H=h \wedge Y=y'}\times \ppr{+}{Y=y'} \]
  \[\geq\]
  \[\ppr{+}{E=e \vert H=h' \wedge Y=y}\times \ppr{+}{Y=y} + \ppr{+}{E=e \vert H=h' \wedge Y=y'}\times \ppr{+}{Y=y'} \]
  We are done since, by the law of total probability and the
  probabilistic dependencies in the graph, (\(C^*\)) is equivalent to
  the above statement.}

There may be cases in which the plausibility ordering is not preserved
because (EQUAL) does not hold. This would be cases of
\textit{transformative} awareness growth. For suppose you have evidence
that---in your judgment---reliably tracks a hypothesis, say you think
that appearance as of hands reliably tracks the presence of hands:
\[\pr{E=\textit{as-of-hands} \vert H=\textit{hands}} > \pr{E=\textit{as-of-hands} \vert H=\textit{no-hands}} \]
You now entertain a `switching hypothesis' in which a demon switches
things around: when you see a hand, there is no hand, and when you do
not see a hand, there is a hand. In this case, (EQUAL) would be violated
since
\[\ppr{+}{E=\textit{as-of-hands} \vert H=\textit{hands} \wedge Y=\textit{switching}} < \ppr{+}{E=\textit{as-of-hands} \vert H=\textit{no-hands} \wedge Y=\textit{switching}}\]
In more common skeptical scenarios without a switching hypothesis,
however, (EQUAL) should still hold. For what skeptical scenarios often
do is to neutralize the value of the evidence, not so much switching its
value.

The preservation of the plausibility ordering formulated by condition
(\(C^*\)) should also hold in Steele and Stefánsson's scenario
\textsc{Movies}. We now briefly explain why. At first, the graphical
network looks like this:
\[\textit{Genre} \rightarrow \textit{Enjoyment} \leftarrow \textit{Language},\]
where each node can take two values: \textit{Genre=comedy} and
\textit{Genre=thriller}; \textit{Language=french} and
\textit{Language=german}; and \textit{Enjoyment=yes} and
\textit{Enjoyment=no}. Assume you are ranking the options in terms of
how they are going to contribute to your enjoyment
(\textit{Enjoyment=yes}). You are more likely to enjoy a comedy in
French than anything else, but you are more likely to enjoy a thriller
in German than one in French, and your lowest preference is for a comedy
in German. This ranking can be encoded by conditional probability
statements of the form
\[\pr{\textit{Enjoyment=x} \vert \textit{Language=y} \wedge \textit{Genre=z}} \geq \pr{\textit{Enjoyment=x} \vert \textit{Language=y'} \wedge \textit{Genre=z'}}.\]

The first episode of awareness growth in \textsc{Movies} consists in
realizing that the linguistic difficulty of the movie could also be a
factor. So the expanded graphical network now becomes: \begin{align*}
\textit{Difficulty} &\\
\downarrow \\
\textit{Genre} \rightarrow \textit{Enjoyment} &\leftarrow \textit{Language}\\
\end{align*} \noindent Your ranking of what is likely to give you
enjoyment should now be updated and made more specific, but much of the
earlier ordering can be retained, that is:
\[\pr{\textit{Enjoyment=x} \vert \textit{Language=y} \wedge \textit{Genre=z}} \geq \pr{\textit{Enjoyment=x} \vert \textit{Language=y'} \wedge \textit{Genre=z'}}\]
\[\textit{ iff }\]
\[\ppr{+}{\textit{Enjoyment=x} \vert \textit{Language=y} \wedge \textit{Genre=z}} \geq \ppr{+}{\textit{Enjoyment=x} \vert \textit{Language=y'} \wedge \textit{Genre=z'}}.\]
The difference with condition (\(C^*\)) is that here two propositions,
not just one, are conditioned on. So (\(C^*\)) should be be generalized,
accordingly, but the general idea remains the same.\footnote{A
  generalzied version iof (\$C\^{}*) wouldbe as follows. \$Let \(E\) be
  a node with several incoming arrows departing from a finite set of
  nodes, \(X_1, X_2, \dots X_k\) that are shared before and after
  awareness growth. Then:
  \[\pr{E=x \vert X_1=x_1 \wedge X_2=x_2 \dots X_k=x_k} \geq \pr{E=x \vert X_1=x_1' \wedge X_2=x_2' \dots X_k=x_k'}\]
  \[\textit{ iff }\]
  \[\ppr{+}{E=x \vert X_1=x_1 \wedge X_2=x_2 \dots X_k=x_k} \geq \ppr{+}{E=x \vert X_1=x_1' \wedge X_2=x_2' \dots X_k=x_k'}.\]}

\hypertarget{conclusion}{%
\section{Conclusion}\label{conclusion}}

We argued that the case against Reverse Bayesianism is much weaker than
one might think. The scenario \textsc{Movies}---which is Steele and
Stefánsson's key counterexample to Reverse Bayesianism---is unconvincing
since it mixes learning and refinement. To avoid this, we constructed a
more clear-cut case of refinement, \textsc{Lighting}, in which both
Awareness Rigidity and Reverse Bayesianism fail unequivocally. However,
we showed that even in cases like this a modest conservative constraint
can still be carved out, vindicating to some extent the spirit of
Reverse Bayesianism. We also showed that there are cases of refinement
like \textsc{Veracity} in which Reverse Bayesianism and Awareness
Rigidity are perfectly fine in their place.

We conclude with a few programmatic observations. We think that the
awareness of agents grows while holding fixed certain material
structural assumptions, based on commonsense, semantic stipulations or
causal dependency. To model awareness growth, we need a formalism that
can express these material structural assumptions. This can done using
Bayesian networks, and we offered some illustrations of this strategy,
for example, by distinguish two forms of refinement on the basis of
different structural assumptions. These material assumptions also guide
us in formulating the adequate conservative constraints, and these will
inevitably vary on a case-by-case basis. Our approach stands in stark
contrast with much of the literature on awareness growth from a Bayesian
perspective. This literature is primarily concerned with a formal,
almost algorithmic solution to the problem. We suspect that seeking such
formal solution is doomed to fail. Insofar as Reverse Bayesianism is an
expression of of this formalistic aspiration, we agree with Steele and
Stefánsson that we are better off looking elsewhere.

\hypertarget{extra-materials-ignore}{%
\section{Extra Materials -- IGNORE}\label{extra-materials-ignore}}

\hypertarget{expansion}{%
\subsection{Expansion}\label{expansion}}

There remains to examine cases of awareness expansion. They consist in
the addition of another proposition not previously in the algebra, but
that is not a refinement of existing propositions. The addition of the
hypothesis \textit{Surprise} is, however, an ambiguous case. For one
thing, \textit{Surprise} is a novel hypothesis that cannot be subsumed
under \textit{Friends/acquaintances} or \textit{Affair}. On the other,
\textit{Surprise} seems a refinement of \textit{Friends/acquaintances},
since a meeting for planning a surprise is a more specific way to
describe a meeting of acquaintances. A more clear-cut case of awareness
expansion would be the following. The police is investigating a murder
case. There are two suspects under investigation: Joe and Sue. They both
have a motive. The incriminating evidence favors one over the other, but
not overwhelmingly. T hen, a new hypothesis is considered: Ela could be
the perpetrator. The evidence incriminates Ela almost without any doubt.
Any theory of awareness growth should be able to model the difference
between the example provided by Steele and Stefánsson and the criminal
case just outlined. They are both, arguably, cases of expansion, but
they are also different.

Steele and Stefánsson provide a formal definition of the difference
between refinement and expansion. Our observations here are largely
confined at the intuitively level. Our point is that there are a number
of intuitively plausible differences that a formal theory should be able
to capture. The coarse distinction between refinement and expansion
might be, in the end, too coarse. Relying on Bayesian networks, we will
illustrate this point more precisely in the next section.

\hypertarget{steele-and-stefuxe1nsson-example}{%
\subsection{Steele and Stefánsson
example}\label{steele-and-stefuxe1nsson-example}}

Before awareness growth, the Bayesian network has a simple form:
\[H \rightarrow E,\] where the hypothesis variable \(H\) takes two
values, \(H=\textit{Affair}\) and \(H=\textit{Friends/acquaintances}\).
The evidence variable \(E\) can take several values, one of them being
\(E=\textit{Secretive}\). You could have seen other things other than
what you saw, but there is no need to specify the other values
exhaustively. Suppose the prior odds ratio of the hypotheses is 1:1,
say, because you suspected your partner might be cheating on you, and
the likelihood ratio
\[\frac{\pr{E=\textit{Secretive}\vert H=\textit{Affair}}}{\pr{E=\textit{Secretive}\vert H=\textit{Friends/acquaintances}}}\]
is 9:1, because the hypothesis \textit{Affair} is a better explanation
of the evidence than the hypothesis \textit{Friends/acquaintances}.
Then, the posterior probability given the evidence
\[\pr{H=\textit{Affair} \vert E=\textit{Secretive}}\] is quite high,
\(\frac{9}{10}=.9\). So
\(\ppr{E=\textit{Secretive}}{H=\textit{Affair}}=.9\).\footnote{This
  calculation presupposes that the two hypotheses \textit{Affair} and
  \textit{Friends/acquaintances} are exclusive and exhaustive. This
  assumption is justified given the initial awareness state of the
  agent.}

After awareness growth, the Bayesian network should be modified as
follows: \[H \leftarrow H' \rightarrow E,\] where the new hypothesis
node now consists of three values instead of two:

\(H'=\textit{Affair}\)

\(H'=\textit{Friends/acquaintances}\wedge \neg \textit{Surprise}\)

\(H'=\textit{Friends/acquaintances}\wedge\textit{Surprise}\).

\noindent The scenario \(\textit{Friends/acquaintances}\) is split into
the scenario in which your partner and best friend met simply as friends
or acquaintances, and the scenario in which they met to prepare a
surprise party for you. On this interpretation, the counterexample by
Steele and Stefánsson is a case of refinement, not expansion. We will
return to this point later.

The network contains a directed arrow between the old hypothesis node
\(H\) and the new hypothesis node \(H'\) This arrow can be interpreted
as a bridge between the old awareness state limited to two hypotheses
and the new awareness state that contains an additional hypothesis. This
bridge is purely conceptual and can be defined by two sets of
constrains. The first set of constrains posits that \textit{Affair}
under \(H\) has the same meaning as \textit{Affair} under \(H'\):

\(\ppr{+}{H=\textit{Affair} \vert H'=\textit{Affair}}=1\)

\(\ppr{+}{H=\textit{Affair} \vert H'=\textit{\textit{Friends/acquaintances}}}=0\)

\(\ppr{+}{H=\textit{Affair} \vert H'=\textit{Surprise}}=0\)

The second set of constrains posits that hypothesis
\(\textit{Friends/acquaintances}\) under \(H\) can be actually be
interpreted in two ways under \(H'\), as
\(\textit{Friends/acquaintances} \wedge \neg \textit{Surprise}\) and
\(\textit{Friends/acquaintances} \wedge \textit{Surprise}\). So, in
other words, the episode of awareness growth consists in the realization
that \(\textit{Friends/acquaintances}\) can be made precise in two more
specific ways:

\(\ppr{+}{H=\textit{Friends/acquaintances} \vert H'=\textit{Affair}}=0\)

\(\ppr{+}{H=\textit{Friends/acquaintances} \vert H'=\textit{Friends/acquaintances} \wedge \neg \textit{Surprise}}=1\)

\(\ppr{+}{H=\textit{Friends/acquaintances} \vert H'=\textit{Friends/acquaintances} \wedge \textit{Surprise}}=0\)

This bridge between \(H\) and \(H'\) justifies the following
conservativity constraint:

\[\frac{\pr{E=\textit{Secretive}\vert H=\textit{Affair}}}{\pr{E=\textit{Secretive}\vert H=\textit{Friends/acquaintances}}} = \frac{\ppr{+}{E=\textit{Secretive}\vert H=\textit{Affair}}}{\ppr{+}{E=\textit{Secretive}\vert H=\textit{Friends/acquaintances}}}=\frac{9}{1} \]

\hypertarget{expansion-criminal-case-example}{%
\subsection{Expansion: criminal case
example}\label{expansion-criminal-case-example}}

\end{document}
