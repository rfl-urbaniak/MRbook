% Options for packages loaded elsewhere
\PassOptionsToPackage{unicode}{hyperref}
\PassOptionsToPackage{hyphens}{url}
\PassOptionsToPackage{dvipsnames,svgnames*,x11names*}{xcolor}
%
\documentclass[
  11pt,
  dvipsnames,enabledeprecatedfontcommands]{scrartcl}
\usepackage{amsmath,amssymb}
\usepackage{lmodern}
\usepackage{ifxetex,ifluatex}
\ifnum 0\ifxetex 1\fi\ifluatex 1\fi=0 % if pdftex
  \usepackage[T1]{fontenc}
  \usepackage[utf8]{inputenc}
  \usepackage{textcomp} % provide euro and other symbols
\else % if luatex or xetex
  \usepackage{unicode-math}
  \defaultfontfeatures{Scale=MatchLowercase}
  \defaultfontfeatures[\rmfamily]{Ligatures=TeX,Scale=1}
\fi
% Use upquote if available, for straight quotes in verbatim environments
\IfFileExists{upquote.sty}{\usepackage{upquote}}{}
\IfFileExists{microtype.sty}{% use microtype if available
  \usepackage[]{microtype}
  \UseMicrotypeSet[protrusion]{basicmath} % disable protrusion for tt fonts
}{}
\usepackage{xcolor}
\IfFileExists{xurl.sty}{\usepackage{xurl}}{} % add URL line breaks if available
\IfFileExists{bookmark.sty}{\usepackage{bookmark}}{\usepackage{hyperref}}
\hypersetup{
  pdftitle={Awareness Growth in Bayesian Networks},
  pdfauthor={Marcello/Rafal},
  colorlinks=true,
  linkcolor=Maroon,
  filecolor=Maroon,
  citecolor=Blue,
  urlcolor=blue,
  pdfcreator={LaTeX via pandoc}}
\urlstyle{same} % disable monospaced font for URLs
\usepackage{graphicx}
\makeatletter
\def\maxwidth{\ifdim\Gin@nat@width>\linewidth\linewidth\else\Gin@nat@width\fi}
\def\maxheight{\ifdim\Gin@nat@height>\textheight\textheight\else\Gin@nat@height\fi}
\makeatother
% Scale images if necessary, so that they will not overflow the page
% margins by default, and it is still possible to overwrite the defaults
% using explicit options in \includegraphics[width, height, ...]{}
\setkeys{Gin}{width=\maxwidth,height=\maxheight,keepaspectratio}
% Set default figure placement to htbp
\makeatletter
\def\fps@figure{htbp}
\makeatother
\setlength{\emergencystretch}{3em} % prevent overfull lines
\providecommand{\tightlist}{%
  \setlength{\itemsep}{0pt}\setlength{\parskip}{0pt}}
\setcounter{secnumdepth}{5}
%\documentclass{article}

% %packages
\usepackage{booktabs}
%\usepackage[left]{showlabels}
\usepackage{multirow}
\usepackage{subcaption}
\usepackage{wrapfig}
\usepackage{graphicx}
\usepackage{longtable}
\usepackage{ragged2e}
\usepackage{etex}
%\usepackage{yfonts}
\usepackage{marvosym}
\usepackage[notextcomp]{kpfonts}
\usepackage{nicefrac}
\newcommand*{\QED}{\hfill \footnotesize {\sc Q.e.d.}}
\usepackage{floatrow}
\usepackage{multicol}

\usepackage[textsize=footnotesize]{todonotes}
\newcommand{\ali}[1]{\todo[color=gray!40]{\textbf{Alicja:} #1}}
\newcommand{\mar}[1]{\todo[color=blue!40]{#1}}
\newcommand{\raf}[1]{\todo[color=olive!40]{#1}}

%\linespread{1.5}
\newcommand{\indep}{\!\perp \!\!\! \perp\!}


\setlength{\parindent}{10pt}
\setlength{\parskip}{1pt}


%language
%\usepackage{times}
\usepackage{mathptmx}
\usepackage[scaled=0.86]{helvet}
\usepackage{t1enc}
%\usepackage[utf8x]{inputenc}
%\usepackage[polish]{babel}
%\usepackage{polski}




%AMS
\usepackage{amsfonts}
\usepackage{amssymb}
\usepackage{amsthm}
\usepackage{amsmath}
\usepackage{mathtools}

\usepackage{geometry}
 \geometry{a4paper,left=35mm,top=20mm,}


%environments
\newtheorem{fact}{Fact}


% allow page breaks in equations
\allowdisplaybreaks


%abbreviations
\newcommand{\ra}{\rangle}
\newcommand{\la}{\langle}
\newcommand{\n}{\neg}
\newcommand{\et}{\wedge}
\newcommand{\jt}{\rightarrow}
\newcommand{\ko}[1]{\forall  #1\,}
\newcommand{\ro}{\leftrightarrow}
\newcommand{\exi}[1]{\exists\, {_{#1}}}
\newcommand{\pr}[1]{\ensuremath{\mathsf{P}(#1)}}
\newcommand{\ppr}[2]{\ensuremath{\mathsf{P}^{#1}(#2)}}
\newcommand{\cost}{\mathsf{cost}}
\newcommand{\benefit}{\mathsf{benefit}}
\newcommand{\ut}{\mathsf{ut}}

\newcommand{\odds}{\mathsf{Odds}}
\newcommand{\ind}{\mathsf{Ind}}
\newcommand{\nf}[2]{\nicefrac{#1\,}{#2}}
\newcommand{\R}[1]{\texttt{#1}}
\newcommand{\prr}[1]{\mbox{$\mathtt{P}_{prior}(#1)$}}
\newcommand{\prp}[1]{\mbox{$\mathtt{P}_{posterior}(#1)$}}



\newtheorem{q}{\color{blue}Question}
\newtheorem{lemma}{Lemma}
\newtheorem{theorem}{Theorem}
\newtheorem{corollary}{Corollary}[fact]


%technical intermezzo
%---------------------

\newcommand{\intermezzoa}{
	\begin{minipage}[c]{13cm}
	\begin{center}\rule{10cm}{0.4pt}



	\tiny{\sc Optional Content Starts}
	
	\vspace{-1mm}
	
	\rule{10cm}{0.4pt}\end{center}
	\end{minipage}\nopagebreak 
	}


\newcommand{\intermezzob}{\nopagebreak 
	\begin{minipage}[c]{13cm}
	\begin{center}\rule{10cm}{0.4pt}

	\tiny{\sc Optional Content Ends}
	
	\vspace{-1mm}
	
	\rule{10cm}{0.4pt}\end{center}
	\end{minipage}
	}
	
	
%--------------------






















\newtheorem*{reply*}{Reply}
\usepackage{enumitem}
\newcommand{\question}[1]{\begin{enumerate}[resume,leftmargin=0cm,labelsep=0cm,align=left]
\item #1
\end{enumerate}}

\usepackage{float}

% \setbeamertemplate{blocks}[rounded][shadow=true]
% \setbeamertemplate{itemize items}[ball]
% \AtBeginPart{}
% \AtBeginSection{}
% \AtBeginSubsection{}
% \AtBeginSubsubsection{}
% \setlength{\emergencystretch}{0em}
% \setlength{\parskip}{0pt}






\usepackage[authoryear]{natbib}

%\bibliographystyle{apalike}



\usepackage{tikz}
\usetikzlibrary{positioning,shapes,arrows}

\ifluatex
  \usepackage{selnolig}  % disable illegal ligatures
\fi

\title{Awareness Growth in Bayesian Networks}
\usepackage{etoolbox}
\makeatletter
\providecommand{\subtitle}[1]{% add subtitle to \maketitle
  \apptocmd{\@title}{\par {\large #1 \par}}{}{}
}
\makeatother
\subtitle{Reply to Steele and Stefánsson}
\author{Marcello/Rafal}
\date{}

\begin{document}
\maketitle

\hypertarget{introduction}{%
\section{Introduction}\label{introduction}}

Learning is modeled in the Bayesian framework by the rule of
conditionalization. This rule posits that the agent's new degree of
belief in a proposition \(A\) after a learning experience \(E\) should
be the same as the agent's old degree of belief in \(A\) conditional on
\(E\). That is, \[\ppr{E}{A}=\pr{A \vert E},\] where \(\pr{}\)
represents the agent's old degree of belief (before the learning
experience \(E\)) and \(\ppr{E}{}\) represents the agent's new degree of
belief (after the learning experience \(E\)).

One assumption here is that \(E\) is learned with certainty. After the
agent learns about \(E\), there is no longer any doubt about the truth
of \(E\). This assumption has been the target of extensive
discussion.\footnote{As is well-known, Jeffrey's conditionalization
  relaxes this assumption.} The other assumption---which we will focus
on---is that \(E\) and \(A\) belong to the agent's algebra of
propositions. This algebra models the propositions that the agent is
aware of and entertains as live possibilities.

The algebra---the agent's awareness state---is fixed once and for all.
The learning experience does not modify it. Crucially, even before
learning about \(E\), the agent already knows the degree of belief in
any proposition conditional on \(E\). In this model, the agent cannot
learn something they have never thought about. This picture forces a
great deal of rigidity on the learning process. It commits the agent to
the specification of their `total possible future experience' (Howson
1976, The Development of Logical Probability), as though learning was
confined to an `initial prison' (Lakatos, 1968, Changes in the Problem
of Inductive Logic).

But, arguably, the learning process is more complex than what
conditionalization allows. Not only do we learn that some propositions
that we were entertaining are true or false, but we may also learn new
propositions that we did not entertain before. Or we may entertain new
propositions---without necessarily learning that they are true or
false---and this change in awareness may in turn change what we already
believe. How should this more complex learning process be modeled by
Bayesianism? Call this the problem of awareness growth.\footnote{This
  problem can perhaps be divided into two parts: (i) how to model
  \textit{learning} a new proposition not in the initial awareness state
  of the agent; (ii) how to model \textit{entertaining} a new
  proposition not in the initial awareness state of the agent (without
  yet learning it). We will return to this distinction in due course.}

Critics of Bayesianism and sympathizers alike have been discussing the
problem of awareness growth under different names for quite some time,
at least since the eighties. This problem arises in a number of
different contexts, for example, new scientific theories (Glymour, 1980,
Why I am not a Bayesian; Chihara 1987, Some Problems for Bayesian
Confirmation Theory; Earmann 1992, Bayes of Bust?), language changes and
paradigm shifts (Williamson 2003, Bayesianism and Language Change), and
theories of induction (Zabell, Predicting the Unpredictable).

Now, of course, the algebra of propositions could in principle be so
rich to contain anything that could possibly be conceived, expressed,
thought of. Such an algebra would not need to change at any point in the
future. God-like agents could be associated with such rich algebra of
propositions, but this is hardly a plausible model of ordinary agents
with bounded resources such as ourselves. A fully comprehensive algebra
of propositions cannot be the answer here.

A more promising proposal is Reverse Bayesianism (Karni and Viero, 2015,
Probabilistic Sophistication and Reverse Bayesianism; Wenmackers and
Romeijn 2016, New Theory About Old Evidence; Bradely 2017, Decision
Theory with A Human Face) . The idea is to model awareness growth as a
change in the algebra while ensuring that the probabilities of the
propositions shared between the old and new algebra remain fixed under
suitable constraints.

Let \(\mathcal{F}\) be the initial algebra of propositions and let
\(\mathcal{F}^+\) the algebra after the agent's awareness has grown. For
reason that will soon become clear, let's pick out subsets of these
algebras which contain only basic propositions, those that do not
contain connectives such as negations, conjunctions or disjunctions.
Call these subsets \(X\) and \(X^+\) respectively. Obviously,
\(\mathcal{F}\subseteq \mathcal{F}^+\) and \(X\subseteq X^+\). Reverse
Bayesianism posits that the ratio of probabilities for any propositions
\(A\) and \(B\) in \(X\)---the basic propositions shared by the old and
new algebra---remain constant through the process of awareness growth:
\[\frac{\pr{A}}{\pr{B}} = \frac{\ppr{+}{A}}{\ppr{+}{B}},\] where
\(\pr{}\) represents the agent's degree of belief before awareness
growth and \(\ppr{+}{}\) represents the agent's degree of belief after
awareness growth.

What is the justification for Reverse Bayesianism? Perhaps the best
justification is pragmatic. As an agent's awareness grows, the agent
might not want to throw away completely the epistemic work they have
done so far. The agent may prefer to retain as much of their old
assignments of degrees of beliefs as possible. Reverse Bayesianism
provides a simple recipe to do that. It also coheres with the
conservative spirit of conditionalization. Reverse Bayesianism preserves
the old probability distribution conditional on the old awareness
state.\footnote{Strictly speaking, this interpretation is what we later
  call Awareness Rigidity.} Similarly, conditionalization preserves the
old probability distribution conditional on what has been learned.
\textbf{SHOW PICTURE BELOW}.

Reverse Bayesianism is an elegant theory that manages to cope with a
seemingly intractable challenge for Bayesianism. Unfortunately, it is
not without difficulties. Steele and Stefánsson (2021, Belief Revision
for Growing Awareness) argue that Reverse Bayesianism, when suitably
formulated, can work in a limited class of cases, what they call cases
of \textit{awareness expansion}. But they claim it cannot work in cases
of \textit{awareness refinement}. The distinction between refinement and
expansion that Steele and Stefánsson draw, albeit a good first
approximation, is too coarse and should be made more precise. We will
show that there are cases of refinement in which Reverse Bayesianism
(or, at least, a suitable formulation of it) can be made to work.

More generally, we believe that Steele and Stefánsson's negative
conclusion about Reverse Bayesianism is overly pessimistic. Much of the
literature on awareness growth is concerned with a formal, algorithmic
solution to the problem. We believe that seeking a formal, algorithmic
solution is not a promising strategy---and in that we agree with Steele
and Stefánsson. At the same time, we think that the awareness of agents
grows while holding fixed certain material structural assumptions, based
on commonsense, semantic stipulations or causal dependency. To model
awareness growth, we need a formalism that can model these material
structural assumptions. We sketch how this can done using Bayesian
networks. This approach---we think---salvages the spirit of Reverse
Bayesianism.

\hypertarget{counterexamples}{%
\section{Counterexamples?}\label{counterexamples}}

To assess the strength of their case against Reverse Bayesianism, we
begin by rehearsing some of the ingenious counterexamples that Steele
and Stefánsson have formulated. The first is this:

\begin{quote}
Suppose you happen to see your partner enter your best friend's house on
an evening when your partner had told you she would have to work late.
At that point, you become convinced that your partner and best friend
are having an affair, as opposed to their being warm friends or mere
acquaintances. You discuss your suspicion with another friend of yours,
who points out that perhaps they were meeting to plan a surprise party
to celebrate your upcoming birthday---a possibility that you had not
even entertained. Becoming aware of this possible explanation for your
partner's behaviour makes you doubt that she is having an affair with
your friend, relative, for instance, to their being warm friends.
(Steele and Stefánsson, 2021, Section 5, Example 2)
\end{quote}

\noindent So, initially, the algebra only contains the hypotheses `my
partner and my best friend met to have an affair' (\textit{Affair}) and
`my partner and my best friend met as friends or acquaintances'
(\textit{Friends/acquaintances}). The other proposition in the algebra
is the evidence at your disposal, that is, the fact that your partner
and your best friend met one night secretively and without telling you
(\textit{Secretive}). There may be other propositions, but these are the
ones to focus on.

Why does this scenario conflict with Reverse Bayesianim? Even though
Steele and Stefánsson do not provide the details, it pays to be explicit
here at the cost of pedantry. Clearly, hypothesis \textit{Affair} better
explains the evidence at your disposal than hypothesis
\textit{Friends/acquaintances}. In probabilistic terms, this can be
expressed by comparing likelihoods:
\[\pr{ \textit{Secretive} \vert \textit{Affair}}> \pr{\textit{Secretive} \vert \textit{Friends/acquaintances}},\]
from which it also follows that \textit{Affair} is more probable than
\textit{Friends/acquaintances}
\[\pr{\textit{Affair} \vert  \textit{Secretive} }> \pr{\textit{Friends/acquaintances} \vert \textit{Secretive}}, \tag{>}\]
so long as the prior probabilities of the two hypotheses are not skewed
in one direction.\footnote{If you were initially nearly certain your
  partner could not possibly have an affair, even the fact they behaved
  very secretively or lied to you might not affect the probability of
  the two hypotheses.}

Next, the algebra changes. A new hypothesis is added which you had not
considered before: your partner and your best friends met to plan a
surprise party for your upcoming birthday (\textit{Surprise}). This is a
game changer. The evidence \textit{Secretive} now makes better sense in
light of this new hypothesis than the hypothesis \textit{Affair}:
\[\ppr{+}{ \textit{Secretive} \vert \textit{Surprise}}> \ppr{+}{\textit{Secretive} \vert \textit{Affair}}.\]
And, this new hypothesis should be more likely than the hypothesis
\textit{Affair}:
\[\ppr{+}{ \textit{Surprise} \vert \textit{Secretive}}> \ppr{+}{ \textit{Affair} \vert \textit{Secretive}}. \tag{*}\]

So far so good. Reverse Bayesianism is not yet in trouble. Steele and
Stefánsson, however, concludes that the probability of
\textit{Friends/acquaintances} should now exceed that of \textit{Affair}
(`Becoming aware of this possible explanation for your partner's
behaviour makes you doubt that she is having an affair with your friend,
relative, for instance, to their being warm friends.'):
\[\ppr{+}{\textit{Affair} \vert  \textit{Secretive} } < \ppr{+}{\textit{Friends/acquaintances} \vert \textit{Secretive}}. \tag{<}\]
Arguably, this holds because \textit{Surprise} implies
\textit{Friends/acquaintances}. In order to prepare a surprise party,
your partner and best friend have to be at least acquaintances. And
given that one implies the other, if \textit{Surprise} is more likely
than \textit{Affair} (by \(*\)), then \textit{Friends/acquaintances}
must also be more likely than \textit{Affair}. And if both (\(>\)) and
(\(<\)) holds, the ratio of the probabilities of basic propositions is
not fixed before and after the episode of awareness growth. This is a
violation of Reverse Bayesianism.

But, as Steele and Stefánsson admits, Reverse Bayesianism is not really
in trouble here. It can still be made to work by replacing it with a
slightly different---though quite similar in spirit---condition, called
Awareness Rigidity: \[\ppr{+}{A \vert T^*}=\pr{A},\] where \(T^*\)
corresponds to a proposition that picks out, from the vantage point of
the new awareness state, what corresponds to the entire possibility
space before the episode of awareness growth. In our running example,
the proposition \(\neg\textit{Surprise}\) picks out the entire
possibility space before the episode of awareness growth. So Awareness
Rigidity would require that:
\[\ppr{+}{\textit{Friends/acquaintances} \vert \neg\textit{Surprise}}=\pr{\textit{Friends/acquaintances}}.\]
Conditional on \(\neg\textit{Surprise}\), it is indeed true that the
probability of \textit{Friends/acquaintances} has not changed before and
after the episode of awareness growth. And it is also true that
\textit{Affair} remains the most likely hypothesis in light of the
evidence (again conditional on \(\neg\textit{Surprise}\)):
\[\ppr{+}{\textit{Affair} \vert  \textit{Secretive} \& \neg\textit{Surprise} } > \ppr{+}{\textit{Friends/acquaintances} \vert \textit{Secretive} \& \neg\textit{Surprise}}. \tag{$>^+$}\]
Awareness Rigidity is vindicated. Reverse Bayesianism---the spirit of
it, not the letter---stands.

This is not the end of the story, however. Steele and Stefánsson offer
another counterexample to Reverse Bayesianism (which also works against
Awareness Rigidity):

\begin{quote}
Suppose you are deciding whether to see a movie at your local cinema.
You know that the movie's predominant language and genre will affect
your viewing experience. The possible languages you consider are French
and German and the genres you consider are thriller and comedy. But then
you realise that, due to your poor French and German skills, your
enjoyment of the movie will also depend on the level of difficulty of
the language. Since it occurs to you that the owner of the cinema is
quite simple-minded, you are, after this realisation, much more
confident that the movie will have low-level language than high-level
language. Moreover, since you associate low-level language with
thrillers, this makes you more confident than you were before that the
movie on offer is a thriller as opposed to a comedy. (Steele and
Stefánsson, 2021, Section 5, Example 3)
\end{quote}

\noindent Admittedly, this counterexample is complex. For the sake of
clarity, it can be split into two episodes of awareness growth. The
first episode consists in considering, as a new variable besides
language and genre, the language difficulty of the movie. Initially, the
algebra contained the propositions \textit{French} and \textit{German},
as well as \textit{Thriller} and \textit{Comedy}. Then, you realize
another variable might be at play, namely the level of difficulty of the
language of the movie, \textit{Difficult} and \textit{Easy}. The second
episode of awareness growth consisting in learning something you did
consider before, namely that the owner is simple-minded.

This counterexample is a case of refinement because, first, you
categorize movies by just language and genre, and then you add a further
category, level of difficulty.\footnote{In the other counterexample,
  instead, the possibility space grew by adding situations in which your
  partner and best friends met neither as lovers nor solely as friends.
  More on expansion below.} The second episode of awareness growth
brings further refinement, namely learning that the owner is
simple-minded. The difference between the two episodes of awareness
growth is that the first did not bring about any change in the
probabilities, only the second did. The realization that the owner is
simple-minded suggests that the level of linguistic difficulty of the
movie will be low. The latter in turn suggests that the movie is more
likely a thriller rather than a comedy (possibly because thrillers are
simpler---linguistically---than comedies).

So, as already noted, the counterexample is complex. But, taken at face
value, it challenges both Reverse Bayesianism and Awareness Rigidity. It
is not true, against Reverse Bayesianism, that
\(\frac{\pr{\textit{Thriller}}}{\pr{\textit{Comedy}}}=\frac{\ppr{+}{\textit{Thriller}}}{\ppr{+}{\textit{Comedy}}}\).
Further, against Awareness Rigidity, it is not true that
\(\pr{\textit{Thriller}}=\ppr{+}{\textit{Thriller} \vert \textit{Thriller}\vee \textit{Comedy}}\).
Since this is a case of refinement, the proposition
\(\textit{Thriller}\vee \textit{Comedy}\) which picks out the entire
possibility space is the the same before and after awareness growth. In
cases of awareness growth by refinement, then, Awareness Rigidity
mandates that all probability assignments stay the same.

This counterexample is likely to leave many unconvinced, or at best
puzzled. Since it consists of two episodes, we are left wondering which
one of the two is essential. The first is a case of mere
refinement---you simply entertain a new way to categorize movies. The
second episode is a case of learning---you learn something your did
consider before, namely that the owner is simple-minded. So awareness
growth is here understood both as (i) \textit{entertaining} a new
proposition not in the initial awareness state of the agent (without yet
learning it) and (ii) \textit{learning} a new proposition not in the
initial awareness state of the agent. We agree with Steele and
Stefánsson that both (i) and (ii) shoud count as instance of awareness
of awareness growth. Our qualms are about their role in the
counterexample. Is only the second episode necessary for the
counterexample to work, while the first just gives added context? What
is going on, exactly? Is there a more straightforward counterexample
that only depicts mere refinement without an episode of learning
intertwined with it? A cleaner picture of awareness growth in cases of
refinement is preferable.

The need for a cleaner picture also applies to the first counterexample.
There remains---we think---the need to further examine cases of
awareness expansion. They consist in the addition of another proposition
not previously in the algebra, but that is not a refinement of existing
propositions. The addition of the hypothesis \textit{Surprise} is,
however, an ambiguous case. For one thing, \textit{Surprise} is a novel
hypothesis that cannot be subsumed under \textit{Friends/acquaintances}
or \textit{Affair}. On the other, \textit{Surprise} seems a refinement
of \textit{Friends/acquaintances}, since a meeting for planning a
surprise is a more specific way to describe a meeting of acquaintances.

A more clear-cut case of awareness expansion would be the following. The
police is investigating a murder case. There are two suspects under
investigation: Joe and Sue. They both have a motive. The evidence
consists in a DNA match and information about how the crime was
committed. Sue genetically matches the traces, but is quite short and
the perpetrator is known to be a tall person. Joe is neither tall nor
does he genetically match the crime traces. In light of the evidence,
Sue seems more likely the culprit than Joe, but matters are still open
ended. Then, a new hypothesis is considered: Ela could be the
perpetrator. As it turns out, Ela genetically matches the traces, is
tall enough to have committed the crime, and does have a motive. This
seems a straightforward case of expansion because Ela, Sue and Joe are
incompatible hypotheses, while \textit{Friends/acquaintances} and
\textit{Surprise} need not be. Any model of awareness growth should be
able to analyze more precisely the difference between the example
provided by Steele and Stefánsson and the criminal case just outlined.
They are both, arguably, cases of expansion, but they are also
different.

Steele and Stefánsson provide a formal definition of the difference
between refinement and expansion. Our observations here are largely
confined at the intuitively level. Our point is that there are a number
of intuitively plausible differences that a formal theory should be able
to capture. The coarse distinction between refinement and expansion
might be in the too coarse. Relying on Bayesian networks, we illustrate
this point more precisely in the next section.

\hypertarget{bayesian-networks}{%
\section{Bayesian Networks}\label{bayesian-networks}}

The key idea is that an awareness state is represented by a graphical
networks, with nodes and arrows, along with a probability distribution
associated with the network. Awareness growth brings a bout a change in
the graphical network---nodes and arrows are added or erased---as well
as a change in the probability distribution from the old to the next
network. Call this network refinement. While this process is not merely
formal or algorithmic, certain plausible conservative constrains guide
network refinement.

ADD BRIEF INFORMAL DESCRIPTION ABOUT BAYESIAN NETWORKS HERE.

\hypertarget{a-simpler-counterexample}{%
\subsection{A simpler counterexample}\label{a-simpler-counterexample}}

As noted earlier, Steele and Stefánsson's counterexample to reverse
Bayesianism for the case of refinement is rather complex, perhaps
unnecessary so. We now present a much simpler one:

\begin{quote}
You have evidence that favors a certain hypothesis, say a witness
testifies they saw the defendant around the crime scene. You give some
weight to this evidence. In your assessment, that the defendant was seen
around the crime scene raises the probability that the defendant was
actually there. But now you wonder, what if it was dark? You come to the
following judgment: if the lighting conditions were good, you should
still trust the evidence, but if they were bad, you should not. Suppose
you do learn that the lighting conditions were bad. In that case, the
evidence at your disposal should no longer favor the hypothesis that the
defendant was actually around the crime scene. After your awareness has
grown, the probabilty of that hypothesis has gone down.
\end{quote}

\noindent This scenario of awareness growth consists of two parts:
first, an episode of mere refinement. You wonder about the lighting
conditions. The witness could have seen the defendant under good or bad
lighting conditions. After that realization, or concomitantly to it, you
learn that the lighting conditions were actually bad. This two-part
structure resembles the more complex scenario by Steele and Stefánsson.
It is instructive to make our counterexample even simpler and focus on
mere refinement without learning. Does mere refinement in this scenario
constitute a counterexample to reverse Bayesian and Awareness Rigidity?
It does, and the theory of Bayesian networks helps to see why.

Initially, your graphical network looks like this: \[H \rightarrow E\]

\noindent Since you trust the evidence, you think that the evidence is
more likely under the hypothesis that the defendant was present at the
crime scene than under the alternative hypothesis:

\[\pr{\textit{E=seen} \vert \textit{H=present}} > \pr{\textit{E=seen} \vert \textit{H=absent}}\]

\noindent It is not necessary to fix exact numerical values for these
conditional probabilities. Think of the inequality as as a qualitative
ordering of how plausible the evidence at your disposal is in light of
competing hypotheses. No matter the numbers, by the probability
calculus, it follows that the evidence raises the probability of the
hypothesis \textit{H=present}:
\[\pr{\textit{H=present}\vert \textit{E=seen}} > \pr{\textit{H=present}}\]

\noindent As you wonder about the lighting conditions, the graph should
be amended: \[H \rightarrow E \leftarrow L,\] where the node \(L\) can
have two values, \textit{L=good} and \textit{L=bad}. A plausible way to
update your assessment of the evidence is as follows:
\[\ppr{+}{\textit{E=seen} \vert \textit{H=present} \wedge \textit{L=good}} > \ppr{+}{\textit{E=seen} \vert \textit{H=absent} \wedge \textit{L=good}}\]
\[\ppr{+}{\textit{E=seen} \vert \textit{H=present} \wedge \textit{bad}} = \ppr{+}{\textit{E=seen} \vert \textit{H=absent} \wedge \textit{L=bad}}\]

\noindent This is what you are thinking: if the lighting conditions were
good, you still trust the evidence like you did before. But if the
lighting conditions were bad, you regard the evidence as no better than
chance. Again, there are no exact numerical values here.

The probability function changes from \(\pr{}\) to \(\ppr{+}{}\), but
the two agree in one respect:
\[\pr{E=x \vert H=x} \geq \pr{E=x \vert H=x'} \textit{ iff } \ppr{+}{E=x \vert H=x} \geq \ppr{+}{E=x \vert H=x'} \tag{C},\]
where both \(E\) and \(H\) are nodes that are part of the graphical
network before and after the awareness growth. So the plausibility
ordering between hypotheses and evidence is preserved. Condition (C) is
a good candidate for a conversativity constraint that governs the
relationship between \(\pr{}\) and \(\ppr{+}{}\) (more on this later).
It would be, however, too strong to require:
\[\frac{\pr{E=x \vert H=x}}{\pr{E=x \vert H=x'}}= \frac{\ppr{+}{E=x \vert H=x}}{\ppr{+}{E=x \vert H=x'}} \tag{CC}.\]
Our counterexample does not violate (C), but does violate (CC). For
suppose:
\[\\pr{\textit{E=seen} \vert \textit{H=present}}=\ppr{+}{\textit{E=seen} \vert \textit{H=present} \wedge \textit{L=good}}=.8\]
\[\pr{\textit{E=seen} \vert \textit{H=absent}}=\ppr{+}{\textit{E=seen} \vert \textit{H=absent} \wedge \textit{L=good}}=0.4\]
\[\ppr{+}{\textit{E=seen} \vert \textit{H=present} \wedge \textit{L=bad}} = \ppr{+}{\textit{E=seen} \vert \textit{H=absent} \wedge \textit{L=bad}}=0.5.\]
So the ratio
\(\frac{\pr{\textit{E=seen} \vert \textit{H=present}}}{\pr{\textit{E=seen} \vert \textit{H=absent}}}=2\).
Before awareness growth, you think the evidence favor the hypothesis
moderly strongly. That seem entirely reasonable. But, after the
awareness growth, the ratio
\(\frac{\ppr{+}{\textit{E=seen} \vert \textit{H=present}}}{\ppr{+}{\textit{E=seen} \vert \textit{H=absent}}}=\frac{.65}{.45}\approx 1.44\).\footnote{The
  calculations here rely on the dependency structure encoded in the
  Bayesian network (se starred step below). \begin{align*}
  \ppr{+}{\textit{E=seen} \vert \textit{H=present}} &= \ppr{+}{\textit{E=seen} \wedge \textit{L=good} \vert \textit{H=present}}+\ppr{+}{\textit{E=seen} \wedge \textit{L=bad} \vert \textit{H=present}}\\
  &= \ppr{+}{\textit{E=seen} \vert \textit{H=present} \wedge \textit{L=good}}  \times \ppr{+}{\textit{L=good} \vert  \textit{H=present} }\\ & +\ppr{+}{\textit{E=seen}  \vert \textit{H=present} \wedge \textit{L=bad}} \times \ppr{+}{\textit{L=bad} \vert  \textit{H=present}}\\
  &=^* \ppr{+}{\textit{E=seen} \vert \textit{H=present} \wedge \textit{L=good}}  \times \ppr{+}{\textit{L=good}}\\ & +\ppr{+}{\textit{E=seen}  \vert \textit{H=present} \wedge \textit{L=bad}} \times \ppr{+}{\textit{L=bad}}\\
  &= .8 \times .5 +.5 *.5 = .65 
  \end{align*} \begin{align*}
  \ppr{+}{\textit{E=seen} \vert \textit{H=absent}} &= \ppr{+}{\textit{E=seen} \wedge \textit{L=good} \vert \textit{H=absent}}+\ppr{+}{\textit{E=seen} \wedge \textit{L=bad} \vert \textit{H=absent}}\\
  &= \ppr{+}{\textit{E=seen} \vert \textit{H=absent} \wedge \textit{L=good}}  \times \ppr{+}{\textit{L=good} \vert  \textit{H=absent} }\\ & +\ppr{+}{\textit{E=seen}  \vert \textit{H=absent} \wedge \textit{L=bad}} \times \ppr{+}{\textit{L=bad} \vert  \textit{H=absent}}\\
  &=^* \ppr{+}{\textit{E=seen} \vert \textit{H=absent} \wedge \textit{L=good}}  \times \ppr{+}{\textit{L=good}}\\ & +\ppr{+}{\textit{E=seen}  \vert \textit{H=absent} \wedge \textit{L=bad}} \times \ppr{+}{\textit{L=bad}}\\
  &= .4 \times .5 +.5 *.5 = .45 
  \end{align*}} This shows that mere refinement can weaken the strength
of the evidence, even without learning anything new. Of course, if you
did learn that the lighting conditions were bad, the evidence would
become even weaker, effectively worthless:
\[\frac{\ppr{+, \textit{L=bad}}{\textit{E=seen} \vert \textit{H=present}}}{\ppr{+, \textit{L=bad}}{\textit{E=seen} \vert \textit{H=absent}}}=1,\]
where \(\ppr{+, \textit{L=bad}}{}\) is the new probability function
after learning that \textit{L=bad}. In any event, after awareness
growth, the evidence at your disposal favors
\textit{H=defendant-present} less strongly, or not at all. Since the
prior probability of the hypothesis should be the same before and after
awareness growth, it follows that
\[\ppr{+}{\textit{H=defendant-present} \vert \textit{E=defendant-seen}} \neq \pr{\textit{H=defendant-present} \vert \textit{E=defendant-seen}}\]
\[\ppr{+, \textit{L=bad} }{\textit{H=defendant-present} \vert \textit{E=defendant-seen}} \neq \pr{\textit{H=defendant-present} \vert \textit{E=defendant-seen}}\]
This outcome violates both Reverse Bayesianism and Awareness
Rigidity.\textbf{WHY? EXPLAIN}.

So we now have a simpler counterexample that works even without
postulating that mere refinement takes place together with learning of
something new. At the same time, this counterexample also show that
something can be preserved, namely the plausibility ordering along the
lines of condition (C). This condition should also hold in Steele and
Stefánsson's scenario. We simply sketch the reasoning here.\\
At first, the graphical network looks like this:
\[\textit{Genre} \rightarrow \textit{Enjoyment} \leftarrow \textit{Language},\]
where each node can take two values: \textit{Genre=comedy} and
\textit{Genre=thriller}; \textit{Language=french} and
\textit{Language=german}; and \textit{Enjoyment=yes} and
\textit{Enjoyment=no}. Assume, you have no reason to think the language
or genre or the movie is one of the two. So you are indifferent about
the different options. But let's assume you are ranking the options in
terms of how they are going to contribute to your enjoyment
(\textit{Enjoyment=yes}). You are more likely to enjoy a comedy in
French more then everything else, but you are more likely to enjoy a
thriller in German than one in French, and your lowest preference is for
a comedy in German. This ranking can be encoded, for all combinations,
by conditional probabilities of the form
\[\pr{\textit{Enjoyment=x} \vert \textit{Language=y} \wedge \textit{Genre=z}} \geq \ppr{+}{\textit{Enjoyment=x} \vert \textit{Language=y'} \wedge \textit{Genre=z'}}.\]

The first episode of awareness growth consists in realizing that the
linguistic difficulty of the movie could also be a factor. So the
expanded graphical network now becomes:
\textbf{DRWA GRAPH WITH AN EXTRA NODE FOR "DIFFICULTY" WITH ARROW POINT INTO "ENJOYMENT" NODE}

\noindent The new node can take two values: \textit{Difficulty=high} and
\textit{Difficulty=low}. Your ranking of what is likely to give you
enjoyment should now be updated and made more specific, but much of the
earlier ordering can be retained, that is:

\[\pr{\textit{Enjoyment=x} \vert \textit{Language=y} \wedge \textit{Genre=z}} \geq \pr{\textit{Enjoyment=x} \vert \textit{Language=y} \wedge \textit{Genre=z}} \textit{ iff } \ppr{+}{\textit{Enjoyment=x} \vert \textit{Language=y} \wedge \textit{Genre=z}} \geq \ppr{+}{\textit{Enjoyment=x} \vert \textit{Language=y} \wedge \textit{Genre=z}}.\]

There may well be cases in which this plausibility ordering is not
preserved. This would be \textit{truly transformative} awareness growth.
But this would perhaps be extremely rare. Suppose you have evidence
that---you think---reliably tracks a hypothesis, say evidence as of hand
reliably tracks the presence of hands before you. You may now entertain
a skeptical hypothesis in which a demon actually switches things around:
when you see a hand, there is no hand at, and when you do not see a
hand, there is actually a hand. Even in this case, it is not obvious
that there would be a violation of a condition like (C). EXPLAIN.

\hypertarget{a-different-refinement}{%
\subsection{A different refinement}\label{a-different-refinement}}

\hypertarget{steele-and-stefuxe1nsson-example}{%
\subsection{Steele and Stefánsson
example}\label{steele-and-stefuxe1nsson-example}}

Before awareness growth, the Bayesian network has a simple form:
\[H \rightarrow E,\] where the hypothesis variable \(H\) takes two
values, \(H=\textit{Affair}\) and \(H=\textit{Friends/acquaintances}\).
The evidence variable \(E\) can take several values, one of them being
\(E=\textit{Secretive}\). You could have seen other things other than
what you saw, but there is no need to specify the other values
exhaustively. Suppose the prior odds ratio of the hypotheses is 1:1,
say, because you suspected your partner might be cheating on you, and
the likelihood ratio
\[\frac{\pr{E=\textit{Secretive}\vert H=\textit{Affair}}}{\pr{E=\textit{Secretive}\vert H=\textit{Friends/acquaintances}}}\]
is 9:1, because the hypothesis \textit{Affair} is a better explanation
of the evidence than the hypothesis \textit{Friends/acquaintances}.
Then, the posterior probability given the evidence
\[\pr{H=\textit{Affair} \vert E=\textit{Secretive}}\] is quite high,
\(\frac{9}{10}=.9\). So
\(\ppr{E=\textit{Secretive}}{H=\textit{Affair}}=.9\).\footnote{This
  calculation presupposes that the two hypotheses \textit{Affair} and
  \textit{Friends/acquaintances} are exclusive and exhaustive. This
  assumption is justified given the initial awareness state of the
  agent.}

After awareness growth, the Bayesian network should be modified as
follows: \[H \leftarrow H' \rightarrow E,\] where the new hypothesis
node now consists of three values instead of two:

\(H'=\textit{Affair}\)

\(H'=\textit{Friends/acquaintances}\wedge \neg \textit{Surprise}\)

\(H'=\textit{Friends/acquaintances}\wedge\textit{Surprise}\).

\noindent The scenario \(\textit{Friends/acquaintances}\) is split into
the scenario in which your partner and best friend met simply as friends
or acquaintances, and the scenario in which they met to prepare a
surprise party for you. On this interpretation, the counterexample by
Steele and Stefánsson is a case of refinement, not expansion. We will
return to this point later.

The network contains a directed arrow between the old hypothesis node
\(H\) and the new hypothesis node \(H'\) This arrow can be interpreted
as a bridge between the old awareness state limited to two hypotheses
and the new awareness state that contains an additional hypothesis. This
bridge is purely conceptual and can be defined by two sets of
constrains. The first set of constrains posits that \textit{Affair}
under \(H\) has the same meaning as \textit{Affair} under \(H'\):

\(\ppr{+}{H=\textit{Affair} \vert H'=\textit{Affair}}=1\)

\(\ppr{+}{H=\textit{Affair} \vert H'=\textit{\textit{Friends/acquaintances}}}=0\)

\(\ppr{+}{H=\textit{Affair} \vert H'=\textit{Surprise}}=0\)

The second set of constrains posits that hypothesis
\(\textit{Friends/acquaintances}\) under \(H\) can be actually be
interpreted in two ways under \(H'\), as
\(\textit{Friends/acquaintances} \wedge \neg \textit{Surprise}\) and
\(\textit{Friends/acquaintances} \wedge \textit{Surprise}\). So, in
other words, the episode of awareness growth consists in the realization
that \(\textit{Friends/acquaintances}\) can be made precise in two more
specific ways:

\(\ppr{+}{H=\textit{Friends/acquaintances} \vert H'=\textit{Affair}}=0\)

\(\ppr{+}{H=\textit{Friends/acquaintances} \vert H'=\textit{Friends/acquaintances} \wedge \neg \textit{Surprise}}=1\)

\(\ppr{+}{H=\textit{Friends/acquaintances} \vert H'=\textit{Friends/acquaintances} \wedge \textit{Surprise}}=0\)

This bridge between \(H\) and \(H'\) justifies the following
conservativity constraint:

\[\frac{\pr{E=\textit{Secretive}\vert H=\textit{Affair}}}{\pr{E=\textit{Secretive}\vert H=\textit{Friends/acquaintances}}} = \frac{\ppr{+}{E=\textit{Secretive}\vert H=\textit{Affair}}}{\ppr{+}{E=\textit{Secretive}\vert H=\textit{Friends/acquaintances}}}=\frac{9}{1} \]

\hypertarget{expansion-criminal-case-example}{%
\subsection{Expansion: criminal case
example}\label{expansion-criminal-case-example}}

\hypertarget{refinement-downward}{%
\subsection{Refinement: downward}\label{refinement-downward}}

\hypertarget{refinement-cross}{%
\subsection{Refinement: cross}\label{refinement-cross}}

\hypertarget{refinement}{%
\subsection{Refinement:}\label{refinement}}

\end{document}
