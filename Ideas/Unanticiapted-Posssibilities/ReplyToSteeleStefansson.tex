% Options for packages loaded elsewhere
\PassOptionsToPackage{unicode}{hyperref}
\PassOptionsToPackage{hyphens}{url}
\PassOptionsToPackage{dvipsnames,svgnames*,x11names*}{xcolor}
%
\documentclass[
  11pt,
  dvipsnames,enabledeprecatedfontcommands]{scrartcl}
\usepackage{amsmath,amssymb}
\usepackage{lmodern}
\usepackage{ifxetex,ifluatex}
\ifnum 0\ifxetex 1\fi\ifluatex 1\fi=0 % if pdftex
  \usepackage[T1]{fontenc}
  \usepackage[utf8]{inputenc}
  \usepackage{textcomp} % provide euro and other symbols
\else % if luatex or xetex
  \usepackage{unicode-math}
  \defaultfontfeatures{Scale=MatchLowercase}
  \defaultfontfeatures[\rmfamily]{Ligatures=TeX,Scale=1}
\fi
% Use upquote if available, for straight quotes in verbatim environments
\IfFileExists{upquote.sty}{\usepackage{upquote}}{}
\IfFileExists{microtype.sty}{% use microtype if available
  \usepackage[]{microtype}
  \UseMicrotypeSet[protrusion]{basicmath} % disable protrusion for tt fonts
}{}
\usepackage{xcolor}
\IfFileExists{xurl.sty}{\usepackage{xurl}}{} % add URL line breaks if available
\IfFileExists{bookmark.sty}{\usepackage{bookmark}}{\usepackage{hyperref}}
\hypersetup{
  pdftitle={Reply to Steeel and Stefansson},
  pdfauthor={Marcello/Rafal},
  colorlinks=true,
  linkcolor=Maroon,
  filecolor=Maroon,
  citecolor=Blue,
  urlcolor=blue,
  pdfcreator={LaTeX via pandoc}}
\urlstyle{same} % disable monospaced font for URLs
\usepackage{graphicx}
\makeatletter
\def\maxwidth{\ifdim\Gin@nat@width>\linewidth\linewidth\else\Gin@nat@width\fi}
\def\maxheight{\ifdim\Gin@nat@height>\textheight\textheight\else\Gin@nat@height\fi}
\makeatother
% Scale images if necessary, so that they will not overflow the page
% margins by default, and it is still possible to overwrite the defaults
% using explicit options in \includegraphics[width, height, ...]{}
\setkeys{Gin}{width=\maxwidth,height=\maxheight,keepaspectratio}
% Set default figure placement to htbp
\makeatletter
\def\fps@figure{htbp}
\makeatother
\setlength{\emergencystretch}{3em} % prevent overfull lines
\providecommand{\tightlist}{%
  \setlength{\itemsep}{0pt}\setlength{\parskip}{0pt}}
\setcounter{secnumdepth}{5}
%\documentclass{article}

% %packages
\usepackage{booktabs}
%\usepackage[left]{showlabels}
\usepackage{multirow}
\usepackage{subcaption}
\usepackage{wrapfig}
\usepackage{graphicx}
\usepackage{longtable}
\usepackage{ragged2e}
\usepackage{etex}
%\usepackage{yfonts}
\usepackage{marvosym}
\usepackage[notextcomp]{kpfonts}
\usepackage{nicefrac}
\newcommand*{\QED}{\hfill \footnotesize {\sc Q.e.d.}}
\usepackage{floatrow}
\usepackage{multicol}

\usepackage[textsize=footnotesize]{todonotes}
\newcommand{\ali}[1]{\todo[color=gray!40]{\textbf{Alicja:} #1}}
\newcommand{\mar}[1]{\todo[color=blue!40]{#1}}
\newcommand{\raf}[1]{\todo[color=olive!40]{#1}}

%\linespread{1.5}
\newcommand{\indep}{\!\perp \!\!\! \perp\!}


\setlength{\parindent}{10pt}
\setlength{\parskip}{1pt}


%language
%\usepackage{times}
\usepackage{mathptmx}
\usepackage[scaled=0.86]{helvet}
\usepackage{t1enc}
%\usepackage[utf8x]{inputenc}
%\usepackage[polish]{babel}
%\usepackage{polski}




%AMS
\usepackage{amsfonts}
\usepackage{amssymb}
\usepackage{amsthm}
\usepackage{amsmath}
\usepackage{mathtools}

\usepackage{geometry}
 \geometry{a4paper,left=35mm,top=20mm,}


%environments
\newtheorem{fact}{Fact}


% allow page breaks in equations
\allowdisplaybreaks


%abbreviations
\newcommand{\ra}{\rangle}
\newcommand{\la}{\langle}
\newcommand{\n}{\neg}
\newcommand{\et}{\wedge}
\newcommand{\jt}{\rightarrow}
\newcommand{\ko}[1]{\forall  #1\,}
\newcommand{\ro}{\leftrightarrow}
\newcommand{\exi}[1]{\exists\, {_{#1}}}
\newcommand{\pr}[1]{\ensuremath{\mathsf{P}(#1)}}
\newcommand{\ppr}[2]{\ensuremath{\mathsf{P}^{#1}(#2)}}
\newcommand{\cost}{\mathsf{cost}}
\newcommand{\benefit}{\mathsf{benefit}}
\newcommand{\ut}{\mathsf{ut}}

\newcommand{\odds}{\mathsf{Odds}}
\newcommand{\ind}{\mathsf{Ind}}
\newcommand{\nf}[2]{\nicefrac{#1\,}{#2}}
\newcommand{\R}[1]{\texttt{#1}}
\newcommand{\prr}[1]{\mbox{$\mathtt{P}_{prior}(#1)$}}
\newcommand{\prp}[1]{\mbox{$\mathtt{P}_{posterior}(#1)$}}



\newtheorem{q}{\color{blue}Question}
\newtheorem{lemma}{Lemma}
\newtheorem{theorem}{Theorem}
\newtheorem{corollary}{Corollary}[fact]


%technical intermezzo
%---------------------

\newcommand{\intermezzoa}{
	\begin{minipage}[c]{13cm}
	\begin{center}\rule{10cm}{0.4pt}



	\tiny{\sc Optional Content Starts}
	
	\vspace{-1mm}
	
	\rule{10cm}{0.4pt}\end{center}
	\end{minipage}\nopagebreak 
	}


\newcommand{\intermezzob}{\nopagebreak 
	\begin{minipage}[c]{13cm}
	\begin{center}\rule{10cm}{0.4pt}

	\tiny{\sc Optional Content Ends}
	
	\vspace{-1mm}
	
	\rule{10cm}{0.4pt}\end{center}
	\end{minipage}
	}
	
	
%--------------------






















\newtheorem*{reply*}{Reply}
\usepackage{enumitem}
\newcommand{\question}[1]{\begin{enumerate}[resume,leftmargin=0cm,labelsep=0cm,align=left]
\item #1
\end{enumerate}}

\usepackage{float}

% \setbeamertemplate{blocks}[rounded][shadow=true]
% \setbeamertemplate{itemize items}[ball]
% \AtBeginPart{}
% \AtBeginSection{}
% \AtBeginSubsection{}
% \AtBeginSubsubsection{}
% \setlength{\emergencystretch}{0em}
% \setlength{\parskip}{0pt}






\usepackage[authoryear]{natbib}

%\bibliographystyle{apalike}



\usepackage{tikz}
\usetikzlibrary{positioning,shapes,arrows}

\ifluatex
  \usepackage{selnolig}  % disable illegal ligatures
\fi

\title{Reply to Steeel and Stefansson}
\author{Marcello/Rafal}
\date{}

\begin{document}
\maketitle

\hypertarget{introduction}{%
\section{Introduction}\label{introduction}}

Learning is modeled in the Bayesian framework by the rule of
conditionalization. This rule posits that the agent's new degree of
belief in a proposition \(A\) after a learning experience \(E\) should
be the same as the agent's old degree of belief in \(A\) conditional on
\(E\). That is, \[\ppr{E}{A}=\pr{A \vert E},\] where \(\pr{}\)
represents the agent's old degree of belief (before the learning
experience) and \(\ppr{E}{}\) represents the agent's new degree of
belief (after learning about \(E\)).

The assumption here is that \(E\) is learned with certainty. After the
agent learns about \(E\), there is no longer any uncertainty about the
truth of \(E\). This assumption has been the target of extensive
discussion. As is well-known, Jeffrey's conditionalization relaxes this
assumption. The other assumption---and this is what we will focus
on---is that \(E\) and \(A\) are propositions that belong to the agent's
algebra of propositions. This algebra models what what the agent is
aware of and entertains as live possibilities.

The algebra---the agent's awareness state---is fixed once and for all.
The learning experience does not modify it. In this model, the agent
cannot learn something they have never thought about. And what the agent
learns cannot have any impact on the degree of belief about propositions
that the agent never thought about. This picture forces a great deal of
rigidity on the learning process. It commits the agent to the
specification of their `total possible future experience' (Howson 1976,
The Development of Logical Probability), as though learning was confined
to an `initial prison' (Lakatos, 1968, Changes in the Problem of
Inductive Logic).

But, arguably, the learning process is more complex than what
conditionalization allows. Not only do we learn that some propositions
that we were actively entertaining are true or false, but we may also
learn new propositions that we did not entertain before. Or we may
entertain new propositions---without necessarily learning that they are
true or false---and this change in awareness may in turn change what we
already believe. How should this more complex learning process be
modeled by Bayesianism? Call this the problem of awareness growth.

Critics of Bayesianism and sympathizers alike have been discussing the
problem of awareness growth under different names for quite some time,
at least since the eighties. This problem arises in a number of
different contexts, for example, new scientific theories (Glymour, 1980,
Why I am not a Bayesian; Chihara 1987, Some Problems for Bayesian
Confirmation Theory; Earmann 1992, Bayes of Bust?), language changes and
paradigm shifts (Williamson 2003, Bayesianism and Language Change), and
theories of induction (Zabell, Predicting the Unpredictable).

Now, of course, the algebra of propositions could in principle be so
rich to contain anything that could possibly be conceived, expressed,
thought of. Such an algebra would not need to change at any point in the
future. God-like agents could be associated with such rich algebra of
propositions, but this is hardly a plausible model of ordinary agents
with bounded resources such as ourselves. A fully comprehensive algebra
of propositions cannot be the answer here.

A more promising proposal is Reverse Bayesianism (Karni and Viero, 2015,
Probabilistic Sophistication and Reverse Bayesianism; Wenmackers and
Romeijn 2016, New Theory About Old Evidence; Bradely 2017, Decision
Theory with A Human Face) . The idea is to model awareness growth as a
change in the algebra while ensuring that the probabilities of the
propositions shared between the old and new algebra remain fixed (under
suitable constraints). Let \(\mathcal{F}\) be the initial algebra of
propositions and let \(\mathcal{F}^+\) the algebra after the agent's
awareness has grown. For reason that will soon become clear, let's pick
out subsets of these algebras which contain only basic propositions,
those that do not contain connectives such as negations, conjunctions or
disjunctions. Call these subsets \(X\) and \(X^+\) respectively.
Obviously, \(\mathcal{F}\subseteq \mathcal{F}^+\) and
\(X\subseteq X^+\). Reverse Bayesianism posits that the ratio of
probabilities for any propositions \(A\) and \(B\) in \(X\)---the basic
propositions shared by the old and new algebra---remain constant through
the process of awareness growth:
\[\frac{\pr{A}}{\pr{B}} = \frac{\ppr{+}{A}}{\ppr{+}{B}},\] where
\(\pr{}\) represents the agent's degree of belief before awareness
growth and \(\ppr{+}{}\) represents the agent's degree of belief after
awareness growth.

What is the justification for Reverse Bayesianism? Perhaps the best
justification is pragmatic. As an agent's awareness grows, the agent
might not want to throw away completely the epistemic work they have
done so far. The agent may prefer to retain as much of their old
assignments of degrees of beliefs as possible. Reverse Bayesian provides
a simple recipe to do that. It also coheres with the conservative spirit
of Bayesian conditionalization. Bayesian conditionalization preserves
the old probability distribution conditional on what has been learned.
Reverse Bayesianism preserves, instead, the old probability distribution
conditional on the old awareness state. \textbf{SHOW PICTURE BELOW}.

Reverse Bayesianism is a simple and elegant theory that manages to cope
with a seemingly intractable problem for Bayesianism. Unfortunately,
Steele and Stefansson (2021, Belief Revision for Growing Awareness) have
provided a few compelling counterexamples. We believe their examples are
ultimately successful, but they are liable to the objection that they
are not genuine example of awareness growth. To block this objection, we
provide simpler and more straightforward counterexamples. In addition,
we believe that Steele and Stefansson's conclusion is too broad and
overly pessimistic. They grant that Reverse Bayesianism, when suitably
formulated, can work in a limited class of cases, what they call cases
of \textit{awareness expansion}. But they claim it cannot work in cases
of \textit{awareness refinement}. We will return to this distinction in
due course. We agree only partly.

Much of the literature on awareness growth is concerned with a formal,
algorithmic solution to the problem. Steele and Stefansson's argument
suggests that a formal, algorithmic solution cannot be found, at least
for cases of awareness refinement. We agree with this. At the same time,
we think that awareness grows while holding fixed certain material
structural assumptions, based on commonsense, semantic stipulations or
causal dependency. To model awareness growth, we need a formalism that
can model these material structural assumptions. We sketch how this can
done using Bayesian networks. The resulting formalism will also justify
when Reverse Bayesianism hold and when it does not. The distinction
between refinement and expansion that Steele and Stefansson draw, albeit
a good first approximation, is too coarse and should be made more
precise. We will see that there are cases of refinement in which Reverse
Bayesianism (or a suitable variation of it) can be made to work.

\hypertarget{counterexamples}{%
\section{Counterexamples?}\label{counterexamples}}

The first counterexample Steele and Stefansson present is this:

\begin{quote}
Suppose you happen to see your partner enter your best friend's house on
an evening when your partner had told you she would have to work late.
At that point, you become convinced that your partner and best friend
are having an affair, as opposed to their being warm friends or mere
acquaintances. You discuss your suspicion with another friend of yours,
who points out that perhaps they were meeting to plan a surprise party
to celebrate your upcoming birthday---a possibility that you had not
even entertained. Becoming aware of this possible explanation for your
partner's behaviour makes you doubt that she is having an affair with
your friend, relative, for instance, to their being warm friends.
(Steele and Styefansson, 2021, Section 5, Example 2)
\end{quote}

Initially, the algebra only contains the hypotheses `my partner and my
best friend met to have an affair' (\textit{Affair}) and `my partner and
my best friend met as friends or acquaintances'
(\textit{Friends/acquaintances}). The other proposition in the algebra
is that your partner and your best friend met one night secretively and
without telling you (\textit{Secretive}). There may be other
propositions in the algebra, but these are the ones to focus on.

In light of the evidence \textit{Secretive}, hypothesis \textit{Affair}
is more plausible than hypothesis \textit{Friends/acquaintances}:
\[\pr{ \textit{Secretive} \vert \textit{Affair}}> \pr{\textit{Secretive} \vert \textit{Friends/acquaintances}},\]
from which it also follows that \textit{Affair} is more probable than
\textit{Friends/acquaintances}
\[\pr{\textit{Affair} \vert  \textit{Secretive} }> \pr{\textit{Friends/acquaintances} \vert \textit{Secretive}}, \tag{>}\]
so long as the prior probabilities of the two hypotheses are not skewed
in one direction.\footnote{If you were initially nearly certain your
  partner could not possisbly have an affair, even the fact they behaved
  very secretively or lied to your to meet one of your friends might not
  affect the probability of the two hypotheses. But this is besides the
  point.}

Next, the algebra changes. A new hypothesis is added which had not been
considered before: your partner and your best friends met to plan a
surprise party for your upcoming birthday (\textit{Surprise}). This is a
game changer. The evidence \textit{Secretive} now makes better sense in
light of this new hypothesis than any of the other two hypotheses:
\[\ppr{+}{ \textit{Secretive} \vert \textit{Surprise}}> \ppr{+}{\textit{Secretive} \vert \textit{Friends/acquaintances}}\]
\[\ppr{+}{ \textit{Secretive} \vert \textit{Surprise}}> \ppr{+}{\textit{Secretive} \vert \textit{Affair}}.\]
And, all things considered, this new hypothesis should be more likely
than any of the other two:
\[\ppr{+}{ \textit{Surprise} \vert \textit{Secretive}}> \ppr{+}{ \textit{Friends/acquaintances} \vert \textit{Secretive} }\]
\[\ppr{+}{ \textit{Surprise} \vert \textit{Secretive}}> \ppr{+}{ \textit{Affair} \vert \textit{Secretive}}. \tag{*}\]

So far so good. Reverse Bayesianism is not yet in trouble. Steele and
Stefansson, however, concludes that the probability of
\textit{Friends/acquaintances} should now exceed that of \textit{Affair}
(`Becoming aware of this possible explanation for your partner's
behaviour makes you doubt that she is having an affair with your friend,
relative, for instance, to their being warm friends.'):
\[\ppr{+}{\textit{Affair} \vert  \textit{Secretive} } < \ppr{+}{\textit{Friends/acquaintances} \vert \textit{Secretive}}. \tag{<}\]
Arguably, this holds because \textit{Surprise} implies
\textit{Friends/acquaintances}. In order to prepare a surprise party,
your partner and best friend have to be at least acquaintances. And
given that one implies the other, if \textit{Surprise} is more likely
than \textit{Affair} (by \(*\)), then \textit{Friends/acquaintances}
must also be more likely than \textit{Affair}. And if both (\(>\)) and
(\(<\)) holds, the ratio of the probabilities of basic propositions is
not fixed before and after the episode of awareness growth. This is a
violation of reverse Bayesianism.

But, as Steele and Stefansson's conced, Reverse Bayesianism is not
really in trouble here. It can still be made to work by replacing it
with a slightly different---though quite similar in spirit---condition,
called Awareness Rigidity: \[\ppr{+}{A \vert T^*}=\pr{A},\] where
\(T^*\) corresponds to a proposition that picks out the entire
possibility space before the episode of awareness growth. In our running
example, the proposition \(\neg\textit{Surprise}\) picks out the entire
possibility space before the episode of awareness growth. So Awareness
Rigidity would require that:
\[\ppr{+}{\textit{Friends/acquaintances} \vert \neg\textit{Surprise}}=\pr{\textit{Friends/acquaintances}}.\]
Conditional on \(\neg\textit{Surprise}\), it is indeed true that the
probability of \textit{Friends/acquaintances} has not changed before and
after the episode of awareness growth. And it is also true that
\textit{Affair} remains the most likely hypothesis in light of the
evidence (again conditional on \(\neg\textit{Surprise}\)):
\[\ppr{+}{\textit{Affair} \vert  \textit{Secretive} \& \neg\textit{Surprise} } > \ppr{+}{\textit{Friends/acquaintances} \vert \textit{Secretive} \& \neg\textit{Surprise}}. \tag{$>^+$}\]
So Awareness Rigidity is vindicated in this example. Reverse
Bayesianism---at least the spirit of it, not the letter---can be
salvaged.

Steele and Stefansson offer what they take to be a more definitive
counterexample to Reverse Bayesianism (even in the form of Awareness
Rigidity):

\begin{quote}
Suppose you are deciding whether to see a movie at your local cinema.
You know that the movie's predominant language and genre will affect
your viewing experience. The possible languages you consider are French
and German and the genres you consider are thriller and comedy. But then
you realise that, due to your poor French and German skills, your
enjoyment of the movie will also depend on the level of difficulty of
the language. Since it occurs to you that the owner of the cinema is
quite simple-minded, you are, after this realisation, much more
confident that the movie will have low-level language than high-level
language. Moreover, since you associate low-level language with
thrillers, this makes you more confident than you were before that the
movie on offer is a thriller as opposed to a comedy. (Steele and
Styefansson, 2021, Section 5, Example 3)
\end{quote}

Admittedly, this example is quite intricate. For analytic clarity, it
can be split into two episodes of awareness growth. The first episode of
awareness growth involved considering the language difficulty of the
movie, as a new variable besides language and genre. Initially, the
algebra contained the propositions \textit{French} and
\textbackslash textit\{\{German\}, as well as \textit{Thriller} and
\textit{Comedy}. Then, you realize another variable might be at play,
namely the level of difficulty of the language of the movie,
\textit{Difficult} and \textit{Easy}. This is a case of refinement
because, first, you categorized movies by just language and genre, and
then you added a further category, level of difficulty. But this
refinement does not seem to bring about any change in the probabilities.
There is no obvious reason why that should be so.

Next, you become aware and learn about something you did consider
before, namely that the owner is simple minded. This learning triggers a
change in degrees of belief that propagates to the genre of the movie
via the language difficulty of the movie. The change in degrees of
belief is triggered by the realization that the owner is simple-minded,
which suggests a low level of language difficulty of the movie. The
latter in turn suggests that the movie is more likely going to be a
thriller rather than a comedy (possibly because thrillers are
simpler---linguistically---than comedies).

Taken at face value, this example is a challenge to Reverse Bayesianism
and Awareness Rigidity. It is not true, for example, that
\(\frac{\pr{\textit{Thriller}}}{\pr{\textit{Comedy}}}=\frac{\ppr{+}{\textit{Thriller}}}{\ppr{+}{\textit{Comedy}}}\)
(against Reverse Bayesianism) nor is it true that
\(\pr{\textit{Thriller}}=\ppr{+}{\textit{Thriller} \vert \textit{Thriller}\vee \textit{Comedy}}\)
(against Awareness Rigidity). Since this is a case of refinement, the
entire possibility space is fixed before and after awareness growth.

But this counterexample is likely to leave many unconvinced, or least
confused. Since it consists of two parts---first the retirement by
language difficulty and second the learning that the wonder is simple
minded--- one is left wondering which one of the two is essential for
the counterexample? Are both necessary? Is only the second part really
necessary while the first just give added context? What is going on
here, exactly? Can we distill a simpler, more straightforward
counterexample that only involves awareness growth without an episode of
learning intertwined with it? For conceptual clarity, we should aspire
to a neater and cleaner picture of awareness growth in cases of
refinement.

The need for a conceptual clearer picture also applies to the first
counterexample. There remains---we think---the need to further examine
cases of awareness expansion. These cases consist in the addition of
another proposition that was not previously in the algebra, but that was
not a refinement of existing propositions. The addition of the
hypothesis \textit{Surprise} is, however, an ambiguous case. For one
thing, \textit{Surprise} is a novel hypothesis that cannot be subsumed
under \textit{Friends/acquaintances} or \textit{Affair}. On the other,
\textit{Surprise} seems a refinement of \textit{Friends/acquaintances},
since a meeting for planning a surprise in a more specific way to
describe a meeting as friends. A more clear-cut case of awareness
expansion would be the following. The police is investing a criminal
cases. There are two suspects under investigation: Joe and Sue. They
both had a motive. The evidence consists in a DNA match and information
about how the crime was committed. Sue genetically matches the traces,
but she is quite short and the perpetrator is known to be a tall person.
Joe is neither tall nor does he genetically match the crime traces. In
light of the evidence, Sue seems more likely the culprit than Joe, but
matters are still open ended. Then, a new hypothesis is considered: Ela
could be the perpetrator. As it turns out, Ela genetically matches the
traces, is tall enough to have committed the crime, and does have a
motive. This seems a straightforward case of expansion because Ela, Sue
and Joe are incompatible hypotheses, while
\textit{Friends/acquaintances} and \textit{Surprise} need not be. Any
model of awareness growth should be able to analyze more precisely the
difference between the exmaple provided by Steele and Stefansson's the
criminal case just outlined. They are both, arguably, cases of
expansion, but they are also different. Does this matter for modelling
awareness growth?

\hypertarget{bayesian-networks}{%
\section{Bayesian Networks}\label{bayesian-networks}}

\end{document}
