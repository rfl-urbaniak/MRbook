% Options for packages loaded elsewhere
\PassOptionsToPackage{unicode}{hyperref}
\PassOptionsToPackage{hyphens}{url}
\PassOptionsToPackage{dvipsnames,svgnames*,x11names*}{xcolor}
%
\documentclass[
  11pt,
  dvipsnames,enabledeprecatedfontcommands]{scrartcl}
\usepackage{amsmath,amssymb}
\usepackage{lmodern}
\usepackage{ifxetex,ifluatex}
\ifnum 0\ifxetex 1\fi\ifluatex 1\fi=0 % if pdftex
  \usepackage[T1]{fontenc}
  \usepackage[utf8]{inputenc}
  \usepackage{textcomp} % provide euro and other symbols
\else % if luatex or xetex
  \usepackage{unicode-math}
  \defaultfontfeatures{Scale=MatchLowercase}
  \defaultfontfeatures[\rmfamily]{Ligatures=TeX,Scale=1}
\fi
% Use upquote if available, for straight quotes in verbatim environments
\IfFileExists{upquote.sty}{\usepackage{upquote}}{}
\IfFileExists{microtype.sty}{% use microtype if available
  \usepackage[]{microtype}
  \UseMicrotypeSet[protrusion]{basicmath} % disable protrusion for tt fonts
}{}
\usepackage{xcolor}
\IfFileExists{xurl.sty}{\usepackage{xurl}}{} % add URL line breaks if available
\IfFileExists{bookmark.sty}{\usepackage{bookmark}}{\usepackage{hyperref}}
\hypersetup{
  pdftitle={Probability, Specificity, Accuracy, Fairness},
  pdfauthor={Marcello/Rafal},
  colorlinks=true,
  linkcolor=Maroon,
  filecolor=Maroon,
  citecolor=Blue,
  urlcolor=blue,
  pdfcreator={LaTeX via pandoc}}
\urlstyle{same} % disable monospaced font for URLs
\usepackage{graphicx}
\makeatletter
\def\maxwidth{\ifdim\Gin@nat@width>\linewidth\linewidth\else\Gin@nat@width\fi}
\def\maxheight{\ifdim\Gin@nat@height>\textheight\textheight\else\Gin@nat@height\fi}
\makeatother
% Scale images if necessary, so that they will not overflow the page
% margins by default, and it is still possible to overwrite the defaults
% using explicit options in \includegraphics[width, height, ...]{}
\setkeys{Gin}{width=\maxwidth,height=\maxheight,keepaspectratio}
% Set default figure placement to htbp
\makeatletter
\def\fps@figure{htbp}
\makeatother
\setlength{\emergencystretch}{3em} % prevent overfull lines
\providecommand{\tightlist}{%
  \setlength{\itemsep}{0pt}\setlength{\parskip}{0pt}}
\setcounter{secnumdepth}{5}
%\documentclass{article}

% %packages
\usepackage{booktabs}
%\usepackage[left]{showlabels}
\usepackage{multirow}
\usepackage{subcaption}
\usepackage{wrapfig}
\usepackage{graphicx}
\usepackage{longtable}
\usepackage{ragged2e}
\usepackage{etex}
%\usepackage{yfonts}
\usepackage{marvosym}
\usepackage[notextcomp]{kpfonts}
\usepackage{nicefrac}
\newcommand*{\QED}{\hfill \footnotesize {\sc Q.e.d.}}
\usepackage{floatrow}
\usepackage{multicol}

\usepackage[textsize=footnotesize]{todonotes}
\newcommand{\ali}[1]{\todo[color=gray!40]{\textbf{Alicja:} #1}}
\newcommand{\mar}[1]{\todo[color=blue!40]{#1}}
\newcommand{\raf}[1]{\todo[color=olive!40]{#1}}

%\linespread{1.5}
\newcommand{\indep}{\!\perp \!\!\! \perp\!}


\setlength{\parindent}{10pt}
\setlength{\parskip}{1pt}


%language
%\usepackage{times}
\usepackage{mathptmx}
\usepackage[scaled=0.86]{helvet}
\usepackage{t1enc}
%\usepackage[utf8x]{inputenc}
%\usepackage[polish]{babel}
%\usepackage{polski}




%AMS
\usepackage{amsfonts}
\usepackage{amssymb}
\usepackage{amsthm}
\usepackage{amsmath}
\usepackage{mathtools}

\usepackage{geometry}
 \geometry{a4paper,left=35mm,top=20mm,}


%environments
\newtheorem{fact}{Fact}


% allow page breaks in equations
\allowdisplaybreaks


%abbreviations
\newcommand{\ra}{\rangle}
\newcommand{\la}{\langle}
\newcommand{\n}{\neg}
\newcommand{\et}{\wedge}
\newcommand{\jt}{\rightarrow}
\newcommand{\ko}[1]{\forall  #1\,}
\newcommand{\ro}{\leftrightarrow}
\newcommand{\exi}[1]{\exists\, {_{#1}}}
\newcommand{\pr}[1]{\ensuremath{\mathsf{P}(#1)}}
\newcommand{\cost}{\mathsf{cost}}
\newcommand{\benefit}{\mathsf{benefit}}
\newcommand{\ut}{\mathsf{ut}}

\newcommand{\odds}{\mathsf{Odds}}
\newcommand{\ind}{\mathsf{Ind}}
\newcommand{\nf}[2]{\nicefrac{#1\,}{#2}}
\newcommand{\R}[1]{\texttt{#1}}
\newcommand{\prr}[1]{\mbox{$\mathtt{P}_{prior}(#1)$}}
\newcommand{\prp}[1]{\mbox{$\mathtt{P}_{posterior}(#1)$}}



\newtheorem{q}{\color{blue}Question}
\newtheorem{lemma}{Lemma}
\newtheorem{theorem}{Theorem}
\newtheorem{corollary}{Corollary}[fact]


%technical intermezzo
%---------------------

\newcommand{\intermezzoa}{
	\begin{minipage}[c]{13cm}
	\begin{center}\rule{10cm}{0.4pt}



	\tiny{\sc Optional Content Starts}
	
	\vspace{-1mm}
	
	\rule{10cm}{0.4pt}\end{center}
	\end{minipage}\nopagebreak 
	}


\newcommand{\intermezzob}{\nopagebreak 
	\begin{minipage}[c]{13cm}
	\begin{center}\rule{10cm}{0.4pt}

	\tiny{\sc Optional Content Ends}
	
	\vspace{-1mm}
	
	\rule{10cm}{0.4pt}\end{center}
	\end{minipage}
	}
	
	
%--------------------






















\newtheorem*{reply*}{Reply}
\usepackage{enumitem}
\newcommand{\question}[1]{\begin{enumerate}[resume,leftmargin=0cm,labelsep=0cm,align=left]
\item #1
\end{enumerate}}

\usepackage{float}

% \setbeamertemplate{blocks}[rounded][shadow=true]
% \setbeamertemplate{itemize items}[ball]
% \AtBeginPart{}
% \AtBeginSection{}
% \AtBeginSubsection{}
% \AtBeginSubsubsection{}
% \setlength{\emergencystretch}{0em}
% \setlength{\parskip}{0pt}






\usepackage[authoryear]{natbib}

%\bibliographystyle{apalike}



\usepackage{tikz}
\usetikzlibrary{positioning,shapes,arrows}

\ifluatex
  \usepackage{selnolig}  % disable illegal ligatures
\fi

\title{Probability, Specificity, Accuracy, Fairness}
\author{Marcello/Rafal}
\date{}

\begin{document}
\maketitle

\hypertarget{the-problem}{%
\section{The Problem}\label{the-problem}}

Does aiming to offer more specific stories (theories, accounts,
narratives, explanations), as opposed to merely highly probable stories,
lead to more accurate and more fair decisions? This is a generic
question and needs to be made precise. It is inspired by a remark made
by Popper about science:

\begin{quote}
Science does not aim, primarily, at high probabilities. It aims at a high informative content, well backed by experience. But a hypothesis may be very probable simply because it tells us nothing, or very little. A high degree of probability is therefore not an indication of goodness. (Popper, 2002, p. 416, Appendix *IX)
\end{quote}

Specificity here is defined as informative content. There is an inverse
correlation between informativeness and probability. Other things being
equal, the more probable a statement, the less informative the
statement. A disjuction \(A\vee B\) is less informative than a single
disjunct, but the disjunction is more probable than the single disjunct.

So how can a policy of adopting more informative (more specific)
theories (stories, narratives, etc.) lead to better accuracy? It would
seem to be the opposite---that someone who adopts less informative
theories is more likely to be right, other things being equal, since
informativeness and probability are inversely proportional.

But this depends on how we define accuracy. Popper's intuition must be
understood against the background fact that theories are challenged,
scrutinized, falsified. So, the claim is that, between a specific theory
that has been challenged and survived challenges and a non-specific
theory that has been challenged and survived challenges, the specific
theory is more likely to be right, more credible, or something to that
effect.

This may be right. But how do we show that it is right?

\hypertarget{simulating-specificity-and-accuracy}{%
\section{Simulating specificity and
accuracy}\label{simulating-specificity-and-accuracy}}

To test Popper's intuition we can use a computer simulation.

\hypertarget{basic-set-up}{%
\subsection{Basic set up}\label{basic-set-up}}

\begin{itemize}
\item
  Suppose we have a plane consisting of 1000 points (or more). A
  maximally informative description of how the world is like is an
  assignment of 0's and 1's to each of these points. This is the
  maximally informative or maximally specific theory.
\item
  Less informative theories will leave undecided assignments of 0's and
  1's to certain points. The more undecided points, the less informative
  the theory. The theory that leaves all points undecided is the least
  informative, corresponding to a tautology.
\item
  The actual world is a maximally specific assignment of 0's and 1's to
  all the points. The goal in science and trial proceedings is to to
  find out what the actual world is like.
\item
  One side, the prosecutor, puts forward a theory---a certain assignment
  of 0's and 1's to the points. The theory need not be maximally
  informative. In fact, it never is, but it does have some degree of
  informativeness.
\item
  The other side, the defense, will proceed to attack this theory. How
  does one do that? We can imagine that the defense has available
  resources to seek evidence. Evidence simply consists in information
  that unravels which points are 0's and which points are 1's. Evidence
  is partial and fragmentary and reveals only some points. Let's assume
  for now the evidence isn't probabilistic. It unequivocally and
  truthfully tells us which points are 0's and which are 1's with no
  error whatsoever. Let's also leave out misleading evidence. All these
  complications will have to be addressed, but one thing at a time! The
  difficulty is to find out such good, truth-conducive evidence.
\item
  Evidence that conflicts with a proposed theory falsifies the theory,
  and evidence that does not, corroborates it. Comparing bits of
  evidence with a proposed theory is the adversarial testing that goes
  on at trial. This process is limited by available resources and time.
  It cannot go on forever. It will end at some point.
\item
  We can think of the following rule of decision: if a theory is
  falsified, reject it; and if the theory is not falsified (within a
  period of time), accept the theory.
\end{itemize}

\hypertarget{questios-reformulated}{%
\subsection{Questios reformulated}\label{questios-reformulated}}

\begin{itemize}
\item
  So the question can now be stated more precisely: Suppose Policy A
  imposes that only very informative theories be discussed at trial and
  contrast this with Policy B that allows less informative theories. Do
  trial systems that adopt Policy A end up accepting theories that are
  false more or less often than systems that adopt Policy B? And what
  about rejection of true theories?
\item
  Suppose a more specific theory has resisted all challenges and so did
  a less specific theory. Do we have good reason to believe one more
  firmly than the other? It would seem that we have better reason to
  believe the more specific theory, but why exactly? Would accepting the
  more specific theory lead to accepting a true theory more often than
  accepting the less specific theory?
\item
  Several scenarios to compare. The proposed theory is true, versus the
  proposed theory is false. Is a more specific theory (that is false)
  more quickly be proven false (=rejected) then a less specific theory
  (that is false)? Perhaps so. What about a more more specific theory
  (that is true) compared to a less specific theory (that is also true)?
\item
  The answer to these questions may depend on time constraints. If there
  were not time constraints, perhaps the two policies would be
  indistinguishable. Is this right? Maybe there is an interesting
  relationships between time, specificity and accuracy.
\end{itemize}

\hypertarget{possisble-answers}{%
\subsection{Possisble answers}\label{possisble-answers}}

\begin{itemize}
\item
  Give limited time, it would seem that, under this simple model, the
  more specific the theory being tested, the more likely it will be
  proven wrong (=rejected) if the theory is actually false. For imagine
  that the theory is maximally informative and it is wrong. Revealing
  one single piece of evidence that conflicts with the theory could be
  enough to prove the theory wrong.
\item
  Anyway these intuitions should be substantiated by the simulation. I
  am already getting confused as to what exactly is going to happen.
\end{itemize}

\hypertarget{complications}{%
\subsection{Complications}\label{complications}}

\begin{itemize}
\item
  So we can ask many questions about the credibility of more or less
  specific theories in the simple set up already. The next step would be
  to introduce the possibility of misleading evidence, or the allocation
  of uneven resources between prosecutor and defense. Do these
  complexities change the general picture?
\item
  The goal here would be to get to a progressively more complex and
  realistic picture of legal decision-making.
\end{itemize}

\hypertarget{specificity-and-fairness}{%
\section{Specificity and fairness}\label{specificity-and-fairness}}

\begin{itemize}
\item
  Is there a relationship between fair decisions and specificity?
  Perhaps one way to tackle this is to compare trials in which innocent
  defendants have available less resources or less time. This presumably
  has a negative impact on accuracy to their detriment. This can be
  checked with the simulation. Seems obvious but good to check!
\item
  In terms of specificity and fairness, one thing to check could be
  whether more specificity can reduce the gap in accuracy between trials
  in which defendants have more resources and more time compared to
  trials in which defendants have less resources and time? In other
  words, is specificity a decently good corrective measure to compare
  for gaps in accuracy between defendants with more and less resources
  and time?
\end{itemize}

\hypertarget{probability-calibration-specificity-and-accuracy}{%
\section{Probability, calibration, specificity and
accuracy}\label{probability-calibration-specificity-and-accuracy}}

\begin{itemize}
\item
  One question that we leave out is whether using probability theory,
  Bayesian networks, etc, has benefits for accuracy. How can this claim
  being tested? It is not obvious.
\item
  The connection might be that probability theory helps to arrive at
  well-calibrated judgments about the probability (credibility) of
  theories. One strategy could be to see what happens if one's
  probability assignments are mis-calibrated. What happens if the
  fact-finders are over-confident or under-confident in their judgments
  about the credibility of a proposed theory? Does mis-calibration
  hamper accuracy? And if so--it seem obvious that it does---how
  exactly?
\item
  A connection with the specificity bit is that, perhaps, aiming to test
  more specific theories is more likely to lead to calibrated judgments
  about the credibility (probability) of these theories compared to less
  specific theories. Is this right? How do we model that?
\end{itemize}

\end{document}
