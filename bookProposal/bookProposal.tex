\PassOptionsToPackage{unicode=true}{hyperref} % options for packages loaded elsewhere
\PassOptionsToPackage{hyphens}{url}
\PassOptionsToPackage{dvipsnames,svgnames*,x11names*}{xcolor}
%
\documentclass[10pt,dvipsnames]{scrartcl}
\usepackage{lmodern}
\usepackage{amssymb,amsmath}
\usepackage{ifxetex,ifluatex}
\usepackage{fixltx2e} % provides \textsubscript
\ifnum 0\ifxetex 1\fi\ifluatex 1\fi=0 % if pdftex
  \usepackage[T1]{fontenc}
  \usepackage[utf8]{inputenc}
  \usepackage{textcomp} % provides euro and other symbols
\else % if luatex or xelatex
  \usepackage{unicode-math}
  \defaultfontfeatures{Ligatures=TeX,Scale=MatchLowercase}
\fi
% use upquote if available, for straight quotes in verbatim environments
\IfFileExists{upquote.sty}{\usepackage{upquote}}{}
% use microtype if available
\IfFileExists{microtype.sty}{%
\usepackage[]{microtype}
\UseMicrotypeSet[protrusion]{basicmath} % disable protrusion for tt fonts
}{}
\IfFileExists{parskip.sty}{%
\usepackage{parskip}
}{% else
\setlength{\parindent}{0pt}
\setlength{\parskip}{6pt plus 2pt minus 1pt}
}
\usepackage{xcolor}
\usepackage{hyperref}
\hypersetup{
            pdftitle={Boosting Legal Probabilism (or Beyond Legal Probabilism 1.01)},
            pdfauthor={Marcello Di Bello and Rafal Urbaniak},
            colorlinks=true,
            linkcolor=Maroon,
            filecolor=Maroon,
            citecolor=Blue,
            urlcolor=blue,
            breaklinks=true}
\urlstyle{same}  % don't use monospace font for urls
\usepackage{graphicx,grffile}
\makeatletter
\def\maxwidth{\ifdim\Gin@nat@width>\linewidth\linewidth\else\Gin@nat@width\fi}
\def\maxheight{\ifdim\Gin@nat@height>\textheight\textheight\else\Gin@nat@height\fi}
\makeatother
% Scale images if necessary, so that they will not overflow the page
% margins by default, and it is still possible to overwrite the defaults
% using explicit options in \includegraphics[width, height, ...]{}
\setkeys{Gin}{width=\maxwidth,height=\maxheight,keepaspectratio}
\setlength{\emergencystretch}{3em}  % prevent overfull lines
\providecommand{\tightlist}{%
  \setlength{\itemsep}{0pt}\setlength{\parskip}{0pt}}
\setcounter{secnumdepth}{5}
% Redefines (sub)paragraphs to behave more like sections
\ifx\paragraph\undefined\else
\let\oldparagraph\paragraph
\renewcommand{\paragraph}[1]{\oldparagraph{#1}\mbox{}}
\fi
\ifx\subparagraph\undefined\else
\let\oldsubparagraph\subparagraph
\renewcommand{\subparagraph}[1]{\oldsubparagraph{#1}\mbox{}}
\fi

% set default figure placement to htbp
\makeatletter
\def\fps@figure{htbp}
\makeatother

%\documentclass{article}

% %packages
 \usepackage{booktabs}
 \usepackage[sort&compress]{natbib}
\usepackage{graphicx}
\usepackage{longtable}
\usepackage{ragged2e}
\usepackage{etex}
%\usepackage{yfonts}
\usepackage{marvosym}
\usepackage[notextcomp]{kpfonts}
\usepackage{nicefrac}
\newcommand*{\QED}{\hfill \footnotesize {\sc Q.e.d.}}
\usepackage{multicol} 
%\usepackage[textsize=footnotesize]{todonotes}
%\linespread{1.5}

\usepackage{xcolor}

\setlength{\parindent}{10pt}
\setlength{\parskip}{1pt}


%language
%\usepackage{times}
\usepackage{mathptmx}
\usepackage[scaled=0.88]{helvet}

\usepackage{t1enc}
%\usepackage[utf8x]{inputenc}
%\usepackage[polish]{babel}
%\usepackage{polski}

\usepackage{subcaption}


%AMS
\usepackage{amsfonts}
\usepackage{amssymb}
\usepackage{amsthm}
\usepackage{amsmath}

%\usepackage{geometry}
 %\geometry{a4paper,left=35mm,top=20mm,}

%abbreviations
\newcommand{\ra}{\rangle}
\newcommand{\la}{\langle}
\newcommand{\n}{\neg}
\newcommand{\et}{\wedge}
\newcommand{\jt}{\rightarrow}
\newcommand{\ko}[1]{\forall  #1\,}
\newcommand{\ro}{\leftrightarrow}
\newcommand{\exi}[1]{\exists\, {_{#1}}}
\newcommand{\pr}{\mathsf{P}}
\newcommand{\odds}{\mathsf{Odds}}
\newcommand{\ind}{\mathsf{Ind}}
\newcommand{\nf}[2]{\nicefrac{#1\,}{#2}}
\newcommand{\R}[1]{\texttt{#1}}

\newtheorem{q}{\color{blue}Question}


%% Rafal's stuff
\usepackage[margin=1.5in,top=1in]{geometry}
\usepackage[textsize=scriptsize, textwidth = 3cm]{todonotes}
% This is my own comment shading, keep \todo for general stuff, feel free to define your own separate comment colors (usually useful if you use margin comments at tall) For instance, I defined one for Patrycja
\newcommand{\rt}[1]{\todo[color = orange!40]{#1}}
\newcommand{\pt}[1]{\todo[color = blue!40]{#1}}




\newtheorem*{reply*}{Reply}
\usepackage{enumitem}
\newcommand{\question}[1]{\begin{enumerate}[resume,leftmargin=0cm,labelsep=0cm,align=left]
\item #1
\end{enumerate}}

\usepackage{float}

% \setbeamertemplate{blocks}[rounded][shadow=true]
% \setbeamertemplate{itemize items}[ball]
% \AtBeginPart{}
% \AtBeginSection{}
% \AtBeginSubsection{}
% \AtBeginSubsubsection{}
% \setlength{\emergencystretch}{0em}
% \setlength{\parskip}{0pt}

\title{Boosting Legal Probabilism (or Beyond Legal Probabilism 1.01)}
\author{Marcello Di Bello and Rafal Urbaniak}
\date{}

\begin{document}
\maketitle

\hypertarget{the-book}{%
\section{The Book}\label{the-book}}

\hypertarget{brief-description}{%
\subsection{Brief Description}\label{brief-description}}

\footnotesize In one or two paragraphs, describe the work, including its
rationale, approach, and pedagogy. (This book is\ldots{} It does\ldots{}
Its distinguishing features are\ldots{})

\normalsize

This book boosts legal probabilism to its limits. It begins with legal
probabilism 1.01, the simplest version of the theory, and examines why
it falls prey to several conceptual difficulties. The bulk of the book
develops legal probabilism 1.02, a more sophisticated version that
addresses many of the limitations of its simpler version. The book also
examines to what extent certain difficulties cannot be met becasue of
inherent limitations of legal robabilism.

Legal probabilism 1.01 was developed in the 70ies and 80ies although its
roots go back to the early days of probability theory to names such as
Bernoulli, Laplace, Condorcet. The 1.01 version comprises a familiar
repertoire: Bayes' theorem, likelihood ratios, reasoning fallacies,
probability thresholds, expected utility maximization. This repertoire
has proven useful in a number of ways, especially in the assessment of
explicitly quantitative evidence such as DNA matches and expert evidence
more generally. The assessment of nonquantitive evidence, such as
eyewitness testimony, and the aggregation of quantitive and
non-quantitive evidence have proven more challenging. Legal probabilism
1.01 is also liable to a host of conceptual difficulties: the
conjunction problem, the problem of priors, paradoxes of naked
statistical evidence. These difficulties are well-known and have
commanded the attention of philosophers and legal scholars. Other
difficulties are less familiar: the problem of complexity, the problem
of soft variables, the problem of corroboration. Familiar and unfamiliar
difficulties parallel objections that Bayesianism faces in contemporary
analytic epistemology: the lottery paradox, the preface paradox, the
problem of coherence, holism, second-order evidence. Legal probabilism
1.01 can hardly answer these challenges. Either we reject legal
probabilism altogether or we move past its 1.01 version. This book takes
the latter route.

A prototype of legal probabilism 1.02 already exists in the literature
in forensic science and artificial intelligence. Bayesian networks have
been added to the repertoire. Bayesian networks can formalize in
probabilistic language notions familiar in argumention theory such as
undercutting and rebuttign defeaters, as well as notions such as
plausibility and coherence familiar from the story of model of judicial
fact-finding. Once stories (narrations or more generall explanation of
the evidence) are represented as Bayesian networks, features of can be
explicated in terms of corresponding formal properties of the network.

The conceptual developments are accompanied by technical accounts.
\textbf{\textsf{R}} code capturing to the technical features developed
is made available to the reader.

A secondary goal is to present a unified introduction to multiple
positive contributions of legal probilism to the status of legal
evidence evaluation, as many such results are scattered over many
professional journals from different fields, and only by understanding
them (rather than focusing only on abstract philosophical thought
experiments) the reader can develop a proper assessment of the
situation.

\hypertarget{outline-new-version}{%
\subsection{Outline (new version)}\label{outline-new-version}}

\renewcommand{\labelenumi}{\Roman{enumi}}
\renewcommand{\labelenumii}{\arabic{enumii}}
\renewcommand{\labelenumiii}{\arabic{enumii}.\arabic{enumiii}}

\begin{enumerate}
\item Legal probabilism and its foes
\begin{enumerate}

  \item The emergence of legal probabilism
  \begin{enumerate}
  \item  Famous cases
  \item  Probabilistic evidence
  \item  Trial by mathematics
  \item  Some history
  \end{enumerate}
  
  
  
  
  
  
  

  \item  A skeptical perspective
  \begin{enumerate}
  \item The difficulty about conjunction
  \item The complexity objection
  \item  The problem of corroboration
  \item  The problem of artificial precision
  \item Naked statistical evidence
  \item  The problem of priors
  \item  The reference class problem
  \item  Non-probabilistic perspectives
  \end{enumerate}


\end{enumerate}
\item  Evidence assessment


\begin{enumerate}


\setcounter{enumii}{2}
  \item  Bayes' Theorem and the usual fallacies
  \begin{enumerate}
  \item  Assuming independence
  \item  The prosecutor's fallacy
  \item  Base rate fallacy
  \item  Defense attorney's fallacy
  \item  Uniqueness fallacy
  \item  Case studies
  \end{enumerate}

  
  
  \item  Complications and caveats
  \begin{enumerate}
  \item  Complex hypotheses and complex bodies of evidence
  \item Source, activity and offense level hypotheses
  \item  Where do the numbers come from?
  \item  Modeling corroboration
  \item  Stories, explanations and coherence
  \end{enumerate}

  
  \item  Likelihood Ratios and Relevance
  \begin{enumerate}
  \item Likelihood ratio as a measure of evidence strength
  \item The risk of false positive and its impact
  \item Hypothesis choice
  \item Levels of hypotheses and the two-stain problem
  \item Relevance and the small-town murder scenario
  \item The cold-hit confusion
  \item  Likelihood ratio and  cold-hit DNA matches
  \end{enumerate}



  \item  Bayesian Networks
  \begin{enumerate}
  \item  Bayesian networks to the rescue
  \item  Legal evidence idioms
  \item Scenario idioms
  \item Modeling relevance
  \item  Case study: Sally Clark
  \item DNA evidence
  \end{enumerate}
  
  \item Corroboration
  \begin{enumerate}
  \item Boole's formula and Cohen's challenge
  \item  Modeling substantial rise in case of agreement
  \item Ekel\"of's corroboration measure and evidentiary mechanisms
  \item General approach with multiple false stories and multiple witnesses
  \end{enumerate}

  \item Coherence
  \begin{enumerate}
  \item  Existing probabilistic coherence measures
  \item  An array of counterexamples
  \item Coherence of structured narrations with Bayesian networks
  \item  Application to legal cases
  \end{enumerate}

  \item  New legal probabilism
    \begin{enumerate}
    \item  Desiderata
    \item  A probabilistic framework for narrations
    \item  Probabilistic explications of the desiderata
    \item  Bayesian network implementation
    \end{enumerate}


\end{enumerate}
\item  Trial Decisions
\begin{enumerate}



\setcounter{enumii}{9}
  \item  The functions of the proof standards
  \begin{enumerate}
  \item  Conceptual desiderata
  \item  Protecting defendants
  \item  Error reduction and error distribution/allocation
  \item  Dispute resolution and public deference
  \item  Justification and answerability
  \end{enumerate}



  \item  Standards of proof
  \begin{enumerate}
  \item  Legal background
  \item  Probabilistic thresholds
  \item  Theoretical challenges
  \item  Specific narratives
  \item The comparative strategy
  \item  The likelihood strategy
  \item Challenges (again)
  \item Probabilistic thresholds revised
  \item  Bayesian networks and probabilistic standard of proof
  \end{enumerate}

  \item  Accuracy and the risk of error
  \begin{enumerate}
  \item  Minimizing expected costs
  \item  Minimizing expected errors
  \item  Expected v.\ actual errors
  \item  Competing accounts of the risk of error
  \item  Bayesian networks and the risk of error
  \end{enumerate}



  \item  Fairness in trial decisions
  \begin{enumerate}
  \item  Procedural v.\ substantive fairness
  \item  Competing measures of substantive fairness
  \item  Bayesian networks and fairnesss
  \end{enumerate}


  \item  Alternative accounts and legal probabilism
  \begin{enumerate}
  \item  Baconian probability
  \item  Relative Plausibility
  \item  Arguments
  \item  Sensitivity
  \item  Normic Support
  \item  Justification/foundherentism
  \item  Completeness
  \item  Relevant alternatives
  \item  Knowledge
  \end{enumerate}

\item Conclusions
\end{enumerate}
\end{enumerate}

\hypertarget{outstanding-features-of-the-book}{%
\subsection{Outstanding Features of the
Book}\label{outstanding-features-of-the-book}}

\begin{itemize}
\item
  (First) comprehensive sustained philosophical discussion of legal
  probabilism.
\item
  Multi-faceted in its incorporation of insights from various
  discussions present in legal, philosophical, and forensic research.
\item
  With a practical accent, due to the implementation of the conceptual
  points by means of bayesian networks and \textbf{\textsf{R}}
  programming language.
\end{itemize}

\todo{what else?}

\hypertarget{apparatus}{%
\subsection{Apparatus}\label{apparatus}}

\footnotesize a. Will the book include photographs, line drawings,
cases, questions, problems, glossaries, bibliography, references,
appendices, etc.?

\vspace{2mm}

\normalsize

Yes, the book will contain various plots, either of Bayesian networks,
or some other data visualisations generated by \texttt{ggplot2}. The
book also will contain bibliography. \vspace{2mm}

\footnotesize b. If the book is a text, do you plan to provide
supplementary material to accompany it? (Teacher's manual, study guide,
solutions, answers, workbook, anthology, or other material.)

\vspace{2mm}

\normalsize

The book will be accompanied by an online-only appendix detailing the
use of the \texttt{R} code in the book and the source code we used.

\hypertarget{competition}{%
\subsection{Competition}\label{competition}}

\footnotesize a. Consider the existing books in this field and discuss
specifically their strengths and weaknesses. Spell out how your book
will be similar to, as well as different from, competing works.

\todo{For now, let's list competition, and discuss key differences}

\normalsize

Three types: BNs in the law, Philosophy \& law, Statistics in law and
forensics

\begin{itemize}
\item
  ``Bayesian Networks and Probabilistic Inference in Forensic Science''
  by Taroni, Aitken, Garbolino and Biedermann.
\item
  ``Risk Assessment and Decision Analysis with Bayesian Networks'' by
  Fenton and Neil.
\item
  ``Bayesian Networks With Examples in R'' by Marco Scutari and
  Jean-Baptiste Denis.
\item
  Alex Stein, foundations of evidence law
\item
  Nance, Burdens of proof
\item
  Schauer, Profiles, \dots
\item
  Ho, Philosophy of evidence law
\item
  Robertson, Vignaux
\item
  Lucy Dawid,
\item
  Statistics for Lawyers etc.
\end{itemize}

\begin{enumerate}
\def\labelenumi{\alph{enumi}.}
\setcounter{enumi}{1}
\item
  Consider what aspects of topical coverage are similar to or different
  from the competition. What topics have been left out of competing
  books and what topics have been left out of yours?
\item
  Please discuss each competing book in a separate paragraph. (If
  possible, please provide us with the publisher and date of publication
  as well.) This information will provide the reviewers and the
  publisher a frame of reference for evaluating your material. Remember,
  you are writing for reviewers and not for publication, so be as frank
  as possible regarding your competition. Give credit where credit is
  due, and show how you can do it better.
\end{enumerate}

\hypertarget{market-considerations}{%
\section{Market Considerations}\label{market-considerations}}

\hypertarget{the-primary-market}{%
\subsection{The Primary Market}\label{the-primary-market}}

\begin{enumerate}
\def\labelenumi{\arabic{enumi}.}
\item
  What is the major market for the book? (Scholarly/professional, text,
  reference, trade?)
\item
  If this is a text, for what course is the book intended? Is the book a
  core text or a supplement? What type of student takes this course?
  What is the level? (Major or non-major; freshman, senior, graduate?)
  Do you offer this course yourself? If so, how many times have you
  given it? Is your text class-tested?
\item
  If the market is scholarly/professional, reference, or trade, how may
  it best be reached? (Direct mail, relevant journals, professional
  associations, libraries, book or music stores?) For what type of
  reader is your book intended?
\end{enumerate}

\hypertarget{status-of-the-work}{%
\section{Status of the Work}\label{status-of-the-work}}

\begin{enumerate}
\def\labelenumi{\arabic{enumi}.}
\tightlist
\item
  Do you have a timetable for completing the book?
\end{enumerate}

\begin{enumerate}
\def\labelenumi{\alph{enumi}.}
\item
  What portion or percentage of the material is now complete?
\item
  When do you expect to have a complete manuscript?
\end{enumerate}

\begin{enumerate}
\def\labelenumi{\arabic{enumi}.}
\setcounter{enumi}{1}
\tightlist
\item
  What do you estimate to be the size of the completed book?
\end{enumerate}

\begin{enumerate}
\def\labelenumi{\alph{enumi}.}
\item
  Double spaced typewritten pages normally reduce about one-third when
  set in type; e.g., 300 typewritten pages make about 200 printed pages.
  There are about 450 words on a printed page.
\item
  Approximately how many photographs do you plan to include?
\item
  Approximately how many line drawings (charts, graphs, diagrams, etc. )
  will you need?
\item
  Do you plan to include material requiring permission (text, music,
  lyrics, illustrations)? To what extent? Have you started the
  permissions request process?
\end{enumerate}

\begin{enumerate}
\def\labelenumi{\arabic{enumi}.}
\setcounter{enumi}{2}
\tightlist
\item
  Do you plan to class-test the material in your own or other sections
  of the course? (Any material distributed to students should be
  protected by copyright notice on the material.)
\end{enumerate}

\hypertarget{sample-chapters}{%
\section{Sample Chapters}\label{sample-chapters}}

Select one or two chapters of the manuscript that are an integral part
of the book. They should be those you consider the best-written ones,
and do not have to be in sequence. For example, you might submit
chapters 3, 7, and 14 of a 20-chapter book, so long as these chapters
represent the content and reflect your writing style and pedagogy in the
best possible light. It is also advisable to submit any chapter that is
particularly innovative or unique. Sample chapters should contain rough
sketches, charts, hand-written musical examples or xerox reproductions,
and description of photographs to be included. The material need not be
in final form, although it should be carefully prepared and represent
your best work. In your preparation, emphasis should be on readability.
Please do not bind your manuscript, as we will have to unbind it in
order to make photocopies for reviewers. Also be sure all pages are
numbered either consecutively or double-numbered by chapter.

\hypertarget{reviews}{%
\section{Reviews}\label{reviews}}

If we are interested in your project, we will commission outside
reviewers to read and evaluate your proposal. We will, of course, obtain
the best available reviewers to consider your work. If you wish to
suggest the names of experts in your field whom you believe to be
ideally suited to evaluate your proposal, you may provide their names,
titles, and email addresses. While we are unlikely to approach these
scholars to act as reviewers themselves, we may ask them for their
suggestions for peer readers. Naturally, we do not reveal the names of
reviewers without their permission.

\hypertarget{author-background}{%
\section{Author Background}\label{author-background}}

Please include a current CV or brief biography of your writing,
teaching, and/or educational background and experience. Be sure to list
any books that you have previously published, and any other information
about yourself on why you are qualified to write this book.

\hypertarget{response-time}{%
\section{Response Time}\label{response-time}}

Please allow at least 6-10 weeks for the manuscript proposal evaluation
and review process. We will contact you as soon as we have had a chance
to thoroughly examine your manuscript proposal. Thank you for your
interest in Oxford University Press. We look forward to reading your
materials.

\end{document}
