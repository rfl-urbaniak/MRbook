% Options for packages loaded elsewhere
\PassOptionsToPackage{unicode}{hyperref}
\PassOptionsToPackage{hyphens}{url}
\PassOptionsToPackage{dvipsnames,svgnames*,x11names*}{xcolor}
%
\documentclass[
  10pt,
  dvipsnames,enabledeprecatedfontcommands]{scrartcl}
\usepackage{amsmath,amssymb}
\usepackage{lmodern}
\usepackage{ifxetex,ifluatex}
\ifnum 0\ifxetex 1\fi\ifluatex 1\fi=0 % if pdftex
  \usepackage[T1]{fontenc}
  \usepackage[utf8]{inputenc}
  \usepackage{textcomp} % provide euro and other symbols
\else % if luatex or xetex
  \usepackage{unicode-math}
  \defaultfontfeatures{Scale=MatchLowercase}
  \defaultfontfeatures[\rmfamily]{Ligatures=TeX,Scale=1}
\fi
% Use upquote if available, for straight quotes in verbatim environments
\IfFileExists{upquote.sty}{\usepackage{upquote}}{}
\IfFileExists{microtype.sty}{% use microtype if available
  \usepackage[]{microtype}
  \UseMicrotypeSet[protrusion]{basicmath} % disable protrusion for tt fonts
}{}
\usepackage{xcolor}
\IfFileExists{xurl.sty}{\usepackage{xurl}}{} % add URL line breaks if available
\IfFileExists{bookmark.sty}{\usepackage{bookmark}}{\usepackage{hyperref}}
\hypersetup{
  pdftitle={The Difficulty With Conjunction},
  pdfauthor={Marcello Di Bello and Rafal Urbaniak},
  colorlinks=true,
  linkcolor=Maroon,
  filecolor=Maroon,
  citecolor=Blue,
  urlcolor=blue,
  pdfcreator={LaTeX via pandoc}}
\urlstyle{same} % disable monospaced font for URLs
\usepackage{graphicx}
\makeatletter
\def\maxwidth{\ifdim\Gin@nat@width>\linewidth\linewidth\else\Gin@nat@width\fi}
\def\maxheight{\ifdim\Gin@nat@height>\textheight\textheight\else\Gin@nat@height\fi}
\makeatother
% Scale images if necessary, so that they will not overflow the page
% margins by default, and it is still possible to overwrite the defaults
% using explicit options in \includegraphics[width, height, ...]{}
\setkeys{Gin}{width=\maxwidth,height=\maxheight,keepaspectratio}
% Set default figure placement to htbp
\makeatletter
\def\fps@figure{htbp}
\makeatother
\setlength{\emergencystretch}{3em} % prevent overfull lines
\providecommand{\tightlist}{%
  \setlength{\itemsep}{0pt}\setlength{\parskip}{0pt}}
\setcounter{secnumdepth}{5}
%\documentclass{article}

% %packages
\usepackage{booktabs}
%\usepackage[left]{showlabels}
\usepackage{multirow}
\usepackage{subcaption}
\usepackage{wrapfig}
\usepackage{graphicx}
\usepackage{longtable}
\usepackage{ragged2e}
\usepackage{etex}
%\usepackage{yfonts}
\usepackage{marvosym}
\usepackage[notextcomp]{kpfonts}
\usepackage{nicefrac}
\newcommand*{\QED}{\hfill \footnotesize {\sc Q.e.d.}}
\usepackage{floatrow}
\usepackage{multicol}

\usepackage[textsize=footnotesize]{todonotes}
\newcommand{\ali}[1]{\todo[color=gray!40]{\textbf{Alicja:} #1}}
\newcommand{\mar}[1]{\todo[color=blue!40]{#1}}
\newcommand{\raf}[1]{\todo[color=olive!40]{#1}}

%\linespread{1.5}
\newcommand{\indep}{\!\perp \!\!\! \perp\!}


\setlength{\parindent}{10pt}
\setlength{\parskip}{1pt}


%language
%\usepackage{times}
\usepackage{mathptmx}
\usepackage[scaled=0.86]{helvet}
\usepackage{t1enc}
%\usepackage[utf8x]{inputenc}
%\usepackage[polish]{babel}
%\usepackage{polski}




%AMS
\usepackage{amsfonts}
\usepackage{amssymb}
\usepackage{amsthm}
\usepackage{amsmath}
\usepackage{mathtools}

\usepackage{geometry}
 \geometry{a4paper,left=35mm,top=20mm,}


%environments
\newtheorem{fact}{Fact}



%abbreviations
\newcommand{\ra}{\rangle}
\newcommand{\la}{\langle}
\newcommand{\n}{\neg}
\newcommand{\et}{\wedge}
\newcommand{\jt}{\rightarrow}
\newcommand{\ko}[1]{\forall  #1\,}
\newcommand{\ro}{\leftrightarrow}
\newcommand{\exi}[1]{\exists\, {_{#1}}}
\newcommand{\pr}[1]{\ensuremath{\mathsf{P}(#1)}}
\newcommand{\cost}{\mathsf{cost}}
\newcommand{\benefit}{\mathsf{benefit}}
\newcommand{\ut}{\mathsf{ut}}

\newcommand{\odds}{\mathsf{Odds}}
\newcommand{\ind}{\mathsf{Ind}}
\newcommand{\nf}[2]{\nicefrac{#1\,}{#2}}
\newcommand{\R}[1]{\texttt{#1}}
\newcommand{\prr}[1]{\mbox{$\mathtt{P}_{prior}(#1)$}}
\newcommand{\prp}[1]{\mbox{$\mathtt{P}_{posterior}(#1)$}}



\newtheorem{q}{\color{blue}Question}
\newtheorem{lemma}{Lemma}
\newtheorem{theorem}{Theorem}
\newtheorem{corollary}{Corollary}[fact]


%technical intermezzo
%---------------------

\newcommand{\intermezzoa}{
	\begin{minipage}[c]{13cm}
	\begin{center}\rule{10cm}{0.4pt}



	\tiny{\sc Optional Content Starts}
	
	\vspace{-1mm}
	
	\rule{10cm}{0.4pt}\end{center}
	\end{minipage}\nopagebreak 
	}


\newcommand{\intermezzob}{\nopagebreak 
	\begin{minipage}[c]{13cm}
	\begin{center}\rule{10cm}{0.4pt}

	\tiny{\sc Optional Content Ends}
	
	\vspace{-1mm}
	
	\rule{10cm}{0.4pt}\end{center}
	\end{minipage}
	}
	
	
%--------------------






















\newtheorem*{reply*}{Reply}
\usepackage{enumitem}
\newcommand{\question}[1]{\begin{enumerate}[resume,leftmargin=0cm,labelsep=0cm,align=left]
\item #1
\end{enumerate}}

\usepackage{float}

% \setbeamertemplate{blocks}[rounded][shadow=true]
% \setbeamertemplate{itemize items}[ball]
% \AtBeginPart{}
% \AtBeginSection{}
% \AtBeginSubsection{}
% \AtBeginSubsubsection{}
% \setlength{\emergencystretch}{0em}
% \setlength{\parskip}{0pt}






\usepackage[authoryear]{natbib}

%\bibliographystyle{apalike}



\usepackage{tikz}
\usetikzlibrary{positioning,shapes,arrows}

\usepackage{booktabs}
\usepackage{longtable}
\usepackage{array}
\usepackage{multirow}
\usepackage{wrapfig}
\usepackage{float}
\usepackage{colortbl}
\usepackage{pdflscape}
\usepackage{tabu}
\usepackage{threeparttable}
\usepackage{threeparttablex}
\usepackage[normalem]{ulem}
\usepackage{makecell}
\usepackage{xcolor}
\ifluatex
  \usepackage{selnolig}  % disable illegal ligatures
\fi
\newlength{\cslhangindent}
\setlength{\cslhangindent}{1.5em}
\newlength{\csllabelwidth}
\setlength{\csllabelwidth}{3em}
\newenvironment{CSLReferences}[2] % #1 hanging-ident, #2 entry spacing
 {% don't indent paragraphs
  \setlength{\parindent}{0pt}
  % turn on hanging indent if param 1 is 1
  \ifodd #1 \everypar{\setlength{\hangindent}{\cslhangindent}}\ignorespaces\fi
  % set entry spacing
  \ifnum #2 > 0
  \setlength{\parskip}{#2\baselineskip}
  \fi
 }%
 {}
\usepackage{calc}
\newcommand{\CSLBlock}[1]{#1\hfill\break}
\newcommand{\CSLLeftMargin}[1]{\parbox[t]{\csllabelwidth}{#1}}
\newcommand{\CSLRightInline}[1]{\parbox[t]{\linewidth - \csllabelwidth}{#1}\break}
\newcommand{\CSLIndent}[1]{\hspace{\cslhangindent}#1}

\title{The Difficulty With Conjunction}
\author{Marcello Di Bello and Rafal Urbaniak}
\date{}

\begin{document}
\maketitle

The standard of proof in criminal or civil trials functions as a
criterion of decision. If the evidence meets the standard of proof, the
defendant should be judged criminally or civilly liable of the wrong
they were accused. If the evidence does not meet the standard of proof,
the defendant should be acquitted. But what should the evidence be like
to meet the standard of proof and warrant a judgement of liability?
According to some legal probabilists, a verdict against the defendant is
warranted in case the evidence establishes, with a probability above a
suitable threshold, that the defendant committed the wrong (see
Bernoulli, 1713; Dekay, 1996; Kaplan, 1968; Kaye, 1979; Laplace, 1814;
Laudan, 2006). In criminal cases, the bar is high, and thus the
threshold will be, say, ``at least .95 probability.'' In civil cases,
the bar is lower, and thus the threshold will be, say, ``at least .5
probability.''

This threshold interpretation of the standard of proof is simple and
elegant. Chapters XX \todo{REFER TO EARLIER CHAPTER} shows how the .95
and .5 thresholds can be formally justified. But, unfortunately, this
interpretation is not without problems. Chapter Chapters ZZ
\todo{REFER TO EARLIER CHAPTER} examines a first theoretical problem:
the paradox of naked statistical evidence. The present chapter examines
a second theoretical problem: what we will call
\textbf{the difficulty with conjunction}, also known as
\textbf{the conjunction paradox}. The chapter is structured as follows.
First, we describe the difficulty with conjunction. Next, we detail
several strategies that legal probabilists have pursued as a response to
this difficulty. These strategies are promising and worth examining, but
we show that they are ultimately unsatisfactory. Finally, we put forward
our own proposal.

Our proposal for addressing the difficulty with conjunction complements
our solution to the problem of naked statistical evidence. As we view
them, both problems suggest that criminal and civil liability should not
be understood as general propositions, but as case-specific stories,
theories or explanations tailored to the defendant on trial. We argue
that this perspective---formalized using Bayesian networks in chapter XY
\todo{REFER TO EARLIER CHAPTER}---affords a better understanding of how
standards of proof operate in a trial.

\hypertarget{introducing-the-difficulty}{%
\section{Introducing the difficulty}\label{introducing-the-difficulty}}

First formulated \ali{R: fix references to Cohen} by J. L. Cohen (1977),
the difficulty with conjunction has enjoyed a great deal of scholarly
attention every since (Allen, 1986; Allen \& Pardo, 2019; Allen \&
Stein, 2013; Cheng, 2012; Clermont, 2012; Haack, 2014; Schwartz \&
Sober, 2017; Spottswood, 2016; Stein, 2005). This difficulty arises when
a claim of wrongdoing, in a civil or criminal proceeding, is broken down
into its constituent elements. By the probability calculus, the
probability of a conjunction is often lower than the probability of the
conjuncts. So, according to the probabilistic interpretation of the
standard of proof, it might happen that even when each constituent
element (each individual conjunct) is established by the required
standard of proof, the overall claim of wrongdoing (the conjunction)
fails to meet the required standard. Cohen and others after him believe
that this outcome is counter-intuitive and runs contrary to trial
practice.

\hypertarget{simple-formulation}{%
\subsection{Simple formulation}\label{simple-formulation}}

An example will help to fix ideas. Suppose that, in order to prevail in
a criminal trial, the prosecution should establish two claims by the
required standard: first, that the defendant caused the victim's death;
and second, that the defendant's action was premeditated.
\ali{R: fix ref} J. L. Cohen (1977) argues that common law systems
subscribe to what he calls a \textbf{conjunction principle}, according
to which if two claims, \(A\) and \(B\), are established according to
the governing standard of proof, so is their conjunction \(A\wedge B\)
(and \emph{vice versa}). If the conjunction principle holds, the
following must be equivalent, where \(S\) is a placeholder for the
standard of proof:

\begin{center}
\begin{tabular}
{@{}ll@{}}
\toprule
\textbf{Separate} &   $A$ is established according to $S$ and $B$ is established according to $S$\\   
\textbf{Overall}  &   The conjunction $A \et B$ is established according to $S$  \\ 
\bottomrule
\end{tabular}
\end{center}

\noindent If we generalize to more than just two constituent claims, the
conjunction principle requires that:

\[S[C_1 \wedge C_2  \wedge \dots \wedge  C_k] \Leftrightarrow S[C_1] \wedge S[C_2]  \wedge \dots \wedge  S[C_k],\]

\noindent where \(S[C_i]\) means that claim or hypothesis \(C_i\) is
established according to standard \(S\). The principle goes in both
directions: call the implication from left to right
\textbf{distribution}, and the one in the opposite direction
\textbf{aggregation}. Aggregation posits that establishing the
individual claims by the requisite standard is enough to establish the
conjunction by the same standard. Distribution posits that establishing
the conjunction by the requisite standard is enough to establish the
individual claims by the same standard. Aggregation and distribution
identify properties that the standard of proof should arguably possess.
The difficulty with conjunction is traditionally concerned with the
failure of aggregation, but we will see later on that on some
probabilistic explications of the standard of proof distribution can
also fail.

Legal scholars disagree about the tenability of the conjunction
principle (more on this toward the end of this chapter). For the time
being, however, let us assume the principle correctly captures two
desidred features of the standard of proof. The principle has some
degree of plausibility and is consistent with the case law. For example,
the United States Supreme Court writes that in criminal cases

\begin{quote}
the accused [is protected] against conviction except upon proof beyond a reasonable doubt of \textit{every fact} necessary to constitute the crime with which he is charged. \linebreak 
(In re Winship, 397 U.S. 358, 364, 1970)
\end{quote}

\noindent A plausible way to interpret this quotation is to assume the
following identity: to establish someone's guilt beyond a reasonable
doubt \textit{just is} to establish each element of the crime beyond a
reasonable doubt (abbreviated as \(\mathsf{BARD}\)). Thus,
\begin{align*}\mathsf{BARD}[C_1 \wedge C_2   \wedge \cdots \wedge C_k] \Leftrightarrow \mathsf{BARD}[C_1] \wedge \mathsf{BARD}[C_2]  \wedge \cdots \wedge \mathsf{BARD}[C_k],
\end{align*}

\noindent where the conjunction \(C_1 \et C_2 \cdots \et C_k\) comprises
all the material facts that, according to the applicable law, constitute
the crime with with the accused is charged. A similar argument could be
run for the standard of proof in civil cases, preponderance of the
evidence or clear and convincing evidence.

The problem for the legal probabilist is that the conjunction principle
conflicts with a threshold-based probabilistic interpretation of the
standard of proof. For suppose the prosecution in a criminal case
presents evidence that establishes claims \(A\) and \(B\), separately,
by the required probability, say about \(.95\) each. Has the prosecution
met its burden of proof? If each claim is established by the requisite
probability threshold, each claim is established by the requisite
standard (assuming the threshold-based interpretation of the standard of
proof). And if each claim is established by the requisite standard, then
liability as a whole is established by the requisite standard (assuming
the conjunction principle). But this cannot be right. Even though each
claim is established by the requisite probability threshold, if the
claims are independent, the probability of their conjunction will be
only \(.95\times .95 \approx .9\), below the required \(.95\) threshold.
So liability as a whole is \textit{not} established by the requisite
standard (assuming a threshold-based probabilistic interpretation of the
standard). This contradicts the conjunction principle.

The difficulty with conjunction shows that a threshold-based
interpretation of the standard of proof violates the conjunction
principle by violating aggregation. Even though aggregation posits that
establishing each conjunct by the required standard of proof is enough
to establish the conjunction as a whole, the probability of each
conjunct, considered in isolation, can meet the required probability
threshold without the conjunction as a whole meeting the threshold.

This difficulty is persistent. It does not subside as the number of
constituent claims increases. If anything, the difficulty becomes more
apparent. Say the prosecution has established three separate claims by
\(.95\) probability. Their conjunction---again if the claims are
independent---would be about \(.85\) probable, even further below the
\(.95\) probability threshold. Nor does the difficulty with conjunction
generally subside if the claims are no longer regarded as independent.
The probability of the conjunction \(A \et B\), without the assumption
of independence, equals \(\pr{A \vert B} \times \pr{B}\). But if claims
\(A\) and \(B\), separately, are established with \(.95\) probability,
enough for each to meet the threshold, the probability of \(A \et B\)
should still be below
\ali{R: double-check that abbreviations in formulae are in textsf (outside of math) and mathsf inside math throughout the chapter.}
the \(.95\) threshold unless \(\pr{A \vert B}=1\).\footnote{For example,
  that someone premeditated a harmful act against another (call it
  \textsf{premed}) makes it more likely that they did cause harm in the
  end (call it \textsf{harm}). Since
  \(\pr{\textsf{harm} \vert \textsf{premed}} > \pr{\textsf{harm}}\), the
  two claims are not independent. Still, premeditation does not always
  lead to harm, so \(\pr{\textsf{harm} \vert \textsf{premed}}\) will
  often be below \(1\). If both claims are established with \(.95\), the
  probability of the conjunction \(\textsf{harm} \et \textsf{premed}\)
  should still be below the \(.95\) threshold so long as
  \(\pr{\textsf{harm} \vert \textsf{premed}}\) is still below \(1\).}

\hypertarget{adding-the-evidence}{%
\subsection{Adding the evidence}\label{adding-the-evidence}}

So far we proceeded without mentioning the evidence in support of the
claims that constitute the wrongdoing. This is a simplification. As we
will see, it is crucial to pay attention to the supporting evidence.
With this in mind, for two claims, the conjunction principle can be
formulated, as follows:
\[\text{S[$a, A$] and S[$b, B$] $\Leftrightarrow$ S[$a \wedge b, A\wedge B$]}.\]

\noindent In the case of more than two claims, the formulation of the
principle can be extended accordingly. Note that \(a\) and \(b\) denote
the evidence for claims \(A\) and \(B\) respectively, and \(S\) denotes
the standard by which the evidence establishes the claim in question.
So, for example, the expression S{[}\(a, A\){]} should be read as
`evidence \(a\) supports claim \(A\) by standard \(S\).' Consistently
with the threshold interpretation of the standard of proof, the
expression S{[}\(a, A\){]} should be read as `evidence \(a\) establishes
claim \(A\) with a probability above a suitable threshold \(t\)
corresponding to standard \(S\)' or in symbols \(\pr{A \vert a}>t\). An
analogous probabilistic reading applies to the expressions
S{[}\(b, B\){]} and S{[}\(a \wedge b, A\wedge B\){]}.

Does a threshold-based probabilistic interpretation of the standard of
proof also conflict with this revised version of the conjunction
principle? The answer is positive, but seeing why requires a bit more
work. We should check whether, if both \(\pr{A \vert a}\) and
\(\pr{B \vert b}\) meet the threshold, say \(.95\), then so does
\(\pr{A\wedge B \vert a\wedge b}\). We are no longer just comparing the
probability of \(A\wedge B\) to the probability of \(A\) and the
probability of \(B\) as such. Rather, we are comparing the probability
of \(A \wedge B\) given the combined evidence \(a \wedge b\) to the
probability of \(A\) given evidence \(a\) and the probability of \(B\)
given evidence \(b\).

Consider an example. In an aggravated assault case, the prosecution
should establish two claims: first, that the defendant injured the
victim; and second, that the defendant knew he was interacting with a
public official. A witness testimony (call it \textsf{witness}) is
offered in support of the proposition that the defendant injured the
victim (call this proposition \textsf{injury}). In addition, the
defendant's call to an emergency number (call it \textsf{emergency}) is
offered as evidence for the proposition that the defendant knew he was
interacting with a firefighter (call this proposition
\textsf{firefighther}). If
\(\pr{\textsf{injury} \vert \textsf{witness}}\) and
\(\pr{\textsf{firefighter} \vert \textsf{emergency} }\) both meet the
required probability threshold, does
\(\pr{\textsf{injury} \wedge \textsf{firefighter} \vert \textsf{witness} \wedge \textsf{emergency}}\)
also necessarily meet the threshold?

The answer is negative, at least provided two independence assumptions
hold. The first is that
\(\pr{\textsf{injury} \vert \textsf{witness}}=\pr{\textsf{injury} \vert \textsf{witness} \wedge \textsf{emergency}}\).
This assumption is plausible: that the defendant called an emergency
number should not make it more (or less) likely that the defendant would
cause injury to someone. The second assumption is that
\(\pr{\textsf{firefighther} \vert \textsf{emergency} }=\pr{\textsf{firefighther} \vert \textsf{witness} \wedge \textsf{emergency} \wedge \textsf{injury}}\).
This assumption is also plausible: that the defendant injured the victim
and there is a testimony to that effect does not make it more (or less)
likely that the victim was a firefighter. Admittedly, more is needed to
justify these assumptions than an appeal to plausibility, a point to
which we will soon return. But, granting for now that the two
assumptions hold, it follows that:\footnote{By the chain rule and the
  independence assumptions \(\pr{A | a}=\pr{A | a \wedge b}\) and
  \(\pr{B | b}=\pr{B | a \wedge b \wedge A}\), the following holds:
  \begin{align*}
  \pr{A \wedge  B \vert a \wedge b}& =\pr{A \vert a \wedge b} \times \pr{B \vert  a \wedge b \wedge A}\\
   & = \pr{A \vert a} \times \pr{B \vert  b}
   \end{align*}}
\[\pr{\textsf{injury} \wedge \textsf{firefighther} \vert \textsf{witness} \wedge \textsf{emergency}}= \pr{\textsf{injury} \vert \textsf{witness}} \times \pr{\textsf{firefighther} \vert \textsf{emergency}}. \]

\noindent If the equality holds, even when
\(\pr{\textsf{injury} \vert \textsf{witness}}\) and
\(\pr{\textsf{firefighther} \vert \textsf{emergency} }\) meet the
required probability threshold,
\(\pr{\textsf{injury} \wedge \textsf{firefighther} \vert \textsf{witness} \wedge \textsf{emergency}}\)
generally will not. An assumption to make here is that both
\(\pr{\textsf{injury} \vert \textsf{witness}}\) and
\(\pr{\textsf{firefighther} \vert \textsf{emergency} }\) are below 1, as
is usually the case given that the evidence offered in a trial is
fallible. The conclusion is that aggregation is violated.

So, in at least one case and under seemingly plausible assumptions, the
revised conjunction principle fails if the standard of proof is
interpreted as a probability threshold. But can we say something more
general? To address this question, we will now represent more
formally---specifically, using Bayesian networks---the relationship
between generic claims \(A\), \(B\) and the conjunction \(A\wedge B\),
as well as the supporting evidence \(a\), \(b\) and the conjunction
\(a\wedge b\).

\hypertarget{independent-hypotheses}{%
\subsection{Independent hypotheses}\label{independent-hypotheses}}

We already studied Bayesian networks in Chapter ZZ.
\todo{REF TO OTHER CHAPETR} Here we only sketch the essential ideas. A
Bayesian network is a formal model that consists of a graphical part (a
directed acyclic graph, \textsf{DAG}) and a numerical part (a
probability table). The nodes in the graph represent random variables
that can take different values. For ease of exposition, we will use
`nodes' and `variables' interchangeably. The nodes are connected by
directed edges (arrows). No loops are allowed, hence the name acyclic.

\begin{wrapfigure}{r}{0.4\textwidth}

\begin{center}\includegraphics{conjunction-paradox4_files/figure-latex/fig:hEDAG-1} \end{center}
\begin{tabular}{c|cc}
& $H$ & $\neg H$ \\
\hline
$E$ & $\pr{E \vert H}$ & $\pr{E\vert \neg H}$ \\
$\neg E$ & $\pr{\neg E\vert H}$ & $\pr{\neg E\vert \neg H}$ \\
\end{tabular}

\caption{DAG of the simplest evidential relation along with a probability table}
\label{fig:hEDAG}
\end{wrapfigure}

\ali{M: Graphs are backwards, arrows go from hypothesis node to evidence nodes, hat is, 
from H to E and a to A and from B to b. Please fix.}

The simplest evidential relation, one of evidence bearing on a
hypothesis, can be represented by the directed graph displayed in Figure
\ref{fig:hEDAG}. The arrow need not have a causal interpretation. The
direction of the arrow indicates which conditional probabilities should
be supplied in the probability table. Since the arrow goes from \(H\) to
\(E\), we should specify the probabilities of the different values of
\(E\) conditional on the different values of \(H\).

Return now to the difficulty with conjunction. We will first examine the
case in which the two hypotheses \(A\) and \(B\) are probabilistically
independent. We will relax this assumption later. The two directed
graphs visualized in Figure \ref{fig:abBDAG} represent two items of
evidence each supporting its own hypothesis: \(a\) supports \(A\) and
\(b\) supports \(B\). To represent the conjunction \(A\wedge B\), a
conjunction node \(AB\) is added and arrows are drawn from nodes \(A\)
and \(B\) into node \(AB\) (Figure \ref{fig:conjunctionDAGchapter}).
This arrangement makes it possible to express the meaning of
\(A\wedge B\) via a probability table that mirrors the truth table for
the conjunction in propositional logic (see Table
\ref{tab:CPTconjunction}).\footnote{The difference is that the values
  \(1\) and \(0\) stand for two different things depending on where they
  are in the table. In the columns corresponding to the nodes they
  represent node states: true and false; in the \(\textsf{Pr}\) column
  they represent the conditional probability of a given state of \(AB\)
  given the states of \(A\) and \(B\) listed in the same row. For
  instance, take a look at row two. It says: if \(A\) and \(B\) are both
  in states 1, then the probability of \(AB\) being in state 0 is 0. In
  principle we could use `true' and `false' instead of 1 and 0 to
  represent states, but the numeric representation is easier to use in
  programming, which we do quite a bit in the background, so the reader
  might as well get used to this harmless ambiguity. For binary nodes,
  we will consistently use `1' and `0' for the states, it's just
  probabilities that in this case end up being extreme.}

\begin{figure}[h]
\begin{subfigure}[!ht]{0.45\textwidth}

\begin{center}\includegraphics[width=0.6\linewidth]{conjunction-paradox4_files/figure-latex/fig:aADAG-1} \end{center}
\end{subfigure}\begin{subfigure}[!ht]{0.45\textwidth}

\begin{center}\includegraphics[width=0.6\linewidth]{conjunction-paradox4_files/figure-latex/fig:bBDAG-1} \end{center}
\end{subfigure}
\caption{\textsf{DAG}s of $a$ supporting $A$ and $b$ supporting $B$.}
\label{fig:abBDAG}
\end{figure}

\begin{figure}[H]

\begin{center}\includegraphics[width=0.6\linewidth]{conjunction-paradox4_files/figure-latex/fig:conjunctionDAG-1} \end{center}

\caption{\textsf{DAG} of the conjunction set-up, with the usual independence assumptions built in (\textsf{DAG 1}).}
\label{fig:conjunctionDAGchapter}
\end{figure}

\pagebreak 
\begin{wraptable}{l}{0.4\textwidth}
\begin{tabular}{lllr}
\toprule
\multicolumn{1}{c}{} & \multicolumn{1}{c}{A} & \multicolumn{1}{c}{B} & \multicolumn{1}{c}{} \\
AB &  &  & Pr\\
\midrule
\cellcolor{gray!6}{1} & \cellcolor{gray!6}{1} & \cellcolor{gray!6}{1} & \cellcolor{gray!6}{1}\\
0 & 1 & 1 & 0\\
\cellcolor{gray!6}{1} & \cellcolor{gray!6}{0} & \cellcolor{gray!6}{1} & \cellcolor{gray!6}{0}\\
0 & 0 & 1 & 1\\
\cellcolor{gray!6}{1} & \cellcolor{gray!6}{1} & \cellcolor{gray!6}{0} & \cellcolor{gray!6}{0}\\
0 & 1 & 0 & 1\\
\cellcolor{gray!6}{1} & \cellcolor{gray!6}{0} & \cellcolor{gray!6}{0} & \cellcolor{gray!6}{0}\\
0 & 0 & 0 & 1\\
\bottomrule
\end{tabular}

\caption{Conditional probability table for the conjunction node.}
\label{tab:CPTconjunction}
\end{wraptable}

The resulting graph, call it \textsf{DAG 1}, satisfies the desired
independence assumptions. First, the two claims \(A\) and \(B\) are
probabilistically independent of one another. Their independence is
guaranteed by the fact that the conjunction node \(AB\) is a collider
and thus no information flows through it.\footnote{A more formal
  treatment of this point is provided in
  \textbf{REFER TO OTHER CHAPTER}.} Second, the supporting items of
evidence \(a\) and \(b\) are also probabilistically independent of one
another. The reason is the same: node \(AB\) blocks any flow of
information between the evidence nodes. Notably, the independence of the
items of evidence is not always explicitly stated in the formulation of
the conjunction paradox. The good thing is that the Bayesian network
makes this assumption explicit.

With this set-up in place, the conjunction paradox arises because
aggregation is violated. By the theory of Bayesian networks,
\textsf{DAG 1} in Figure \ref{fig:conjunctionDAGchapter}

ensures the following:\footnote{This is because the only path between
  \textsf{A} and \textsf{B} goes through \textsf{AB}, which is a
  collider; as long as we do not condition on it, all paths between
  \textsf{A} and \textsf{B} remain blocked. See our chapter introducing
  Bayesian networks for details on this issue.
  \textbf{REFER TO APPROPRIATE CHAPTER}} \begin{align*}
\pr{A \wedge  B \vert a \wedge b}& =\pr{A \vert a \wedge b} \times \pr{B \vert  a \wedge b \wedge A}\\
 & = \pr{A \vert a} \times \pr{B \vert  b}
 \end{align*}

\noindent Thus, even when claims \(A\) and \(B\) are sufficiently
probable given their supporting evidence \(a\) and \(b\) (for a fixed
threshold \(t\))---in symbols, \(\pr{A \vert a}>t\) and
\(\pr{B \vert b}>t\)---it does not generally follow that \(A \et B\) is
sufficiently probable given combined evidence \(a\et b\) provided (as is
normally the case) neither \(\pr{A \vert a}\) nor \(\pr{B \vert b}\)
equal 1. As before, the conjunction principle fails because aggregation
fails.

The argument here goes beyond the specific example about aggravated
assault in the previous section. The argument only assumes that the
directed graph in Figure \ref{fig:conjunctionDAGchapter} is an adequate
representation of a situation in which two items of evidence, \(a\) and
\(b\), support their own hypothesis, \(A\) and \(B\). The graph encodes
two plausible relations of probabilistic independence: between
hypotheses \(A\) and \(B\) and between items of evidence \(a\) and
\(b\). The theory of Bayesian networks does the rest of the work.

\hypertarget{dependent-hypotheses}{%
\subsection{Dependent hypotheses}\label{dependent-hypotheses}}

Consider now what happens if claims \(A\) and \(B\) are not regarded as
probabilistic independent. To represent this, it is enough to draw an
arrow between nodes \(A\) and \(B\). The modified graph is displayed in
Figure \ref{fig:conjunctionDAG2chapter}, call it \textsf{DAG 2}. The
open path between nodes \(A\) and \(B\) no longer guarantees the
probabilistic independence of \(A\) and \(B\) or the independence of
evidence nodes \(a\) and \(b\). Note, however, that there is still no
\emph{direct} dependence between the items of evidence. The items of
evidence are still probabilistically independent of one another
\textit{conditional} on their respective hypothesis. That is,
\(\pr{a \vert A}=\pr{a \vert A \wedge b}\) and
\(\pr{b \vert B}=\pr{b \vert B \wedge a}\). So \(a\) and \(b\) still
counts as independent lines of evidence despite not being
(unconditionally) probabilistically independent.\footnote{Here is an
  illustration of the idea of independent lines of evidence without
  unconditional independence. Suppose the same phenomenon (say blood
  pressure) is measured by two instruments. The reading of the two
  instruments (say `high' blood pressure) should be
  \textit{probabilistically dependent} of one another. After all, if the
  instruments were both infallible and they were measuring the same
  phenomenon, they should give the exact same reading. On the other
  hand, the two instruments measuring the same phenomenon should count
  as \textit{independent lines of evidence}. This fact is rendered in
  probabilistic terms by means of probabilistic independence conditional
  on the hypothesis of interest. These ideas can be worked out more
  systematically in the language of Bayesian networks. Roughly, two
  variables are probabilistically dependent if there is an open path
  between them. On the other hand, an open path can be closed by
  conditioning on one of the variables along the path. For a more
  rigorous exposition of the notions of open and closed paths, see
  \textbf{CITE EARLIER CHAPTERS}.}

\begin{figure}[h]

\begin{center}\includegraphics[width=0.6\linewidth]{conjunction-paradox4_files/figure-latex/fig:conjunctionDAG2-1} \end{center}
\caption{\textsf{DAG} of the conjunction set-up, without independence between $A$ and $B$ (\textsf{DAG 2}).}
\label{fig:conjunctionDAG2chapter}
\end{figure}

The difficulty with conjunction arises even without the independence of
hypotheses \(A\) and \(B\), at least in a number of circumstances. To
see why this is the case analytically is a bit cumbersome (see appendix
for details).
\raf{M: I think we should move the analytical arguemnt to the appendix or remove it entirely. It's muddled. Commented it out}
But we can study how often, in principle, the joint posterior
\(\pr{A\wedge B \vert a \wedge b}\) is below both of the individual
posteriors \(\pr{A \vert a}\) and \(\pr{B \vert b}\). To this end, we
simulated 10,000 random Bayesian networks based on \textsf{DAG 1} and
\textsf{DAG 2}. Assuming that each possible Bayesian network has an
equal probability of occurring, the joint posterior is lower than both
individual posteriors 68\% of the time for \textsf{DAG 1}, and around
60\% for \textsf{DAG 2} (see the appendix for details). This result
agrees with Schwartz \& Sober (2017) who pointed out that, if hypotheses
\(A\) and \(B\) are probabilistically dependent on one another,
aggregation will fail less often. Still, postulating a dependence
between hypotheses does little for solving the difficulty with
conjunction since a drop in the failure rate from 68\% to 60\% is
limited.

\vspace{1mm}
\footnotesize

\normalsize

\hypertarget{evidential-strength}{%
\section{Evidential Strength}\label{evidential-strength}}

The failures of the conjunction principles so far are failures of
aggregation. When the probability of \(A\) and the probability of \(B\)
are both above a given threshold, the probability of the conjunction
\(A\wedge B\) often is not. This can happen whether or not \(A\) and
\(B\) are probabilistically independent. These failures of aggregation
occur if the standard of proof is understood as a posterior probability
threshold.

Some have regarded these failures as strong enough ground to advocate a
different conception of probability, along the lines of so-called
Baconian probability (L. J. Cohen, 2008), fuzzy logic or belief
functions (Clermont, 2013), or reject legal probabilism altogether. We
do not rehearse these arguments here. Our focus is on
\textit{probabilistic} strategies for handling the difficulty with
conjunction. Our working hypothesis is that probability theory need not
be revised. It is a well-established theory. We should not reject it
unless strong reasons mandate it. Since we have not yet explored any
alternative strategies within the probabilistic approach, the failures
of aggregation are not strong enough reasons to reject probability
theory in this context.

Now, as the previous section demonstrated, aggregation fails in a large
class of cases if the standard of proof is understood as a posterior
probability threshold. What can legal probabilists do? Posterior
probability thresholds are not the only way of understanding standards
of proof from a probabilistic perspective. Standards of proof can also
be modeled using probabilistic measures of evidential strength. This is
the approach we explore in this section. As we shall see, this
alternative way of modeling proof standards succeeds at validating the
principle of aggregation (at least, given some additional assumptions),
but it is still not entirely satisfactory.

\hypertarget{a-dilemma}{%
\subsection{A dilemma}\label{a-dilemma}}

We begin by providing an outline of our argument. Two common
probabilistic measures of evidential strength are the Bayes factor and
the likelihood ratio. We discussed them earlier in Chapter XX.
\todo{REFER TO EARLIER CHAPTERS} We will show that, under plausible
assumptions, these measures of evidential strength validate one
direction of the conjunction principle: aggregation. If \(a\) is
sufficiently strong evidence in favor of \(A\) and \(b\) is sufficiently
strong evidence in favor of \(B\), then \(a\wedge b\) is sufficiently
strong evidence in favor of the conjunction \(A \wedge B\). In fact, the
evidential support for the conjunction will often exceed that for the
individual claims, a point already made by Dawid (1987):

\begin{quote} suitably measured, the support supplied by the conjunction of several independent testimonies exceeds that supplied by any of its constituents.
 \end{quote}

\noindent Dawid thought that vindicating aggregation was enough for the
conjunction paradox to `evaporate.' Unfortunately, we will show that on
the evidential strength interpretation of the standard of proof, the
other direction of the conjunction principle, distribution, does not
hold. If \(a \wedge b\) is sufficiently strong evidence in favor of
\(A \wedge B\), it does not follow that \(a\wedge b\) is sufficiently
strong evidence in favor of \(A\) or that \(a\wedge b\) is sufficiently
strong evidence in favor of \(B\). This is odd. It would mean that,
given a body of evidence, one can establish beyond a reasonable doubt
that \(A \wedge B\) (say the defendant killed the victim \textit{and}
acted intentionally) while failing to establish one of the conjuncts.

We face a dilemma. If the standard of proof is understood as a posterior
probability threshold, the conjunction principle fails because
aggregation fails while distribution succeeds. If, on the other hand,
the standard of proof is understood as a threshold relative to
evidential strength, the conjunction principle fails because
distribution fails while aggregation succeeds. From a probabilistic
perspective, it seems impossible to capture both directions of the
conjunction principle.

In what follows, we develop more precisely the argument that, on the
evidential strength approach, (a) aggregation succeeds but (b)
distribution fails. The argument for these two claims is tedious. The
reader should arm themselves with patience or take our word for it and
jump ahead. On the other hand, the curious reader is welcome to read the
appendix for the technical details.

\hypertarget{combined-support-bayes-factor}{%
\subsection{Combined support: Bayes
factor}\label{combined-support-bayes-factor}}

The first step in the argument shows that the combined support supplied
by multiple pieces of evidence (e.g.~\(a\wedge b\)) for a conjunctive
claim (e.g.~\(A\wedge B\)) typically exceeds the individual support
supplied by individual pieces of evidence for individual claims. This
claim holds for the Bayes factor and to some extent for the likelihood
ratio. We start with the Bayes factor \(\pr{E \vert H}/\pr{E}\) as our
measure of the support of \(E\) in favor of \(H\). Since by Bayes'
theorem

\[\pr{H \vert E} = \frac{\pr{E \vert H}}{\pr{E}}\times \pr{H},\]

\noindent the Bayes factor measures the extent to which a piece of
evidence increases the probability of a hypothesis, as compared to its
prior probability. The greater the Bayes factor (for values above one),
the stronger the support of \(E\) in favor of \(H\). Putting aside
reservations about this measure of evidential support (see Chapter XX
\todo{REFER TO EARLIER CHAPTER}), the Bayes factor
\(\pr{E \vert H}/\pr{E}\), unlike the conditional probability
\(\pr{H \vert E}\), offers a potential way to overcome the difficulty
with conjunction by vindicating aggregation.

\hypertarget{independent-hypotheses-1}{%
\subsubsection{Independent hypotheses}\label{independent-hypotheses-1}}

Suppose items of evidence \(a\) and \(b\) positively support \(A\) and
\(B\), separately. In other words, both Bayes factors
\(\nicefrac{\pr{a \vert A}}{\pr{a}}\) (abbreviated \(BF_A\)) and
\(\nicefrac{\pr{b \vert B}}{\pr{b}}\) (abbreviated \(BF_B\)) are greater
than one. Does the combined evidence \(a \wedge b\) provide at least as
much support in favor of the joint claim \(A \wedge B\) as the
individual support by \(a\) and \(b\) in favor of \(A\) and \(B\)
considered separately? The combined support here is: \begin{align*}
\frac{\pr{a \wedge b \vert A \wedge B}}{\pr{a \wedge b}} &= \frac{\pr{a \vert A}}{\pr{a}} \times \frac{\pr{b \vert B}}{\pr{b}}\\
BF_{AB} &= BF_{A} \times BF_{B}
\end{align*} \noindent (See the appendix for a proof.) This claim holds
assuming (roughly) that hypotheses \(A\) and \(B\) are independent and
that items of evidence \(a\) and \(b\) are independent. These
assumptions are plausible insofar as the Bayesian network in Figure
\ref{fig:conjunctionDAGchapter} is a plausible representation of the
situation at hand. Thus, the combined support \(BF_{AB}\) will always be
higher than the individual support so long as \(BF_{A}\) and \(BF_{B}\)
are greater than one, that is, if the individual piece of evidence
positively support their respective hypotheses.

This result generalizes beyond two pieces of evidence. Figure
\ref{fig:bfconjunction5} compares the Bayes factor of one item of
evidence, say \(\nicefrac{\pr{a \vert A}}{\pr{a}}\) with the combined
Bayes factor for five items of evidence, say
\(\nicefrac{\pr{a_1 \wedge \dots \wedge a_5 \vert A_1 \wedge \dots \wedge A_5}}{\pr{a_1\wedge \dots \wedge a_5}}\),
for different values of sensitivity and specificity of the
evidence.\footnote{The \textbf{sensitivity} of a piece of evidence \(E\)
  relative to a hypothesis \(H\) is \(\pr{E \vert H}\), while its
  \textbf{specificity} is \(\pr{\neg E \vert \neg H}\).} The combined
Bayes factor always exceeds the individual Bayes factors provided, as
usual, the individual pieces of evidence positively support the
individual hypotheses.\footnote{The order is reversed if the items of
  evidence oppose the individual hypotheses. Neutral evidence results in
  a combined Bayes factor of 1, no matter the prior or the number of
  items of evidence} Under these conditions, Dawid's claim that `the
support supplied by the conjunction of several independent testimonies
exceeds that supplied by any of its constituents' (if support is to be
measured in terms of Bayes factor) is vindicated.

\ali{R: recode lty scale so that the order is five-two-one in the legend.}
\begin{figure}[h]

\begin{center}\includegraphics[width=0.9\linewidth]{conjunction-paradox4_files/figure-latex/bfconjunction5-1} \end{center}
\caption{Bayes factor for one, two and five items of evidence and the corresponding claims, given different degrees of specificity and sensitivity of the evidence. The independence assumptions in Figure \ref{fig:conjunctionDAGchapter} hold.}
\label{fig:bfconjunction5}
\end{figure}

\hypertarget{dependent-hypotheses-1}{%
\subsubsection{Dependent hypotheses}\label{dependent-hypotheses-1}}

If \(A\) and \(B\) are not necessarily probabilistically independent as
in the Bayesian network in Figure \ref{fig:conjunctionDAG2chapter}, the
combined Bayes factor \(BF_{AB}\) is still greater than both the
individual Bayes factor \(BF_{A}\) and \(BF_{B}\) if the probabilistic
measure fits \textsf{DAG 2}. To see why, first note that the following
holds:

\begin{align}
\frac{\pr{a \wedge b\vert A\wedge B}}{\pr{a \wedge b}} & = \frac{\pr{a |A}}{\pr{a}} \times \frac{\pr{b |B}}{\pr{b|a}} \nonumber \\
BF_{AB}& =  BF_{A}\times BF^{'}_{B}  \label{eq:BFdep}
 \end{align}

\noindent(See the appendix for a proof.) The difference from the case of
independent hypotheses is that
\(BF_B=\nicefrac{\pr{b \vert B}}{\pr{b}}\) was replaced by
\(BF^{'}_B=\nicefrac{\pr{b \vert B}}{\pr{b\vert a}}\). Since \(b\) need
not be probabilistically independent of \(a\), there is no guarantee
that \(\pr{b \vert a}=\pr{b}\). However, if the probabilistic measure
fits \textsf{DAG 2}, if \(BF_B\) is greater than 1, then so is
\(BF^{'}_B\). In such circumstances, \eqref{eq:BFdep} entails that the
joint Bayes factor, \(BF_{AB}\), will be greater than any of the
individual Bayes factors\textgreater{} Interestingly, if the underlying
\textsf{DAG} allows for a direct dependence between the items of
evidence, the claim fails, and the joint Bayes factor can be lower than
either of the individual Bayes factors (see the appendix for proof).

\mar{R: I though hard about including the graphics here, but at this point I don't think this would contribute to clarity, give it a thought.}

\hypertarget{combined-support-likelihood-ratio}{%
\subsection{Combined support: likelihood
ratio}\label{combined-support-likelihood-ratio}}

The likelihood ratio is another probabilistic measure of evidential
support, extensively discussed in Chapter XX.
\todo{REFER TO EARLIER CHAPTER} The likelihood ratio compares the
probability of the evidence on the assumption that a hypothesis of
interest is true (\textit{sensitivity}) and the probability of the
evidence on the assumption that the negation of the hypothesis is true
(\textit{1- specificity}). That is,
\[\frac{\pr{E \vert H}}{\pr{E \vert \neg H}}=\frac{\textit{sensitivity}}{\textit{1- specificity}}\]

\noindent The greater the likelihood ratio (for values above one), the
stronger the evidential support in favor of the hypothesis (as
contrasted to the its negation). Unlike the Bayes factor, the likelihood
ratio does not vary depending on the prior probability of the hypothesis
so long as sensitivity and specificity do not.\footnote{Whether the
  sensitivity and specificity of the evidence depends on the prior
  probability of the hypothesis is debated in the literature
  \textbf{CITE}. Further, we will see that specificity does depend on
  the prior probability in the case of conjunctive hypotheses such as
  \(A \wedge B\).}

The question to examine is whether the combined support measured by the
combined likelihood ratio
\[\frac{\pr{a\wedge b \vert A\wedge B}}{\pr{a \wedge b \vert \neg (A\wedge B)}}\]

\noindent exceeds the individual support measured by the individual
likelihood ratios \(\nicefrac{\pr{a \vert A}}{\pr{a \vert \neg A}}\) and
\(\nicefrac{\pr{b \vert B}}{\pr{b \vert \neg B}}\). Under suitable
assumptions, the answer is positive. So, details aside, Bayes factor and
likelihood ratio agree here. The argument for the likelihood ratio,
however, is more laborious.

We start with observing that, in a large class of cases captured by the
Bayesian network in Figure \ref{fig:conjunctionDAGchapter} or Figure
\ref{fig:conjunctionDAG2chapter}, the following holds:

\begin{align*}
LR_{AB} & = \frac{\pr{a\wedge b \vert A\wedge B}}{\pr{a \wedge b \vert \neg (A\wedge B)}} = \frac{\pr{a \vert A} \times \pr{b \vert B}}
 {\frac{\pr{\neg A}\pr{B \vert \neg A} \pr{a \vert \neg A}\pr{b \vert B} + \pr{A}\pr{\neg B \vert A} \pr{a \vert A }\pr{b \vert \neg B} + \pr{\neg A}\pr{\neg B \vert \neg A } \pr{a \vert \neg A}\pr{b \vert \neg B}}{\pr{\neg A}\pr{B \vert \neg A} + \pr{A}\pr{\neg B \vert A } + \pr{\neg A}\pr{\neg B \vert \neg A} }}
\end{align*}

\noindent The equality is sufficiently general and it holds whether or
not hypotheses \(A\) and \(B\) are probabilistically independent. (See
the appendix for a proof.)

\hypertarget{same-sensitivity-and-specificity-and-independence}{%
\subsubsection{Same sensitivity and specificity, and
independence}\label{same-sensitivity-and-specificity-and-independence}}

For illustrative purposes, we first consider a simplified picture. Let
the sensitivity and specificity of the items of evidence be the same and
equal \(x\). Let also \(A\) and \(B\) be probabilistically independent
in agreement with Figure \ref{fig:conjunctionDAGchapter}. The combined
likelihood ratio can now be plotted as a function of \(x\).\footnote{In
  this simplified set-up, the combined likelihood ratio reduces to the
  following, where \(\pr{A}=k\) and \(\pr{B}=t\): \begin{align*}
  \frac{\pr{a \wedge b \vert A\wedge B}}{\pr{a \et b\vert \neg (A\et B)}} & = \frac{x^2}{\frac{(1-k)t(1-x)x + k(1-t)x(1-x) + (1-k)(1-t)(1-x)(1-x)}{ \left(1-k\right) t +\left(1-t\right) k+\left(1-k\right) \left(1-t\right)}}
   \end{align*}} Figure \ref{fig:jointLRMarcello} shows that the
combined likelihood ratio always exceeds the individual likelihood
ratios whenever they are greater than one (or in other words, as is
usually assumed, the two pieces of evidence provide positive support for
their respective hypotheses). Interestingly, the combined likelihood
ratio varies depending on the prior probabilities \(\pr{A}\) and
\(\pr{B}\).

\begin{figure}

\begin{center}\includegraphics[width=0.9\linewidth]{conjunction-paradox4_files/figure-latex/unnamed-chunk-3-1} \end{center}

\caption{Combined likelihood ratios exceeds individual Likelihood ratios as soon as sensitivity is above .5. Changes in the prior probabilities $t$ and $k$ do not invalidate this result.}
\label{fig:jointLRMarcello}
\end{figure}

\noindent As with the Bayes factor, the combined likelihood ratio
exceeds the individual likelihood ratios. But the graph only covers
cases in which the two pieces of evidence have the same sensitivity and
specificity and hypotheses \(A\) and \(B\) are independent. What happens
if these assumptions are relaxed?

\hypertarget{different-sensitivity-and-specificity}{%
\subsubsection{Different sensitivity and
specificity}\label{different-sensitivity-and-specificity}}

If the items of evidence have different levels of sensitivity and
specificity, the combined likelihood ratio never goes below the lower of
the two individual likelihood ratios, but can be lower than the higher
individual likelihood ratio. We established this claim by means of a
computer simulation (see the appendix for details). This holds if the
probabilistic measure fits \textsf{DAG 1} and \textsf{DAG 2}, but fails
if there is direct dependence between the pieces of evidence. The
simplified set-up in the previous situation does not contradict this
claim but follows from it. In the simplified set-up, both individual
likelihood ratios were the same, so whenever the joint likelihood was
higher than the minimum of the individual likelihood ratios, it was
higher than the both of them. In this sense, the joint likelihood ratio
behaves differently than the joint Bayes factor in that it is greater
than the lower of the individual likelihood ratios, rather than being
greater than both of them. Here Dawid's claim that `the support supplied
by the conjunction of several independent testimonies exceeds that
supplied by any of its constituents' should therefore be weakened. That
is, the evidential support (as measured by the likelihood ratio)
supplied by the conjunction of several independent items of evidence
exceeds the support supplied by at least one individual item of evidence
but possibly not all, and only if there is no direct dependence between
the items of evidence.

\hypertarget{vindicating-aggregation}{%
\subsection{Vindicating aggregation}\label{vindicating-aggregation}}

We have seen that under suitable assumptions the combined evidential
support for the conjunction exceeds the individual support for at least
one of the individual claims. More precisely:

\[STR[a \wedge b, A\wedge A] \geq min(STR[a, A], STR[b, B]),\]

\noindent where \(STR[...]\) stands for the strength of the evidential
support of a piece of evidence toward a hypothesis, measured by the
Bayes factor or the likelihood ratio. This fact can be used to justify
aggregation. By the principle above, the following implication holds:

\[STR[a, A]>t_{STR} \wedge STR[b, B]>t_{STR} \Rightarrow  STR[a \wedge b, A\wedge B]>t_{STR},\]

\noindent where \(t_{STR}\) is a threshold on the strength of evidential
support. This implication resembles the aggregation principle, repeated
below for convenience:

\[S[a, A] \wedge S[b, B] \Rightarrow  S[a \wedge b, A\wedge B],\]

\noindent  where \(S\) stands for the standard of proof by which a claim
must be established based on the evidence.

The argument for justifying the principle of aggregation, however, is
not immediate. Aggregation is a principle about the standard of proof.
The Bayes factor and the likelihood ratio are instead measures of
evidential strength. The standard of proof must be connected to
evidential strength. Naturally enough, the decision criterion should be
formulated in terms of evidential strength rather than posterior
probabilities. In criminal trials, for example, the rule of decision
could be: guilt is proven beyond a reasonable doubt if and only if the
evidential support in favor of \(H\)---as measured by the Bayes factor
\(\nicefrac{\pr{E \vert H}}{\pr{E}}\) or the likelihood ratio
\(\nicefrac{\pr{E\vert H}}{\pr{E \vert \n H}}\)---meets a suitably high
threshold \(t_{BF}\) or \(t_{LR}\). The new threshold is no longer be a
probability between 0 and 1, but a number somewhere above one. The
greater this number, the more stringent the standard of proof, for any
value above one. The question at this point is, how to identify the
appropriate evidential strength threshold? The answer to this question
is not obvious. Below we examine two possible approaches.

\hypertarget{variable-threhsold}{%
\subsubsection{Variable threhsold}\label{variable-threhsold}}

The first approach we consider will soon turn out to be inadequate. It
is still useful to understand why this approach does not work to
formulate one that is more promising. A threshold on evidential strength
can be derived from the threshold on posterior probability. The
advantage of the posterior probability threshold is that its stringency
can be determined in a decision-theoretic manner via the minimization of
expected costs (see Chapter XX \todo{REFER TO ERALIER CHAPPER}). The
threshold for the Bayes factor and the likelihood ratio, call them
\(t_{BF}\) or \(t_{LR}\), can be derived from the threshold \(t\) for
the posterior probability by manipulating a simple equation.

Consider \(t_{BF}\) first. Since \begin{align*}
\mathsf{ Bayes \: factor }=\frac{\mathsf{posterior }}{\mathsf{ prior}},
\end{align*}

\noindent the Bayes factor threshold can be defined as follows:
\begin{align*}t_{BF} & = \frac{t}{\textsf{prior}}
\end{align*}

\noindent Note that \(t_{BF}\) will depend on the prior probability of
the hypothesis of interest. The higher its prior probability, the lower
\(t_{BF}\). Whether this is a desirable property for a decision
threshold can be questioned, but a similar point holds about the
posterior threshold \(t\): the higher the prior probability, the easier
to meet the threshold. The same strategy works for the threshold
\(t_{LR}\). By the odds version of Bayes' theorem,
\begin{align*}\textsf{likelihood ratio}=\frac{\textsf{posterior odds}}{\textsf{prior odds}}.
\end{align*} If the posterior ratio is fixed at, say \(t/1-t\),
\(t_{LR}\) can be obtained as follows:
\begin{align*}t_{LR} & = \frac{\nicefrac{t}{1-t}}{\textsf{prior odds}}.
\end{align*}

\noindent Once again, the higher the prior, the lower the likelihood
ratio threshold.

Unfortunately, this approach incurs a major shortcoming: aggregation
still fails. There will be cases in which the conjuncts taken separately
satisfy the decision standard \(t_{BF}\) or \(t_{LR}\), while the
conjunction does not.
\raf{Question: how does Kaplow think about it? Does he only derive threshold for the ultimate claim? I don't remember now.}
The culprit is the fact that \(t_{BF}\) and \(t_{LR}\) have different
absolute values when applied to individual claims \(A\) and \(B\)
compared to the composite claim \(A \wedge B\). To illustrate this
point, consider the principle of aggregation formulated in terms of the
Bayes factor threshold:
\begin{align*} \frac{\pr{a \vert A }}{\pr{a}}>t^A_{BF} &\mbox{ and } 
\frac{\pr{ b \vert B}}{\pr{b}}>t^B_{BF} \Rightarrow  
\frac{\pr{a \et b \vert A \et B}}{\pr{a \et b}}>t^{A\wedge B}_{BF} 
\end{align*}

\noindent Note the superscripts \(A\), \(B\) or \(A\wedge B\). The Bayes
factor threshold \(t_{BF}\) is now indexed by the superscript to the
claim of interest because the threshold \(T_{BF}\) is prior-dependent
and thus claim-dependent (since different claims have different prior
probabilities). Now, for simplicity, suppose the individual claims \(A\)
and \(B\) are probabilistically independent. Consider a posterior
threshold of \(.95\) as might be appropriate in a criminal case. If
\(A\) and \(B\) both have a prior probability of, say \(.1\), the
threshold \(t^A_{BF}=t_{BF}^B=\nicefrac{.95}{.1}=9.5\) for \(A\) or
\(B\) individually. The composite claim \(A \wedge B\) will be
associated with the threshold
\(t^{A\wedge B}_{BF}=\nicefrac{.95}{(.1\times .1)}=95\), a much higher
value. But if each individual claim meets its Bayes factor threshold of
9.5 and the two claims are independent, the joint Bayes factor would
result from the multiplication of the individual Bayes factors, that is,
\(9.5 \times 9.5=90.25\). This is not quite enough to meet
\(t^{A\wedge B}_{BF}=95\). So aggregation fails. An analogous point
holds for the likelihood ratio threshold.\footnote{Say \(A\) and \(B\)
  have prior probabilities of \(.2\) and \(.3\) respectively. On this
  approach, the likelihood ratio threshold for \(A\) and \(B\) will be
  \(t_{LR}^{A}\approx 76\) and \(t_{LR}^{B}a\approx 44\). The likelihood
  ratio threshold for the composite claim \(A \wedge B\) will be
  \(t^{A\wedge B}_{LR}\approx 297\). Now suppose the individual
  likelihood ratios meet their threshold and respective sensitivity and
  specificity are identical. For \(t_{LR}^{A}\) to be met, evidence
  \(a\) should have sensitivity of at least 0.988. For \(t_{LR}^{B}\) to
  be met, evidence \(b\) should have sensitivity 0.978. Now with these
  separate sensitivies, The combined likelihood ratio equals about 145,
  far short that what the threshold \(t^{A\wedge B}_{LR}\) requires,
  namely a likelihood ratio of 297. Once again, aggregation fails.}

Perhaps, this is not the path that the proponent of the evidential
strength approach would take anyway. The variable threshold for Bayes
factor or the likelihood ratio is parasitic on the posterior probability
threshold. It should not be surprising that, if there are reasons to
reject the posterior probability threshold, these reasons also apply to
other thresholds that are parasitic to it.

\hypertarget{fixed-threshold}{%
\subsubsection{Fixed threshold}\label{fixed-threshold}}

The second approach we consider consists in fixing the evidential
strength threshold regardless of the prior probability of the hypothesis
of interest. This approach raises the question of how the threshold
should be fixed irrespective of the priors. Standard decision theory can
no longer help here. But assuming the question can be answered, it can
be shown that the fixed threshold approach vindicates aggregation.

If the standard of proof is formalized using a fixed threshold for the
Bayes factor or the likelihood ratio, the conjunction principle boils
down to one of these:
\begin{align*} \frac{\pr{a \vert A }}{\pr{a}}>t_{BF} &\mbox{ and } 
\frac{\pr{ b \vert B}}{\pr{b}}>t_{BF} \Leftrightarrow 
\frac{\pr{a \et b \vert A \et B}}{\pr{a \et b}}>t_{BF} \\
 \frac{\pr{a \vert A }}{\pr{a \vert \n A}}>t_{LR} &\mbox{ and } 
\frac{\pr{ b \vert B}}{\pr{b\vert B}}>t_{LR} \Leftrightarrow 
\frac{\pr{a \et b \vert A \et B}}{\pr{a \et b\vert \n (A \et B)}}>t_{LR}
\end{align*}

\noindent The superscripts have been dropped since the threshold is the
same for individual and conjunctive claims. We now show that the
left-to-right direction---the principle of aggregation---holds for any
thresholds \(t_{BF}\) or \(t_{LR}\) greater than one. As seen
previously, the combined evidential support is usually greater than at
least one, if not all, individual evidential supports, whether measured
by the Bayes factor or the likelihood ratio. So whenever \(BF_A\) and
\(BF_B\) meet the threshold \(t_{BF}\), then usually also \(BF_{AB}\)
meets \(t_{BF}\). The same applies for the likelihood ratio threshold.
Whenever \(LR_A\) and \(LR_B\) meet the threshold \(t_{LR}\), then
usually also \(LR_{AB}\) meets \(t_{LR}\). Success. Aggregation is
finally vindicated. Since aggregation could not be justified using
posterior probabilities \(\pr{A \vert a}\) and \(\pr{B \vert b}\) or
using a variable evidential strength threshold, this result militates in
favor of the fixed evidential strength threshold.

Unfortunately, the right-to-left direction---the seemingly
uncontroversial principle of distribution---has now become problematic.
Suppose the combined Bayes factor,
\(\nicefrac{\pr{a \et b \vert A \et B}}{\pr{a \et b}}\), barely meets
the threshold. The individual support, say
\(\nicefrac{\pr{a \vert A}}{\pr{a}}\), could be still below the
threshold unless \(\nicefrac{\pr{b \vert B}}{\pr{b}}=1\) (which should
not happen if \(b\) positively supports \(B\)). The problem for
likelihood ratio is analogous. For suppose evidence \(a \et b\) supports
\(A \et B\) to the required threshold \(t\). The threshold in this case
should be some order of magnitude greater than one. If the combined
likelihood ratio meets the threshold \(t_{LR}\), one of the individual
likelihood ratios may well be below \(t_{LR}\). So---if the standard of
proof is interpreted using evidential support measured by the likelihood
ratio---even though the conjuction \(A \et B\) was proven according to
the desired standard, one of individual claims might not.

To be specific, we have just shown that the following distribution
principles fails: \begin{align}
S[a \wedge b, A\wedge B] \Rightarrow S[a, A] \wedge S[b, B], \tag{DIS1}
\end{align} where \(S\) is a placeholder for the standard of proof.
Perhaps, the problem is with principle (DIS1). Should it be rejected? It
might not be as essential as we thought at first. Since the evidence is
not held constant, the support supplied by \(a\wedge b\) could be
stronger than that supplied by \(a\) and \(b\) individually. So even
when the conjunction \(A \wedge B\) is established by the requisite
standard given evidence \(a\wedge b\), it might still be that \(A\) does
not meet the requisite standard (given \(a\)) nor does \(B\) (given
\(b\)). Or at least one might argue this way.

To accommodate this line of reasoning, consider this other distribution
principle: \begin{align}
S[a \wedge b, A\wedge B] \Rightarrow S[a \wedge b, A] \wedge S[a\wedge b, B]. \tag{DIS2}
\end{align}

\noindent Principle (DIS2) is less controversial because it holds the
evidence constant. This principle is harder to deny: one would not want
to claim that, while holding fixed evidence \(a\wedge b\), establishing
the conjunction might not be enough for establishing one of the
conjuncts. It seems that any formalization of the standard of proof
should obey (DIS2). Yet, (DIS2) also fails both for Bayes factor and
likelihood ratio threshold. After all, if \(A\) and \(B\) are
probabilistically independent, (DIS1) and (DIS2) are in fact equivalent
principles so long as the standard of proof is interpreted as a fixed
evidential strength
threshold.\raf{M: Need proof of this. Appendix? What about if they are not independent. Need proof. Maybe simulation?}
\mar{R: should we get into this?} So the counterexamples to (DIS1) also
work against (DIS2).

Curiously, on the fixed evidential strength threshold approach, no
matter whether one uses Bayes factor or likelihood ratio, there would be
cases in which, even though the conjunction \(A\et B\) is established by
the desired standard of proof, one of the individual claims fails to
meet the standard (see the appendix). This is odd.

\hypertarget{the-comparative-strategy}{%
\section{The comparative strategy}\label{the-comparative-strategy}}

Instead of thinking in terms of absolute thresholds---whether relative
to posterior probabilities, the Bayes factor or the likelihood
ratio---the standard of proof can be understood comparatively. This
suggestion has been advanced by Cheng (2012) following the theory of
relative plausibility by Pardo \& Allen (2008). Say the prosecutor or
the plaintiff puts forward a hypothesis \(H_p\) about what happened. The
defense offers an alternative hypothesis, call it \(H_d\). On this
approach, rather than directly evaluating the support of \(H_p\) given
the evidence and comparing it to a threshold, we compare the support
that the evidence provides for two competing hypotheses \(H_p\) and
\(H_d\), and decide for the one for which the evidence provides better
support.

It is controversial whether this is what happens in all trial
proceedings, especially in criminal trials, if one thinks of the defense
hypothesis \(H_d\) as a substantial account of what has happened. The
defense may elect to challenge the hypothesis put forward by the other
party without proposing one of its own. For example, in the O.J.~Simpson
trial the defense did not advance its own story about what happened, but
simply argued that the evidence provided by the prosecution, while
significant on its face to establish OJ's guilt, was riddled with
problems and deficencies. This defense strategy was enough to secure an
acquittal. So, in order to create a reasonable doubt about guilt, the
defense does not always provide a full-fledged alternative hypothesis.
The supporters of the comparative approach, however, will respond that
this could happen in a small number of cases, even though in
general---especially for tactical reasons---the defense will provide an
alternative hypothesis.

\hypertarget{comparing-posteriors}{%
\subsection{Comparing posteriors}\label{comparing-posteriors}}

Setting aside this controversy for the time being, we first work out the
comparative strategy using posterior probabilities. More specifically,
the standard of proof is understood comparatively as follows: given a
body of evidence \(E\) and two competing hypotheses \(H_p\) and \(H_d\),
the probability \(\pr{H_p \vert E}\) should be suitably higher than
\(\pr{H_d \vert E}\), or in other words, the ratio
\(\nicefrac{\pr{H_p \vert E}}{\pr{H_d \vert E}}\) should be above a
suitable threshold. Presumably, the ratio threshold should be higher for
criminal than civil cases. In fact, in civil cases it seems enough to
require that the ratio \(\nicefrac{\pr{H_p \vert E}}{\pr{H_d \vert E}}\)
be above 1, or equivalently, that \(\pr{H_p \vert E}\) should be higher
than \(\pr{H_d \vert E}\). Note that \(H_p\) and \(H_d\) need not be one
the negation of the other. But, if two hypotheses are exclusive and
exhaustive, \(\nicefrac{\pr{H_p \vert E}}{\pr{H_d \vert E}}>1\) implies
that \(\pr{H_p \vert E}>.5\), the standard probabilistic interpretation
of the preponderance standard.

One advantage of the comparative aapproach---as Cheng (2012) shows---is
that expected utility theory can set the appropriate comparative
threshold \(t\) as a function of the costs and benefits of trial
decisions. For simplicity, suppose that if the decision is correct, no
costs result, but incorrect decisions have their price.
\todo{REFERENCE TO EARLIER CHAPTER 
FOR MORE COMPLEX COST STRUCTURE} The costs of a false positive is
\(c_{FP}\) and that of a false negative is \(C_{FN}\), both greater than
zero. Intuitively, the decision rule should minimize the expected costs.
That is, a finding against the defendant would be acceptable whenever
its expected costs---\(\pr{H_d \vert E} \times c_{FP}\)---are smaller
than the expected costs of an
acquittal---\(\pr{H_p \vert E}\times c_{FN}\)--- or in other words:

\[\frac{\pr{H_p \vert E}}{\pr{H_d \vert E}} > \frac{c_{FP}}{c_{FN}}.\]

\noindent In civil cases, it is customary to assume the costs ratio of
false positives to false negatives equals one. So the rule of decision
would be: find against the defendant whenever
\(\frac{\pr{H_p \vert E}}{\pr{H_d \vert E}} > 1\) or in other words
\(\pr{H_p \vert E}\) is greater than \(\pr{H_d \vert E}\). In criminal
trials, the costs ratio is usually considered higher, since convicting
an innocent (false positive) should be more harmful or morally
objectionable than acquitting a guilty defendant (false negative). Thus,
the rule of decision in criminal proceedings would be: convict whenever
\(\pr{H_p \vert E}\) is appropriately greater than \(\pr{H_d \vert E}\).

Does the comparative strategy just outlined solve the difficulty with
conjunction? We will work through a stylized case used by Cheng himself.
Suppose, in a civil case, the plaintiff claims that the defendant was
speeding (\(S\)) and that the crash caused her neck injury (\(C\)).
Thus, the plaintiff's hypothesis \(H_p\) is \(S\et C\). Given the total
evidence \(E\), the conjuncts, taken separately, meet the decision
threshold: \begin{align}
 \nonumber 
 \frac{\pr{S\vert E}}{\pr{\neg S \vert E}} > 1   & & \frac{\pr{C\vert E}}{\pr{\neg C \vert E}} > 1
\end{align} \noindent The question is whether
\(\nicefrac{\pr{S\et C\vert E}}{\pr{H_d \vert E}}>1\). To answer it, we
have to decide what the defense hypothesis \(H_d\) should be. Cheng
reasons that there are three alternative defense scenarios:
\(H_{d_1}= S\et \n C\), \(H_{d_2}=\n S \et C\), and
\(H_{d_3}=\n S \et \n C\). How does the hypothesis \(H_p\) compare to
each of them? Assuming independence between \(C\) and \(S\), we have

\begin{align}\label{eq:cheng-multiplication}
\frac{\pr{S\et C\vert E}}{\pr{S\et \n C\vert E}} & = \frac{\pr{S\vert E}\pr{C\vert E}}{\pr{S \vert E}\pr{\n C \vert E}}  =\frac{\pr{C\vert E}}{\pr{\n C \vert E}} > 1 \\
\nonumber
\frac{\pr{S\et C\vert E}}{\pr{\n S\et C\vert E}} & = \frac{\pr{S\vert E}\pr{C\vert E}}{\pr{\n S \vert E}\pr{C\vert E}}  = \frac{\pr{S\vert E}}{\pr{\n S \vert E}} > 1 \\
\nonumber
\frac{\pr{S\et C\vert E}}{\pr{\n S\et \n C\vert E}} & = \frac{\pr{S\vert E}\pr{C\vert E}}{\pr{\n S \vert E}\pr{\n C \vert E}}   > 1 
\end{align}

\noindent So, whatever the defense hypothesis, the plaintiff's
hypothesis is more probable. At least in this case, whenever the
elements of a plaintiff's claim satisfy the decision threshold, so does
their conjunction. The left-to-right direction of the conjunction
principle---what we called aggregation---has been vindicated, at least
for simple cases involving independence. Success.

What about the opposite direction, distribution? Distribution is not
generally satisfied. Suppose
\(\nicefrac{\pr{S\et C\vert E}}{\pr{H_d \vert E}}>1\), or in other
words, the combined hypothesis \(S \et C\) has been established by
preponderance of the evidence. The question is whether the individual
hypotheses have been established by the same standard, specifically,
whether \(\frac{\pr{C\vert E}}{\pr{\n C \vert E}} > 1\) and
\(\frac{\pr{S\vert E}}{\pr{\n S \vert E}} > 1\). If
\(\nicefrac{\pr{S\et C\vert E}}{\pr{H_d \vert E}}>1\), the combined
hypothesis is assumed to be more probable than any of the competing
hypotheses, in particular,
\(\nicefrac{\pr{S\et C\vert E}}{\pr{\neg S \et C \vert E}}>1\),
\(\nicefrac{\pr{S\et C\vert E}}{\pr{S \et \neg C \vert E}}>1\) and
\(\nicefrac{\pr{S\et C\vert E}}{\pr{\neg S \et \neg C \vert E}}>1\). We
have: \begin{align}\label{eq:cheng-multiplication-two}
1 < \frac{\pr{S\et C\vert E}}{\pr{S\et \n C\vert E}} & = \frac{\pr{S\vert E}\pr{C\vert E}}{\pr{S \vert E}\pr{\n C \vert E}}  =\frac{\pr{C\vert E}}{\pr{\n C \vert E}} \\
\nonumber
1 < \frac{\pr{S\et C\vert E}}{\pr{\n S\et C\vert E}} & = \frac{\pr{S\vert E}\pr{C\vert E}}{\pr{\n S \vert E}\pr{C\vert E}}  = \frac{\pr{S\vert E}}{\pr{\n S \vert E}}  \\
\nonumber
1 < \frac{\pr{S\et C\vert E}}{\pr{\n S\et \n C\vert E}} & = \frac{\pr{S\vert E}\pr{C\vert E}}{\pr{\n S \vert E}\pr{\n C \vert E}}   
\end{align}

\noindent In the first two cases, clearly, if the composite hypothesis
meets the threshold, so do the individual claims. But consider the third
case.
\(\nicefrac{\pr{S\vert E}\pr{C\vert E}}{\pr{\n S \vert E}\pr{\n C \vert E}}\)
might be strictly greater than
\(\nicefrac{\pr{C\vert E}}{\pr{\n C \vert E}}\) or
\(\nicefrac{\pr{S\vert E}}{\pr{\n S \vert E}}\). It is possible that
\(\nicefrac{\pr{S\vert E}\pr{C\vert E}}{\pr{\n S \vert E}\pr{\n C \vert E}}\)
is greater than one, while
either\(\nicefrac{\pr{C\vert E}}{\pr{\n C \vert E}}\) or
\(\nicefrac{\pr{S\vert E}}{\pr{\n S \vert E}}\) are not, say when they
are 3 and 0.5, respectively. Distribution fails. And the same problem
would arise with a more stringent threshold as might be appropriate in
criminal cases.

There is a more general problem with Cheng's comparative approach. Much
of the heavy lifting here is done by the strategic splitting of the
defense line into multiple scenarios. Suppose, for illustrative
purposes, \(\pr{H_p\vert E}=0.37\) and the probability of each of the
defense lines given \(E\) is \(0.21\). This means that \(H_p\) wins with
each of the scenarios, so on this approach we should find against the
defendant. But should we? Given the evidence, the accusation is very
likely to be false, because \(\pr{\n H_p \vert E}=0.63\). The problem
generalizes. If, as here, we individualize scenarios by Boolean
combinations of elements of a case, the more elements, the more
alternative scenarios into which \(\n H_p\) needs to be divided. This
normally would lead to lowering even further the probability of each of
them (because now \(\pr{\n H_p}\) needs to be split between more
scenarios). So, if we take this approach seriously, the more elements a
case has, the more at a disadvantage the defense is. This seems
undesirable.

\hypertarget{comparing-likelihoods}{%
\subsection{Comparing likelihoods}\label{comparing-likelihoods}}

Instead of posterior probabilities, likelihoods can also be compared.
The standard of proof would then be as follows: the ratio between
likelihoods \(\nicefrac{\pr{E \vert H_p}}{\pr{E \vert H_d}}\) should be
above a suitable threshold. Note that the posterior ratio
\(\nicefrac{\pr{H_p \vert E}}{\pr{H_d \vert E}}\) from before was
replaced by the likelihood ratio
\(\nicefrac{\pr{E \vert H_p}}{\pr{E \vert H_d}}\) where \(H_d\) and
\(H_p\), as before, need not be exhaustive hypotheses. In civil cases,
the likelihood ratio should perhaps be just be above 1, meaning that the
evidence supports \(H_p\) more strongly than it supports \(H_d\). In
criminal cases, the ratio should be several orders of magnitude above
one. This approach runs into the same problem as Cheng's. It cannot
justify distribution.

We do not provide all the details of the argument. The reasoning is
analogous. Consider the car crash example from before, where \(S\)
stands for the defendant's speeding, \(C\) stands for the statement that
the crash caused neck injury, and \(E\) stands for the total evidence.
The plaintiff's hypothesis \(H_p\) is \(S\et C\). Suppose
\(\nicefrac{\pr{E \vert S\et C}}{\pr{E \vert H_d}}>1\), or in other
words, the combined hypothesis \(S \et C\) has been established by
preponderance of the evidence. The question is whether the individual
hypotheses have been established by the same standard, specifically,
whether \(\frac{\pr{E \vert C}}{\pr{E\vert \neg C}} > 1\) and
\(\frac{\pr{E \vert S}}{\pr{E\vert \neg S}} > 1\). Focusing on a
specific defense hypothesis, \(\n S\et \n C\), the following
holds:\footnote{NEED PROOF}
\begin{align}\label{eq:lr-multiplication-two}
1 < \frac{\pr{E \vert S\et C}}{\pr{E\vert \n S\et \n C}} & = \frac{\pr{E\vert S}\pr{E\vert C}}{\pr{ E \vert \n S}\pr{E \vert \n C}}   
\end{align}

\noindent Note that
\(\nicefrac{\pr{E\vert S}\pr{E\vert C}}{\pr{E \vert \n S}\pr{E\vert \n C}}\)
might be strictly greater than
\(\nicefrac{\pr{E\vert C}}{\pr{E\vert \n C}}\) or
\(\nicefrac{\pr{E\vert S}}{\pr{E\vert \n S}}\). It is possible that
\(\nicefrac{\pr{E \vert S\et C}}{\pr{E \vert H_d}}\) is greater than
one, while either \(\frac{\pr{E \vert C}}{\pr{E\vert \neg C}}\) and
\(\frac{\pr{E \vert S}}{\pr{E\vert \neg S}}\) are not, say when they are
3 and 0.5, respectively. Once again, distribution fails.

\mar{R: take a look at this additional argument, is this good enough?}

A general difficulty remains, related to how the comparative likelihood
strategy is sensitive to the choice of the hypotheses. In many plausible
situations it might be the case there might be hypotheses that one
wishers to compare \(H_1, H_2\) such that \(\pr{E\vert H_1}\) is much
(say, at least a few times) larger than \(\pr{E\vert H_2}\) while
\(\pr{E \vert H_1}\) is still smaller than \(\pr{E \vert \n H_1}\) in
such circumstances, the comparative likelihood strategy seems to be
recommending the acceptance of \(H_1\), even though, intuitively, the
evidence seems to support \(\n H_1\) to a larger extent.

\hypertarget{rejecting-the-conjunction-principle}{%
\section{Rejecting the conjunction
principle?}\label{rejecting-the-conjunction-principle}}

A number of strategies that legal probabilists can pursue to address the
difficulty with conjunction have proven problematic. Perhaps, a
different perspective should be taken here. Observe that the problem
would not arise without the conjunction principle. Should legal
probabilists simply reject this principle? So far we have not challenged
it, but it is time to scrutinize it more closely. In this section, we
provide an epistemic argument and a legal argument to question the
conjunction principle. We also caution that merely rejecting the
conjunction principle will not dissolve the difficulty with conjunction.
More work needs to be done. We take it on in the final section.

\hypertarget{the-legal-argument}{%
\subsection{The legal argument}\label{the-legal-argument}}

Before moving further, it is worth asking what the law says about the
conjunction principle. The answer, perhaps unsurprisingly is, not very
much. We have been assuming that the law agrees with the conjunction
principle. At least, this is what Cohen thought. Matters, however, are
not so clear-cut. Looking at legal practice, the conjunction principle
is an uncertain principle at best.

The best place to look is how jury instructions are formulated. Do they
obey the conjunction principle? To some extent, they do. For example,
here are jury instructions about negligence claims in civil cases:

\begin{quote}
A negligence claim has three elements:
\end{quote}

\begin{quote}
\begin{enumerate}
\def\labelenumi{\arabic{enumi}.}
\tightlist
\item
  {[}Defendant{]} did not use ordinary care;
\end{enumerate}
\end{quote}

\begin{quote}
\begin{enumerate}
\def\labelenumi{\arabic{enumi}.}
\setcounter{enumi}{1}
\tightlist
\item
  {[}Defendant's{]} failure to use ordinary care caused
  {[}Plaintiff's{]} harm; and
\end{enumerate}
\end{quote}

\begin{quote}
\begin{enumerate}
\def\labelenumi{\arabic{enumi}.}
\setcounter{enumi}{2}
\tightlist
\item
  {[}Plaintiff{]} is entitled to damages as compensation for that harm.
\end{enumerate}
\end{quote}

\begin{quote}
{[}Plaintiff{]} must prove each element by a preponderance of the
evidence---that each element is more likely so than not so. If
{[}Plaintiff{]} proves each element, your verdict must be for
{[}Plaintiff{]}. If {[}Plaintiff{]} does not prove each element, your
verdict must be for {[}Defendant{]}.\footnote{Standardized Civil Jury
  Instructions for the District of Columbia (Civil Jury Instructions,
  revised edition 2017), Sec. 5.01.}
\end{quote}

\noindent The elements are explicitly separated and the standard of
proof is applied to each element separately. This seems to confirm the
conjunction principle. Other jury instructions are more ambiguous:

\begin{quote}
In order to find that the plaintiff is entitled to recover, you must
decide it is more likely true than not true that:
\end{quote}

\begin{quote}
\begin{enumerate}
\def\labelenumi{\arabic{enumi}.}
\tightlist
\item
  the defendant was negligent;
\end{enumerate}
\end{quote}

\begin{quote}
\begin{enumerate}
\def\labelenumi{\arabic{enumi}.}
\setcounter{enumi}{1}
\tightlist
\item
  the plaintiff was harmed; and
\end{enumerate}
\end{quote}

\begin{quote}
\begin{enumerate}
\def\labelenumi{\arabic{enumi}.}
\setcounter{enumi}{2}
\tightlist
\item
  the defendant's negligence was a substantial factor in causing the
  plaintiff's harm.\footnote{Alaska Civil Pattern Jury Instructions,
    Sec. 3.01 (Civil Pattern Jury Instructions 2017).)}
\end{enumerate}
\end{quote}

\noindent The elements are still separated, but the standard of proof
(`more likely than not') applies to the conjunction as a whole, not the
individual claims. At least, these jury instruction are at best
ambiguous between an atomistic reading (the standard applies to each
claim separately) and a holistic reading of the standard of proof (the
standard applies to the conjunction). Only the atomistic reading would
justify the conjunction principle.

This quick survey of jury instructions gives some reassurance that,
should we decide to reject the conjunction principle, we would not
violate a well-entrenched, indispensable legal
principle.\todo{CITE SOBER AND SCHAWRTZ WHI DIUD THE EMPIRICAL ANALYSIS OF JURY INSTRUCTIONS}

\hypertarget{risk-accumulation}{%
\subsection{Risk accumulation}\label{risk-accumulation}}

Beside legal uncertainty about the conjunction principle, there are also
independent theoretical reasons to question the principle. In current
discussions in epistemology about knowledge or justification, a
principle similar to the conjunction principle has been contested
because it appears to deny the fact that risks of error accumulate
(Kowalewska, 2021). If one is reasonably sure about the truth of each
claim considered separately, one should not be equally reasonably sure
of their conjunction. You have checked each page of a book and found no
error. So, for each page, you are reasonably sure there is no error.
Having checked each page and found no error, can you be equally
reasonably sure that the book as a whole contains no error? Not really.
As the number of pages grow, it becomes virtually certain that there is
at least one error in the book you have overlooked, although for each
page you are reasonably sure there is no error (Makinson, 1965). A
reasonable doubt about the existence of an error, in one page or
another, creeps up as one considers more and more pages. The same
observation applies to other contexts, say product quality control. You
may be reasonably sure, for each product you checked, that it is free
from defects. But you cannot, on this basis alone, be equally reasonably
sure that all products you checked are free from defects. Since the
risks of error accumulate, you must have missed at least one defective
product.

Risk accumulation challenges aggregation: even if the probability of
several claims, considered individually, is above a threshold \(t\),
their conjunction need not be above \(t\). It does not, however,
challenge distribution. If, all risks considered, you have good reasons
to accept a conjunction, no further risk is involved in accepting any of
the conjuncts separately. This is also mirrored by what happens with
probabilities. If the probability of the conjunction of several claims
is above \(t\), so is the probability of each individual claim.
\mar{R: What's our take on risk accummulation?}

The standard of proof in criminal or civil cases can be understood as a
criterion concerning the degree of risk that judicial decisions should
not exceed. If this understanding of the standard of proof is correct,
the phenomenon of risk accumulation would invalidate the conjunction
principle, specifically, it would invalidate aggregation. It would
longer be correct to assume that, if each element is proven according to
the applicable standard, the case as a whole is proven according to the
same standard. And, in turn, if the conjunction principle no longer
holds, the conjunction paradox will disappear. Or will it?

\hypertarget{atomistic-and-holistic-approches}{%
\subsection{Atomistic and holistic
approches}\label{atomistic-and-holistic-approches}}

Matters are not so straightfoward, however. Suppose legal probabilists
do away with the conjunction principle. Now what? How should they define
standards of proof? Two immediate options come to mind, but neither is
without problems.

Let's stipulate that, in order to establish the defendant's guilt beyond
a reasonable doubt (or civil liability by preponderance of the evidence
or clear and convincing evidence), the party making the accusation
should establish each claim, separately, to the requisite probability,
say at least .95 (or .5 in a civil case), without needing to establish
the conjunction to the requisite probability. Call this the
\textit{atomistic account}. On this view, the prosecution could be in a
position to establish guilt beyond a reasonable doubt without
establishing the conjunction of different claims with a sufficiently
high probability. This account would allow convictions in cases in which
the probability of the defendant's guilt is relatively low, just because
guilt is a conjunction of several independent claims that separately
satisfy the standard of proof. For example, if each constituent claim is
established with .95 probability, a composite claim consisting of five
subclaims---assuming, as usual, probabilistic independence between the
subclaims---would only be established with probability equal to .77, a
far cry from proof beyond a reasonable doubt. This is counterintuitve,
as it would allow convictions when the defendant is not very likely to
have committed the crime. A similar argument can be run for the civil
standard of proof `preponderance of the evidence.' Under the atomistic
account, the composite claim representing the case as a whole would
often be established with a probability below the required threshold.
The atomistic approach is a non-starter.

Another option is to require that the prosecution in a criminal case (or
the plaintiff in a civil case) establish the accusation as a whole---say
the conjunction of \(A\) and \(B\)---to the requisite probability. Call
this the \textit{holistic account}. This account is not without problems
either.

The standard that applies to one of the conjuncts would depend on what
has been achieved for the other conjuncts. For instance, assuming
independence, if \(\pr{A}\) is \(.96\), then \(\pr{B}\) must be at least
\(.99\) so that \(\pr{A\et B}\) is above a \(.95\) threshold. But if
\(\pr{A}\) is \(.9999\), then \(\pr{B}\) must only be slightly greater
than \(.95\) to reach the same threshold. Thus, the holistic account
might require that the elements of an accusation be proven to different
probabilities---and thus different standards---depending on how well
other claims have been established. This result runs counter to the
tacit assumption that each element should be established to the same
standard of proof. \raf{M: Cite Ubaniak's paper on this point.}

Fortunately, this challenge can be addressed. It is true that different
elements will be established with different probabilities, depending on
the probabilities of the other elements. But this follows from the fact
that the prosecution or the plaintiff may choose different strategies to
argue their case. They may decide that, since they have strong evidence
for one element and weaker evidence for the other, one element should be
established with a higher probability than the other. What matters is
that the case as a whole meets the required threshold, and this
objective can be achieved via different means. What will never happen is
that, while the case as a whole meets the threshold, one of the
constituent elements does not. As seen earlier, the probability of the
conjunction never exceeds the probability of one of the conjunct, or in
other words, distribution is never violated.

A more difficult challenge is the observation that the proof of
\(A\et B\) would impose a higher requirement on the separate
probabilities of the conjuncts. If the conjunction \(A\et B\) is to be
proven with at least .95 probability, the individual conjuncts should be
established with probability higher than .95. So the more constituent
claims, the higher the posterior probability for each claim needed for
the conjunction to meet the requisite probability threshold.

This difficulty is best appreciated by running some numbers. Assume, for
the sake of illustration, the independence and equiprobability of the
constituent claims. If a composite claim consists of \(k\) individual
claims, these individual claims will have to be established with
probability of at least \(t^{1/k}\), where \(t\) is the threshold to be
applied to the composite
claim.\footnote{Let $p$ the probability of each constituent claim. To meet threshold $t$, the probability of the composite claim, $p^k$, should satisfy the constraint $p^k>t$, or in other words, $p>t^{1/k}$.}
For example, if there are ten constituent claims, they will have to be
proven with \(.5^{1/10}=.93\) even if the probability threshold is only
\(.5\). If the threshold is more stringent, as is appropriate in
criminal cases, say \(.95\), each individual claim will have to be
proven with near certainty. This would make the task extremely demanding
on the prosecution, if not downright impossible. If there are ten
constituent claims, they will have to be proven with
\(.95^{1/10}=.995\). So the plaintiff or the prosecution would face the
demanding task of establishing each element of the accusation beyond
what the standard of proof would seem to require.

We reached an impasse. Under the atomistic approach, the standard is too
lax because it allows for findings of liability when the defendant quite
likely committed no wrong. Under the holistic approach, the standard is
too demanding on the prosecution (or the plaintiff) because it requires
the individual claims to be established with extremely high
probabilities. Bummer.

\hypertarget{not-asking-too-much}{%
\subsection{Not asking too much}\label{not-asking-too-much}}

Consider again the holistic approach. It is true that the individual
elements (the individual conjuncts) should be established with a higher
probability than the case as a whole (the conjunction). This would seem
to impose an unreasonably stringent burden of proof on the prosecution
or the plaintiff. But the burden might not be as unreasonable as it
appears at first. As Dawid (1987) pointed out, in one of the earliest
attempts to solve the conjunction paradox from a probabilistic
perspective, the prior probabilities of the conjuncts will also be
higher than the prior probability of their conjunction:

\begin{quote}
\dots it is not asking too much of the plaintiff to establish the case as a whole with a posterior probability exceeding one half, even though this means  that the several component issues must be established with much larger posterior probabilities; for the \textit{prior}  probabilities of the components will also be correspondingly larger, compared with that of their conjunction~[p.~97].
 \end{quote}

Dawid's proposal is compelling. Still, why exactly is it `not asking too
much' to establish the individual conjuncts by a higher threshold than
the case as a whole? The prior probabilities of the conjuncts are surely
higher than the prior probability of the conjunction. But what is the
notion of `not asking too much' at work here? Dawid might be
recommending---as the rest of his paper suggests---that the standard of
proof no longer be understood in terms of just posterior probabilities.
Measures of how strongly each claim is being supported by the evidence,
such as the Bayes' factor of the likelihood ratio, account for the
difference between prior and posterior probabilities. So, presumably,
Dawid is recommending these measures as better suited to formalize the
standard of proof.

Now, as the reader will have realized, we have pursued Dawid's strategy
already. This strategy can justify, on purely probabilistic grounds, one
direction of the conjunction principle: aggregation. The evidential
support---measured by the Bayes' factor or the likelihood ratio---for
the conjunction often exceeds the individual support for (at least one
of) the individual claims. This is a success, especially because the
failure of aggregation motivated Cohen's formulation of the conjunction
paradox. Unfortunately, we have seen that this strategy invalidates a
previously unchallenged direction of the conjunction principle:
distribution.

\hypertarget{the-proposal}{%
\section{The proposal}\label{the-proposal}}

Here is where we have gotten so far. There might be good reasons to
reject the conjunction principle, but rejecting it does not
automatically solve the difficulty with conjunction. We still need a
theory that explains how individual claims are combined, together with
the available evidence, to form more complex claims, say the claim that
the defendant committed the crime for which they were charged. The
conjunction principle provides a recipe---a very simple one at that---to
combine individual claims and form conjunctive claims. That recipe might
not be right. If it is not, a good theory of the standard of proof
should still provide an alternative recipe for combing individual
claims.

Our proposal is inspired by the story model of adjudication (Pennington
\& Hastie, 1993; Wagenaar, Van Koppen, \& Crombag, 1993) and the
relative plausibility theory (Allen \& Pardo, 2019; Pardo \& Allen,
2008). It posits that prosecutors and plaintiffs should aim to establish
a unified narrative of what happened or explanation of the evidence, not
establish each individual element of wrongdoing separately. As we shall
see, any attempt to proceed in a piecemeal manner implicitly requires,
sooner or later, to weave the different elements together into a unified
whole.

Our argument consists of two parts. First, the guilt or civil liability
of a defendant on trial cannot be equated with a generic claim of guilt
or civil liability as defined in the law. The claim against a defendant
facing trial should always be grounded in specific details. Call this
the specificity argument. Second, it is erroneous to think of someone's
guilt or civil liability as the mere conjunction of separate claims. The
separate claims must be unified, not just added up in a conjunction.
Call this the unity argument.

\hypertarget{the-specificity-argument}{%
\subsection{The specificity argument}\label{the-specificity-argument}}

We start with the specificity argument. The probabilistic interpretation
of proof standards usually posits a threshold that applies to the
posterior probability of a \emph{generic} hypothesis, such as the
defendant is guilty of a crime, call it \(G\), or civilly liable, call
it \(L\). In criminal cases, the requirement is formulated as follows:
the evidence \(E\) presented at trial establishes guilt beyond a
reasonable doubt provided \(\pr{G \vert E}\) is above a suitable
threshold, say .95. The threshold is lower in civil trials. Civil
liability is proven by preponderance provided \(\pr{L \vert E}\) is
above a suitable threshold, say .5.

This formulation conflates two things. The wrongdoing as defined in the
applicable law is one thing. The way in which the wrongdoing is
established in court is another thing. The wrongdoing is defined in a
generic manner and is applicable across a class of situations, whereas
the way the wrongdoing is established in court is specific to a
situation and tailored to the individual defendant. A prosecutor in a
criminal case does not just establish that the defendant assaulted the
victim in one way or another, but rather, that the defendant behaved in
such and such a manner in this time and place, and that the behavior in
question fulfills the legal definition of assault. The requirement of
specificity is, for one thing, a consequence of the fact that defendants
have a right to be informed with sufficient degree of detail, and also
that they should be in a position to prepare a defense.\footnote{ADD
  CASE LAW REFERENCES TO BUTRESS THIS POINT.}

If this is correct, the probabilistic interpretation of proof standards
should be revised. The generic claim that the defendant is guilty or
civilly liable should be replaced by a more fine-grained hypothesis,
call it \(H_p\), the hypothesis put forward by the prosecutor (or the
plaintiff in a civil case), for example, that the defendant, given
reasonably well-specified circumstances, approached the victim, pushed
and kicked the victim to the ground, and then run away. Hypothesis
\(H_p\) is a more precise description of what happened that entails the
defendant committed the wrong. In defining proof standards, instead of
saying---generically---that \(\pr{G \vert E}\) or \(\pr{L \vert E}\)
should be above a suitable threshold, a probabilistic interpretation
should read: civil or criminal liability is proven by the applicable
standard provided \(\Pr(H_p \vert E)\) is above a suitable threshold,
where \(H_p\)is a reasonably specific description of what happened
according to the prosecutor or the plaintiff.

This revision of the probabilistic interpretation of standards of proof
may appear inconsequential, but it is not. It is the revision we invoked
to address the puzzles of naked statistical evidence in Chapter XX.
\todo{REFERENCE TO EARLIER CHAPTER} Recall the gist of the argument.
Consider the prisoner hypothetical, a standard example of naked
statistical evidence. The naked statistics \(E_s\) make the prisoner on
trial .99 likely to be guilty, that is, \(\pr{G \vert E_s} =.99\). It is
\(.99\) likely that the prisoner on trial is one of those who attacked
and killed the guard. This is a generic claim. It merely asserts that
the prisoner was -- with very high probability -- one of those who
killed the guard, without specifying what he did, how he partook in the
killing, what role he played in the attack, etc. If the prosecution
offered a more specific incriminating hypothesis, call it \(H_p\), the
probability \(\pr{H_p \vert E_{s}}\) of this hypothesis based on the
naked statistical evidence \(E_s\) would be well below \(.99\), even
though \(\pr{G \vert E_s}=.99\). That the prisoner on trial is most
likely guilty is an artifact of the choice of a generic hypothesis
\(G\). When this hypothesis is made more specific---as should be---this
probability drops significantly. And the puzzle of naked statistical
evidence disappears.

\hypertarget{the-unity-argument}{%
\subsection{The unity argument}\label{the-unity-argument}}

The specificity argument addresses the problem of naked statistical
evidence, but also provides the necessary background for addressing the
difficultly about conjunction. In the traditional formulation, not only
are \(G\) and \(L\) understood as generic claims. They are also
understood as \emph{mere conjunctions} of simpler claims that correspond
to the elements of wrongdoing in the applicable law. Since the
probability of a conjunction is often lower than the probability of its
conjuncts, the individual claims can be established with a suitably high
probability that meets the required threshold even though the
conjunction as a whole fails to meet the same threshold. This mismatch
gives rise to the difficulty with conjunction.

But someone's guilt (and the same applies to civil liability) cannot be
the mere conjunction of the claims corresponding to the elements of
wrongdoing as defined in the law. Someone's guilt is a state of affairs
that is described by a well-specified series of events that possess a
coherent, structured unity. These events, taken as a whole, can be
subsumed under the legal definition that consists of several discrete
elements. The conjunction paradox assumes that criminal or civil
wrongdoings are the mere collection of separate elements. The law is
more complicated. Legal definitions often impose a structure on how the
different elements relate to one another. They are not merely separate
items that should be proven one by one. They often form a structured
unity.\footnote{Consider, for example, the allegation of negligent
  misrepresentation to be established by clear and convincing evidence.
  As an illustration, we follow the jury instructions of the State of
  Washington. (EXACT REFERENCE) These are the elements a plaintiff
  should establish: (E1) defendant supplied false information to guide
  plaintiff in their business transaction; (E2) defendant knew the false
  information was supplied to guide plaintiff; (E3) defendant was
  negligent in obtaining or communicating the false information; (E4)
  plaintiff relied on the false information; (E5) plaintiff's reliance
  on the false information was reasonable; and (E6) the false
  information proximately caused damages to the plaintiff. For the
  plaintiff to prevail in this type of case, they should prove each
  element by the required standard. What does that actually require? The
  plaintiff should first establish, at a minimum, that the defendant
  supplied false information for the guidance of the plaintiff in the
  course of a business transaction. Say the defendant, a owner of a
  vacation resort, told plaintiff, a travel agent, that the resort
  included amenities that were not actually there, and the plaintiff
  decided to book several clients at the defendant's resort instead of
  other resorts. The plaintiff could offer copies of emails
  communication or screenshots from the resort's website. This would
  take care of the first element. The second element qualifies the
  first, in that the defendant \emph{knew} they were supplying false
  information to guide the business transaction. Clearly, establishing
  the second element presupposes having already established the first
  since the second element is a qualification of the first. The truth of
  the second element entails the truth of the first. If the defendant
  knowingly supplied false information (second element), they did supply
  false information (first element). The third element requires to show
  that the defendant was negligent in obtaining or communicating the
  false information. This is another qualification that applies to the
  first element and cannot be proven without having already proven the
  first element. The fourth and fifth element should be understood
  together. The plaintiff relied on the false information (fourth
  element) and such reliance was reasonable (fifth element). In turn,
  establishing the fourth and fifth element presupposes that the
  plaintiff did supply false information to begin with (first element).
  Finally, the sixth element concerns causation of harm. This is again a
  qualification of the first element and cannot be established without
  having already established the first element.}

But what if the law does not impose any structure among the different
elements of wrongdoing? Consider this case in which only two elements
must be proven. Element 1: the defendant's conduct caused a bodily
injury to victim. Element 2: the defendant's conduct consisted in
reckless driving. Call this criminal offense ``vehicular
assault.''\footnote{CITE RELATED LAWS} The two elements are not
independent, but they each add novel information. It could be that the
defendant's driving caused an injury to victim, but the driving was not
reckless, or the driving was reckless, but no injury ensued. Neither
element is presupposed by the other. The law does not impose any
specific structure between the elements. But the same unity argument
would still apply at a conceptual level.

Consider how one could go about establishing the claim of reckless
driving that caused injury to the victim. One option is to offer a
detailed reconstruction of what happened. The reconstruction could go
something like this. The defendant was driving above the speed limit,
veering left and right. The defendant's reached a school crosswalk when
children were getting out of school. The defendant hit a child on the
crosswalk who was then pushed against a light pole on the sidewalk
incurring a head injury. This story is supported by plenty of evidence:
other children, people standing around, police officers, paramedics.
There is plenty of supporting evidence as the incident occurred in the
middle of the day. Taken at face value, this story does establish both
elements: reckless driving and cause of injury. Parts of the story are
relevant for element 1 (reckless driving) and others are relevant for
element 1 (cause of injury). The two cannot be neatly separated,
however. Still, what is crucial is that the different parts of the story
are part of the same episode, the same unit of wrongdoing.

Could the prosecutor prove vehicular assault in a piecemeal manner?
Suppose the prosecutor attempted to do that, by establishing, first,
that the defendant drove recklessly, and second---\emph{separately from
the first element}---that the defendant's action caused injury. As noted
before in the specificity argument, it is not enough to establish that
the defendant drove recklessly at some point in time somewhere. Nor is
it enough to establish that the defendant's action caused injury. The
prosecutor should offer a specific story detailing what happened, a
story relevant for the first element and a story relevant for the second
element. Say this expectation of specificity is met. Suppose the
prosecutor did not simply establish generic element 1 and generic
element 2 of the charge, but rather, a reasonably detailed story which,
if true, would establish element 1 and a story which, if true, would
establish element 2. Wouldn't that be enough? It wouldn't. Even if each
element---more precisely, each story associated with each element---was
established by the required standard, there would be something missing
here.

The prosecutor should establish that the two elements---reckless driving
and injury, or the two stories associated with the two elements---are
part of the same unity of wrongdoing. It must me \emph{this} reckless
driving that caused \emph{this} injury. So, under the piecemeal
approach, the prosecutor would be tasked with establishing three claims:
(1) the defendant, in some well-specified circumstances, was driving
reckless; (2) the defendant, in some well-specified circumstances,
caused injury to the victim; and (3) the well-specified circumstances in
(1) and (2) are part of the same episode. But once (3) is established,
the prosecutor would have effectively established the charge by the
required standard in accordance with the holistic approach. The
prosecutor did not only establish each separate element (two separate
stories) but also combine the two elements (the two stories) together.
Once the piecemeal approach is pursued to its logical conclusion, it
coincides with the holistic approach.\footnote{We should be clear that
  it is not enough for the prosecutor or plaintiff to provide
  well-specified narrative in support of their allegations, even when
  they are well-supported by the evidence. When the two narratives are
  combined into one narrative, its probability could well be below the
  threshold. If we only require that each element-specific narrative be
  proven, a defendant could be found criminally or civilly liable even
  though it is unlikely that they committed the alleged wrongful act.
  This counter-intuitive result is similar to the one that arose with
  the atomistic approach.}

Let's summarize the unity argument in schematic form. If the prosecutor
or the plaintiff is expected to establish claim \(A\) and \(B\) by the
required standard, what the law actually requires---even in terms of the
piecemeal approach---is (1) to establish \(A\); (2) to establish \(B\);
and (3) to establish \(A\) and \(B\) are part of the same unit of
wrongdoing by the required standard. Item (3) is often implicit, which
leaves the impression that the law only requires to establish (1) and
(2) separately. Interestingly, (3) entails (1) and (2). In fact, (3)
amounts to establishing a unified story, narrative or theory about what
happened. Such a narrative should be subsumed under the different
elements of wrongdoing as defined in the law. The piecemeal approach and
the revised holistic approach, therefore, converge.

To be sure, not all wrongful acts, in civil or criminal cases, require
the prosecutor or the plaintiff to establish a unified
\emph{spatio-temporal} narrative. It might not be necessary to show that
all elements of an offense occurred at the same point in time or in
close succession one after the other. Some wrongful acts may consist of
a pattern of acts that stretches for several days, months or even years.
There may be temporal and spatial gaps that cannot not be filled. We
consider several of these examples in our discussion of naked
statistical evidence in Chapter XX. \todo{SEE PREVIOUS CHAPTER} Be that
as it may, an accusation of wrongdoing in a criminal or civil case
should still have a degree of cohesive unity. The acts and occurrences
that constitute the wrongdoing should belong to the same wrongful act.
It is this unity which the plaintiff and the prosecution must establish
when they make their case. One way to establish this unity is by
providing a unifying narrative, but this need not be the only way.
Perhaps the expressions `theory' or `explanation' are more apt than
`narrative' or `story.'

\hypertarget{probability-specificity-and-completeness}{%
\subsection{Probability, specificity and
completeness}\label{probability-specificity-and-completeness}}

We emphasize the distinction between a narrative (or theory, story,
explanation, account) and a mere conjunctions of elements of wrongdoing
\(E_1\wedge E_1 \wedge \dots \wedge E_k\). The narrative describes one
way among many of instantiating the conjunction. This distinction is
important. The claims that constitute a narrative or unified explanation
need not map neatly onto the elements of the wrongdoing. The narrative
will comprise claims about the evidence itself and how the evidence
supports other claims in the narrative, say that witnesses were standing
around when the defendant's car hit the child. The narrative or
explanation will not only comprise a description of what happened but
also of how we know this is what happened.

The distinction between narrative and the mere conjunction of elements
matters for how we should understand the standard of proof. Other things
being equal, the conjunction is more probable on the evidence than the
narrative, and each conjunct even more probable. But this does not mean
that the mere conjunction is established by a higher standard of proof
than the narrative. As we argued in Chapter XX on naked statistical
evidence, \todo{REFERENCE TO EARLIER CHAPTERS}
\mar{R: remember to add your paper to the folder.} a highly probable
narrative that nevertheless lacks the desired degree of specificity will
fail to meet the standard of proof. By contrast, a more specific
narrative that is otherwise less probable than the mere conjunction
might well meet the standard. On this account, the standard of proof
consists of two criteria: (1) the posterior probability of the proposed
narrative (or theory, story, explanation) given the evidence presented
at trial; and (2) the degree of specificity, coherence and unity of the
narrative or explanation.

\mar{R: added this brief survey, take a look!}

Are we giving up on legal probabilism, then? We are giving up on
\emph{traditional} legal probabilism. Even though ideas such as
specificity, coherence and unity cannot be captured by the posterior
probability alone, they can be formalized as properties of Bayesian
networks. Early development of this approach can be found in (Urbaniak,
2018). While we will develop this approach further later on
\todo{REF CHAPTER}, the general idea is this. Once we represent the
relevant claims and pieces of evidence and their interaction as a
Bayesian network (or a set thereof), some of the binary nodes
corresponding to various propositions are marked as evidence nodes, and
some qualify as narration nodes---these are the nodes that various sides
or proposed scenarios disagree about. Within this setup various
intuitively needed notions can be explicated. For instance, an accusing
narration should ``make sense'' of evidence in the following sense: any
item of evidence presented, should have sufficiently high posterior
probability in the network updated with the total evidence obtained. In
contrast, a defending narration is supposed to explain evidence in a
different sense: if the defense story is rather minimal and mostly
constitutes in rebutting the accusations, it isn't reasonable to expect
the defense to explain all pieces of evidence, and it is not reasonable
to expect the defense to provide a story explaining how each piece of
evidence came into existence. Rather, the defense should argue that the
probability of the evidence being as it is while the defense's narration
is true isn't below a rejection threshold. As another example, a piece
of evidence is missing if there is an evidential proposition such that
the probability of it being instantiated (given what is known about
cases of a given type and about the particularities of a given case) is
non-negligible, but it is not included in the evidence. Yet another
example: a narration contains gaps just in case there are claims that
the narration should choose from (nodes that a narration should consider
instantiated), but it does not do so. This, of course, is very hand-wavy
at this stage, but at this point we just want to leave the reader with
the impression that more can be explicated with Bayesian networks than
one might initially expect, leaving detailed development to a different
chapter. \todo{REF TO APPROPRIATE CHAPTER}

Further, this analysis of the standard of proof---which combines two
criteria, posterior probability and specificity---can be evaluated using
concepts from probability theory that are not posterior probabilities.
To illustrate, compare a trial system that convicts defendants on the
basis of claims that generic but highly probable, as opposed to a trial
system that convicts defendants on the basis of claims that are more
specific but less probable. A natural question to ask at this point is,
which trial system will make fewer mistakes---fewer false convictions
and false acquittals---in the long run? The answer is not obvious. But
the question can be made precise in the language of probability. The
question concerns the diagnostic properties of the two trial systems,
such as their rate of false positives and false negatives. We examine
this question in Chapter YY, including a simulationist study of the
impact of various features of narrations in their expected accuracy.
\todo{REFERENCE TO LATER CHAPTER} To anticipate, we argue that more
specific claims are liable to more extensive \emph{direct} adversarial
scrutiny than generic claims. The more specific someone's claim, the
more liable to be attacked. At the same time, if a specific claim
resists adversarial scrutiny, it becomes more firmly established than a
less specific claim that survived scrutiny by evading the questions. So
specificity plays an accuracy-conducive role even though more specific
claims are, other things being equal, less probable than more generic
claims.\footnote{CITE POPPER HERE} That is why specificity should be an
important ingredient in any theory of the standard of proof.

Another ingredient worth adding to posterior probability and specificity
is the completeness of the evidence presented at trial. Could the
probability of someone's guilt be extremely high just because the
evidence presented is one-sided and missing crucial pieces of
information? It certainly can. If the probability of guilt is high
because the evidence is partial, guilt was not proven beyond a
reasonable doubt. It is a matter of dispute whether knowledge about the
partiality of the evidence should affect the posterior probability.
After all, if we know that evidence about a hypothesis is missing,
shouldn't we revise the assessment of the posterior probability of the
hypothesis? This may be true, but the problem is that the content of the
missing evidence is unknown. The missing evidence might increase or
decrease this probability. We cannot know that without knowing what the
evidence turns out to be. If we knew how the missing evidence would
affect our judgment about the defendant's guilt, the evidence would no
longer be---strictly speaking---missing.

\mar{R: do you still want to add these?}

THIGNS TO ADD: 1. ROLE OF COMPLETENESS OF EVIDENCE, SEE OREGON DNA CASE,
MEANT TO SHOW THAT HOLISTIC AOPPROACH IS NOT AD HOC, WEIGHT, RESILIENCE
IMPORTANT FOR ASESSING STRENGHT OF EVIDENCE AND STANDART OF PROOF, GUILT
COULD BE HIGHLY PROBABLE GIVEN AVAILABLE EVIDEMCE, BUT THIS NEED NOT BE
ENOUGH IF EVIDENCE THAT SHOULD BE THERE IS MISSING

\hypertarget{the-conjuction-principle-revised}{%
\subsection{The conjuction principle
revised}\label{the-conjuction-principle-revised}}

What does this discussion tell us about the conjunction paradox? Say we
take seriously the idea that the standard of proof is a legal device to
optimize the satisfaction of the following three criteria:

\begin{enumerate}
\def\labelenumi{\arabic{enumi}.}
\item
  The defendant's civil or criminal liability must be sufficiently high
\item
  The narrative, story, theory, explanation that details the defendant's
  civil or criminal liability should be sufficiently (reasonably)
  specific
\item
  The supporting evidence should be sufficiently (reasonably) complete
\end{enumerate}

The conjunction paradox no longer arises given this conception of the
standards of proof. That prosecutors and plaintiffs should aim to
establish a well-specified, unified account of the wrongdoing would
trivialize the conjunction principle and dissolve the difficulty about
conjunction. Suppose the prosecutor established a narrative \(N\) by a
very high probability, say above the required threshold for proof beyond
a reasonable doubt. Denote the elements of wrongdoing by
\(E_1, E_2, \dots\). Then,
\[\text{ $\pr{E_1\wedge E_1 \wedge \dots \wedge E_k \vert N}=\pr{C_i \vert N} = 1$ for any $i=\{1, 2, ..., k\}$}.\]
\noindent Both directions of the conjunction principle, aggregation and
distribution, are now trivially satisfied. Once we condition on the
narrative \(N\), each individual claim has a probability of one and thus
their conjunction also has a probability of one. The conjunction
principle is reduced to a deductive check that the elements of the
wrongdoing follow deductively from the narrative put forward. The
narrative, however, has a probability short of one, up to whatever value
is required to meet the governing standard of proof. The standard
applies to the narrative as a whole, and only indirectly---via a
deductive check---to the individual elements. This trivialization of the
conjunction principle is unsurprising and desirable given that no lawyer
has ever been concerned with the reliability of conjunction elimination
or introduction.

\hypertarget{references}{%
\section*{References}\label{references}}
\addcontentsline{toc}{section}{References}

\hypertarget{refs}{}
\begin{CSLReferences}{1}{0}
\leavevmode\hypertarget{ref-Allen1986A-Reconceptuali}{}%
Allen, R. J. (1986). A reconceptualization of civil trials. \emph{Boston
University Law Review}, \emph{66}, 401--437.

\leavevmode\hypertarget{ref-AllenPardo2019relative}{}%
Allen, R. J., \& Pardo, M. (2019). Relative plausibility and its
critics. \emph{The International Journal of Evidence {\&} Proof},
\emph{23}(1-2), 5--59. \url{https://doi.org/10.1177/1365712718813781}

\leavevmode\hypertarget{ref-allen2013}{}%
Allen, R. J., \& Stein, A. (2013). Evidence, probability and the burden
of proof. \emph{Arizona Law Journal}, \emph{55}, 557--602.

\leavevmode\hypertarget{ref-Bernoulli1713Ars-conjectandi}{}%
Bernoulli, J. (1713). \emph{Ars conjectandi}.

\leavevmode\hypertarget{ref-cheng2012reconceptualizing}{}%
Cheng, E. (2012). Reconceptualizing the burden of proof. \emph{Yale LJ},
\emph{122}, 1254.

\leavevmode\hypertarget{ref-clermont2012aggregation}{}%
Clermont, K. M. (2012). Aggregation of probabilities and illogic.
\emph{Ga. L. Rev.}, \emph{47}, 165--180.

\leavevmode\hypertarget{ref-clermont2013paradox}{}%
Clermont, K. M. (2013). Death of paradox: The killer logic beneath the
standards of proof. \emph{Notre Dame Law Review}, \emph{88}(3),
1061--1138.

\leavevmode\hypertarget{ref-Cohen1977The-probable-an}{}%
Cohen, J. L. (1977). \emph{The probable and the provable}. Oxford
University Press. \url{https://doi.org/10.2307/2219193}

\leavevmode\hypertarget{ref-cohen1981can}{}%
Cohen, L. J. (2008). Can human irrationality be experimentally
demonstrated? In J. E. Adler \& L. J. Rips (Eds.), \emph{REASONING
studies of human inference and its foundations} (pp. 136--155).
Cambridge University Press.

\leavevmode\hypertarget{ref-dawid1987difficulty}{}%
Dawid, A. P. (1987). The difficulty about conjunction. \emph{The
Statistician}, 91--97.

\leavevmode\hypertarget{ref-Dekay1996}{}%
Dekay, M. L. (1996). The difference between {B}lackstone-like error
ratios and probabilistic standards of proof. \emph{Law and Social
Inquiry}, \emph{21}, 95--132.

\leavevmode\hypertarget{ref-haack2011legal}{}%
Haack, S. (2014). Legal probabilism: An epistemological dissent. In
\emph{{Haack2014-HAAEMS}} (pp. 47--77).

\leavevmode\hypertarget{ref-Kaplan1968decision}{}%
Kaplan, J. (1968). Decision theory and the fact-finding process.
\emph{Stanford Law Review}, \emph{20}(6), 1065--1092.

\leavevmode\hypertarget{ref-kaye79}{}%
Kaye, D. H. (1979). The laws of probability and the law of the land.
\emph{The University of Chicago Law Review}, \emph{47}(1), 34--56.

\leavevmode\hypertarget{ref-Kowalewska2021conjunction}{}%
Kowalewska, A. (2021). Reasoning without the conjunction closure.
\emph{Episteme}, 1--14. \url{https://doi.org/10.1017/epi.2020.53}

\leavevmode\hypertarget{ref-Laplace1814}{}%
Laplace, P.-S. (1814). \emph{Essai philosophique sur les probabilités}.

\leavevmode\hypertarget{ref-laudan2006truth}{}%
Laudan, L. (2006). \emph{Truth, error, and criminal law: An essay in
legal epistemology}. Cambridge University Press.

\leavevmode\hypertarget{ref-Makinson1965-MAKTPO-2}{}%
Makinson, D. C. (1965). The paradox of the preface. \emph{Analysis},
\emph{25}(6), 205--207.

\leavevmode\hypertarget{ref-Pardo2008judicial}{}%
Pardo, M. S., \& Allen, R. J. (2008). Judicial proof and the best
explanation. \emph{Law and Philosophy}, \emph{27}(3), 223--268.

\leavevmode\hypertarget{ref-penn1993}{}%
Pennington, N., \& Hastie, R. (1993). Reasoning in explanation-based
decision making. \emph{Cognition}, \emph{49}, 123--163.

\leavevmode\hypertarget{ref-schwartz2017ConjunctionProblemLogic}{}%
Schwartz, D. S., \& Sober, E. R. (2017). The {Conjunction Problem} and
the {Logic} of {Jury Findings}. \emph{William \& Mary Law Review},
\emph{59}(2), 619--692.

\leavevmode\hypertarget{ref-spottswood2016}{}%
Spottswood, M. (2016). Unraveling the conjunction paradox. \emph{Law,
Probability and Risk}, \emph{15}(4), 259--296.

\leavevmode\hypertarget{ref-Stein05}{}%
Stein, A. (2005). \emph{Foundations of evidence law}. Oxford University
Press.

\leavevmode\hypertarget{ref-urbaniak2018narration}{}%
Urbaniak, R. (2018). Narration in judiciary fact-finding: A
probabilistic explication. \emph{Artificial Intelligence and Law},
1--32. \url{https://doi.org/10.1007/s10506-018-9219-z}

\leavevmode\hypertarget{ref-wagenaar1993anchored}{}%
Wagenaar, W., Van Koppen, P., \& Crombag, H. (1993). \emph{Anchored
narratives: The psychology of criminal evidence.} St Martin's Press.

\end{CSLReferences}

\end{document}
