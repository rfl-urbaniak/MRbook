% Options for packages loaded elsewhere
\PassOptionsToPackage{unicode}{hyperref}
\PassOptionsToPackage{hyphens}{url}
\PassOptionsToPackage{dvipsnames,svgnames*,x11names*}{xcolor}
%
\documentclass[
  10pt,
  dvipsnames,enabledeprecatedfontcommands]{scrartcl}
\usepackage{amsmath,amssymb}
\usepackage{lmodern}
\usepackage{ifxetex,ifluatex}
\ifnum 0\ifxetex 1\fi\ifluatex 1\fi=0 % if pdftex
  \usepackage[T1]{fontenc}
  \usepackage[utf8]{inputenc}
  \usepackage{textcomp} % provide euro and other symbols
\else % if luatex or xetex
  \usepackage{unicode-math}
  \defaultfontfeatures{Scale=MatchLowercase}
  \defaultfontfeatures[\rmfamily]{Ligatures=TeX,Scale=1}
\fi
% Use upquote if available, for straight quotes in verbatim environments
\IfFileExists{upquote.sty}{\usepackage{upquote}}{}
\IfFileExists{microtype.sty}{% use microtype if available
  \usepackage[]{microtype}
  \UseMicrotypeSet[protrusion]{basicmath} % disable protrusion for tt fonts
}{}
\usepackage{xcolor}
\IfFileExists{xurl.sty}{\usepackage{xurl}}{} % add URL line breaks if available
\IfFileExists{bookmark.sty}{\usepackage{bookmark}}{\usepackage{hyperref}}
\hypersetup{
  pdftitle={Notes on},
  pdfauthor={Marcello Di Bello and Rafal Urbaniak},
  colorlinks=true,
  linkcolor=Maroon,
  filecolor=Maroon,
  citecolor=Blue,
  urlcolor=blue,
  pdfcreator={LaTeX via pandoc}}
\urlstyle{same} % disable monospaced font for URLs
\usepackage{graphicx}
\makeatletter
\def\maxwidth{\ifdim\Gin@nat@width>\linewidth\linewidth\else\Gin@nat@width\fi}
\def\maxheight{\ifdim\Gin@nat@height>\textheight\textheight\else\Gin@nat@height\fi}
\makeatother
% Scale images if necessary, so that they will not overflow the page
% margins by default, and it is still possible to overwrite the defaults
% using explicit options in \includegraphics[width, height, ...]{}
\setkeys{Gin}{width=\maxwidth,height=\maxheight,keepaspectratio}
% Set default figure placement to htbp
\makeatletter
\def\fps@figure{htbp}
\makeatother
\setlength{\emergencystretch}{3em} % prevent overfull lines
\providecommand{\tightlist}{%
  \setlength{\itemsep}{0pt}\setlength{\parskip}{0pt}}
\setcounter{secnumdepth}{5}
\usepackage{booktabs}
\usepackage{longtable}
\usepackage{array}
\usepackage{multirow}
\usepackage{wrapfig}
\usepackage{float}
\usepackage{colortbl}
\usepackage{pdflscape}
\usepackage{tabu}
\usepackage{threeparttable}
\usepackage{threeparttablex}
\usepackage[normalem]{ulem}
\usepackage{makecell}
\usepackage{xcolor}
\ifluatex
  \usepackage{selnolig}  % disable illegal ligatures
\fi

\title{Notes on}
\usepackage{etoolbox}
\makeatletter
\providecommand{\subtitle}[1]{% add subtitle to \maketitle
  \apptocmd{\@title}{\par {\large #1 \par}}{}{}
}
\makeatother
\subtitle{Chapter 5: Assessing evidential strength with the likelihood
ratio}
\author{Marcello Di Bello and Rafal Urbaniak}
\date{}

\begin{document}
\maketitle

\hypertarget{paper-likelihood-ratio-as-weight-of-forensic-evidence-a-closer-look-by-lund-and-iyer}{%
\section{paper: ``Likelihood Ratio as Weight of Forensic Evidence: A
Closer Look'' by Lund and
Iyer}\label{paper-likelihood-ratio-as-weight-of-forensic-evidence-a-closer-look-by-lund-and-iyer}}

It is relevant for our work in three ways:

\begin{enumerate}
\def\labelenumi{\arabic{enumi})}
\item
  The authors show the LR varies depending on priors in virtually any
  realistic case of forensic identification evidence. So I think we
  might need to revise some of the claims we make in the chapter on
  likelihood ratios.
\item
  They also argue that likelihood ratios are dependent on specific
  modelling assumptions and so they cannot be given alone, but in a
  range of possible values paired with modelling assumptions, against
  claims made by Taroni etc. I am wonderin whether Rafal's approach to
  ``weight of evidence'' can capture some of the observations they make
  about the need to take into account modeling assumptions.
\item
  They give two realistic examples of forensic identification evidence
  -- glass evidence and fingerprints -- which are quite instructive and
  described with sophistication. I think we would benefit from
  incorporating their examples somewhere.
\end{enumerate}

Rafal's response (as discussed August 18, 2022):

\begin{itemize}
\tightlist
\item
  mere equivalence does not entail epistemic ordering
\item
  toy models don't usually employ the disjuncts (think about citizens in
  a city), if they do, separate LRs are given and dependence is
  indicated (like, DNA match depending on race)
\item
  putting together with LR is easy in toy model BNs which are modular
\end{itemize}

\end{document}
